\begin{center}\section*{A}
\end{center}
\addcontentsline{toc}{section}{A}

% Stile per i termini del glossario
\newcommand{\term}[1]{%
  \par % chiude il paragrafo precedente
  \vspace{1.2em} % spazio prima del termine
  \begin{center}
    \textbf{\large #1}
  \end{center}
  \vspace{0.4em} % piccolo spazio dopo il termine
  \noindent % inizio della definizione senza rientro
}


\term{Amministratore}
	L'Amministratore si occupa della configurazione e gestione dell'infrastruttura IT di supporto al progetto.
	Il suo ruolo è particolarmente importante nelle fasi iniziali e durante il deployment, dove raggiunge il picco di impegno per garantire un deploy corretto.
	Con il progredire del progetto, il suo contributo diminuisce man mano che i membri del gruppo diventano autonomi nell'uso degli strumenti predisposti.
\term{Analisi dei Requisiti}
    Documento contenente i Casi d'Uso identificati da un Team per i requisiti di un determinato progetto.
\term{Analista}
	L'Analista è cruciale durante le fasi iniziali del progetto; si occupa di identificare e chiarire i requisiti, interpretando le esigenze degli utilizzatori finali per garantire una corretta definizione delle funzionalità.\\
	L'analisi viene svolta in collaborazione con l'azienda proponente e successivamente rielaborata dal gruppo, che redigerà il documento dei requisiti.\\
	Con l’avanzare del progetto, il monte ore dedicato a questo ruolo diminuirà, pur restando attivo per eventuali aggiornamenti o adattamenti dei requisiti in base al confronto con il proponente.
\term{Attore}
  	Un'entità esterna e non controllabile dal progetto, ma che interagisce con esso, con attività o obiettivi specifici da soddisfare.
\term{Automazione}
	Esecuzione deterministica e ripetibile di processi (build, test, deploy, verifiche, provisioning) attivati da eventi del repository o da pianificazioni, 
	definita tramite workflow dichiarativi in YAML, eseguita in ambienti isolati, senza intervento manuale.
\term{Azione}
  	L'insieme di processi intrapresi per mettere in pratica una determinata Decisione.




