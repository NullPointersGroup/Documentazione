\begin{center}
\section*{V}
\end{center}
\addcontentsline{toc}{section}{V}

\term{Validazione}
Insieme di controlli per assicurarsi che un prodotto (o una parte di esso) soddisfi in modo congruo ciò per cui è stato creato.

\term{VE}
Acronimo di \textit{Verbale Esterno}, viene utilizzato per tracciare nel tempo decisioni e azioni intraprese in un incontro esterno. Il codice associato ai VE segue la forma \texttt{VE x.y}, dove \texttt{x} identifica il numero progressivo del verbale e \texttt{y} identifica una specifica decisione o voce registrata all’interno del verbale.

\term{Verifica}
Insieme di processi atti a garantire che una funzione all'interno di un prodotto soddisfi determinati requisiti.

\term{Verificatore}
Il Verificatore si occupa di assicurare la qualità dei prodotti e dei processi adottati, effettuando revisioni e test.

\term{Versionamento}
Metodo per gestire e tenere traccia di documenti, codice o altri artefatti digitali, organizzandoli in versioni identificate da un codice univoco, in modo da poter monitorare modifiche, confrontare stati diversi e ripristinare versioni precedenti.

\term{VI}
Acronimo di \textit{Verbale Interno}, viene utilizzato per tracciare nel tempo decisioni e azioni intraprese in un incontro interno. Il codice associato ai VI segue la forma \texttt{VI x.y}, dove \texttt{x} identifica il numero progressivo del verbale e \texttt{y} identifica una specifica decisione o voce registrata all’interno del verbale.

\term{ViT}
Acronimo di \textit{Vision Transformer}, modello di visione artificiale basato sull’architettura Transformer che consente l’analisi del contenuto visivo delle immagini tramite rappresentazioni sequenziali. Nel contesto di sistemi multimodali, i ViT possono essere utilizzati per estrarre feature visive da integrare con informazioni testuali o audio.
