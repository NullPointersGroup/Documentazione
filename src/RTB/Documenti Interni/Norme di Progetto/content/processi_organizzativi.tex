\section{Processi Organizzativi}
I processi organizzativi definiscono un insieme di operazioni di supporto per lo sviluppo software che operano trasversalmente rispetto al ciclo di vita del software garantendo che il gruppo possieda l’organizzazione, le infrastrutture e le competenze necessarie per sostenere i processi primari.
Assicurano la buona esecuzione di tutti i processi adottati e eventuali miglioramenti.\\
Si individuano i seguenti processi:
\begin{itemize}
    \item Gestione dei Processi;
    \item Gestione dell'Infrastruttura;
    \item Processo di Miglioramento;
    \item Processo di Formazione.
\end{itemize}

\subsection{Gestione dei Processi}
Secondo lo standard ISO 12207:1995, \vr{La gestione dei processi comprende le attività e i compiti che possono essere svolti da qualsiasi soggetto che debba gestire i propri processi}.\\
Sulla base di questo principio, il suo scopo principale è stabilire come un processo deve essere pianificato e monitorato secondo le relative responsabilità dei membri del gruppo.\\
Un altro obiettivo fondamentale è garantire un flusso comunicativo efficace, sia interno che esterno assicurando: coerenza, controllo e miglioramento continuo.
 \subsubsection{Attività previste}
    \subsubsubsection{Inizializzazione}
    L’avvio del processo avviene tramite la selezione dei requisiti, presenti nel documento Analisi dei requisiti, da portare a termine tramite le attività di tale processo.\\
    Il responsabile valuta preliminarmente la fattibilità del processo: se alcuni requisiti risultano irrealizzabili per vincoli di tempo, risorse o competenze, e previo accordo di tutto il gruppo, i requisiti del processo possono essere modificati in questa fase per garantire il raggiungimento dei criteri di completamento.
    \subsubsubsection{Pianificazione}
    L’attività di pianificazione, portata a termine dal responsabile, ha lo scopo di preparare il piano di esecuzione delle attività del processo. 
    In particolare deve verificare la disponibilità delle risorse necessarie (budget residuo, della disponibilità dei componenti del gruppo, competenza, etc.) per completare il processo entro i tempi stabiliti.\\
    Infine assegna le attività del processo ai membri del team in base ai loro ruoli.
    \subsubsubsection{Esecuzione e controllo}
    Durante l’esecuzione: i membri del team portano a termine le attività assegnategli, mentre il responsabile ha il compito di monitorare l’andamento delle attività.\\
    Qualora si presentassero problemi, il responsabile deve essere immediatamente notificato e in caso di stallo, contattare la proponente o il committente per un chiarimento.
    \subsubsubsection{Verifica}
    La verifica del prodotto realizzato dai membri del team è di competenza del verificatore, che assicura la corretta realizzazione del singolo requisito tramite la procedura di verifica definita dall'analisi dei requisiti. 
    \subsubsubsection{Chiusura}
    La chiusura del processo consegue la terminazione di tutte le attività che ne hanno preso parte.
    È compito del responsabile approvare il merge della pull request nel branch main; una volta fatto il merge, il processo è da definirsi chiuso. %da chiarire se è il responsabile 

\subsubsection{Ruoli}

\begin{table}[h!]
\centering
\begin{tabular}{ |p{0.25\textwidth}|p{0.65\textwidth}|}
    \hline
    \textbf{Ruolo} & \textbf{Compiti} \\
    \hline
    Responsabile & Il Responsabile coordina le attività del gruppo garantendo una pianificazione efficace. \\
    \hline
    Amministicatore & L’Amministratore si occupa della configurazione e gestione dell’infrastruttura IT di supporto al progetto. \\
    \hline
    Analista & L’Analista si occupa di identificare e chiarire i requisiti, interpretando le esigenze degli utilizzatori finali per garantire una corretta definizione delle funzionalità. \\
    \hline
    Verificatore & Il Verificatore si occupa di assicurare la qualità dei prodotti e dei processi adottati, effettuando revisioni e test. \\
    \hline
    Programmatore & Il Programmatore è responsabile dello sviluppo del codice sorgente del progetto, traducendo il design in codice funzionante e testabile dal proponente. \\
    \hline
    Progettista & Il Progettista traduce i requisiti del sistema in un’architettura software dettagliata, definendo moduli, interfacce, flussi dati e vincoli tecnici. \\
    \hline
\end{tabular}
\end{table}

\subsection{Gestione dell'infrastruttura}
Il processo di infrastruttura ha lo scopo di fornire, configurare e mantenere l’ambiente di lavoro necessario all'esecuzione di tutti i processi di sviluppo e documentazione. 
Esso comprende la gestione delle risorse, siano esse hardware o software, garantendone la disponibilità e l'efficienza per l'intera durata del progetto.
\subsubsection{Attività previste}
\subsubsubsection{Implementazione}
Per supportare il lavoro asincrono, la tracciabilità e la qualità dei prodotto NullPointers Group adotta i seguenti strumenti che costituiscono l’infrastruttura del progetto:
\begin{itemize}
    \item Gestione del versionamento: Git
    \item Piattaforma: GitHub
    \item Automazione: GitHub Actions e Script Python e Lua
    \item Comunicazione: Discord e Whatsapp
\end{itemize}

\subsubsubsection{Predisposizione}
L’attività di predisposizione stabilisce le regole di interazione tra i membri del gruppo e l’ambiente di lavoro, inoltre definisce la natura dell’infrastruttura utilizzata. 
L’infrastruttura adottata è finalizzata a minimizzare gli errori e a garantire la coerenza del prodotto. Vengono riportati gli strumenti principali:

\begin{table}[!h]
\centering
\begin{tabular}{ |p{0.25\textwidth}|p{0.65\textwidth}|}
    \hline
    \textbf{Strumento} & \textbf{Predisposizione} \\
    \hline
    Git & Definizione di un file .gitignore condiviso per escludere i file temporanei e di build garantendo che la repository contenga solamente i file sorgente. \\
    \hline
    GitHub & È stata creata una repository dedicata alla documentazione del progetto. È stata applicata una  branch protection rule sul ramo main: ogni modifica deve provenire da una pull request e richiede l’approvazione di un Verificatore per il suo merge.\\
    \hline
    Labels GitHub & Sono state implementate delle Labels per categorizzare le attività e Milestones per tracciare l’avanzamento del progetto.\\
    \hline
    GitHub Actions e Script Python & Sono state configurate le GitHub Actions per l’esecuzione automatica degli script che compilano i file sorgente LaTeX ad ogni push garantendo che la versione PDF visibile sul sito sia sempre sincronizzata con l’ultima versione dei documenti. \\
    \hline
    Discord e Whatsapp & Per consentire al gruppo di riunirsi settimanalmente, e venire in contro al fatto che ci sono significative distanze tra i membri, è stato creato un server sulla piattaforma Discord. Un’applicazione che consente videochiamate e scambio di messaggi; ideale per il nostro scopo.
                            È stato inoltre creato un gruppo Whatsapp per questioni minori che non richiedono una videochiamata. \\
    \hline
\end{tabular}
\end{table}

\newpage
\subsubsubsection{Manutenzione}
Data la complessità del progetto è probabile che l’infrastruttura subisca dei cambiamenti nel corso del tempo per l’aggiornamento o il miglioramento delle sue funzionalità. 
E’ compito dell’amministratore la manutenzione dell’infrastruttura ovvero le attività di controllo delle funzionalità ed aggiornamento/creazione degli script di automazione.\\
Successivamente verranno illustrate le norme da seguire per mantenere e aggiornare l’infrastruttura affinché il flusso di lavoro non venga spezzato:

\subsubsubsubsection{Git}
Git non ha bisogno di particolari configurazioni, è sufficiente accedere localmente con le credenziali che il membro usa per accedere a Github.

\subsubsubsubsection{GitHub}
Su Github, l’account di NullPointers Group è gestito come organizzazione, ovvero un account che serve da contenitore per il lavoro condiviso tra membri di un team.
Sono state create 3 repository dentro l’organizzazione:
\begin{itemize}
    \item \textbf{Documentazione: }Repository dove viene salvata e versionata tutta la documentazione in merito al capitolato SmartOrder e non solo.
    \item \textbf{SmartOrder: }Repository dove viene salvato e versionato il codice sorgente dell’applicativo SmartOrder.
    \item \textbf{Test: }Repository che i sviluppatori ed amministratori utilizzano per la sperimentazione di nuove impostazioni di infrastruttura e nuove automazioni.
\end{itemize}
L’amministratore sperimenta le nuove impostazioni di infrastruttura nella repository di test.\\
Affinché le nuove impostazioni vengano effettivamente applicate nelle repository di documentazione o di codice sorgente, il verificatore dovrà assicurarsi che la nuova infrastruttura proposta superi le metriche di qualità di processo definite in seguito.

\subsubsubsubsection{Actions e script ausiliari}
Nella repository di Documentazione e SmartOrder, si impiegano strumenti di CI messi a disposizione da Github: le Github Actions, le quali vengono definite in un file “.yml” dentro la cartella “.github/workflows”.\\
Affinché le Github Actions portassero a termine lo scopo per cui sono state configurate, sono stati sviluppati script ausiliari, in Python e Lua, da far eseguire a quest’ultime.\\
La configurazione di nuove Github Actions o la modifica di Github Action esistenti spetta agli amministratori, che le sperimentano, come accennato in precedenza, nella repo di test.\\
La creazione di nuove Github Actions viene richiesta agli amministratori dal responsabile, sotto comune accordo dai membri del gruppo.\\
La creazione o la modifica di eventuali script da far eseguire alle Github Actions sarà compito degli sviluppatori, anche questi, in collaborazione con gli amministratori, sperimenteranno nella repository di Test.

\subsubsubsubsection{Issue, Project Board, Labels}
%da fare

\subsubsubsubsection{Discord}
Nel server discord sono presenti canali di comunicazione testuali suddivisi per argomento di discussione, principalmente documentazione e dubbi sul codice sorgente. 
Qualora vi sia necessità, è compito dell’amministratore, su richiesta del gruppo, di aggiungere eventuali canali testuali. Questi dovranno essere accessibili da tutti i membri del gruppo, la moderazione è affidata all’amministratore.

\subsection{Processo di Miglioramento}
Il processo di miglioramento ha lo scopo di analizzare l’efficacia e l'efficienza dei processi adottati, identificando le aree di debolezza ed implementando azioni correttive e preventive per l’ottimizzazione dell’esecuzione. 
%da continuare
\subsection{Processo di Formazione}
Il processo di formazione è un’attività di supporto volta a garantire che tutti i membri del gruppo possiedano le competenze necessarie per svolgere i compiti assegnati e gestire le tecnologie richieste dal progetto.
%da continuare