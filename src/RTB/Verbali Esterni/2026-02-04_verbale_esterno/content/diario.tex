\section{Diario della riunione}
\subsection{Test di Accettazione}
Il gruppo ha presentato i test di Accettazione elaborati. L'azienda ha approvato tutti i test proposti, tuttavia è emersa la necessità di riformulare il test \textbf{TA\_13} per migliorarne la comprensione e la chiarezza. 

\subsection{Presentazione del Proof of Concept}
Il gruppo ha mostrato i progressi del PoC realizzato fino a questo momento.
L'azienda ha valutato positivamente il lavoro svolto, commentando che il tempo di risposta da noi ritenuto problematico rientra nei limiti accettabili.
L'azienda ha inoltre suggerito di rendere visibile la query generata dall'agente per avere un maggior controllo sul processo e migliorarne l'ottimizzazione, consentendo inoltre di controllare se l'agente genera una singola query o query multiple.

\subsection{Dubbi riguardanti le tecnologie}
Il gruppo ha esposto alcuni dubbi emersi durante lo sviluppo del Proof of Concept.
\subsubsection{LangChain e utilizzo degli agenti}
Il gruppo ha posto domande relative all'utilizzo di LangChain all'interno del progetto e sulla necessità di implementare agenti multipli o un singolo agente.\\
L'azienda ha chiarito che per l'utilizzo che viene fatto all'interno del progetto un solo agente è sufficiente e non è necessario implementare un sistema multi-agente più complesso.

\subsubsection{Utilizzo del database vettoriale}
Il gruppo ha sollevato il dubbio se l'utilizzo di LangChain renda superfluo il database vettoriale.\\
L'azienda ha spiegato che esistono due approcci validi:
\begin{itemize}
    \item Utilizzare un database relazionale classico con un agente che costruisce query per interrogare il database
    \item Utilizzare un database vettoriale che semplifica la gestione della cronologia e garantisce maggiore coerenza nelle risposte
\end{itemize}
L'azienda ha confermato che attualmente il database vettoriale rimane la scelta più frequente, ma l'approccio con agente e database relazionale è altrettanto valido per il progetto.

\subsection{Ottimizzazione e diagnostica}
L'azienda ha suggerito l'aggiunta di indici anche sulle operazioni e richieste effettuate più frequentemente per migliorare le performance del sistema. 
È stato inoltre consigliato l'utilizzo di profiler per il database PostgreSQL al fine di effettuare diagnostica e analizzare le query che vengono eseguite.