% Configurazione
\documentclass{article} 

\usepackage{titling} % Required for inserting the subtitle
\usepackage{graphicx} % Required for inserting images
\usepackage{tabularx} % Per l'ambiente tabularx (tabelle)
\usepackage{calc} % Sempre per le tabelle
\usepackage[hidelinks]{hyperref} % Per i collegamenti ipertestuali, ad esempio sulla table of contents
\usepackage{xcolor} % Per colorare il testo
\usepackage{colortbl} % Per colorare le celle delle tabelle
\usepackage{lipsum} % Per generare lorem ipsum
\usepackage[normalem]{ulem} % Per sottolineare il testo
\usepackage{array} % Per la visualizzazione fluttuante di array di domande e risposte
\usepackage{ragged2e} % Pacchetto necessario per \justifying che giustifica il testo di tabelle
\usepackage{placeins}
\usepackage{hyperref}
\usepackage{xcolor}
\usepackage{fancyhdr} % Per l'intestazione e piè di pagina
\usepackage{lastpage} % Per ottenere il numero totale di pagine

\newcommand{\ulhref}[2]{\href{#1}{\uline{#2}}} % Nuovo comando per sottolineare i link
\newcommand{\ulref}[1]{\uline{\ref{#1}}} % Nuovo comando per sottolineare i collegamenti a immagini
\setlength{\parindent}{0pt} % Rimuove il rientro automatico dei paragrafi

\graphicspath{ {immagini/} {../shared/images} }

% Configurazione piè di pagina
\pagestyle{fancy}
\fancyhf{} % Cancella intestazione e piè di pagina predefiniti
\renewcommand{\headrulewidth}{0pt} % Rimuove la linea nell'intestazione
\fancyfoot[C]{\thepage} % Numero di pagina centrato in fondo

% Struttura
\begin{document}
\vspace*{0.05\textheight}
    \begin{center}
	\begin{minipage}{0.4\textwidth}
		\centering
		\includegraphics[width=0.7\textwidth]{logo_gruppo}
	\end{minipage}
	\begin{minipage}{0.45\textwidth}
		\centering
		\textbf{NullPointers Group} \\
		\textsf{nullpointersg@gmail.com}
	\end{minipage}
	\end{center}
    
    \vspace{2cm}
    
    {
        \centering
        \Huge\bfseries Presentazione RTB\par
        \vspace{0.5cm}
    }

    \begin{center}
        \vspace{3.5em}
        {\LARGE\textbf{Componenti}}\\
		\textcolor{gray}{\rule{4.5cm}{0.4pt}}\\
        \vspace{2em}
        \begin{tabular}{l l l}
			Brunello & Marco & 2110997 \\[0.5em]
			Casagrande & Lisa & 2116440 \\[0.5em]
            Ceron & Tommaso & 2101045 \\[0.5em]
            Marcuzzo & Luca & 2113198 \\[0.5em]
            Mazzaretto & Matteo & 2111005 \\[0.5em]
            Pieripolli & Laura & 2048057 \\[0.5em]
        \end{tabular}
    \end{center}

    \newpage
    
    \begin{center}
        \begin{minipage}{1.0\textwidth}
            Destinatari:\\
            Prof. Vardanega Tullio,\\
            Prof. Cardin Riccardo.\\
        \end{minipage}
    \end{center}
    
    Egregi Professori Vardanega e Cardin,\\
    con il presente documento il team \textbf{NullPointers Group} desidera comunicarVi formalmente la propria intenzione di sottoporsi alla revisione della Requirements and Technology Baseline.\\
	La revisione riguarderà il progetto sviluppato per il capitolato C8 (SmartOrder: Analisi multimodale per la creazione automatica di ordini) proposto da Ergon Informatica Srl:
    
    \begin{center}
        \url{https://www.math.unipd.it/~tullio/IS-1/2025/Progetto/C8.pdf}
    \end{center}

    Nel corso della fase di sviluppo, \textbf{NullPointers Group} ha realizzato un \textit{Proof of Concept}, reso disponibile al seguente link:
    
    \begin{center}
        \url{https://github.com/NullPointersGroup/PoC}
    \end{center}

    Tutta la documentazione prodotta per la RTB è consultabile e sempre aggiornata nella relativa sezione del nostro sito: 
    \begin{center}
        \url{https://nullpointersgroup.github.io/Documentazione/}
    \end{center}

    All'interno si possono trovare, tra gli altri, i seguenti documenti:
    \begin{itemize}
        \item Lettera di presentazione;
        \item Piano di Qualifica v1.0.0
        \item Piano di Progetto v1.0.0
        \item Analisi dei Requisiti v1.0.0
        \item Norme di Progetto v1.0.0
        \item Glossario v1.0.0
    \end{itemize}

    Infine, riconfermiamo la data ultima di consegna del progetto al \textbf{30 Aprile 2026}, con costo stimato massimo di \textbf{€11.440}.\\[0.5cm]

    Cordiali Saluti,\\[0.3cm]
    NullPointers Group
    
\end{document}
