\section{Decisioni e Azioni}

\begin{table}[h]
	\centering
	\renewcommand{\arraystretch}{1.15}
	\resizebox{\textwidth}{!}{
		\begin{tabular}{|p{0.12\textwidth}|p{0.65\textwidth}|}
			\hline
			\rowcolor[gray]{0.9}
			Codice & Descrizione \\
			\hline
			VI$^G$ 15.1 & Creazione repository dedicata in vista del PoC.\\
			\hline
			VI$^G$ 15.2 & Dichiarare eventuali casi d'uso derivanti da altri casi d'uso.\\
			\hline
			VI$^G$ 15.3 & Utilizzare sempre elenco puntato per sezioni "Pre-Condizioni" e "Post-Condizioni".\\
			\hline
			VI$^G$ 15.4 & Utilizzare elenco numerato per la sezione "Sezione Principale".\\
			\hline
			VI$^G$ 15.5 & Utilizzare il comando latex "hyperref" per i collegamenti ipertestuali".\\
			\hline
			VI$^G$ 15.6 & Utilizzare sempre caption per l'immagine raffigurante lo scema UML del caso d'uso.\\
			\hline
			\href{https://github$^G$.com/NullPointersGroup/Documentazione/issues/108}{SMD$^G$ 22} & Continuazione Analisi di Requisiti: sezione Casi d'Uso (1-6). \\
			\hline
			\href{https://github$^G$.com/NullPointersGroup/Documentazione/issues/109}{SMD$^G$ 23} & Continuazione Analisi di Requisiti: sezione Casi d'Uso (7-18). \\
			\hline
			\href{https://github$^G$.com/NullPointersGroup/Documentazione/issues/110}{SMD$^G$ 24} & Continuazione Analisi di Requisiti: sezione Casi d'Uso (19-29). \\
			\hline
			\href{https://github$^G$.com/NullPointersGroup/Documentazione/issues/111}{SMD$^G$ 25} & Continuazione Analisi di Requisiti: sezione Casi d'Uso (36-43). \\
			\hline
			\href{https://github$^G$.com/NullPointersGroup/Documentazione/issues/112}{SMD$^G$ 26} & Continuazione Analisi di Requisiti: sezione Casi d'Uso (44-51). \\
			\hline
		\end{tabular}
	}
	\vspace{0.3cm}
\end{table}


Qualsiasi modifica$^G$ alle issue$^G$ relativa all'Analisi dei Requisiti$^G$, dovute a ricontrolli durante la stesura, saranno tracciate nel sistema di progetto$^G$.
