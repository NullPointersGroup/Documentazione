\section{Diario della riunione}
\subsection{Allineamento tecniche e stili di stesura dei casi d'uso}
Per garantire chiarezza e coerenza nella stesura dei casi d'uso, il gruppo ha concordato convenzioni di stile e regole sull'utilizzo di specifici comandi da applicare ai singoli casi d'uso in presenza di particolari condizioni.\\
In particolare il gruppo ha concordato:
\begin{itemize}
    \item Se un caso d'uso ha delle precondizioni derivanti da un'altro caso d'uso queste devono essere dichiarate nella sezione "Pre-Condizioni" del caso d'uso interessato.
    \item Viene lo stesso utilizzato l'elenco puntato qualora il caso d'uso presenti un singolo elemento nella sezione "Pre-Condizioni" e "Post-Condizioni"
    \item Nella sezione "Scenario Principale" viene usato l'elenco numerato in quanto si tratta di una sequenza ordinata
    \item Eventuali collegamenti ipertestuali ad altri casi d'uso vengono effettuati con il comando latex "hyperref[<label uc>]\{UC\_XX\}"
    \item L'immagine che raffigura lo schema UML del caso d'uso deve sempre avere una caption, in particolare deve avere il formato "UC\_XX: Descrizione" se si sta trattando un singolo caso d'uso; "Inclusioni / Estensioni UC\_XX: <elenco casi d'uso>" qualora si sta trattando un caso d'uso che ha una relazione di inclusione/estensione con un altro caso d'uso.
\end{itemize}

\subsection{Discussione e pianificazione PoC}
Nel corso della riunione ci si è prefissati di iniziare il PoC il giorno 12 gennaio, a seguito dell'incontro formativo con l'azienda.\\
In previsione del PoC il gruppo ha deciso di creare un repository dedicato.

\subsection{Rotazione ruoli}
In questo sprint$^G$, fino al 12/01/2026, i ruoli saranno i seguenti: \\
\begin{center}
	Marco Brunello $\rightarrow$ Responsabile$^G$ \\
	Tommaso Ceron $\rightarrow$ Analista$^G$ \\
	Matteo Mazzareto $\rightarrow$ Analista$^G$ \\
	Laura Pieripolli $\rightarrow$ Analista$^G$\\
	Lisa Casagrande $\rightarrow$ Analista$^G$, Verificatore$^G$ \\
	Luca Marcuzzo $\rightarrow$ Analista$^G$, Verificatore$^G$ \\
\end{center}