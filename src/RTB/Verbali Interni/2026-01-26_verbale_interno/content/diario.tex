\section{Diario della riunione}
\subsection{Revisione Test di Accettazione}
Il gruppo ha esaminato i test$^G$ di Accettazione precedentemente definiti, verificandone la completezza e la coerenza con i requisiti funzionali.
I test$^G$ coprono le principali funzionalità$^G$ del sistema, inclusi input testuale e vocale, gestione ordini, autenticazione, gestione carrello, gestione errori e persistenza della memoria dell'AI. \\
È stato inoltre pianificato un incontro con la proponente$^G$ per discutere e validare tali test$^G$, verificando che siano adatti ad accertare la completezza del prodotto finale.

\subsection{Accertamento stato di avanzamento}
Nel precedente sprint$^G$ il gruppo aveva pianificato di iniziare le attività di programmazione riguardanti il PoC$^G$ prima del giorno odierno.
Tuttavia il gruppo ha riscontrato un rallentamento causato dalla sessione invernale, rimandando così l'inizio delle attività di programmazione del PoC$^G$ all'attuale sprint$^G$.\\
Di conseguenza, si è deciso di posticipare la data di presenzazione del PoC$^G$ per il 16 febbraio.

\subsection{Rotazione ruoli}
In questo sprint$^G$, fino al 09/02/2026, i ruoli saranno i seguenti:
\begin{center}
	Laura Pieripolli $\rightarrow$ Responsabile$^G$ \\
	Tommaso Ceron $\rightarrow$ Programmatore$^G$, Verificatore$^G$ \\
	Matteo Mazzaretto $\rightarrow$ Programmatore\\
	Marco Brunello $\rightarrow$ Programmatore$^G$ \\
	Lisa Casagrande $\rightarrow$ Programmatore$^G$ \\
	Luca Marcuzzo $\rightarrow$ Programmatore$^G$ \\
\end{center}
