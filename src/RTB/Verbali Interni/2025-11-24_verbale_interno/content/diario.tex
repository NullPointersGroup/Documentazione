\section{Diario della riunione}
\subsection{Nuove task e modifica incarico}
A segiuto di un esame delle task da completare è emersa la necessità di un analista$^G$ aggiuntivo per l’analisi dei casi d’uso. Di conseguenza, il ruolo di Analista sarà ricoperto anche da:
\begin{center}
Matteo Mazzaretto $\rightarrow$ Analista
\end{center}

\subsection{Workflow e Organizzazione}
Il gruppo ha deciso di organizzare il tracciamento delle attività$^G$ mediante la creazione di issue$^G$ etichettate in base al tipo di documento da produrre. \\
Ogni issue sarà inoltre associata a una milestone$^G$, corrispondente allo sprint di riferimento, così da garantire una pianificazione ordinata e una visione temporale delle consegne.
Parallelamente, nei project board$^G$ verranno inserite le attività raggruppate rispetto alle due principali scadenze previste, identificate come RTB$^G$ e PB$^G$.\\
In questo modo il project board fungerà da quadro di sintesi delle attività critiche e delle tappe fondamentali del progetto, mentre la gestione delle issue consentirà il monitoraggio 
delle singole produzioni documentali.

\subsection{Definizione ruoli per redazione dei documenti}
Sono stati definiti i ruoli per la redazione dei documenti che verranno prodotti nella PB, è stato stabilito quanto segue: 
\begin{center}
\begin{tabular}{r c l}
Diario di Bordo & $\rightarrow$ & Responsabile$^G$ \\
Verbali Esterni & $\rightarrow$ & Responsabile \\
Verbali Interni & $\rightarrow$ & Responsabile \\
Piano di Progetto & $\rightarrow$ & Responsabile \\
Norme di Progetto & $\rightarrow$ & Amministratore$^G$ \\
Piano di Qualifica & $\rightarrow$ & Amministratore \\
Analisi dei Requisiti & $\rightarrow$ & Analista \\
Glossario & $\rightarrow$ & Analista \\
\end{tabular}
\end{center}

Tale decisione non è retroattiva, dunque si applica a seguito dell'approvazione di questo documento.