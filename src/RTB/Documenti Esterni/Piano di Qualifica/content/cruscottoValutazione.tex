\section{Cruscotto di Valutazione}

\subsection{MQC\_01: Earned Value e MQC\_02: Planned Value}
\begin{figure}[H]
	\centering 
	\includegraphics[width=0.8\textwidth]{PianoQualifica/EarnedPlanned}
	\caption{MQC\_01: Earned Value e MQC\_02: Planned Value}
\end{figure}

Dal grafico si nota l'abilità del gruppo nel completare tutte le attività individuate ogni sprint, portando alla coincidenza fra le metriche \vr{Earned Value} e \vr{Planned Value}.\\
Comunque si nota un progressivo aumento del lavoro pianificato da svolgere, escluso per gli ultimi sprint che hanno denotato una diminuzione del carico del lavoro dovuto alla sessione invernale.

\subsection{MQC\_03: Actual Cost e MQC\_07: Estimate to Complete}
\begin{figure}[H]
	\centering 
	\includegraphics[width=0.8\textwidth]{PianoQualifica/ActualEstimate}
	\caption{MQC\_03: Actual Cost e MQC\_07: Estimate to Complete}
\end{figure}
Dal grafico si nota che il gruppo è riuscito sempre a tenersi sotto la soglia definita dalla metrica \vr{Estimate at Completion}, evidenziando la capacità di gestire i costi.\\
All'inizio grafico si può vedere come la differenza fra i preventivi e i consuntivi sia evidente, ma nel corso del tempo la capacità del gruppo di stendere i preventivi è migliorato. Questo fatto è documentato nel \href{https://nullpointersgroup.github.io/Documentazione/}{Piano di Progetto}.

\subsection{MQC\_04: Cost Performance Index e MQC\_05: Schedule Performance Index}
\begin{figure}[H]
	\centering 
	\includegraphics[width=0.8\textwidth]{PianoQualifica/CostSchedulePerformance}
	\caption{MQC\_04: Cost Performance Index e MQC\_05: Schedule Performance Index}
\end{figure}
Si nota come il Cost Performance Index sia sempre sopra il valore ottimo, mentre lo Schedule Performance Index sia uguale al valore ottimo, denotando la bravura del gruppo nel riuscire a restare sotto la soglia del budget e non rischiando mai, secondo i dati, di superare a fine progetto il budget.

\subsection{MQC\_06: Estimate at Completion}
\begin{figure}[H]
	\centering 
	\includegraphics[width=0.8\textwidth]{PianoQualifica/EstimateAtCompletion}
	\caption{MQC\_06: Estimate at Completion}
\end{figure}
Il costo stimato di fine progetto si attesta sempre sotto la soglia del budget definito in sede di Candidatura.
Il picco del costo stimato è stato raggiunto allo Sprint 5, periodo in cui il gruppo era impegnato a definire correttamente l'Analisi dei Requisiti.

\subsection{MQC\_08: Time Estimate at Completion}
\begin{figure}[H]
	\centering 
	\includegraphics[width=0.8\textwidth]{PianoQualifica/TimeEstimate}
	\caption{MQC\_08: Time Estimate at Completion}
\end{figure}
Siccome lo Schedule Performance Index si è sempre attenuto al valore 1, anche il tempo stimato per la conclusione del progetto non si è spostato dal valore 176, ovvero i giorni che intercorrono fra la data di aggiudicazione appalto e la data di fine prevista del progetto, cioè il 30 aprile.


\subsection{MQC\_09: Completeness Issue}
\begin{figure}[H]
	\centering 
	\includegraphics[width=0.8\textwidth]{PianoQualifica/Completeness}
	\caption{MQC\_09: Completeness Issue}
\end{figure}
Il gruppo nel corso del progetto è riuscito sempre a chiudere nel periodo di sprint le attività delineate, evidenziando la capacità di assegnare e distribuire correttamente le attività fra i membri del gruppo.


\subsection{MQC\_10: Indice di Gulpease}
\subsubsection{Indice di Gulpease Analisi dei Requisiti}
\begin{figure}[H]
	\centering 
	\includegraphics[width=0.8\textwidth]{PianoQualifica/GulpeaseAnalisi}
	\caption{Indice di Gulpease Analisi dei Requisiti}
\end{figure}
L'indice di Gulpease dell'Analisi dei Requisiti non è mai riuscito a superare la soglia del valore ottimo, probabilmente determinato dal fatto che il documento ha molte figure, tabelle ed elenchi, i quali vengono esclusi dallo script di calcolo per l'indice, in quanto non rilevanti.\\
Infatti il contenuto puramente testuale è molto limitato ma descrittivo e verboso, determinando la non capacità di superare il valore ottimo.

\subsubsection{Indice di Gulpease Piano di Qualifica}
\begin{figure}[H]
	\centering 
	\includegraphics[width=0.8\textwidth]{PianoQualifica/GulpeaseQualifica}
	\caption{Indice di Gulpease Piano di Qualifica}
\end{figure}
L'indice di Gulpease del Piano di Qualifica è determinato dalle stesse motivazioni descritte precedentemente, impedendo anche al presente documento di riuscire a superare la soglia ottima definita dal gruppo.

\subsubsection{Indice di Gulpease Piano di Progetto}
%\begin{figure}[H]
%	\centering 
%	\includegraphics[width=0.8\textwidth]{PianoQualifica/GulpeaseProgetto}
%	\caption{Indice di Gulpease Piano di Progetto}
%\end{figure}
Anche l'indice di Gulpease del Piano di Progetto non è ottimo, essendo composto da molti elenchi e tabelle, i quali non vengono conteggiati.

\subsubsection{Indice di Gulpease Norme di Progetto}
\begin{figure}[H]
	\centering 
	\includegraphics[width=0.8\textwidth]{PianoQualifica/GulpeaseNorme}
	\caption{Indice di Gulpease Norme di Progetto}
\end{figure}
Le Norme di Progetto si sono stabilizzate dallo sprint 3 in poi, in quanto in quel periodo il gruppo era riuscito a definire tutte le principali procedure del gruppo e da quel momento ci sono state pochissime modifiche, come dimostrato dal \vr{Registro delle Modifiche} del documento \href{https://nullpointersgroup.github.io/Documentazione/output/RTB/Documenti%20Interni/Norme_di_Progetto.pdf}{Norme di Progetto}

\subsubsection{Indice di Gulpease Glossario}
\begin{figure}[H]
	\centering 
	\includegraphics[width=0.8\textwidth]{PianoQualifica/GulpeaseGlossario}
	\caption{Indice di Gulpease Glossario}
\end{figure}
L'indice di Gulpease del Glossario contiene per definizione termini specialistici e tecnici che sono spesso parole lunghe e composte, abbassando naturalmente l'indice.

\subsection{MQC\_13: Quality Metrics Satisfied}
%\begin{figure}[H]
%	\centering 
%	\includegraphics[width=0.8\textwidth]{PianoQualifica/QualityMetrics}
%	\caption{MQC\_13: Quality Metrics Satisfied}
%\end{figure}
