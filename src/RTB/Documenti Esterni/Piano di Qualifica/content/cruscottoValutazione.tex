\section{Cruscotto di Valutazione}

\subsection{MQC\_01: Earned Value e MQC\_02: Planned Value}
\begin{figure}[H]
	\centering 
	\includegraphics[width=0.8\textwidth]{PianoQualifica/EarnedPlanned}
	\caption{MQC\_01: Earned Value e MQC\_02: Planned Value}
\end{figure}

Dal grafico emerge la capacità del gruppo nel completare tutte le attività previste in ciascuno sprint, portando alla coincidenza fra le metriche \vr{Earned Value} e \vr{Planned Value}.\\
Si osserva in oltre un progressivo aumento del lavoro pianificato da svolgere, fatta eccezione per gli ultimi sprint che hanno denotato una diminuzione del carico del lavoro dovuto alla sessione invernale.

\subsection{MQC\_03: Actual Cost e MQC\_07: Estimate to Complete}
\begin{figure}[H]
	\centering 
	\includegraphics[width=0.8\textwidth]{PianoQualifica/ActualEstimate}
	\caption{MQC\_03: Actual Cost e MQC\_07: Estimate to Complete}
\end{figure}
Dal grafico si nota che il gruppo ha mantenuto i costi effettivi al di sotto della soglia definita dalla metrica \vr{Estimate at Completion}, evidenziando una buona capacità di gestire dei costi.\\
Nella prima fase del progetto si può vedere come la differenza fra i preventivi e i consuntivi sia più evidente, tuttavia nel corso del tempo la capacità di pianificazione del gruppo è migliorata, riducendo progressivamente questo scarto. Questa evoluzione è documentata nel \href{https://nullpointersgroup.github.io/Documentazione/}{Piano di Progetto}.

\subsection{MQC\_04: Cost Performance Index e MQC\_05: Schedule Performance Index}
\begin{figure}[H]
	\centering 
	\includegraphics[width=0.8\textwidth]{PianoQualifica/CostSchedulePerformance}
	\caption{MQC\_04: Cost Performance Index e MQC\_05: Schedule Performance Index}
\end{figure}
Dall'analisi si evince che il Cost Performance Index (CPI) si mantiene stabilmente al di sopra del valore ottimale, mentre lo Schedule Performance Index (SPI) vi coincide perfettamente. Questo risultato testimonia l'efficacia del gruppo nel controllare i costi, mantenendoli sistematicamente al di sotto del budget preventivato e, secondo i dati, senza mai rischiare di superarlo alla conclusione del progetto.

\subsection{MQC\_06: Estimate at Completion}
\begin{figure}[H]
	\centering 
	\includegraphics[width=0.8\textwidth]{PianoQualifica/EstimateAtCompletion}
	\caption{MQC\_06: Estimate at Completion}
\end{figure}
Il costo stimato di fine progetto si attesta sempre sotto la soglia del budget definito in sede di Candidatura.
Il picco del costo stimato è stato raggiunto allo Sprint 5, periodo in cui il gruppo era interamente impegnato nella definizione dell'Analisi dei Requisiti.

\subsection{MQC\_08: Time Estimate at Completion}
\begin{figure}[H]
	\centering 
	\includegraphics[width=0.8\textwidth]{PianoQualifica/TimeEstimate}
	\caption{MQC\_08: Time Estimate at Completion}
\end{figure}
Poiché lo Schedule Performance Index si è sempre mantenuto al valore ottimale di 1, anche la stima del tempo necessario per la conclusione del progetto è rimasta invariata, confermando i 176 giorni previsti tra la data di aggiudicazione dell'appalto e la scadenza del 30 aprile.

\subsection{MQC\_09: Completeness Issue}
\begin{figure}[H]
	\centering 
	\includegraphics[width=0.8\textwidth]{PianoQualifica/Completeness}
	\caption{MQC\_09: Completeness Issue}
\end{figure}
Il gruppo nel corso del progetto è sempre riuscito a completare nel periodo di sprint le attività delineate, evidenziando la capacità di assegnare e distribuire correttamente le attività fra i membri del gruppo.

\subsection{MQC\_10: Indice di Gulpease}
\subsubsection{Indice di Gulpease Analisi dei Requisiti}
\begin{figure}[H]
	\centering 
	\includegraphics[width=0.8\textwidth]{PianoQualifica/GulpeaseAnalisi}
	\caption{Indice di Gulpease Analisi dei Requisiti}
\end{figure}
L'indice di Gulpease dell'Analisi dei Requisiti non ha mai superato la soglia del valore ottimo. Questo è probabilmente dovuto al fatto che il documento contiene molte figure, tabelle ed elenchi, elementi che lo script di calcolo esclude in quanto non pertinenti al computo dell'indice stesso.\\
Di conseguenza, il contenuto puramente testuale è molto limitato seppur descrittivo e dettagliato, determinando l'impossibilità strutturale di raggiungere il valore ottimale della metrica.

\subsubsection{Indice di Gulpease Piano di Qualifica}
\begin{figure}[H]
	\centering 
	\includegraphics[width=0.8\textwidth]{PianoQualifica/GulpeaseQualifica}
	\caption{Indice di Gulpease Piano di Qualifica}
\end{figure}
Per le stesse motivazioni esposte in precedenza, anche l'indice di Gulpease del Piano di Qualifica non è riuscito a superare la soglia ottimale definita dal gruppo.

\subsubsection{Indice di Gulpease Piano di Progetto}
\begin{figure}[H]
	\centering 
	\includegraphics[width=0.8\textwidth]{PianoQualifica/GulpeaseProgetto}
	\caption{Indice di Gulpease Piano di Progetto}
\end{figure}
Anche l'indice di Gulpease del Piano di Progetto non è ottimo, essendo composto da molti elenchi e tabelle, i quali non vengono conteggiati.

\subsubsection{Indice di Gulpease Norme di Progetto}
\begin{figure}[H]
	\centering 
	\includegraphics[width=0.8\textwidth]{PianoQualifica/GulpeaseNorme}
	\caption{Indice di Gulpease Norme di Progetto}
\end{figure}
Le Norme di Progetto si sono stabilizzate a partire dallo sprint 3.\\
In quel periodo, infatti, il gruppo aveva completato la definizione delle principali procedure operative, e da allora le modifiche successive sono state minime, come documentato nel \vr{Registro delle Modifiche} del documento \href{https://nullpointersgroup.github.io/Documentazione/output/RTB/Documenti%20Interni/Norme_di_Progetto.pdf}{Norme di Progetto}

\subsubsection{Indice di Gulpease Glossario}
\begin{figure}[H]
	\centering 
	\includegraphics[width=0.8\textwidth]{PianoQualifica/GulpeaseGlossario}
	\caption{Indice di Gulpease Glossario}
\end{figure}
Il Glossario contiene per definizione termini specialistici e tecnici che sono spesso lunghi e corposi, influendo strutturalmente sul suo indice di Gulpease, mantenendolo naturalmente su valori più bassi.

\subsection{MQC\_12: Quality Metrics Satisfied}
\begin{figure}[H]
	\centering 
	\includegraphics[width=0.8\textwidth]{PianoQualifica/QualityMetrics}
	\caption{MQC\_13: Quality Metrics Satisfied}
\end{figure}
Il grafico mostra la stabilità del gruppo nel soddisfacimento delle metriche, come evidenziato dai grafici precedenti, l'unica metrica non rispettata è la capacità del gruppo di mantenere l'\vr{Estimate at Completion} all'interno della soglia del $\pm 10\%$, anche se non è necessariamente un punto negativo in quanto il gruppo sfrutta efficientemente le ore produttive e riesce a non consumare troppe risorse.
