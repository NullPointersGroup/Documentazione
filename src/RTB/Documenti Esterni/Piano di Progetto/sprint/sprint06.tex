\subsubsection{Sprint 6}

\subsubsubsection{Informazioni generali e attività da svolgere}

\begin{tabular}{p{0.25\textwidth} p{0.2\textwidth}}
	\textbf{Inizio} & 12/01/2026 \\
	\textbf{Fine prevista} & 26/01/2026 \\
	\textbf{Fine reale} &  26/01/2026 \\
	\textbf{Giorni di ritardo} & 0
\end{tabular}
\newline \newline 
Le nostre attività da svolgere, definite nel \href{https://nullpointersgroup.github.io/Documentazione/output/RTB/Verbali%20Interni/2026-01-12_verbale_interno.pdf}{Verbale interno del 12/01/2026} sono:
\begin{itemize}[itemsep=5pt, parsep=1pt, label=$\scriptstyle\bullet$]
    \item Definizione Test$^G$ nelle Norme di Progetto.
    \item Scrittura Test$^G$ di Sistema nel Piano di Qualifica.
    \item Scrittura Test$^G$ di Regressione nel Piano di Qualifica.
    \item Aggiunta termini nel Glossario.
\end{itemize}
Inoltre, sebbene non tracciata nel verbale precedente, è emersa la seguente attività aggiuntiva:
\begin{itemize}
	\item Scrittura dei Test$^G$ di Accettazione nel Piano di Qualifica.
\end{itemize}

\subsubsubsection{Rischi attesi}
I possibili rischi attesi sono:
\begin{itemize}[itemsep=5pt, parsep=1pt, label=$\scriptstyle\bullet$]
	\item RP1: sessione invernale in corso 
	\item RP3: rischio d'incomprensione di qualche task
	\item RA1: rischio di qualche imprevisto personale dovuto alle prove di esame
	\item RE1: rischio dovuto alla poca conoscenza sulle nuove tecnologie da affrontare
\end{itemize}

\subsubsubsection{Preventivo}
Si prospetta l'utilizzo delle seguenti risorse:
\begin{table}[H]
	\centering
	\makebox[\textwidth][c]{%
		\begin{minipage}{1.2\textwidth}
			\centering
			% Prima tabella senza bordi per le intestazioni ruotate
			\begin{tabular}{p{0.2\textwidth}p{0.12\textwidth}p{0.12\textwidth}p{0.12\textwidth}p{0.12\textwidth}p{0.12\textwidth}p{0.12\textwidth}}
				& \rotatebox{45}{\textbf{Responsabile$^G$}} &
				\rotatebox{45}{\textbf{Amministratore$^G$}} & \rotatebox{45}{\textbf{Analista$^G$}} & \rotatebox{45}{\textbf{Programmatore$^G$}} & \rotatebox{45}{\textbf{Verificatore$^G$}} &
				\rotatebox{45}{\textbf{Progettista$^G$}} \\
			\end{tabular}
			
			\vspace{0.2cm} % Piccolo spazio tra le due tabelle
			
			% Seconda tabella con i dati e i bordi
			\begin{tabular}{|p{0.2\textwidth}|p{0.12\textwidth}|p{0.12\textwidth}|p{0.12\textwidth}|p{0.12\textwidth}|p{0.12\textwidth}|p{0.12\textwidth}|}
				\hline
				M. Mazzaretto & - & - & 1 & 3 & - & - \\ \hline
				T. Ceron      & 3 & - & - & - & - & - \\ \hline
				L. Pieripolli & - & 3 & - & 3 & - & - \\ \hline
				L. Marcuzzo   & - & - & 1 & 3 & - & - \\ \hline
				M. Brunello   & - & - & - & 3 & 2 & - \\ \hline
				L. Casagrande & - & - & - & 3 & - & - \\ \hline
			\end{tabular}
		\end{minipage}
	}
	\caption{Sprint 6: Preventivo}
\end{table}

Il totale preventivato per lo sprint è \textbf{455€}.\\

\begin{figure}[H]
	\centering
	\includegraphics[width=0.8\textwidth]{PianoProgetto/sprint06_preventivo}
	\caption{Sprint 6: Preventivo}
\end{figure}

\subsubsubsection{Consuntivo}
In questo sprint sono state utilizzate le seguenti risorse:
\begin{table}[H]
	\centering
	\makebox[\textwidth][c]{%
		\begin{minipage}{1.2\textwidth}
			\centering
			% Prima tabella senza bordi per le intestazioni ruotate
			\begin{tabular}{p{0.2\textwidth}p{0.12\textwidth}p{0.12\textwidth}p{0.12\textwidth}p{0.12\textwidth}p{0.12\textwidth}p{0.12\textwidth}}
				& \rotatebox{45}{\textbf{Responsabile$^G$}} &
				\rotatebox{45}{\textbf{Amministratore$^G$}} & \rotatebox{45}{\textbf{Analista$^G$}} & \rotatebox{45}{\textbf{Programmatore$^G$}} & \rotatebox{45}{\textbf{Verificatore$^G$}} &
				\rotatebox{45}{\textbf{Progettista$^G$}} \\
			\end{tabular}
			
			\vspace{0.2cm} % Piccolo spazio tra le due tabelle
			
			% Seconda tabella con i dati e i bordi
			\begin{tabular}{|p{0.2\textwidth}|p{0.12\textwidth}|p{0.12\textwidth}|p{0.13\textwidth}|p{0.12\textwidth}|p{0.12\textwidth}|p{0.12\textwidth}|}
				\hline
				M. Mazzaretto & - & - & 1 & \textcolor{green}{(-3)} & - & - \\ \hline
				T. Ceron      & 2 \textcolor{green}{(-1)} & - & - & - & - & - \\ \hline
				L. Pieripolli & - & 3 & - & \textcolor{green}{(-3)} & - & - \\ \hline
				L. Marcuzzo   & - & - & 1 & \textcolor{green}{(-3)} & - & - \\ \hline
				M. Brunello   & - & - & - & 1 \textcolor{green}{(-2)} & \textcolor{green}{(-2)} & - \\ \hline
				L. Casagrande & - & - & 1 \textcolor{red}{(+1)} & \textcolor{green}{(-3)} & - & - \\ \hline
			\end{tabular}
		\end{minipage}
	}
	\caption{Sprint 6: Consuntivo}
\end{table}
Il totale consuntivato per lo sprint è \textbf{210€}.\\

In questo sprint la distribuzione delle ore non è stata rispettata, a causa di una sottostima dell'impegno richiesto per la sessione invernale.

\begin{figure}[H]
	\centering
	\includegraphics[width=0.8\textwidth]{PianoProgetto/sprint06_consuntivo}
	\caption{Sprint 6: Consuntivo}
\end{figure}

\subsubsubsection{Aggiornamento delle risorse rimanenti}
\begin{table}[H]
	\centering
	\begin{tabular}{|p{0.2\textwidth}|p{0.1\textwidth}|p{0.05\textwidth}|p{0.1\textwidth}|p{0.15\textwidth}|p{0.17\textwidth}|}
		\hline
		\rowcolor{gray!25}
		Ruolo & Costo & Ore & Costo \newline effettivo & Ore \newline rimanenti & Budget \newline rimanente \\ \hline
		Responsabile       & 30(€/h) & 2 & 60 & 32 \textcolor{red}{(-2)} & 960 \textcolor{red}{(-60)} \\ \hline
		Amministratore$^G$ & 20(€/h) & 3 & 60 & 8 \textcolor{red}{(-3)} & 160 \textcolor{red}{(-60)} \\ \hline
		Analista$^G$       & 25(€/h) & 3 & 75 & 4.5 \textcolor{red}{(-3)} & 112.5 \textcolor{red}{(-75)}\\ \hline
		Progettista$^G$    & 25(€/h) & - & - & 125 & 3125 \\ \hline
		Verificatore$^G$   & 15(€/h) & 1 & 15 & 94 \textcolor{red}{(-1)} & 1410 \textcolor{red}{(-15)} \\ \hline
		Programmatore$^G$  & 15(€/h) & - & - & 125 & 1875 \\ \hline
		Totale         & - & 28 & 655 & 388.5 \textcolor{red}{(-9)} & 7642.5 \textcolor{red}{(-210)}\\ \hline
	\end{tabular}
	\caption{Sprint 6: Aggiornamento risorse}
\end{table}


\subsubsubsection{Rischi incontrati}
In questo sprint si è incontrato il rischio RP1 che si è manifestato con un'intensità superiore a quanto preventivato. 
La concomitanza di diverse prove d'esame ha ridotto drasticamente la disponibilità oraria dei membri del gruppo, portando a un rallentamento generale delle attività previste.

\subsubsubsection{Efficacia delle strategie di gestione di mitigazione dei rischi}
Per il rischio RP1 si è pianificato di ridurre il carico di lavoro preventivato rimandando le attività più complesse. 
Nonostante la riduzione, l'impatto della sessione d'esame è stato superiore alle aspettative, rendendo necessaria un'ulteriore contrazione delle ore produttive effettive. 

\subsubsubsection{Migliorie da attuare per le attività future}
Nei prossimi sprint il gruppo dovrà valutare con maggiore attenzione l'impatto dei periodi di esame sulla disponibilità effettiva, considerando una riduzione del carico di lavoro preventivato durante periodi di sessione accademica.

\subsubsubsection{Retrospettiva}
Le attività pianificate sono state completate solo parzialmente a causa dell'elevato impegno richiesto dalla sessione invernale, che ha ridotto significativamente la disponibilità di tutti i membri rendendo inevitabile la ripianificazione di alcune attività che erano state previste.
Il rallentamento ha permesso di definire con maggior precisione gli obiettivi del prossimo sprint, dove sarà previsto un maggiore impegno per riallinearsi con gli obiettivi previsti.