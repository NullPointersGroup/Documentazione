\subsubsection{Sprint 6}

\subsubsubsection{Informazioni generali e attività da svolgere}

\begin{tabular}{p{0.25\textwidth} p{0.2\textwidth}}
	\textbf{Inizio} & 12/01/2026 \\
	\textbf{Fine prevista} & 26/01/2026 \\
	\textbf{Fine reale} &  \\
	\textbf{Giorni di ritardo} &
\end{tabular}
\newline \newline 
Le nostre attività da svolgere, definite nel \href{https://nullpointersgroup.github.io/Documentazione/output/RTB/Verbali%20Interni/2026-01-12_verbale_interno.pdf}{Verbale interno del 12/01/2026} sono:
\begin{itemize}[itemsep=5pt, parsep=1pt, label=$\scriptstyle\bullet$]
    \item Definizione Test nelle Norme di Progetto.
    \item Scrittura Test di Sistema nel Piano di Qualifica.
    \item Scrittura Test di Regressione nel Piano di Qualifica.
    \item Aggiunta termini nel Glossario.
\end{itemize}
Inoltre, sebbene non tracciata nel verbale precedente, è emersa la seguente attività aggiuntiva:
\begin{itemize}
	\item Scrittura dei test di Accettazione nel Piano di Qualifica.
\end{itemize}

\subsubsubsection{Rischi attesi}
I possibili rischi attesi sono:
\begin{itemize}[itemsep=5pt, parsep=1pt, label=$\scriptstyle\bullet$]
	\item RP1: Sessione invernale in corso 
	\item RP3: rischio d'incomprensione di qualche task$^G$
	\item RA1: rischio di qualche imprevisto personale dovuto alle prove di esame
	\item RE1: rischio dovuto alla poca conoscenza sulle nuove tecnologie da affrontare
\end{itemize}

\subsubsubsection{Preventivo}
Si prospetta l'utilizzo delle seguenti risorse:\newpage
\begin{table}[H]
	\centering
	% Prima tabella senza bordi per le intestazioni ruotate
	\begin{tabular}{p{0.2\textwidth}p{0.12\textwidth}p{0.12\textwidth}p{0.12\textwidth}p{0.12\textwidth}p{0.12\textwidth}p{0.12\textwidth}}
		& \rotatebox{45}{\textbf{Responsabile$^G$}} &
		\rotatebox{45}{\textbf{Amministratore$^G$}} & \rotatebox{45}{\textbf{Analista$^G$}} & \rotatebox{45}{\textbf{Programmatore$^G$}} & \rotatebox{45}{\textbf{Verificatore$^G$}} &
		\rotatebox{45}{\textbf{Progettista$^G$}} \\
	\end{tabular}
	
	\vspace{0.2cm} % Piccolo spazio tra le due tabelle
	
	% Seconda tabella con i dati e i bordi
	\begin{tabular}{|p{0.2\textwidth}|p{0.12\textwidth}|p{0.12\textwidth}|p{0.12\textwidth}|p{0.12\textwidth}|p{0.12\textwidth}|p{0.12\textwidth}|}
		\hline
		M. Mazzaretto & - & - & 1 & 3 & - & - \\ \hline
		T. Ceron      & 3 & - & - & - & - & - \\ \hline
		L. Pieripolli & - & 3 & - & 3 & - & - \\ \hline
		L. Marcuzzo   & - & - & 1 & 3 & - & - \\ \hline
		M. Brunello   & - & - & - & 3 & 2 & - \\ \hline
		L. Casagrande & - & - & - & 3 & - & - \\ \hline
	\end{tabular}
	\caption{Sprint 6: Preventivo}
\end{table}

Il totale preventivato per lo sprint$^G$ è \textbf{455€}.\\

\begin{figure}[H]
	\centering
	\includegraphics[width=0.8\textwidth]{PianoProgetto/sprint06_preventivo}
	\caption{Sprint 6: Preventivo}
\end{figure}

\subsubsubsection{Consuntivo}



\subsubsubsection{Aggiornamento delle risorse rimanenti}

\subsubsubsection{Rischi incontrati}

\subsubsubsection{Efficacia delle strategie di gestione di mitigazione dei rischi}

\subsubsubsection{Migliorie da attuare per le attività future}

\subsubsubsection{Retrospettiva}
