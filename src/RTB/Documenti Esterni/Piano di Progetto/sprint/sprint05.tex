\subsubsection{Sprint 5}

\subsubsubsection{Informazioni generali e attività da svolgere}

\begin{tabular}{p{0.25\textwidth} p{0.2\textwidth}}
	\textbf{Inizio} & 29/12/2025 \\
	\textbf{Fine prevista} & 12/01/2026 \\
	\textbf{Fine reale} &  12/01/2026\\
	\textbf{Giorni di ritardo} & 0
\end{tabular}
\newline \newline 
Le nostre attività da svolgere, definite nel \href{https://nullpointersgroup.github.io/Documentazione/output/RTB/Verbali\%20Interni/2025-12-29\_verbale\_interno.pdf}{Verbale interno del 29/12/2025} e nel \href{https://nullpointersgroup.github.io/Documentazione/output/RTB/Verbali\%20Interni/2026-01-05\_verbale\_interno.pdf}{Verbale interno del 05/01/2026} sono:
\begin{itemize}[itemsep=5pt, parsep=1pt, label=$\scriptstyle\bullet$]
	\item Continuazione Analisi dei Requisiti: sezione Casi d’Uso (30-35)
	\item Continuazione Analisi dei Requisiti: sezione requisiti funzionali
	\item Continuazione Analisi dei Requisiti: sezione tracciamento requisiti
\end{itemize}

\subsubsubsection{Rischi attesi}
I possibili rischi attesi sono:
\begin{itemize}[itemsep=5pt, parsep=1pt, label=$\scriptstyle\bullet$]
	\item RP1: Inizio sessione invernale
	\item RP2: attività extra-universitarie quali festività natalizie
	\item RP3: rischio d'incomprensione di qualche task$^G$
\end{itemize}

\subsubsubsection{Preventivo}
Si prospetta l'utilizzo delle seguenti risorse:\newpage
\begin{table}[H]
	\centering
	% Prima tabella senza bordi per le intestazioni ruotate
	\begin{tabular}{p{0.2\textwidth}p{0.12\textwidth}p{0.12\textwidth}p{0.12\textwidth}p{0.12\textwidth}p{0.12\textwidth}p{0.12\textwidth}}
		& \rotatebox{45}{\textbf{Responsabile$^G$}} &
		\rotatebox{45}{\textbf{Amministratore$^G$}} & \rotatebox{45}{\textbf{Analista$^G$}} & \rotatebox{45}{\textbf{Programmatore$^G$}} & \rotatebox{45}{\textbf{Verificatore$^G$}} &
		\rotatebox{45}{\textbf{Progettista$^G$}} \\
	\end{tabular}
	
	\vspace{0.2cm} % Piccolo spazio tra le due tabelle
	
	% Seconda tabella con i dati e i bordi
	\begin{tabular}{|p{0.2\textwidth}|p{0.12\textwidth}|p{0.12\textwidth}|p{0.12\textwidth}|p{0.12\textwidth}|p{0.12\textwidth}|p{0.12\textwidth}|}
		\hline
		M. Mazzaretto & - & - & 4 & - & - & - \\ \hline
		T. Ceron      & - & - & 4 & - & - & - \\ \hline
		L. Pieripolli & - & - & 3 & - & - & - \\ \hline
		L. Marcuzzo   & - & - & 5 & - & 2 & - \\ \hline
		M. Brunello   & 3 & - & - & - & - & - \\ \hline
		L. Casagrande & - & - & 2 & - & 3 & - \\ \hline
	\end{tabular}
	\caption{Sprint 5: Preventivo}
\end{table}

Il totale preventivato per lo sprint$^G$ è \textbf{615€}.\\

\begin{figure}[H]
	\centering
	\includegraphics[width=0.8\textwidth]{PianoProgetto/sprint05_preventivo}
	\caption{Sprint 5: Preventivo}
\end{figure}

\subsubsubsection{Consuntivo}
In questo sprint sono state utilizzate le seguenti risorse:
\begin{table}[H]
	\centering
	% Prima tabella senza bordi per le intestazioni ruotate
	\begin{tabular}{p{0.2\textwidth}p{0.12\textwidth}p{0.12\textwidth}p{0.12\textwidth}p{0.12\textwidth}p{0.12\textwidth}p{0.12\textwidth}}
		& \rotatebox{45}{\textbf{Responsabile$^G$}} &
		\rotatebox{45}{\textbf{Amministratore$^G$}} & \rotatebox{45}{\textbf{Analista$^G$}} & \rotatebox{45}{\textbf{Programmatore$^G$}} & \rotatebox{45}{\textbf{Verificatore$^G$}} &
		\rotatebox{45}{\textbf{Progettista$^G$}} \\
	\end{tabular}
	
	\vspace{0.2cm} % Piccolo spazio tra le due tabelle
	
	% Seconda tabella con i dati e i bordi
	\begin{tabular}{|p{0.2\textwidth}|p{0.12\textwidth}|p{0.12\textwidth}|p{0.13\textwidth}|p{0.12\textwidth}|p{0.12\textwidth}|p{0.12\textwidth}|}
		\hline
		M. Mazzaretto & - & - & 5 \textcolor{red}{(+1)} & - & - & - \\ \hline
		T. Ceron      & - & - & 3,5 \textcolor{green}{(-0.5)} & - & - & - \\ \hline
		L. Pieripolli & - & - & 3,5 \textcolor{red}{(+0.5)}& - & - & - \\ \hline
		L. Marcuzzo   & - & - & 3 \textcolor{green}{(-2)} & - & 2 & - \\ \hline
		M. Brunello   & 3 & - & - & - & - & - \\ \hline
		L. Casagrande & - & - & 4 \textcolor{red}{(+2)} & - & 4 \textcolor{green}{(+1)} & - \\ \hline
	\end{tabular}
	\caption{Sprint 5: Consuntivo}
\end{table}
In questo sprint la distribuzione delle ore non è stata accuratamente rispettata, a causa di una leggera sottostima dell'impegno richiesto per ultimare l'Analisi dei Requisiti.
\begin{figure}[H]
	\centering
	\includegraphics[width=0.8\textwidth]{PianoProgetto/sprint05_consuntivo}
	\caption{Sprint 4: Consuntivo}
\end{figure}



\subsubsubsection{Aggiornamento delle risorse rimanenti}
\begin{table}[H]
	\centering
	\begin{tabular}{|p{0.2\textwidth}|p{0.1\textwidth}|p{0.05\textwidth}|p{0.1\textwidth}|p{0.13\textwidth}|p{0.12\textwidth}|}
		\hline
		\rowcolor{gray!25}
		Ruolo & Costo & Ore & Costo \newline effettivo & Ore \newline rimanenti & Budget \newline rimanente \\ \hline
		Responsabile$^G$   & 30(€/h) & 3 & 90 & 34 \textcolor{red}{(-3)} & 1020 \textcolor{red}{(-90)} \\ \hline
		Amministratore$^G$ & 20(€/h) & - & - & 11 & 220 \\ \hline
		Analista$^G$       & 25(€/h) & 19 & 475 & 7.5 \textcolor{red}{(-19)} & 187.5 \newline \textcolor{red}{(-475)} \\ \hline
		Progettista$^G$    & 25(€/h) & - & - & 125 & 3125 \\ \hline
		Verificatore$^G$   & 15(€/h) & 6 & 90 & 95 \textcolor{red}{(-6)} & 1425 \textcolor{red}{(-95)} \\ \hline
		Programmatore$^G$  & 15(€/h) & - & - & 125 & 1875 \\ \hline
		Totale         & - & 28 & 655 & 397.5 \textcolor{red}{(-28)} & 7852.5 \newline \textcolor{red}{(-655)} \\ \hline
	\end{tabular}
	\caption{Sprint 4: Aggiornamento risorse}
\end{table}
\subsubsubsection{Rischi incontrati}
In questo sprint è stato incontrato il rischio RE2, dato dalla poca esperienza nell'attività di Analisi dei Requisiti, che ha portato a una stima errata del carico di lavoro per alcuni membri in fase di preventivo.

\subsubsubsection{Efficacia delle strategie di gestione di mitigazione dei rischi}
Per il rischio RE2 dobbiamo cercare di preventivare le ore in modo più accurato, anche se, come visto nei precedenti sprint, siamo ad un buon punto.\\
Il gruppo ha inoltre migliorato ulteriormente la gestione individuale delle task e la comunicazione interna, consentendo una gestione più efficiente del lavoro.

\subsubsubsection{Migliorie da attuare per le attività future}
È necessario migliorare la stima delle ore per evitare ulteriori sottostime o sovrastime.

\subsubsubsection{Retrospettiva}
Tutte le attività previste sono state completate nei tempi previsti.
