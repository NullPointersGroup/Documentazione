\subsubsection{Sprint 4}

\subsubsubsection{Informazioni generali e attività da svolgere}

\begin{tabular}{p{0.25\textwidth} p{0.2\textwidth}}
	\textbf{Inizio} & 15/12/2025 \\
	\textbf{Fine prevista} & 29/12/2025 \\
	\textbf{Fine reale} & \\
	\textbf{Giorni di ritardo} &
\end{tabular}
\newline \newline 
Le nostre attività da svolgere, definite nel \href{https://nullpointersgroup.github$^G$.io/Documentazione/output/RTB/Verbali\%20Interni/2025-12-15\_verbale\_interno.pdf}{Verbale interno del 15/12/2025}, \href{https://nullpointersgroup.github$^G$.io/Documentazione/output/RTB/Verbali\%20Interni/2025-12-17\_verbale\_interno.pdf}{Verbale interno del 17/12/2025} e nel \href{https://nullpointersgroup.github$^G$.io/Documentazione/output/RTB/Verbali\%20Interni/2025-12-23\_verbale\_interno.pdf}{Verbale interno del 23/12/2025} sono:
\begin{itemize}[itemsep=5pt, parsep=1pt, label=$\scriptstyle\bullet$]
	\item Continuazione Analisi di Requisiti: sezione Attori, Requisiti di qualità e di vincolo
	\item Continuazione Analisi di Requisiti: sezione Casi d’Uso (1-29/36-51)
\end{itemize}


\subsubsubsection{Rischi attesi}
I possibili rischi attesi sono:
\begin{itemize}[itemsep=5pt, parsep=1pt, label=$\scriptstyle\bullet$]
	\item RP1: esame universitario a qualche componente
	\item RP2: attività extra-universitarie quali festività natalizie
	\item RP3: rischio d'incomprensione di qualche task
\end{itemize}


\subsubsubsection{Preventivo}
Si prospetta l'utilizzo delle seguenti risorse:\newpage
\begin{table}[H]
	\centering
	% Prima tabella senza bordi per le intestazioni ruotate
	\begin{tabular}{p{0.2\textwidth}p{0.12\textwidth}p{0.12\textwidth}p{0.12\textwidth}p{0.12\textwidth}p{0.12\textwidth}p{0.12\textwidth}}
		& \rotatebox{45}{\textbf{Responsabile}} &
		\rotatebox{45}{\textbf{Amministratore}} & \rotatebox{45}{\textbf{Analista}} & \rotatebox{45}{\textbf{Programmatore}} & \rotatebox{45}{\textbf{Verificatore}} &
		\rotatebox{45}{\textbf{Progettista}} \\
	\end{tabular}
	
	\vspace{0.2cm} % Piccolo spazio tra le due tabelle
	
	% Seconda tabella con i dati e i bordi
	\begin{tabular}{|p{0.2\textwidth}|p{0.12\textwidth}|p{0.12\textwidth}|p{0.12\textwidth}|p{0.12\textwidth}|p{0.12\textwidth}|p{0.12\textwidth}|}
		\hline
		M. Mazzaretto & - & - & 6 & - & - & - \\ \hline
		T. Ceron      & - & - & 6 & - & 1 & - \\ \hline
		L. Pieripolli & - & - & 6 & - & 1 & - \\ \hline
		L. Marcuzzo   & 4 & - & - & - & - & - \\ \hline
		M. Brunello   & - & - & 6 & - & 1 & - \\ \hline
		L. Casagrande & - & - & 6 & - & - & - \\ \hline
	\end{tabular}
	\caption{Sprint 4: Preventivo}
\end{table}

Il totale preventivato per lo sprint$^G$ è \textbf{915€}.\\

\begin{figure}[H]
	\centering
	\includegraphics[width=0.8\textwidth]{PianoProgetto/sprint04_preventivo}
	\caption{Sprint 4: Preventivo}
\end{figure}

\subsubsubsection{Consuntivo}

\subsubsubsection{Aggiornamento delle risorse rimanenti}

\subsubsubsection{Rischi incontrati}

\subsubsubsection{Efficacia delle strategie di gestione di mitigazione dei rischi}

\subsubsubsection{Migliorie da attuare per le attività future}

\subsubsubsection{Retrospettiva}
