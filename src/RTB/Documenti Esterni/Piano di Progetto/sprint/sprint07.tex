\subsubsection{Sprint 7}

\subsubsubsection{Informazioni generali e attività da svolgere}

\begin{tabular}{p{0.25\textwidth} p{0.2\textwidth}}
	\textbf{Inizio} & 26/01/2026 \\
	\textbf{Fine prevista} & 08/02/2026 \\
	\textbf{Fine reale} &  08/02/2026\\
	\textbf{Giorni di ritardo} & 0
\end{tabular}
\newline \newline 
Le nostre attività da svolgere, definite nel \href{https://nullpointersgroup.github.io/Documentazione/output/RTB/Verbali%20Interni/2026-01-26_verbale_interno.pdf}{Verbale interno del 26/01/2026} e \href{https://nullpointersgroup.github.io/Documentazione/output/RTB/Verbali%20Interni/2026-02-04_verbale_interno.pdf}{Verbale interno del 04/02/2026} sono:
\begin{itemize}[itemsep=5pt, parsep=1pt, label=$\scriptstyle\bullet$]
    \item Inizio implementazione PoC
    \item Modifica Analisi dei Requisiti: sezione Casi d'Uso (50-53)
\end{itemize}

\subsubsubsection{Rischi attesi}
I possibili rischi attesi sono:
\begin{itemize}[itemsep=5pt, parsep=1pt, label=$\scriptstyle\bullet$]
	\item RP1: Sessione invernale in corso 
	\item RP3: rischio d'incomprensione di qualche task
	\item RA1: rischio di qualche imprevisto personale
	\item RE1: rischio dovuto alla poca conoscenza sulle nuove tecnologie da affrontare
\end{itemize}

\subsubsubsection{Preventivo}
Si prospetta l'utilizzo delle seguenti risorse:
\begin{table}[H]
	\centering
	\makebox[\textwidth][c]{%
		\begin{minipage}{1.2\textwidth}
			\centering
			% Prima tabella senza bordi per le intestazioni ruotate
			\begin{tabular}{p{0.2\textwidth}p{0.12\textwidth}p{0.12\textwidth}p{0.12\textwidth}p{0.12\textwidth}p{0.12\textwidth}p{0.12\textwidth}}
				& \rotatebox{45}{\textbf{Responsabile}} &
				\rotatebox{45}{\textbf{Amministratore}} & \rotatebox{45}{\textbf{Analista}} & \rotatebox{45}{\textbf{Programmatore}} & \rotatebox{45}{\textbf{Verificatore}} &
				\rotatebox{45}{\textbf{Progettista}} \\
			\end{tabular}
			
			\vspace{0.2cm} % Piccolo spazio tra le due tabelle
			
			% Seconda tabella con i dati e i bordi
			\begin{tabular}{|p{0.2\textwidth}|p{0.12\textwidth}|p{0.12\textwidth}|p{0.12\textwidth}|p{0.12\textwidth}|p{0.12\textwidth}|p{0.12\textwidth}|}
				\hline
				M. Mazzaretto & - & - & - & 3 & - & - \\ \hline
				T. Ceron      & - & - & - & 3 & - & - \\ \hline
				L. Pieripolli & 3 & - & - & - & - & - \\ \hline
				L. Marcuzzo   & - & - & - & 3 & - & - \\ \hline
				M. Brunello   & - & - & - & 3 & - & - \\ \hline
				L. Casagrande & - & - & - & 3 & - & - \\ \hline
			\end{tabular}
		\end{minipage}
	}
	\caption{Sprint 7: Preventivo}
\end{table}

Il totale preventivato per lo sprint è \textbf{330€}.\\

\begin{figure}[H]
	\centering
	\includegraphics[width=0.8\textwidth]{PianoProgetto/sprint07_preventivo}
	\caption{Sprint 7: Preventivo}
\end{figure}

\subsubsubsection{Consuntivo}
In questo sprint sono state utilizzate le seguenti risorse:
\begin{table}[H]
	\centering
	\makebox[\textwidth][c]{%
		\begin{minipage}{1.2\textwidth}
			\centering
			% Prima tabella senza bordi per le intestazioni ruotate
			\begin{tabular}{p{0.2\textwidth}p{0.12\textwidth}p{0.12\textwidth}p{0.12\textwidth}p{0.12\textwidth}p{0.12\textwidth}p{0.12\textwidth}}
				& \rotatebox{45}{\textbf{Responsabile}} &
				\rotatebox{45}{\textbf{Amministratore}} & \rotatebox{45}{\textbf{Analista}} & \rotatebox{45}{\textbf{Programmatore}} & \rotatebox{45}{\textbf{Verificatore}} &
				\rotatebox{45}{\textbf{Progettista}} \\
			\end{tabular}
			
			\vspace{0.2cm} % Piccolo spazio tra le due tabelle
			
			% Seconda tabella con i dati e i bordi
			\begin{tabular}{|p{0.2\textwidth}|p{0.12\textwidth}|p{0.12\textwidth}|p{0.13\textwidth}|p{0.12\textwidth}|p{0.12\textwidth}|p{0.12\textwidth}|}
				\hline
				M. Mazzaretto & - & 1\textcolor{red}{(+1)} & 1 \textcolor{red}{(+1)} & 3 & - & - \\ \hline
				T. Ceron      & - & - & - & 3 & 1 \textcolor{red}{(+1)} & - \\ \hline
				L. Pieripolli & 3 & - & - & - & - & - \\ \hline
				L. Marcuzzo   & - & - & - & 3 & - & - \\ \hline
				M. Brunello   & - & 1 \textcolor{red}{(+1)} & - & 4 \textcolor{red}{(+1)} & - & - \\ \hline
				L. Casagrande & - & - & - & 3 & - & - \\ \hline
			\end{tabular}
		\end{minipage}
	}
	\caption{Sprint 7: Consuntivo}
\end{table}
Il totale consuntivato per lo sprint è \textbf{410€}.\\

In questo sprint, il carico di lavoro effettivo ha superato leggermente le stime iniziali, in oltre il team ha speso diverse ore per la formazione individuale non inclusa nelle stime.

\begin{figure}[H]
	\centering
	\includegraphics[width=0.8\textwidth]{PianoProgetto/sprint07_consuntivo}
	\caption{Sprint 7: Consuntivo}
\end{figure}

\subsubsubsection{Aggiornamento delle risorse rimanenti}
\begin{table}[H]
	\centering
	\begin{tabular}{|p{0.2\textwidth}|p{0.1\textwidth}|p{0.05\textwidth}|p{0.1\textwidth}|p{0.15\textwidth}|p{0.17\textwidth}|}
		\hline
		\rowcolor{gray!25}
		Ruolo & Costo & Ore & Costo \newline effettivo & Ore \newline rimanenti & Budget \newline rimanente \\ \hline
		
		Responsabile
		& 30(€/h) & 3 & 90
		& 29 \textcolor{red}{(-3)}
		& 870 \textcolor{red}{(-90)} \\ \hline
		
		Amministratore
		& 20(€/h) & 2 & 40
		& 6 \textcolor{red}{(-2)}
		& 120 \textcolor{red}{(-40)} \\ \hline
		
		Analista
		& 25(€/h) & 1 & 25
		& 3.5 \textcolor{red}{(-1)}
		& 87.5 \textcolor{red}{(-25)} \\ \hline
		
		Progettista
		& 25(€/h) & - & -
		& 125
		& 3125 \\ \hline
		
		Verificatore
		& 15(€/h) & 1 & 15
		& 93 \textcolor{red}{(-1)}
		& 1395 \textcolor{red}{(-15)} \\ \hline
		
		Programmatore
		& 15(€/h) & 16 & 240
		& 109 \textcolor{red}{(-16)}
		& 1635 \textcolor{red}{(-240)} \\ \hline
		
		Totale
		& - & 23 & 410
		& 365.5 \textcolor{red}{(-23)}
		& 7232.5 \textcolor{red}{(-410)} \\ \hline
	\end{tabular}
	\caption{Sprint 7: Aggiornamento risorse}
\end{table}

\subsubsubsection{Rischi incontrati}
Durante questo sprint si è concretizzato il rischio RP1, che si è però manifestato con un'intensità inferiore rispetto al precedente periodo, grazie al fatto che il suo avveramento era già stato anticipato.
Inoltre, è emerso il rischio RE1, anch'esso già previsto nella valutazione iniziale. La sua insorgenza è stata mitigata da un maggiore impegno individuale nello studio e nell'approfondimento delle nuove tecnologie coinvolte.

\subsubsubsection{Efficacia delle strategie di gestione di mitigazione dei rischi}
Il gruppo ha stimato con buona accuratezza il monte ore da destinare al ruolo di programmatore, anche in considerazione del fatto che il periodo coincidente con la sessione d'esame ha influito sulla disponibilità temporale.

\subsubsubsection{Migliorie da attuare per le attività future}
Nei prossimi sprint, il gruppo dovrà pianificare con maggiore attenzione l'assegnazione dei ruoli necessari, assicurandosi di non sottostimare il carico di lavoro associato a ciascuno di essi.

\subsubsubsection{Retrospettiva}
Il team ha completato in modo efficiente le attività pianificate, riuscendo anche a completare il PoC. Ciò ha consentito il pieno soddisfacimento degli obiettivi prefissati nel Piano di Progetto.