\section{Casi d'Uso}

\subsection{Descrizione associata ai Casi d'Uso}
Ogni Caso d'Uso$^G$ si compone di tutte le informazioni presenti nella seguente tabella, salvo i casi in cui uno specifico campo non risulti rilevante.\\

\begin{table}[h]
	\centering
	\renewcommand{\arraystretch}{1.15}
	\resizebox{\textwidth}{!}{
		\begin{tabular}{|p{0.25\textwidth}|p{0.65\textwidth}|}
			\hline
			\rowcolor[gray]{0.9}
			\textbf{Campo} & \textbf{Descrizione} \\
			\hline
			Grafico UML$^G$ & Rappresenta lo scenario dei Casi d'Uso in oggetto. \\
			\hline
			Attore$^G$ & Rappresenta coloro che interagiscono in quel sistema, senza il controllo da parte del sistema. \\
			\hline
			Scenario principale & La sequenza ragionevole delle operazioni che l'Attore$^G$ deve effettuare per portare a compimento lo scenario. \\
			\hline
			Precondizioni & Lista di elementi necessari per far sì che l'Attore$^G$ possa soddisfare il Caso d'Uso$^G$ in oggetto. \\
			\hline
			Postcondizioni & Lista di elementi che descrive le modifiche effettuate internamente dopo il corretto avvenimento dello scenario principale. \\
			\hline
			Scenario alternativo & Rappresenta un comportamento valido ma non principale, che devia dal flusso base a causa di errori. \\
			\hline
			Inclusioni & Ulteriori Casi d'Uso che l'Attore$^G$ deve compiere per portare a termine lo scenario. \\
			\hline
			Estensioni & Relazione che aggiunge comportamento opzionale o alternativo a un caso d’uso completo, attivata solo al verificarsi di una condizione specifica.  \\
			\hline
			Generalizzazioni & Relazione che descrive una specializzazione di un caso d'uso$^G$ rispetto a uno più generale. In UML$^G$ indica esclusività: i casi d'uso specializzati rappresentano varianti del caso d'uso$^G$ generale e non vincolano il percorso di esecuzione \\
			\hline
			User Story & Descrizione sintetica di una funzionalità$^G$ dal punto di vista dell’utente, focalizzata sugli obiettivi. \\
			\hline
		\end{tabular}
	}
	\caption{Campi dei Casi d'Uso}
	\vspace{0.3cm}
\end{table}

\subsection{Attori}
\begin{figure}[H]
	\centering
	\includegraphics[width=0.5\textwidth]{AnalisiRequisiti/Utenti}
	\caption{Attori principali}
\end{figure}


\begin{table}[h]
	\centering
	\renewcommand{\arraystretch}{1.15}
	\resizebox{\textwidth}{!}{
		\begin{tabular}{|p{0.25\textwidth}|p{0.65\textwidth}|}
			\hline
			\rowcolor[gray]{0.9}
			\textbf{Autore} & \textbf{Descrizione} \\
			\hline
			Utente & Rappresenta qualsiasi persona che può interagire con il sistema. \\
			\hline
			Cliente & Utente registrato con accesso a funzionalità$^G$ dedicate agli utenti finali. \\
			\hline
			Admin & Utente con privilegi speciali per gestire e configurare il sistema. \\
			\hline
		\end{tabular}
	}
	\caption{Descrizione Attori}
	\vspace{0.3cm}
\end{table}

\newpage

\subsection{Lista dei Casi d'Uso}

\subsubsection{UC\_01: Registrazione}
\label{uc_01}

\begin{figure}[H]
    \centering 
    \includegraphics[width=0.8\textwidth]{AnalisiRequisiti/UC_01}
    \caption{UC\_01: Registrazione}
\end{figure}

\begin{itemize}
    \item \textbf{Attore Principale}: Cliente
    \item \textbf{Scenario Principale}: 
    \begin{enumerate}
        \item Inserimento dello Username $\rightarrow$ Vedi \hyperref[uc_01.1]{UC\_01.1}
        \item Inserimento della Password $\rightarrow$ Vedi \hyperref[uc_01.2]{UC\_01.2}
        \item Inserimento della Password nel campo di \vr{conferma password} $\rightarrow$ Vedi \hyperref[uc_01.3]{UC\_01.3}
        \item Inserimento della email $\rightarrow$ Vedi \hyperref[uc_01.4]{UC\_01.4}
    \end{enumerate}
    \item \textbf{Precondizioni}:
    \begin{itemize}
        \item Il sistema è online
        \item Il Cliente non è registrato nel sistema
    \end{itemize}
    \item \textbf{Postcondizioni}:
    \begin{itemize}
        \item Il Cliente è ora registrato ed è riconosciuto come tale
    \end{itemize}
    \item \textbf{Inclusioni}:
    \begin{itemize}
        \item \hyperref[uc_01.1]{UC\_01.1}
        \item \hyperref[uc_01.2]{UC\_01.2}
        \item \hyperref[uc_01.3]{UC\_01.3}
        \item \hyperref[uc_01.4]{UC\_01.4}
    \end{itemize}
    \item \textbf{User Story}: Il Cliente vuole registrarsi nel sistema
\end{itemize}

\subsubsubsection{UC\_01.1: Inserimento username}
\label{uc_01.1}

\begin{figure}[H]
    \centering 
    \includegraphics[width=0.8\textwidth]{AnalisiRequisiti/UC_01.1}
    \caption{Estensioni UC\_01.1: UC\_01.1.1}
\end{figure}

\begin{itemize}
    \item \textbf{Attore Principale}: Cliente
    \item \textbf{Scenario Principale}:
    \begin{enumerate}
        \item Il Cliente inserisce uno username con il quale vuole registrarsi
    \end{enumerate}
    \item \textbf{Precondizioni}: 
    \begin{itemize}
        \item Il sistema è online
        \item Il Cliente non è registrato nel sistema
        \item Lo username che Il Cliente inserisce non deve essere già presente nel sistema
    \end{itemize}
    \item \textbf{Postcondizioni}:
    \begin{itemize}
        \item Username inserito, pronto per essere registrato
    \end{itemize}
    \item \textbf{Scenari Alternativi}:
    \begin{itemize}
        \item Il Cliente ha inserito uno username esistente $\rightarrow$ Vedi \hyperref[uc_01.1.1]{UC\_01.1.1}
    \end{itemize}
    \item \textbf{Estensioni}:
    \begin{itemize}
        \item \hyperref[uc_01.1.1]{UC\_01.1.1}
    \end{itemize}
    \item \textbf{User Story}: Il Cliente vuole registrarsi nel sistema
\end{itemize}

\subsubsubsubsection{UC\_01.1.1: Errore inserimento username}
\label{uc_01.1.1}

\begin{itemize}
    \item \textbf{Attore Principale}: Cliente
    \item \textbf{Precondizioni}:
    \begin{itemize}
        \item Il sistema è online
        \item Il Cliente non è registrato nel sistema
        \item Il Cliente ha inserito uno username già esistente come proprio username
    \end{itemize}
    \item \textbf{Postcondizioni}:
    \begin{itemize}
        \item Link HTML per andare alla pagina di login
    \end{itemize}
\end{itemize}

\subsubsubsection{UC\_01.2: Inserimento password}
\label{uc_01.2}

\begin{figure}[H]
    \centering 
    \includegraphics[width=0.8\textwidth]{AnalisiRequisiti/UC_01.2}
    \caption{Estensioni UC\_01.2: UC\_01.2.1}
\end{figure}

\begin{itemize}
    \item \textbf{Attore Principale}: Cliente
    \item \textbf{Scenario Principale}:
    \begin{enumerate}
        \item Il Cliente inserisce la password, che insieme allo username userà per autenticarsi
    \end{enumerate}
    \item \textbf{Precondizioni}: 
    \begin{itemize}
        \item Il sistema è online
        \item Il Cliente non è registrato nel sistema
        \item Il sistema non conosce la password
    \end{itemize}
    \item \textbf{Postcondizioni}:
    \begin{itemize}
        \item Il sistema riceve la password e ne conserva l'hash
    \end{itemize}
    \item \textbf{Scenari Alternativi}:
    \begin{itemize}
        \item Il Cliente ha inserito una password che non è conforme $\rightarrow$ Vedi \hyperref[uc_01.2.1]{UC\_01.2.1}
    \end{itemize}
    \item \textbf{Estensioni}:
    \begin{itemize}
        \item \hyperref[uc_01.2.1]{UC\_01.2.1}
    \end{itemize}
    \item \textbf{User Story}: Il Cliente vuole registrarsi nel sistema
\end{itemize}


\subsubsubsubsection{UC\_01.2.1: Errore inserimento password}
\label{uc_01.2.1}

\begin{itemize}
    \item \textbf{Attore Principale}: Cliente
    \item \textbf{Precondizioni}:
    \begin{itemize}
        \item Il sistema è online
        \item Il Cliente non è registrato nel sistema
        \item Il Cliente ha inserito una password e quest'ultima non rispetta i criteri di sicurezza
    \end{itemize}
    \item \textbf{Postcondizioni}:
    \begin{itemize}
        \item Messaggio di errore visibile al Cliente
    \end{itemize}
\end{itemize}

\subsubsubsection{UC\_01.3: Conferma password}
\label{uc_01.3}

\begin{figure}[H]
    \centering 
    \includegraphics[width=0.8\textwidth]{AnalisiRequisiti/UC_01.3}
    \caption{Estensioni UC\_01.3: UC\_01.3.1}
\end{figure}

\begin{itemize}
    \item \textbf{Attore Principale}: Cliente
    \item \textbf{Scenario Principale}:
    \begin{enumerate}
        \item Il Cliente inserisce nuovamente la password nel campo \vr{conferma password}
    \end{enumerate}
    \item \textbf{Precondizioni}:
    \begin{itemize}
        \item Il sistema è online
        \item Il Cliente non è registrato nel sistema
        \item Il sistema conosce la password inserita in precedenza
    \end{itemize}
    \item \textbf{Postcondizioni}:
    \begin{itemize}
        \item Password confermata, la registrazione può essere conclusa
    \end{itemize}
    \item \textbf{Scenari Alternativi}:
    \begin{itemize}
        \item Il Cliente ha inserito nel campo \vr{conferma password} una password che non coincide con quella inserita precedentemente $\rightarrow$ Vedi \hyperref[uc_01.3.1]{UC\_01.3.1}
    \end{itemize}
    \item \textbf{Estensioni}:
    \begin{itemize}
        \item \hyperref[uc_01.3.1]{UC\_01.3.1}
    \end{itemize}
    \item \textbf{User Story}: Il Cliente vuole registrarsi nel sistema
\end{itemize}

\subsubsubsubsection{UC\_01.3.1: Errore conferma password}
\label{uc_01.3.1}

\begin{itemize}
    \item \textbf{Attore Principale}: Cliente
    \item \textbf{Precondizioni}:
    \begin{itemize}
        \item Il sistema è online
        \item Il Cliente non è registrato nel sistema
        \item Il sistema conosce la password inserita in precedenza
        \item Il Cliente ha inserito una password non coincidente con quella inserita precedentemente
    \end{itemize}
    \item \textbf{Postcondizioni}:
    \begin{itemize}
        \item Messaggio di errore visibile al Cliente
    \end{itemize}
\end{itemize}

\subsubsubsection{UC\_01.4: Inserimento email}
\label{uc_01.4}

\begin{figure}[H]
    \centering 
    \includegraphics[width=0.8\textwidth]{AnalisiRequisiti/UC_01.4}
    \caption{Estensioni UC\_01.4: UC\_01.4.1, UC\_01.4.2}
\end{figure}

\begin{itemize}
    \item \textbf{Attore Principale}: Cliente
    \item \textbf{Scenario Principale}:
    \begin{enumerate}
        \item Il Cliente inserisce l’email che utilizzerà per il recupero della password. $\rightarrow$ Vedi \hyperref[uc_04]{UC\_04}
    \end{enumerate}
    \item \textbf{Precondizioni}:
    \begin{itemize}
        \item Il sistema è online
        \item Il Cliente non è registrato nel sistema
        \item Il sistema non conosce l'email
    \end{itemize}
    \item \textbf{Postcondizioni}:
    \begin{itemize}
        \item Email inserita, pronta per essere registrata
    \end{itemize}
    \item \textbf{Scenari Alternativi}:
    \begin{itemize}
        \item Il Cliente ha inserito una email non valida $\rightarrow$ Vedi \hyperref[uc_01.4.1]{UC\_01.4.1}
        \item Il Cliente ha inserito una email già presente nel sistema $\rightarrow$ Vedi \hyperref[uc_01.4.2]{UC\_01.4.2}
    \end{itemize}
    \item \textbf{Estensioni}:
    \begin{itemize}
        \item \hyperref[uc_01.4.1]{UC\_01.4.1}
        \item \hyperref[uc_01.4.2]{UC\_01.4.2}
    \end{itemize}
    \item \textbf{User Story}: Il Cliente vuole registrarsi nel sistema
\end{itemize}

\subsubsubsubsection{UC\_01.4.1: Email non valida}
\label{uc_01.4.1}

\begin{itemize} 
    \item \textbf{Attore Principale}: Cliente
    \item \textbf{Precondizioni}:
    \begin{itemize}
        \item Il sistema è online
        \item Il Cliente non è registrato nel sistema
        \item Il Cliente ha inserito una email in un formato non corretto
    \end{itemize}
    \item \textbf{Postcondizioni}:
    \begin{itemize}
        \item Messaggio di errore visibile al Cliente 
    \end{itemize}
\end{itemize}


\subsubsubsubsection{UC\_01.4.2: Email già presente}
\label{uc_01.4.2}

\begin{itemize}
    \item \textbf{Attore Principale}: Cliente
    \item \textbf{Precondizioni}:
    \begin{itemize}
        \item Il sistema è online
        \item Il Cliente non è registrato nel sistema
        \item Il Cliente ha inserito una email già presente nel sistema
    \end{itemize}
    \item \textbf{Postcondizioni}:
    \begin{itemize}
        \item Messaggio di errore visibile al Cliente 
    \end{itemize}
\end{itemize}

\subsubsection{UC\_02: Autenticazione}
\label{uc_02}

\begin{figure}[H]
    \centering 
    \includegraphics[width=0.8\textwidth]{AnalisiRequisiti/UC_02}
    \caption{UC\_02: Autenticazione}
\end{figure}

\begin{itemize}
    \item \textbf{Attore$^G$ Principale}: Utente
    \item \textbf{Scenario Principale}:
    \begin{enumerate}
        \item L'Utente ha selezionato dal menù la possibilità di autenticarsi nel sistema 
        \item L'Utente inserisce lo username $\rightarrow$ Vedi \hyperref[uc_02.1]{UC\_02.1}
        \item L'Utente inserisce la password $\rightarrow$ Vedi \hyperref[uc_02.2]{UC\_02.2}
    \end{enumerate}
    \item \textbf{Precondizioni}:
    \begin{itemize}
        \item Il sistema è online
        \item L'Utente non è autenticato nel sistema
    \end{itemize}
    \item \textbf{Postcondizioni}:
    \begin{itemize}
        \item L'Utente è ora riconosciuto come Cliente
    \end{itemize}
    \item \textbf{Scenari Alternativi}:
    \begin{itemize}
        \item L'autenticazione fallisce $\rightarrow$ Vedi \hyperref[uc_03]{UC\_03}
    \end{itemize}
    \item \textbf{Inclusioni}:
    \begin{itemize}
        \item \hyperref[uc_02.1]{UC\_02.1}
        \item \hyperref[uc_02.2]{UC\_02.2}
    \end{itemize}
    \item \textbf{Estensioni}:
    \begin{itemize}
        \item \hyperref[uc_03]{UC\_03}
    \end{itemize}
    \item \textbf{Trigger}: L'Utente vuole autenticarsi nel sistema
\end{itemize}


\subsubsubsection{UC\_02.1: Inserimento username}
\label{uc_02.1}

\begin{itemize}
    \item \textbf{Attore$^G$ Principale}: Utente
    \item \textbf{Scenario Principale}:
    \begin{enumerate}
        \item L'Utente inserisce lo username con il quale vuole autenticarsi
    \end{enumerate}
    \item \textbf{Precondizioni}:
    \begin{itemize}
        \item Il sistema è online
        \item L'Utente non è ancora autenticato
    \end{itemize}
    \item \textbf{Postcondizioni}:
    \begin{itemize}
        \item Username inserito correttamente
    \end{itemize}
    \item \textbf{Trigger}: L'Utente vuole inserire lo username per l'autenticazione
\end{itemize}


\subsubsubsection{UC\_02.2: Inserimento password}
\label{uc_02.2}

\begin{itemize}
    \item \textbf{Attore$^G$ Principale}: Utente
    \item \textbf{Scenario Principale}:
    \begin{enumerate}
        \item L'Utente inserisce la propria password
    \end{enumerate}
    \item \textbf{Precondizioni}:
    \begin{itemize}
        \item Il sistema è online
        \item L'Utente non è ancora autenticato
    \end{itemize}
    \item \textbf{Postcondizioni}:
    \begin{itemize}
        \item Password inserita correttamente
    \end{itemize}
    \item \textbf{Trigger}: L'Utente vuole inserire la password per l'autenticazione
\end{itemize}

\subsubsection{UC\_03: Autenticazione non riuscita}
\label{uc_03}

\begin{itemize}
    \item \textbf{Attore Principale}: Utente
    \item \textbf{Scenario Principale}:
    \begin{enumerate}
        \item Il sistema ha ricevuto lo username e la password che l'Utente vuole usare per autenticarsi, ma non è riuscito a trovare una corrispondenza valida 
    \end{enumerate}
    \item \textbf{Precondizioni}:
    \begin{itemize}
        \item Il sistema è online
        \item L'Utente non è registrato nel sistema
        \item L'Utente ha inserito un username non valido oppure ha inserito una password non corretta
    \end{itemize}
    \item \textbf{Postcondizioni}:
    \begin{itemize}
        \item Il messaggio di errore \vr{Username o Passowrd errata} diventa visibile all'utente
    \end{itemize}
\end{itemize}

\subsubsection{UC\_04: Recupero password dimenticata}
\label{uc_04}

\begin{figure}[H]
    \centering 
    \includegraphics[width=1\textwidth]{AnalisiRequisiti/UC_04}
    \caption{UC\_04: Recupero password dimenticata}
\end{figure}

\begin{itemize}
    \item \textbf{Attore Principale}: Utente
    \item \textbf{Scenario Principale}:
    \begin{enumerate}
        \item L'Utente seleziona l'opzione \vr{password dimenticata} nel sistema
        \item L'Utente inserisce l'email di registrazione per recuperare la password $\rightarrow$ Vedi \hyperref[uc_04.1]{UC\_04.1}
    \end{enumerate}
    \item \textbf{Precondizioni}:
    \begin{itemize}
        \item Il sistema è online
        \item Utente non autenticato
        \item Accesso alla pagina di login
    \end{itemize}
    \item \textbf{Postcondizioni}:
    \begin{itemize}
        \item L'Utente ha inviato una richiesta di recupero password valida.
        \item Il sistema ha registrato la richiesta e generato le credenziali temporanee per il ripristino.
    \end{itemize}
    \item \textbf{Inclusioni}:
    \begin{itemize}
        \item \hyperref[uc_04.1]{UC\_04.1}
    \end{itemize}
    \item \textbf{Trigger}: L'Utente vuole ripristinare la propria password
\end{itemize}

\subsubsubsection{UC\_04.1: Inserimento email di contatto}
\label{uc_04.1}

\begin{itemize}
    \item \textbf{Attore Principale}: Utente
    \item \textbf{Scenario Principale}:
    \begin{enumerate}
        \item L'Utente inserisce l'email con cui si è registrato
    \end{enumerate}
    \item \textbf{Precondizioni}:
    \begin{itemize}
        \item Procedura di recupero password avviata
    \end{itemize}
    \item \textbf{Postcondizioni}:
    \begin{itemize}
        \item Email inserita correttamente
    \end{itemize}
    \item \textbf{Scenari Alternativi}:
    \begin{itemize}
        \item L'Utente ha inserito una email non valida $\rightarrow$ Vedi \hyperref[uc_04.1.1]{UC\_04.1.1}
        \item L'Utente ha inserito una email non presente nel sistema $\rightarrow$ Vedi \hyperref[uc_04.1.2]{UC\_04.1.2}
    \end{itemize}
    \item \textbf{Estensioni}:
    \begin{itemize}
        \item \hyperref[uc_04.1.1]{UC\_04.1.1}
        \item \hyperref[uc_04.1.2]{UC\_04.1.2}
    \end{itemize}
    \item \textbf{Trigger}: L'Utente vuole inserire l'email per il recupero password
\end{itemize}

UC\_04.1 presenta due ulteriori estensioni, come mostrato in figura:
\begin{figure}[H]
    \centering 
    \includegraphics[width=1\textwidth]{AnalisiRequisiti/UC_04.1}
    \caption{Estensioni UC\_04.1: UC\_04.1.1, UC\_04.1.2}
\end{figure}

\subsubsubsubsection{UC\_04.1.1: Email non valida}
\label{uc_04.1.1}

\begin{itemize}
    \item \textbf{Attore Principale}: Utente
    \item \textbf{Precondizioni}:
    \begin{itemize}
        \item Procedura di recupero password avviata
        \item L'Utente ha inserito una email in un formato non corretto
    \end{itemize}
    \item \textbf{Postcondizioni}:
    \begin{itemize}
        \item Messaggio d'errore visibile all'Utente
    \end{itemize}
\end{itemize}

\subsubsubsubsection{UC\_04.1.2: Email non presente nel sistema}
\label{uc_04.1.2}

\begin{itemize}
    \item \textbf{Attore Principale}: Utente
    \item \textbf{Precondizioni}:
    \begin{itemize}
        \item Procedura di recupero password avviata
        \item L'Utente ha inserito una email 
    \end{itemize}
    \item \textbf{Postcondizioni}:
    \begin{itemize}
        \item L'Utente visualizza un messaggio d'errore per email non presente nel sistema
    \end{itemize}
\end{itemize}

\begin{figure}[H]
	\centering 
	\includegraphics[width=1\textwidth]{AnalisiRequisiti/DA-UC_04}
	\caption{Diagramma di attività che descrive UC\_02, UC\_03, UC\_04}
\end{figure}



\subsubsection{UC\_05: Eliminazione account}
\label{uc_05}

\begin{figure}[H]
    \centering 
    \includegraphics[width=0.8\textwidth]{AnalisiRequisiti/UC_05}
    \caption{UC\_05: Eliminazione account}
\end{figure}

\begin{itemize}
    \item \textbf{Attore Principale}: Utente 
    \item \textbf{Scenario Principale}:
    \begin{enumerate}
        \item L'Utente clicca il pulsante \vr{Elimina Account}
    \end{enumerate}
    \item \textbf{Precondizioni}:
    \begin{itemize}
        \item Il sistema è online
        \item L'Utente è autenticato
        \item L'Utente si trova nella pagina di info del profilo
    \end{itemize}
    \item \textbf{Postcondizioni}:
    \begin{itemize}
        \item L'Utente è stato eliminato
    \end{itemize}
    \item \textbf{User story}: L'Utente vuole eliminare il proprio account
\end{itemize}

\subsubsection{UC\_06: Logout}
\label{uc_06}

\begin{figure}[H]
    \centering 
    \includegraphics[width=0.8\textwidth]{AnalisiRequisiti/UC_06}
    \caption{UC\_06: Logout}
\end{figure}

\begin{itemize}
    \item \textbf{Attore$^G$ Principale}: Utente
    \item \textbf{Scenario Principale}:
    \begin{enumerate}
        \item L'Utente clicca il pulsante \vr{Logout}
    \end{enumerate}
    \item \textbf{Precondizioni}:
    \begin{itemize}
        \item L'Utente è autenticato
        \item L'Utente si trova nella pagina di info del profilo
    \end{itemize}
    \item \textbf{Postcondizioni}:
    \begin{itemize}
        \item L'Utente esce dal sistema
    \end{itemize}
    \item \textbf{Trigger}: L'Utente desidera terminare la sessione corrente
\end{itemize}

\subsubsection{UC\_07: Visualizza informazioni profilo}
\label{uc_07}

\begin{figure}[H]
        \centering
        \includegraphics[width=0.8\textwidth]{AnalisiRequisiti/UC_07}
        \caption{UC\_07: Visualizza informazioni profilo}
\end{figure}

\begin{itemize}
    \item \textbf{Attore$^G$ Principale}: Cliente
    \item \textbf{Scenario principale}: 
            \begin{enumerate}
                \item Il Cliente accede alla sezione info del proprio profilo
                \item Il Cliente visualizza il proprio username $\rightarrow$ Vedi \hyperref[uc_07.1]{UC\_07.1}
                \item Il Cliente visualizza la propria ragione sociale $\rightarrow$ Vedi \hyperref[uc_07.2]{UC\_07.2}
                \item Il Cliente visualizza il proprio indirizzo email $\rightarrow$ Vedi \hyperref[uc_07.3]{UC\_07.3}
            \end{enumerate}
    \item \textbf{Precondizioni}:   
            \begin{itemize}
                \item Il sistema è online
                \item Il Cliente è autenticato
            \end{itemize}      
    \item \textbf{Postcondizioni}: 
            \begin{itemize}
                \item Il sistema mostra una pagina HTML con le informazioni
            \end{itemize}
    \item \textbf{Inclusioni}: 
            \begin{itemize}
                \item \hyperref[uc_07.1]{UC\_07.1}
		        \item \hyperref[uc_07.2]{UC\_07.2}
		        \item \hyperref[uc_07.3]{UC\_07.3}
            \end{itemize}
    \item \textbf{Trigger}: Il Cliente vuole visualizzare i propri dati registrati nel sistema
\end{itemize}

\subsubsubsection{UC\_07.1: Visualizza username}
\label{uc_07.1}
\begin{itemize}
    \item \textbf{Attore$^G$ Principale}: Cliente
    \item \textbf{Scenario principale}:
    \begin{enumerate}
        \item Il Cliente visualizza lo username
    \end{enumerate}
    \item \textbf{Precondizioni}:
    \begin{itemize}
        \item Il sistema è online
        \item Il Cliente è autenticato
        \item Il Cliente si trova nella sezione info del proprio profilo
    \end{itemize}      
    \item \textbf{Postcondizioni}:
    \begin{itemize}
        \item Il Cliente visualizza lo username
    \end{itemize}
    \item \textbf{Trigger}: Il Cliente vuole visualizzare la sezione informazioni del profilo
\end{itemize}

\subsubsubsection{UC\_07.2: Visualizza ragione sociale}
\label{uc_07.2}
\begin{itemize}
    \item \textbf{Attore$^G$ Principale}: Cliente
    \item \textbf{Scenario principale}:
            \begin{enumerate}
                \item Il Cliente visualizza la ragione sociale
            \end{enumerate}
    \item \textbf{Precondizioni}:
            \begin{itemize}
                \item Il sistema è online
                \item Il Cliente è autenticato
                \item Il Cliente si trova nella sezione info del proprio profilo
            \end{itemize}      
    \item \textbf{Postcondizioni}:
            \begin{itemize}
                \item Il Cliente visualizza la ragione sociale
            \end{itemize}
    \item \textbf{Trigger}: Il Cliente vuole visualizzare la sezione informazioni del profilo
\end{itemize}

\subsubsubsection{UC\_07.3: Visualizza indirizzo email}
\label{uc_07.3}
\begin{itemize}
    \item \textbf{Attore$^G$ Principale}: Cliente
    \item \textbf{Scenario principale}:
            \begin{enumerate}
                \item Il Cliente visualizza l'indirizzo email
            \end{enumerate}
    \item \textbf{Precondizioni}:
            \begin{itemize}
                \item Il sistema è online
                \item Il Cliente è autenticato
                \item Il Cliente si trova nella sezione info del proprio profilo
            \end{itemize}      
    \item \textbf{Postcondizioni}:
            \begin{itemize}
                \item Il Cliente visualizza l'indirizzo email
            \end{itemize}
    \item \textbf{Trigger}: Il Cliente sta visualizzando la sezione informazioni del profilo.
\end{itemize}

\subsubsection{UC\_08: Reimposta password}
\label{uc_08}

\begin{figure}[H]
        \centering
        \includegraphics[width=0.8\textwidth]{AnalisiRequisiti/UC_08}
        \caption{UC\_08: Reimposta password}
\end{figure}

\begin{itemize}
    \item \textbf{Attore principale}: Utente
    \item \textbf{Scenario principale}:
        \begin{itemize}
            \item L'utente clicca il tasto per reimpostare la password
        \end{itemize}
    L'utente vuole cambiare la propria password di accesso
    \item \textbf{Precondizioni}:
        \begin{itemize}
            \item Il sistema è online
            \item L'utente è autenticato
            \item L'utente conosce la precedente password
        \end{itemize}
    \item \textbf{Postcondizioni}:
        \begin{itemize}
            \item La password viene aggiornata correttamente
        \end{itemize}
    \item \textbf{User Story}: L'utente vuole aggiornare le proprie credenziali di accesso
\end{itemize}

\subsubsection{UC\_09: Inserimento ordine via testo}
\label{uc_09}

\begin{figure}[H]
        \centering
        \includegraphics[width=0.8\textwidth]{AnalisiRequisiti/UC_09}
        \caption{UC\_09: Inserimento ordine via testo}
\end{figure}

\begin{itemize}
    \item \textbf{Attore principale}: Utente
    \item \textbf{Scenario principale}:
        \begin{itemize}
            \item L'utente inserisce un ordine tramite campo di testo
        \end{itemize}
    \item \textbf{Precondizioni}:
        \begin{itemize}
            \item Il sistema è online
            \item L'utente visualizza il campo di testo
            \item L'utente visualizza il pulsante d'invio
        \end{itemize}
    \item \textbf{Postcondizioni}:
        \begin{itemize}
            \item Il testo ricevuto viene inviato al pre-processing
            \item Il campo di testo viene svuotato dopo l'invio
        \end{itemize}
    \item \textbf{Scenari alternativi}:
        \begin{itemize}
            \item L'utente riceve un messaggio di errore per il superamento limite caratteri. $\rightarrow$ Vedi [UC\_10, §\ref{uc_10}]
        \end{itemize}
    \item \textbf{Estensioni}:
        \begin{itemize}
            \item $[$UC\_10, §\ref{uc_10}$]$
        \end{itemize}    
    \item \textbf{User Story}: L'utente desidera ordinare dei prodotti scrivendo un messaggio
\end{itemize}


\subsubsection{UC\_10: Errore inserimento ordine testuale}
\label{uc_10}

\begin{itemize}
   \item \textbf{Attore principale}: Cliente
   \item \textbf{Precondizioni}:
		\begin{itemize}
			\item Il Cliente ha inserito del testo nell'area di input
			\item Il Cliente ha inserito una quantità di caratteri superiore al limite previsto
		\end{itemize}
	\item \textbf{Postcondizioni}:
	\begin{itemize}
		\item Il sistema visualizza un messaggio di errore che informa del superamento del limite di caratteri
		\item Il testo inserito dal Cliente non viene accettato per l'invio
	\end{itemize}
\end{itemize}

\subsubsection{UC\_11: Inserimento ordine via immagine}
\label{uc_11}

\begin{figure}[H]
        \centering
        \includegraphics[width=0.8\textwidth]{AnalisiRequisiti/UC_11}
        \caption{UC\_11: Inserimento ordine via immagine}
\end{figure}

\begin{itemize}
    \item \textbf{Attore principale}: Utente
    \item \textbf{Scenario principale}:
        \begin{itemize}
            \item L'utente inserisce un ordine tramite immagine
        \end{itemize}
    \item \textbf{Precondizioni}:
        \begin{itemize}
            \item Il sistema è online
            \item L'Utente visualizza il pulsante per allegare le immagini
            \item L'utente visualizza il pulsante d'invio
        \end{itemize}
    \item \textbf{Postcondizioni}:
        \begin{itemize}
            \item L'immagine viene archiviata
            \item L'immagine viene inviata all'OCR
        \end{itemize}
    \item \textbf{Scenari alternativi}:
        \begin{itemize}
            \item L'utente riceve un messaggio di errore per il caricamento dell'immagine. $\rightarrow$ Vedi [UC\_12, §\ref{uc_12}]
        \end{itemize}
    \item \textbf{Inclusioni}:
        \begin{itemize}
            \item $[$UC\_11.1, §\ref{uc_11.2}$]$
            \item $[$UC\_11.2, §\ref{uc_11.1}$]$
        \end{itemize}
    \item \textbf{Estensioni}:
    \begin{itemize}
            \item $[$UC\_12, §\ref{uc_12}$]$
        \end{itemize}
    \item \textbf{User Story}: L'Utente vuole inserire un ordine scattando una foto o selezionando un'immagine
\end{itemize}


\subsubsection{UC\_11.1: Scatto foto con fotocamera}
\label{uc_11.1}

\begin{itemize}
    \item \textbf{Attore principale}: Utente
    \item \textbf{Scenario principale}:
        \begin{itemize}
            \item L'Utente ha premuto il pulsante per usare la fotocamera del dispositivo
        \end{itemize}
    \item \textbf{Precondizioni}:
        \begin{itemize}
            \item La fotocamera viene attivata
        \end{itemize}
    \item \textbf{Postcondizioni}: 
        \begin{itemize}
            \item L'immagine viene scattata
            \item L'immagine viene archiviata
            \item Il sistema invia l'immagine appena scattata all'OCR   
        \end{itemize}
    \item \textbf{User Story}: L'utente vuole inserire un ordine scattando una foto
\end{itemize}



\subsubsection{UC\_11.2: Selezionare immagine da dispositivo}
\label{uc_11.2}

\begin{itemize}
    \item \textbf{Attore principale}: Utente
    \item \textbf{Scenario principale}:
        \begin{itemize}
            \item L'Utente ha premuto il pulsante per selezionare un file dal dispositivo
        \end{itemize}
    \item \textbf{Precondizioni}:
        \begin{itemize}
            \item Viene aperto il dialogo di selezione file del dispositivo
        \end{itemize}
    \item \textbf{Postcondizioni}: 
        \begin{itemize}
            \item L'immagine viene selezionata
            \item L'immagine viene archiviata
            \item Il sistema invia l'immagine appena scattata all'OCR   
        \end{itemize}
    \item \textbf{User Story}: L'utente vuole inserire un ordine mediante un'immagine presente nel dispositivo
\end{itemize}

\subsubsection{UC\_12: Errore inserimento immagine}
\label{uc_12}

\begin{itemize}
    \item \textbf{Attore$^G$ Principale}: Cliente
   \item \textbf{Precondizioni}:
		\begin{itemize}
			\item Il Cliente ha allegato un'immagine
			\item L'immagine eccede le dimensioni massime consentite
			\item L'immagine non rispetta i formati accettati
			\item L'immagine non rispetta la qualità minima consentita
		\end{itemize}
	\item \textbf{Postcondizioni}:
   \begin{itemize}
      \item Il sistema visualizza un messaggio di errore appropriato (per dimensioni, formato o qualità non conformi)
      \item L'immagine non viene accettata per l'elaborazione
   \end{itemize}
\end{itemize}

\subsubsection{UC\_13: Inserimento ordine via audio}
\label{uc_13}
\begin{figure}[H]
        \centering
        \includegraphics[width=0.8\textwidth]{AnalisiRequisiti/UC_13}
        \caption{UC\_13: Inserimento ordine via audio}
\end{figure}

\begin{itemize}
    \item \textbf{Attore principale}: Utente
    \item \textbf{Scenario principale}:
        \begin{itemize}
            \item L'utente inserisce un ordine tramite audio
        \end{itemize}
    \item \textbf{Precondizioni}:
        \begin{itemize}
            \item Il sistema è online
            \item L'Utente visualizza il pulsante per allegare i file audio
            \item LUtente visualizza il pulsante per registrare un messaggio audio
            \item L'utente visualizza il pulsante d'invio
        \end{itemize}
    \item \textbf{Postcondizioni}:
        \begin{itemize}
            \item L'audio viene archiviato
            \item L'audio viene inviato allo Speech-to-Text
        \end{itemize}
    \item \textbf{Scenari alternativi}:
        \begin{itemize}
            \item L'utente riceve un messaggio di errore per il caricamento dell'audio. $\rightarrow$ Vedi [UC\_14, §\ref{uc_14}]
        \end{itemize}
    \item \textbf{Inclusioni}:
        \begin{itemize}
            \item $[$UC\_13.1, §\ref{uc_13.2}$]$
            \item $[$UC\_13.2, §\ref{uc_13.1}$]$
        \end{itemize}
    \item \textbf{Estensioni}:
    \begin{itemize}
            \item $[$UC\_14, §\ref{uc_14}$]$
        \end{itemize}
    \item \textbf{User Story}: L'utente desidera inserire un ordine registrando un audio o selezionandone uno dal dispositivo
\end{itemize}


\subsubsection{UC\_13.1: Registrare file audio}
\label{uc_13.1}

\begin{itemize}
    \item \textbf{Attore principale}: Utente
    \item \textbf{Scenario principale}:
        \begin{itemize}
            \item L'Utente ha premuto il pulsante per registrare un audio
        \end{itemize}
    \item \textbf{Precondizioni}:
        \begin{itemize}
            \item La registrazione viene avviata
            \item Viene visualizzato il timer del tempo rimanente
            \item Viene visualizzato il pulsante per fermare la registrazione
        \end{itemize}
    \item \textbf{Postcondizioni}: 
        \begin{itemize}
            \item L'audio viene registrato
            \item L'audio viene archiviato
            \item L'audio viene inviato allo Speech-to-Text
        \end{itemize}
    \item \textbf{User Story}: L'utente vuole inserire un ordine registrando un messaggio audio
\end{itemize}


\subsubsection{UC\_13.2: Selezionare file audio}
\label{uc_13.2}

\begin{itemize}
    \item \textbf{Attore principale}: Utente
    \item \textbf{Scenario principale}:
        \begin{itemize}
            \item L'Utente ha premuto il pulsante per selezionare un file dal dispositivo
        \end{itemize}
    \item \textbf{Precondizioni}:
        \begin{itemize}
            \item Viene aperto il dialogo di selezione file del dispositivo
        \end{itemize}
    \item \textbf{Postcondizioni}: 
        \begin{itemize}
            \item Il file audio viene selezionato
            \item L'audio viene archiviato
            \item L'audio viene inviato allo Speech-to-Text 
        \end{itemize}
    \item \textbf{User Story}: L'utente vuole inserire un ordine mediante un file audio presente nel dispositivo
\end{itemize}


\subsubsection{UC\_14: Errore inserimento audio}
\label{uc_14}

\begin{itemize}
   \item \textbf{Attore principale}: Utente
   \item \textbf{Precondizioni}:
		\begin{itemize}
			\item L'Utente tenta di inserire un file audio
			\item L'audio audio eccede le dimensioni massime consentite
			\item L'audio audio non rispetta i formati accettati
			\item L'audio non rispetta la lunghezza massima consentita
		\end{itemize}
	\item \textbf{Postcondizioni}:
		\begin{itemize}
			\item L'inserimento dell'ordine fallisce
		\end{itemize}
\end{itemize}

\subsubsection{UC\_15: Aggiunta articoli al carrello}
\label{uc_15}
\begin{figure}[H]
        \centering
        \includegraphics[width=0.8\textwidth]{AnalisiRequisiti/UC_15}
        \caption{UC\_15: Aggiunta articoli al carrello}
\end{figure}

\begin{itemize}
    \item \textbf{Attore principale}: Cliente
    \item \textbf{Scenario principale}:
    \begin{enumerate}
        \item Il Cliente invia un ordine tramite input testuale, immagine o audio
        \item Il sistema elabora l'input, identifica gli articoli e le quantità
        \item Il sistema aggiunge gli articoli identificati al carrello del Cliente
        \item Il Cliente visualizza il carrello aggiornato
    \end{enumerate}
    \item \textbf{Precondizioni}:
        \begin{itemize}
            \item Il sistema è online
            \item Il Cliente è autenticato
        \end{itemize}
    \item \textbf{Postcondizioni}:
        \begin{itemize}
            \item Gli articoli identificati sono stati aggiunti al carrello del Cliente
        \end{itemize}
    \item \textbf{Scenari alternativi}:
        \begin{itemize}
            \item Il Cliente riceve un messaggio di errore di aggiunta articoli al carrello $\rightarrow$ Vedi \hyperref[uc_16]{UC\_16}
        \end{itemize}
    \item \textbf{Estensioni}:
    \begin{itemize}
            \item \hyperref[uc_16]{UC\_16}
        \end{itemize}
    \item \textbf{Trigger}: Il Cliente vuole aggiungere degli articoli al carrello mediante uno dei metodi possibili
\end{itemize}

\subsubsection{UC\_16: Errore aggiunta articoli al carrello}
\label{uc_16}

\begin{itemize}
    \item \textbf{Attore principale}: Cliente
    \item \textbf{Precondizioni}:
        \begin{itemize}
            \item Il Cliente ha richiesto l'aggiunta di articoli al carrello tramite una delle forme di input
            \item Il sistema non ha riconosciuto gli articoli richiesti
        \end{itemize}
	\item \textbf{Postcondizioni}:
        \begin{itemize}
            \item Il sistema visualizza un messaggio di errore di riconoscimento
            \item Gli articoli non vengono aggiunti al carrello
        \end{itemize}
\end{itemize}

\subsubsection{UC\_17: Rimozione articoli dal carrello}
\label{uc_17}
\begin{figure}[H]
        \centering
        \includegraphics[width=0.8\textwidth]{AnalisiRequisiti/UC_17}
        \caption{UC\_17: Rimozione articoli dal carrello}
\end{figure}

\begin{itemize}
    \item \textbf{Attore principale}: Utente
    \item \textbf{Scenario principale}:
        \begin{itemize}
            \item L'Utente rimuove degli articoli tramite input testuale
            \item L'Utente rimuove degli articoli tramite immagine
            \item L'Utente rimuove degli articoli tramite audio
        \end{itemize}
    \item \textbf{Precondizioni}:
        \begin{itemize}
            \item Il sistema è online
            \item L'Utente ha effettuato l’accesso al sistema
        \end{itemize}
    \item \textbf{Postcondizioni}:
        \begin{itemize}
            \item Gli articoli richiesti vengono rimossi al carrello
        \end{itemize}
    \item \textbf{Scenari alternativi}:
        \begin{itemize}
            \item L'utente riceve un messaggio di errore di rimozione articolo dal carrello. $\rightarrow$ Vedi [UC\_18, §\ref{uc_18}]
        \end{itemize}
    \item \textbf{Estensioni}:
    \begin{itemize}
            \item $[$UC\_18, §\ref{uc_18}$]$
        \end{itemize}
    \item \textbf{User Story}: L'utente vuole rimuovere degli articoli dal carrello
\end{itemize}

\subsubsection{UC\_18: Errore carrello vuoto}
\label{uc_18}

\begin{itemize}
    \item \textbf{Attore$^G$ Principale}: Cliente
	 \item \textbf{Scenario Principale}:
    \begin{enumerate}
        \item Il Cliente cerca di eliminare il contenuto del carrello quando questo è vuoto
    \end{enumerate}
   \item \textbf{Precondizioni}:
	\begin{itemize}
		\item Il sistema è online
		\item Il Cliente è autenticato
		\item Il Cliente ha richiesto la rimozione di articoli dal carrello.
		\item Il carrello è vuoto
    \end{itemize}
	\item \textbf{Postcondizioni}:
	\begin{itemize}
		\item Il sistema visualizza un messaggio di errore per carrello vuoto
		\item Nessun articolo viene rimosso dal carrello
	\end{itemize}
\end{itemize}

\subsubsection{UC\_19: Visualizza anteprima carrello}
\label{uc_19}

\begin{figure}[H]
	\centering
	\includegraphics[width=0.8\textwidth]{AnalisiRequisiti/UC_19}
	\caption{UC\_19: Visualizza anteprima carrello}
\end{figure}

\begin{itemize}
    \item \textbf{Attore$^G$ Principale}: Cliente
    \item \textbf{Scenario principale}:
	\begin{enumerate}
		\item Il Cliente invia il comando \texttt{/carrello} nella chat
		\item Il Cliente visualizza il riepilogo del carrello, elencando ogni articolo aggiunto $\rightarrow$ Vedi \hyperref[uc_19.1]{UC\_19.1} (per ciascun articolo)
	\end{enumerate}
	\item \textbf{Precondizioni}:
	\begin{itemize}
		\item Il sistema è online
		\item Il Cliente è autenticato
		\item La sessione chat del Cliente è attiva
		\item Sono presenti articoli nel carrello
	\end{itemize}
	\item \textbf{Postcondizioni}:
	\begin{itemize}
		\item Il Cliente visualizza correttamente l'anteprima del proprio carrello
	\end{itemize}
	\item \textbf{Scenari alternativi}:
	\begin{itemize}
		\item Il carrello è vuoto, il Cliente visualizza un messaggio di avviso specifico $\rightarrow$ Vedi \hyperref[uc_20]{UC\_20}
	\end{itemize}
	\item \textbf{Inclusioni}:
	\begin{itemize}
		\item \hyperref[uc_19.1]{UC\_19.1}
	\end{itemize}
	\item \textbf{Estensioni}:
	\begin{itemize}
		\item \hyperref[uc_20]{UC\_20}
	\end{itemize}
    \item \textbf{Trigger}: Il Cliente vuole visualizzare il riepilogo dell'ordine corrente
\end{itemize}

%--------------------------------------------------------------------
\subsubsubsection{UC\_19.1: Visualizza singolo elemento carrello}
\label{uc_19.1}

\begin{itemize}
    \item \textbf{Attore$^G$ Principale}: Cliente
	\item \textbf{Scenario principale}: 
	\begin{enumerate}
		\item Per ogni articolo presente nel carrello, il sistema ne visualizza le informazioni
	\end{enumerate}
	\item \textbf{Precondizioni}: 
	\begin{itemize}
		\item Il Cliente sta visualizzando il riepilogo del carrello (UC\_19)
		\item Esiste almeno un articolo nel carrello
	\end{itemize}
	\item \textbf{Postcondizioni}:
	\begin{itemize}
		\item Le informazioni di ogni articolo nel carrello sono state visualizzate
	\end{itemize}
	\item \textbf{Inclusioni}:
	\begin{itemize}
		\item \hyperref[uc_19.1.1]{UC\_19.1.1}
		\item \hyperref[uc_19.1.2]{UC\_19.1.2}
		\item \hyperref[uc_19.1.3]{UC\_19.1.3}
	\end{itemize}
\end{itemize}

%--------------------------------------------------------------------
\subsubsubsubsection{UC\_19.1.1: Visualizza nome articolo}
\label{uc_19.1.1}
\begin{itemize}
    \item \textbf{Attore$^G$ Principale}: Cliente
	\item \textbf{Scenario principale}: 
	\begin{enumerate}
		\item Il Cliente visualizza il nome dell'articolo
	\end{enumerate}
	\item \textbf{Precondizioni}: 
	\begin{itemize}
		\item Il Cliente sta visualizzando un elemento del carrello (UC\_19.1)
	\end{itemize}
	\item \textbf{Postcondizioni}:
	\begin{itemize}
		\item Il nome dell'articolo è visibile
	\end{itemize} 
\end{itemize}

%--------------------------------------------------------------------
\subsubsubsubsection{UC\_19.1.2: Visualizza prezzo articolo}
\label{uc_19.1.2}
\begin{itemize}
    \item \textbf{Attore$^G$ Principale}: Cliente
	\item \textbf{Scenario principale}: 
	\begin{enumerate}
		\item Il Cliente visualizza il prezzo dell'articolo
	\end{enumerate}
	\item \textbf{Precondizioni}: 
	\begin{itemize}
		\item Il Cliente sta visualizzando un elemento del carrello (UC\_19.1)
	\end{itemize}
	\item \textbf{Postcondizioni}:
	\begin{itemize}
		\item Il prezzo dell'articolo è visibile
	\end{itemize} 
\end{itemize}

%--------------------------------------------------------------------
\subsubsubsubsection{UC\_19.1.3: Visualizza quantità articolo}
\label{uc_19.1.3}
\begin{itemize}
    \item \textbf{Attore$^G$ Principale}: Cliente
	\item \textbf{Scenario principale}: 
	\begin{enumerate}
		\item Il Cliente visualizza la quantità dell'articolo
	\end{enumerate}
	\item \textbf{Precondizioni}: 
	\begin{itemize}
		\item Il Cliente sta visualizzando un elemento del carrello (UC\_19.1)
	\end{itemize}
	\item \textbf{Postcondizioni}:
	\begin{itemize}
		\item La quantità dell'articolo è visibile
	\end{itemize} 
\end{itemize}

\subsubsection{UC\_20: Errore visualizzazione anteprima}
\label{uc_20}

\begin{itemize}
    \item \textbf{Attore$^G$ Principale}: Cliente
	 \item \textbf{Scenario Principale}:
    \begin{enumerate}
        \item Il Cliente inserisce il comando \texttt{/carrello} quando il carrello è vuoto
    \end{enumerate}
	\item \textbf{Precondizioni}:
	\begin{itemize}
		\item Il sistema è online
		\item Il Cliente è autenticato
	\end{itemize}
	\item \textbf{Postcondizioni}:
	\begin{itemize}
		\item Il Cliente visualizza un avviso che comunica che il carrello è vuoto
	\end{itemize}
\end{itemize}

\subsubsection{UC\_21: Gestione ambiguità prodotto}
\label{uc_21}

\begin{figure}[H]
	\centering
	\includegraphics[width=0.8\textwidth]{AnalisiRequisiti/UC_21}
	\caption{UC\_21: Gestione ambiguità prodotto}
\end{figure}

\begin{itemize}
	\item \textbf{Attore principale}: Cliente
	\item \textbf{Scenario principale}:
	\begin{enumerate}
		\item Al Cliente vengono segnalate le ambiguità rilevate dal chatbot 
		\item Il Cliente riformula i prodotti ambigui $\rightarrow$ Vedi \hyperref[uc_21.1]{UC\_21.1}
		\item Il Cliente riceve conferma dell'aggiornamento dell'ordine $\rightarrow$ Vedi \hyperref[uc_21.2]{UC\_21.2}
	\end{enumerate}
	\item \textbf{Precondizioni}:
	\begin{itemize}
		\item Il sistema è online
		\item Il Cliente vuole aggiungere uno o più prodotti all'ordine tramite una forma di input
	\end{itemize}
	\item \textbf{Postcondizioni}:
	\begin{itemize}
		\item I dati dell'ordine vengono aggiornati e reinviati alla pipeline per la prosecuzione dell'elaborazione
	\end{itemize}
	\item \textbf{Scenari alternativi}:
	\begin{itemize}
		\item Il Cliente può decidere di uscire dalla disambiguazione annullandola. $\rightarrow$ Vedi \hyperref[uc_22]{UC\_22}
	\end{itemize}
	\item \textbf{Inclusioni}:
	\begin{itemize}
		\item \hyperref[uc_21.1]{UC\_21.1}
		\item \hyperref[uc_21.2]{UC\_21.2}
	\end{itemize}
	\item \textbf{Estensioni}:
	\begin{itemize}
		\item \hyperref[uc_22]{UC\_22}
	\end{itemize}
	\item \textbf{Trigger}: Il Cliente, quando effettua una richiesta che risulta ambigua al chatbot, deve risolvere l'ambiguità su proposta del chatbot
\end{itemize}

\subsubsubsection{UC\_21.1: Riformulazione prodotti ambigui}
\label{uc_21.1}

\begin{itemize}
	\item \textbf{Attore principale}: Cliente
	\item \textbf{Scenario principale}:
	\begin{enumerate}
		\item Il Cliente ha riformulato l'input $\rightarrow$ Vedi \hyperref[uc_21]{UC\_21}.
	\end{enumerate}
	\item \textbf{Precondizioni}:
	\begin{itemize}
		\item Il sistema è online
		\item Al Cliente viene segnalata almeno una ambiguità
	\end{itemize}
	\item \textbf{Postcondizioni}:
	\begin{itemize}
		\item Il Cliente ha riformulato l'input.
	\end{itemize}
	\item \textbf{Trigger}: Il Cliente riceve una conferma visiva dell'aggiunta degli articoli al carrello
\end{itemize}

\subsubsubsection{UC\_21.2: Visualizza conferma aggiunta articoli}
\label{uc_21.2}

\begin{itemize}
	\item \textbf{Attore principale}: Cliente
	\item \textbf{Scenario principale}:
	\begin{enumerate}
		\item Il Cliente visualizza un messaggio di conferma dell’avvenuta aggiunta dei prodotti al carrello
	\end{enumerate}
	\item \textbf{Precondizioni}:
	\begin{itemize}
		\item Il Cliente ha riformulato l'input $\rightarrow$ Vedi \hyperref[uc_21.1]{UC\_21.1}.
		\item La disambiguazione è risolta
	\end{itemize}
	\item \textbf{Postcondizioni}:
	\begin{itemize}
		\item Il Cliente visualizza la conferma dell'aggiunta degli articoli nel carrello
	\end{itemize}
	\item \textbf{Trigger}: Il Cliente riceve una conferma visiva dell'aggiunta degli articoli al carrello
\end{itemize}

\subsubsection{UC\_22: Annullamento disambiguazione}
\label{uc_22}

\begin{figure}[H]
	\centering
	\includegraphics[width=0.8\textwidth]{AnalisiRequisiti/UC_22}
	\caption{UC\_22: Annullamento disambiguazione}
\end{figure}

\begin{itemize}
	\item \textbf{Attore$^G$ principale}: Cliente
	\item \textbf{Scenario principale}:
	\begin{enumerate}
		\item Il Cliente decide di annullare il processo$^G$ di disambiguazione
		\item Il Cliente visualizza il messaggio di conferma per l'interruzione del processo$^G$ di disambiguazione
	\end{enumerate}
	\item \textbf{Precondizioni}:
	\begin{itemize}
		\item Il processo$^G$ di gestione ambiguità è attivo $\rightarrow$ Vedi \hyperref[uc_21]{UC\_21}
	\end{itemize}
	\item \textbf{Postcondizioni}:
	\begin{itemize}
		\item Il processo$^G$ di disambiguazione viene annullato
		\item L'ordine rimane nello stato precedente all'ambiguità
	\end{itemize}
	\item \textbf{Trigger}: Il Cliente durante la chiarificazione di un prodotto ambiguo, vuole annullare l'operazione di disambiguazione
\end{itemize}

\subsubsubsection{UC\_22.1: Invio comando}
\label{uc_22.1}

\begin{itemize}
	\item \textbf{Attore$^G$ principale}: Cliente
	\item \textbf{Scenario principale}:
	\begin{enumerate}
		\item Il Cliente seleziona il comando \texttt{/annulla}
		\item Il Cliente invia il comando \texttt{/annulla}
	\end{enumerate}
	\item \textbf{Precondizioni}:
	\begin{itemize}
		\item È attivo un processo$^G$ di disambiguazione $\rightarrow$ Vedi \hyperref[uc_21]{UC\_21}
	\end{itemize}
	\item \textbf{Postcondizioni}:
	\begin{itemize}
		\item La richiesta di annullamento è stata ricevuta dal sistema
	\end{itemize}
\end{itemize}

\subsubsubsection{UC\_22.2: Notifica di conferma annullamento}
\label{uc_22.2}

\begin{itemize}
	\item \textbf{Attore$^G$ principale}: Cliente
	\item \textbf{Scenario principale}:
	\begin{enumerate}
		\item Il Cliente visualizza la notifica di conferma dell'avvenuto annullamento della disambiguazione
	\end{enumerate}
	\item \textbf{Precondizioni}:
	\begin{itemize}
		\item Il Cliente ha richiesto l'annullamento del processo$^G$ di disambiguazione $\rightarrow$ Vedi \hyperref[uc_22.1]{UC\_22.1}
	\end{itemize}
	\item \textbf{Postcondizioni}:
	\begin{itemize}
		\item Il Cliente è informato del corretto annullamento del processo$^G$
	\end{itemize}
\end{itemize}

\subsubsection{UC\_23: Fornire feedback all'AI}
\label{uc_23}

\begin{figure}[H]
	\centering
	\includegraphics[width=0.8\textwidth]{AnalisiRequisiti/UC_23}
	\caption{UC\_23: Fornire feedback all'AI}
\end{figure}

\begin{itemize}
    \item \textbf{Attore Principale}: Cliente
	\item \textbf{Scenario principale}:
	\begin{enumerate}
		\item Il Cliente, tramite il pulsante affianco all'ultimo messaggio dell'AI, apre il form per la compilazione del feedback
		\item Il Cliente compila il campo tipologia del feedback $\rightarrow$ Vedi \hyperref[uc_23.1]{UC\_23.1}
		\item Il Cliente compila il campo descrizione del feedback $\rightarrow$ Vedi \hyperref[uc_23.2]{UC\_23.2}
        \item Il Cliente invia il feedback $\rightarrow$ Vedi \hyperref[uc_23.3]{UC\_23.3}
        \item Il Cliente riceve conferma dell'invio del feedback $\rightarrow$ Vedi \hyperref[uc_23.4]{UC\_23.4}
	\end{enumerate}
	\item \textbf{Precondizioni}:
	\begin{itemize}
		\item Il sistema è online
		\item Il Cliente ha ricevuto una risposta dal chatbot AI
	\end{itemize}
	\item \textbf{Postcondizioni}:
	\begin{itemize}
		\item Il feedback è stato registrato nel sistema
		\item Il Cliente ha ricevuto conferma dell'operazione
	\end{itemize}
	\item \textbf{Inclusioni}:
	\begin{itemize}
		\item \hyperref[uc_23.1]{UC\_23.1}
		\item \hyperref[uc_23.2]{UC\_23.2}
		\item \hyperref[uc_23.3]{UC\_23.3}
		\item \hyperref[uc_23.4]{UC\_23.4}
	\end{itemize}
	\item \textbf{Trigger}: Il Cliente vuole valutare le risposte del chatbot tramite feedback, per contribuire a migliorarne l'accuratezza
\end{itemize}

\subsubsubsection{UC\_23.1: Inserimento tipologia feedback}
\label{uc_23.1}
\begin{itemize}
    \item \textbf{Attore Principale}: Cliente
	\item \textbf{Scenario principale}:
	\begin{enumerate}
		\item Il Cliente compila il campo relativo alla tipologia del feedback
	\end{enumerate}
	\item \textbf{Precondizioni}:
	\begin{itemize}
		\item Il modulo di feedback è aperto e il campo tipologia è visibile al Cliente
	\end{itemize}
	\item \textbf{Postcondizioni}:
	\begin{itemize}
		\item Il campo tipologia del feedback è compilato
	\end{itemize}
\end{itemize}

\subsubsubsection{UC\_23.2: Inserimento descrizione feedback}
\label{uc_23.2}
\begin{itemize}
    \item \textbf{Attore Principale}: Cliente
	\item \textbf{Scenario principale}:
	\begin{enumerate}
		\item Il Cliente compila il campo descrizione del feedback
	\end{enumerate}
	\item \textbf{Precondizioni}:
	\begin{itemize}
		\item Il modulo di feedback è aperto e il campo descrizione è visibile al Cliente
	\end{itemize}
	\item \textbf{Postcondizioni}:
	\begin{itemize}
		\item Il campo descrizione del feedback è compilato
	\end{itemize}
\end{itemize}

\subsubsubsection{UC\_23.3: Invio feedback}
\label{uc_23.3}
\begin{itemize}
    \item \textbf{Attore Principale}: Cliente
	\item \textbf{Scenario principale}:
	\begin{enumerate}
		\item Il Cliente preme il pulsante di invio del feedback
	\end{enumerate}
	\item \textbf{Precondizioni}:
	\begin{itemize}
		\item Il modulo di feedback è compilato (campi tipologia e descrizione inseriti)
	\end{itemize}
	\item \textbf{Postcondizioni}:
	\begin{itemize}
		\item La richiesta di invio del feedback è stata trasmessa al sistema
	\end{itemize}
\end{itemize}

\subsubsubsection{UC\_23.4: Notifica invio feedback}
\label{uc_23.4}
\begin{itemize}
    \item \textbf{Attore Principale}: Cliente
	\item \textbf{Scenario principale}:
	\begin{enumerate}
		\item Il Cliente visualizza la conferma dell'invio del feedback
	\end{enumerate}
	\item \textbf{Precondizioni}:
	\begin{itemize}
		\item Il Cliente ha inviato il feedback (UC\_23.3)
	\end{itemize}
	\item \textbf{Postcondizioni}:
	\begin{itemize}
		\item Il sistema ha confermato l'avvenuta registrazione del feedback
	\end{itemize}
\end{itemize}

\subsubsection{UC\_24: Segnalazione bug o supporto}
\label{uc_24}

\begin{figure}[H]
	\centering
	\includegraphics[width=0.8\textwidth]{AnalisiRequisiti/UC_24}
	\caption{UC\_24: Segnalare bug o supporto}
\end{figure}

\begin{itemize}
	\item \textbf{Attore principale}: Cliente
	\item \textbf{Scenario principale}:
	\begin{enumerate}
		\item Il Cliente apre il modulo di segnalazione
		\item Il Cliente compila la tipologia di segnalazione $\rightarrow$ Vedi \hyperref[uc_24.1]{UC\_24.1}
		\item Il Cliente compila la descrizione del problema $\rightarrow$ Vedi \hyperref[uc_24.2]{UC\_24.2}
		\item Il Cliente invia la segnalazione $\rightarrow$ Vedi \hyperref[uc_24.3]{UC\_24.3}
		\item Il sistema mostra una conferma di avvenuto invio
	\end{enumerate}
	\item \textbf{Precondizioni}:
	\begin{itemize}
		\item Il sistema è attivo.
	\end{itemize}
	\item \textbf{Postcondizioni}:
	\begin{itemize}
		\item Un ticket di segnalazione viene registrato nel sistema
	\end{itemize}
	\item \textbf{Inclusioni}:
	\begin{itemize}
		\item \hyperref[uc_24.1]{UC\_24.1}
		\item \hyperref[uc_24.2]{UC\_24.2}
		\item \hyperref[uc_24.3]{UC\_24.3}
	\end{itemize}
	\item \textbf{Trigger}: Il Cliente vuole segnalare problemi o richiedere supporto, per risolvere eventuali difficoltà incontrate durante l'utilizzo del servizio.
\end{itemize}

\subsubsubsection{UC\_24.1: Inserimento tipologia segnalazione}
\label{uc_24.1}
\begin{itemize}
	\item \textbf{Attore principale}: Cliente
	\item \textbf{Scenario principale}:
	\begin{enumerate}
		\item Il Cliente visiona il modulo di segnalazione
		\item Il Cliente compila il campo relativo alla tipologia
	\end{enumerate}
	\item \textbf{Precondizioni}:
	\begin{itemize}
		\item Il modulo di segnalazione è aperto
	\end{itemize}
	\item \textbf{Postcondizioni}:
	\begin{itemize}
		\item La tipologia di segnalazione viene impostata
	\end{itemize}
\end{itemize}

\subsubsubsection{UC\_24.2: Inserimento descrizione problema}
\label{uc_24.2}
\begin{itemize}
	\item \textbf{Attore principale}: Cliente
	\item \textbf{Scenario principale}:
	\begin{enumerate}
		\item Il Cliente visiona il modulo di segnalazione 
		\item Il Cliente inserisce una descrizione testuale del problema o della richiesta di supporto
	\end{enumerate}
	\item \textbf{Precondizioni}:
	\begin{itemize}
		\item Il modulo di segnalazione è aperto
		\item Campo tipologia completato
	\end{itemize}
	\item \textbf{Postcondizioni}:
	\begin{itemize}
		\item Il Cliente ha inserito correttamente la descrizione del problema nel modulo
	\end{itemize}
\end{itemize}

\subsubsubsection{UC\_24.3: Invio segnalazione}
\label{uc_24.3}
\begin{itemize}
	\item \textbf{Attore principale}: Cliente
	\item \textbf{Scenario principale}:
	\begin{enumerate}
		\item Il Cliente invia la segnalazione
	\end{enumerate}
	\item \textbf{Precondizioni}:
	\begin{itemize}
		\item Il modulo di segnalazione è compilato in tutti i campi obbligatori
	\end{itemize}
	\item \textbf{Postcondizioni}:
	\begin{itemize}
		\item Il Cliente riceve una conferma
	\end{itemize}
\end{itemize}

\subsubsubsection{UC\_24.4: Notifica invio segnalazione}
\label{uc_23.4}
\begin{itemize}
	\item \textbf{Attore principale}: Cliente
	\item \textbf{Scenario principale}:
	\begin{enumerate}
		\item Il Cliente riceve la conferma dell'invio della segnalazione
	\end{enumerate}
	\item \textbf{Precondizioni}:
	\begin{itemize}
		\item Il Cliente ha premuto il pulsante di invio della segnalazione
	\end{itemize}
	\item \textbf{Postcondizioni}:
	\begin{itemize}
		\item Il Cliente riceve una conferma dell'avvenuto invio
	\end{itemize}
\end{itemize}

\subsubsection{UC\_25: Visualizza pagina performance}
\label{uc_25}

\begin{figure}[H]
	\centering
	\includegraphics[width=0.8\textwidth]{AnalisiRequisiti/UC_25}
	\caption{UC\_25: Visualizza pagina performance}
\end{figure}

\begin{itemize}
	\item \textbf{Attore principale}: Admin
	\item \textbf{Scenario principale}:
	\begin{enumerate}
		\item L'Admin accede alla pagina web di monitoraggio delle performance
		\item L'Admin visualizza durata media di risposta del chatbot $\rightarrow$ Vedi \hyperref[uc_25.1]{UC\_25.1}
		\item L'Admin visualizza durata media di permanenza nella web-app $\rightarrow$ Vedi \hyperref[uc_25.2]{UC\_25.2}
	\end{enumerate}
	\item \textbf{Precondizioni}:
	\begin{itemize}
		\item L'Admin è autenticato come Admin.
		\item Il sistema è online
	\end{itemize}
	\item \textbf{Postcondizioni}:
	\begin{itemize}
		\item Vengono visualizzate le metriche di performance
	\end{itemize}
	\item \textbf{Inclusioni}:
	\begin{itemize}
		\item \hyperref[uc_25.1]{UC\_25.1}
		\item \hyperref[uc_25.2]{UC\_25.2}
	\end{itemize}
	\item \textbf{Trigger}: L'Admin deve poter visualizzare le metriche di utilizzo del sistema, per valutarne le prestazioni e l'engagement degli utenti.
\end{itemize}

\subsubsubsection{UC\_25.1: Visualizza durata media di risposta del chatbot}
\label{uc_25.1}
\begin{itemize}
	\item \textbf{Attore principale}: Admin
	\item \textbf{Scenario principale}:
	\begin{enumerate}
		\item L'Admin visiona la durata media di risposta del chatbot
	\end{enumerate}
	\item \textbf{Precondizioni}:
	\begin{itemize}
		\item L'Admin visiona la pagina web delle performance
	\end{itemize}
	\item \textbf{Postcondizioni}:
	\begin{itemize}
		\item L'Admin visiona la durata media di risposta del chatbot
	\end{itemize}
\end{itemize}

\subsubsubsection{UC\_25.2: Visualizza durata media di permanenza nella web-app}
\label{uc_25.2}
\begin{itemize}
	\item \textbf{Attore principale}: Admin
	\item \textbf{Scenario principale}:
	\begin{enumerate}
		\item L'Admin visiona la durata media di permanenza degli utenti nella web-app
	\end{enumerate}
	\item \textbf{Precondizioni}:
	\begin{itemize}
		\item L'Admin visiona la pagina web delle performance
	\end{itemize}
	\item \textbf{Postcondizioni}:
	\begin{itemize}
		\item L'Admin visiona la durata media di permanenza nella web-app
	\end{itemize}
\end{itemize}

\subsubsection{UC\_26: Invio Ordine}
\label{uc_26}

\begin{figure}[H]
	\centering
	\includegraphics[width=0.8\textwidth]{AnalisiRequisiti/UC_26&UC_27}
	\caption{UC\_26: Invio Ordine}
\end{figure}

\begin{itemize}
	\item \textbf{Attore principale}: Utente
	\item \textbf{Scenario principale}:
	\begin{enumerate}
		\item L'Utente digita il comando \texttt{/invio} nella chat per confermare l'ordine
		\item L'Utente visiona il riepilogo dell'ordine $\rightarrow$ \hyperref[uc_26.1]{UC\_26.1}
		\item L'Utente conferma l'invio dell'ordine $\rightarrow$ \hyperref[uc_26.2]{UC\_26.2}
		\item L'Utente visualizza la conferma dell'acquisto
	\end{enumerate}
	\item \textbf{Precondizioni}:
	\begin{itemize}
		\item Il carrello dell'Utente non è vuoto
	\end{itemize}
	\item \textbf{Postcondizioni}:
	\begin{itemize}
		\item L'ordine viene confermato e inserito nello storico
		\item L'Utente riceve conferma dell'invio
	\end{itemize}
	\item \textbf{Scenari alternativi}:
	\begin{itemize}
		\item Annullamento ordine a procedura avviata $\rightarrow$ \hyperref[uc_26.3]{UC\_26.3}
		\item Tentativo di invio di un ordine nullo $\rightarrow$ \hyperref[uc_27]{UC\_27}
	\end{itemize}
	\item \textbf{Inclusioni}:
	\begin{itemize}
		\item \hyperref[uc_26.1]{UC\_26.1} $\rightarrow$ Visualizza riepilogo ordine
		\item \hyperref[uc_26.2]{UC\_26.2} $\rightarrow$ Conferma ordine $\rightarrow$ 
	\end{itemize}
	\item \textbf{Estensioni}:
	\begin{itemize}
		\item \hyperref[uc_26.3]{UC\_26.3}
		\item \hyperref[uc_27]{UC\_27}
	\end{itemize}
	\item \textbf{User Story}: l'Utente deve poter confermare e inviare l'ordine con un semplice comando, per completare l'acquisto in modo rapido.
\end{itemize}

\subsubsubsection{UC\_26.1: Visualizza riepilogo ordine}
\label{uc_26.1}
\begin{itemize}
    \item \textbf{Attore principale}: Utente
    \item \textbf{Scenario principale}:
    \begin{enumerate}
        \item L'Utente vede il riepilogo dell'ordine con un numero identificativo.
    \end{enumerate}
    \item \textbf{Precondizioni}:
    \begin{itemize}
        \item Il carrello non è vuoto.
        \item L'Utente ha digitato \texttt{/invio}
    \end{itemize}
    \item \textbf{Postcondizioni}:
    \begin{itemize}
        \item Il riepilogo dell'ordine è visualizzato correttamente.
    \end{itemize}
\end{itemize}

\subsubsubsection{UC\_26.2: Conferma ordine}
\label{uc_26.2}
\begin{itemize}
    \item \textbf{Attore principale}: Utente
    \item \textbf{Scenario principale}:
    \begin{enumerate}
        \item L'Utente conferma l'invio dell'ordine dal riepilogo.
    \end{enumerate}
    \item \textbf{Precondizioni}:
    \begin{itemize}
        \item Il carrello non è vuoto.
        \item Il riepilogo dell'ordine è visualizzato.
    \end{itemize}
    \item \textbf{Postcondizioni}:
    \begin{itemize}
        \item L'ordine è confermato.
        \item L'Utente visualizza la conferma dell'invio.
    \end{itemize}
\end{itemize}

\subsubsubsection{UC\_26.3: Annullamento ordine}
\label{uc_26.3}
\begin{itemize}
    \item \textbf{Attore principale}: Utente
    \item \textbf{Scenario principale}:
    \begin{enumerate}
        \item L'Utente annulla l'ordine tramite comando "annulla".
        \item L'Utente visiona un messaggio di conferma di annullamento dell'invio dell'ordine
    \end{enumerate}
    \item \textbf{Precondizioni}:
    \begin{itemize}
        \item Il carrello non è vuoto.
        \item Il riepilogo dell'ordine è visualizzato.
    \end{itemize}
    \item \textbf{Postcondizioni}:
    \begin{itemize}
        \item L'ordine non viene inviato.
        \item Il contenuto del carrello rimane invariato.
    \end{itemize}
\end{itemize}

\subsubsection{UC\_27: Errore ordine nullo}
\label{uc_27}

\begin{itemize}
    \item \textbf{Attore$^G$ Principale}: Cliente
	 \item \textbf{Scenario principale}:
    \begin{enumerate}
        \item Il Cliente invia un ordine con il carrello vuoto
    \end{enumerate}
	\item \textbf{Precondizioni}:
	\begin{itemize}
		\item Il sistema è online
		\item Il Cliente è autenticato
	\end{itemize}
	\item \textbf{Postcondizioni}:
	\begin{itemize}
		\item Il sistema visualizza un messaggio di errore per mancanza di articoli nel carrello
	\end{itemize}
\end{itemize}

\subsubsection{UC\_28: Inizio nuova chat}
\label{uc_28}

\begin{figure}[H]
	\centering
	\includegraphics[width=0.8\textwidth]{AnalisiRequisiti/UC_28}
	\caption{UC\_28: Inizio nuova chat}
\end{figure}

\begin{itemize}
	\item \textbf{Attore$^G$ principale}: Cliente
	\item \textbf{Scenario principale}:
	\begin{enumerate}
		\item Il Cliente visualizza il pulsante per iniziare una nuova chat
		\item Il Cliente preme il pulsante per confermare l'inizio di una nuova chat
		\item Il sistema avvia una nuova sessione di chat con contesto resettato
		\item Il Cliente visualizza la nuova sessione di chat vuota
	\end{enumerate}
	\item \textbf{Scenari alternativi}:
	\begin{itemize}
		\item Il Cliente decide di non iniziare una nuova chat; la sessione corrente rimane attiva
	\end{itemize}
	\item \textbf{Precondizioni}:
	\begin{itemize}
		\item Il sistema è online
		\item Il Cliente è autenticato
		\item Una sessione chat del Cliente è attiva
	\end{itemize}
	\item \textbf{Postcondizioni}:
	\begin{itemize}
		\item Una nuova sessione di chat è stata avviata
		\item La cronologia della chat precedente non è più visibile
		\item Il carrello associato alla sessione è vuoto
	\end{itemize}
	\item \textbf{Trigger}: Il Cliente vuole ricominciare da zero con una nuova chat, resettando e il carrello
\end{itemize}

\subsubsection{UC\_29: Esportare log}
\label{uc_29}

\begin{figure}[H]
	\centering
	\includegraphics[width=0.8\textwidth]{AnalisiRequisiti/UC_29}
	\caption{UC\_29: Esportare log}
\end{figure}

\begin{itemize}
	\item \textbf{Attore$^G$ principale}: Admin
	\item \textbf{Scenario principale}:
	\begin{enumerate}
		\item L'Admin clicca il pulsante relativo all'esportazione dei log
		\item Il sistema genera ed esporta il file di log
		\item L'Admin visualizza/scarica il file esportato
	\end{enumerate}
	\item \textbf{Precondizioni}:
	\begin{itemize}
		\item L'Admin è autenticato ed autorizzato
		\item Sono presenti log esportabili nel sistema
	\end{itemize}
	\item \textbf{Postcondizioni}:
	\begin{itemize}
		\item Il file è stato esportato e reso disponibile all'Admin
	\end{itemize}
	\item \textbf{Trigger}: L'Admin deve poter esportare i log delle attività, per poterli analizzare con strumenti esterni
\end{itemize}

\subsubsection{UC\_30: Esportazione storico ordini dei Clienti}
\label{uc_30}

\begin{figure}[H]
	\centering
	\includegraphics[width=0.8\textwidth]{AnalisiRequisiti/UC_30}
	\caption{UC\_30: Esportazione storico ordini (JSON)}
\end{figure}

\begin{itemize}
	\item \textbf{Attore principale}: Admin
	\item \textbf{Scenario principale}:
	\begin{enumerate}
		\item L'Admin richiede l'esportazione dello storico ordini in formato JSON
		\item Il Sistema genera e rende disponibile il file JSON
		\item L'Admin scarica lo storico ordini in formato JSON
		\item Il sistema mostra una notifica di download riuscito $\rightarrow$ Vedi \hyperref[uc_30.1]{UC\_30.1}
	\end{enumerate}
	\item \textbf{Precondizioni}:
	\begin{itemize}
		\item Il Sistema è online
		\item L'Admin è autenticato ed autorizzato
		\item Sono presenti ordini esportabili
	\end{itemize}
	\item \textbf{Postcondizioni}:
	\begin{itemize}
		\item Il file JSON con lo storico ordini è stato generato e reso disponibile per il download
		\item L'Admin ha scaricato il file
	\end{itemize}
	\item \textbf{Scenari alternativi}:
	\begin{itemize}
		\item Si verifica un errore nel download dello storico ordini $\rightarrow$ Vedi \hyperref[uc_31]{UC\_31}
	\end{itemize}
	\item \textbf{Inclusioni}:
	\begin{itemize}
		\item \hyperref[uc_30.1]{UC\_30.1}
	\end{itemize}
	\item \textbf{Estensioni}:
	\begin{itemize}
		\item \hyperref[uc_31]{UC\_31}
	\end{itemize}
	\item \textbf{Trigger}: L'Admin vuole esportare lo storico ordini in formato JSON
\end{itemize}

\subsubsubsection{UC\_30.1: Visualizzazione notifica (download riuscito)}
\label{uc_30.1}

\begin{itemize}
	\item \textbf{Attore principale}: Admin
	\item \textbf{Scenario principale}:
	\begin{enumerate}
		\item Il sistema mostra una notifica di download riuscito
		\item L'Admin visualizza la notifica
	\end{enumerate}
	\item \textbf{Precondizioni}:
	\begin{itemize}
		\item L'Admin ha completato il download del file JSON (UC\_30, passo 3)
		\item Il file JSON è stato generato correttamente
	\end{itemize}
	\item \textbf{Postcondizioni}:
	\begin{itemize}
		\item La notifica di download riuscito è stata visualizzata
	\end{itemize}
	\item \textbf{Trigger}: Il sistema conferma il completamento dell'operazione di esportazione
\end{itemize}

\subsubsection{UC\_31: Errore download}
\label{uc_31}

\begin{itemize}
	\item \textbf{Attore$^G$ principale}: Admin
	\item \textbf{Precondizioni}:
	\begin{itemize}
		\item Il Sistema è online
		\item L'Admin è autenticato ed autorizzato
		\item L'Admin ha richiesto l'esportazione dello storico ordini $\rightarrow$ Vedi \hyperref[uc_30]{UC\_30}
	\end{itemize}
	\item \textbf{Postcondizioni}:
	\begin{itemize}
		\item Il sistema visualizza un messaggio di errore che impedisce il completamento del download
		\item Il file JSON$^G$ non viene scaricato
	\end{itemize}
\end{itemize}

\subsubsection{UC\_32: Creazione nuovi Utenti}
\label{uc_32}

\begin{figure}[H]
	\centering
	\includegraphics[width=0.8\textwidth]{AnalisiRequisiti/UC_32}
	\caption{UC\_32: Creazione nuovi Utenti}
\end{figure}


\begin{itemize}
    \item \textbf{Attore$^G$ Principale}: Admin
	\item \textbf{Scenario principale}:
	\begin{enumerate}
		\item L'Admin inserisce il ruolo $\rightarrow$ Vedi \hyperref[uc_32.1]{UC\_32.1}
		\item L'Admin inserisce lo username $\rightarrow$ Vedi \hyperref[uc_32.2]{UC\_32.2}
		\item L'Admin inserisce l'email $\rightarrow$ Vedi \hyperref[uc_32.3]{UC\_32.3}
		\item L'Admin conferma la creazione del nuovo Utente
	\end{enumerate}
	\item \textbf{Precondizioni}:
	\begin{itemize}
		\item Il sistema è online
		\item L'Admin è autenticato ed autorizzato
	\end{itemize}
	\item \textbf{Postcondizioni}:
	\begin{itemize}
		\item Il nuovo Utente è stato creato nel Sistema
	\end{itemize}
	\item \textbf{Scenari alternativi}:
	\begin{itemize}
		\item La creazione Utente fallisce e il sistema mostra una notifica con spiegazione dell'errore $\rightarrow$ Vedi \hyperref[uc_33]{UC\_33}
	\end{itemize}
	\item \textbf{Inclusioni}:
	\begin{itemize}
		\item \hyperref[uc_32.1]{UC\_32.1}
		\item \hyperref[uc_32.2]{UC\_32.2}
		\item \hyperref[uc_32.3]{UC\_32.3}
	\end{itemize}
	\item \textbf{Estensioni}:
	\begin{itemize}
		\item \hyperref[uc_33]{UC\_33}
	\end{itemize}
	\item \textbf{Trigger}: L'Admin vuole creare un nuovo Utente
\end{itemize}

\subsubsubsection{UC\_32.1: Inserimento ruolo}
\label{uc_32.1}

\begin{itemize}
    \item \textbf{Attore$^G$ Principale}: Admin
	\item \textbf{Scenario principale}:
	\begin{enumerate}
		\item L'Admin inserisce il ruolo del nuovo Utente
	\end{enumerate}
	\item \textbf{Precondizioni}:
	\begin{itemize}
        \item Il sistema è online
        \item L'Utente non è registrato nel sistema
		\item L'Admin sta creando un nuovo Utente $\rightarrow$ Vedi \hyperref[uc_32]{UC\_32}
	\end{itemize}
	\item \textbf{Postcondizioni}:
	\begin{itemize}
		\item Il ruolo del nuovo Utente è stato inserito
	\end{itemize}
	\item \textbf{Trigger}: L'Admin deve specificare il ruolo del nuovo Utente
\end{itemize}

\subsubsubsection{UC\_32.2: Inserimento username}
\label{uc_32.2}

\begin{itemize}
    \item \textbf{Attore$^G$ Principale}: Admin
    \item \textbf{Scenario Principale}:
    \begin{enumerate}
        \item L'Admin inserisce lo username del nuovo Utente
    \end{enumerate}
    \item \textbf{Precondizioni}: 
    \begin{itemize}
        \item Il sistema è online
        \item L'Utente non è registrato nel sistema
		\item L'Admin sta creando un nuovo Utente $\rightarrow$ Vedi \hyperref[uc_32]{UC\_32}
    \end{itemize}
    \item \textbf{Postcondizioni}:
    \begin{itemize}
        \item Username inserito correttamente
    \end{itemize}
    \item \textbf{Scenari Alternativi}:
    \begin{itemize}
        \item L'Admin ha inserito uno username non valido $\rightarrow$ Vedi \hyperref[uc_32.2.1]{UC\_32.2.1}
        \item L'Admin ha inserito uno username già esistente $\rightarrow$ Vedi \hyperref[uc_32.2.2]{UC\_32.2.2}
    \end{itemize}
    \item \textbf{Estensioni}:
    \begin{itemize}
        \item \hyperref[uc_32.2.1]{UC\_32.2.1}
        \item \hyperref[uc_32.2.2]{UC\_32.2.2}
    \end{itemize}
    \item \textbf{Trigger}: L'Admin vuole inserire l'username per la registrazione
\end{itemize}

UC\_32.2 presenta due ulteriori estensioni, come mostrato in figura:

\begin{figure}[H]
    \centering 
    \includegraphics[width=0.8\textwidth]{AnalisiRequisiti/UC_32.2}
    \caption{Estensioni UC\_32.2: UC\_32.2.1, UC\_32.2.2}
\end{figure}

\subsubsubsubsection{UC\_32.2.1: Username non valido}
\label{uc_32.2.1}

\begin{itemize}
    \item \textbf{Attore$^G$ Principale}: Admin
    \item \textbf{Scenario Principale}:
    \begin{enumerate}
        \item L'Admin visualizza un errore per username non valido
    \end{enumerate}
    \item \textbf{Precondizioni}:
    \begin{itemize}
        \item Il sistema è online
        \item L'Utente non è registrato nel sistema
        \item L'Admin ha inserito uno username non valido
    \end{itemize}
    \item \textbf{Postcondizioni}:
    \begin{itemize}
        \item L'Admin visualizza un messaggio di errore per username non valido
    \end{itemize}
\end{itemize}

\subsubsubsubsection{UC\_32.2.2: Username già esistente}
\label{uc_32.2.2}

\begin{itemize}
    \item \textbf{Attore$^G$ Principale}: Admin
    \item \textbf{Scenario Principale}:
    \begin{enumerate}
        \item L'Admin visualizza un errore per username già esistente
    \end{enumerate}
    \item \textbf{Precondizioni}:
    \begin{itemize}
        \item Il sistema è online
        \item L'Utente non è registrato nel sistema
        \item L'Admin ha inserito uno username già esistente
    \end{itemize}
    \item \textbf{Postcondizioni}:
    \begin{itemize}
        \item L'Admin visualizza un messaggio di errore per username già esistente
    \end{itemize}
\end{itemize}

\subsubsubsection{UC\_32.3: Inserimento email}
\label{uc_32.3}

\begin{itemize}
    \item \textbf{Attore$^G$ Principale}: Admin
    \item \textbf{Scenario Principale}:
    \begin{enumerate}
        \item L'Admin inserisce l’email che l'Utente registrato utilizzerà
    \end{enumerate}
    \item \textbf{Precondizioni}:
    \begin{itemize}
        \item Il sistema è online
        \item L'Utente non è registrato nel sistema
		\item L'Admin sta creando un nuovo Utente $\rightarrow$ Vedi \hyperref[uc_32]{UC\_32}
    \end{itemize}
    \item \textbf{Postcondizioni}:
    \begin{itemize}
        \item Email inserita correttamente
    \end{itemize}
    \item \textbf{Scenari Alternativi}:
    \begin{itemize}
        \item L'Admin ha inserito una email non valida $\rightarrow$ Vedi \hyperref[uc_32.3.1]{UC\_32.3.1}
        \item L'Admin ha inserito una email già presente nel sistema $\rightarrow$ Vedi \hyperref[uc_32.3.2]{UC\_32.3.2}
    \end{itemize}
    \item \textbf{Estensioni}:
    \begin{itemize}
        \item \hyperref[uc_32.3.1]{UC\_32.3.1}
        \item \hyperref[uc_32.3.2]{UC\_32.3.2}
    \end{itemize}
    \item \textbf{Trigger}: L'Admin vuole inserire l'email del nuovo Utente
\end{itemize}

UC\_32.3 presenta due ulteriori estensioni, come mostrato in figura:
\begin{figure}[H]
    \centering 
    \includegraphics[width=0.8\textwidth]{AnalisiRequisiti/UC_32.3}
    \caption{Estensioni UC\_32.3: UC\_32.3.1, UC\_32.3.2}
\end{figure}

\subsubsubsubsection{UC\_32.3.1: Email non valida}
\label{uc_32.3.1}

\begin{itemize} 
    \item \textbf{Attore$^G$ Principale}: Admin
    \item \item \textbf{Scenario Principale}:
    \begin{enumerate}
        \item L'Admin visualizza un errore per email non valida
    \end{enumerate}
    \item \textbf{Precondizioni}:
    \begin{itemize}
        \item Il sistema è online
        \item L'Utente non è registrato nel sistema
        \item L'Admin ha inserito una email in un formato non corretto
    \end{itemize}
    \item \textbf{Postcondizioni}:
    \begin{itemize}
        \item Messaggio di errore visibile all'Admin 
    \end{itemize}
\end{itemize}


\subsubsubsubsection{UC\_32.3.2: Email già presente}
\label{uc_32.3.2}

\begin{itemize}
    \item \textbf{Attore$^G$ Principale}: Admin
    \item \item \textbf{Scenario Principale}:
    \begin{enumerate}
        \item L'Admin visualizza un errore per email già presente nel sistema
    \end{enumerate}
    \item \textbf{Precondizioni}:
    \begin{itemize}
        \item Il sistema è online
        \item L'Utente non è registrato nel sistema
        \item L'Admin ha inserito una email già presente nel sistema
    \end{itemize}
    \item \textbf{Postcondizioni}:
    \begin{itemize}
        \item L'Admin visualizza un messaggio di errore per email già presente nel sistema
    \end{itemize}
\end{itemize}

\subsubsection{UC\_33: Errore creazione utenti}
\label{uc_33}

\begin{itemize}
	\item \textbf{Attore principale}: Admin
	\item \textbf{Precondizioni}:
	\begin{itemize}
		\item L'Admin ha inviato una richiesta di creazione nuovo utente $\rightarrow$ Vedi \hyperref[uc_32]{UC\_32}
		\item La creazione utente fallisce (es per dati duplicati o non validi)
	\end{itemize}
	\item \textbf{Postcondizioni}:
	\begin{itemize}
		\item Il sistema visualizza una notifica di errore che spiega il motivo del fallimento
		\item Il nuovo utente non viene creato nel sistema
	\end{itemize}
\end{itemize}

\subsubsection{UC\_34: Duplicazione ordine tramite chatbot}
\label{uc_34}

\begin{figure}[H]
	\centering
	\includegraphics[width=0.8\textwidth]{AnalisiRequisiti/UC_34}
	\caption{UC\_34: Duplicazione ordine tramite chatbot}
\end{figure}

\begin{itemize}
	\item \textbf{Attore principale}: Utente
	\item \textbf{Scenario principale}:
	\begin{enumerate}
		\item L'Utente richiede la duplicazione di un ordine tramite chatbot
		\item L'Utente visualizza la conferma della duplicazione dell'ordine
	\end{enumerate}
	\item \textbf{Precondizioni}:
	\begin{itemize}
		\item Esiste l'ordine del quale L'Utente ha richiesto la duplicazione
	\end{itemize}
	\item \textbf{Postcondizioni}:
	\begin{itemize}
		\item È stato duplicato l'ordine richiesto
	\end{itemize}
	\item \textbf{Scenari alternativi}:
	\begin{itemize}
		\item Il sistema non riesce a duplicare l'ordine e mostra un messaggio di errore $\rightarrow$ Vedi \hyperref[uc_35]{UC\_35}
	\end{itemize}
	\item \textbf{Estensioni}:
	\begin{itemize}
		\item \hyperref[uc_35]{UC\_35}
	\end{itemize}
	\item \textbf{User Story}: L'Utente vuole duplicare un ordine tramite chatbot.
\end{itemize}


\subsubsection{UC\_35: Errore duplicazione ordine tramite chatbot}
\label{uc_35}

\begin{itemize}
	\item \textbf{Attore principale}: Cliente
	\item \textbf{Scenario principale}:
	\begin{enumerate}
		\item Il Cliente richiede la duplicazione di un ordine tramite chatbot
		\item Il sistema rileva che l'ordine richiesto non esiste
		\item Il sistema mostra un messaggio di errore al Cliente
	\end{enumerate}
	\item \textbf{Precondizioni}:
	\begin{itemize}
		\item Il Cliente ha avviato la duplicazione ordine $\rightarrow$ Vedi \hyperref[uc_34]{UC\_34}
		\item L'ordine richiesto non esiste
	\end{itemize}
	\item \textbf{Postcondizioni}:
	\begin{itemize}
		\item La duplicazione dell'ordine non viene eseguita
	\end{itemize}
	\item \textbf{Estensioni}:
	\begin{itemize}
		\item Estensione di \hyperref[uc_34]{UC\_34}
	\end{itemize}
	\item \textbf{User Story}: Il Cliente vuole ricevere un messaggio di errore se l'ordine da duplicare non esiste.
\end{itemize}


\subsubsection{UC\_36: Visualizza storico ordini}
\label{uc_36}

\begin{figure}[H]
	\centering
	\includegraphics[width=0.8\textwidth]{AnalisiRequisiti/UC_36}
	\caption{UC\_36: Visualizza storico ordini }
\end{figure}

\begin{itemize}
	\item \textbf{Attore principale}: Utente
	\item \textbf{Scenario principale}:
	\begin{enumerate}
		\item L'Utente clicca il pulsante per visualizzare lo storico ordini
	\end{enumerate}
	\item \textbf{Precondizioni}:
	\begin{itemize}
		\item Il sistema è online
		\item L'Utente si trova nella home
	\end{itemize}
	\item \textbf{Postcondizioni}:
	\begin{itemize}
		\item L'Utente si trova nella pagina web dedicata alla visualizzazione dello storico ordini
	\end{itemize}
	\item \textbf{Scenari alternativi}:
	\begin{itemize}
		\item L'Utente decide di impostare almeno un filtro per la visualizzazione di ordini specifici $\rightarrow$ Vedi \hyperref[uc_36.1]{UC\_36.1}
		\item L'Utente decide di visualizzare il dettaglio di un ordine partendo dalla pagina di visualizzazione storico ordini $\rightarrow$ Vedi \hyperref[uc_37]{UC\_37}
		\item L'Utente decide di duplicare un ordine partendo dalla pagina di visualizzazione storico ordini $\rightarrow$ Vedi \hyperref[uc_38]{UC\_38}
	\end{itemize}
	\item \textbf{Estensioni}:
	\begin{itemize}
		\item \hyperref[uc_36.1]{UC\_36.1}
		\item \hyperref[uc_37]{UC\_37}
		\item \hyperref[uc_38]{UC\_38}
	\end{itemize}
	\item \textbf{User Story}: L'Utente vuole visualizzare tutto lo storico dei suoi ordini effettuati fino a quel momento
\end{itemize}

Oltretutto UC\_36 ha un'ulteriore estensione descritta con il seguente diagramma UML, non descritto nel precedente per semplicità
\begin{figure}[H]
	\centering
	\includegraphics[width=0.8\textwidth]{AnalisiRequisiti/UC_36.1}
	\caption{UC\_36.1: Imposta filtri }
\end{figure}

\subsubsubsection{UC\_36.1: Imposta filtri}
\label{uc_36.1}
\begin{itemize}
	\item \textbf{Attore principale}: Utente
	\item \textbf{Scenario principale}:
	\begin{enumerate}
		\item L'Utente imposta almeno un filtro fra quelli disponibili
		\item L'Utente visualizza solo gli ordini che rispettano i filtri impostati
	\end{enumerate}
	\item \textbf{Precondizioni}:
	\begin{itemize}
		\item Il sistema è online
		\item L'Utente sta visualizzando la pagina dello storico ordini
	\end{itemize}
	\item \textbf{Postcondizioni}:
	\begin{itemize}
		\item L'Utente visualizza lo storico ordini filtrato
	\end{itemize}
	\item \textbf{Generalizzazioni}:
	\begin{itemize}
		\item \hyperref[uc_36.1.1]{UC\_36.1.1}
		\item \hyperref[uc_36.1.2]{UC\_36.1.2}
		\item \hyperref[uc_36.1.3]{UC\_36.1.3}
	\end{itemize}
	\item \textbf{User Story}: L'Utente vuole visualizzare lo storico ordini filtrando con delle preferenze, possibili scelte di filtro sono per Cliente (accessibile solo dall'Admin), per data o per prodotti
\end{itemize}

\subsubsubsubsection{UC\_36.1.1: Seleziona filtro Cliente}
\label{uc_36.1.1}
\begin{itemize}
	\item \textbf{Attore principale}: Admin
	\item \textbf{Scenario principale}:
	\begin{enumerate}
		\item L'Admin si trova nella pagina dedicata per la visualizzazione dello storico ordini
		\item L'Admin imposta il filtro Cliente
	\end{enumerate}
	\item \textbf{Precondizioni}:
	\begin{itemize}
		\item Il sistema è online
		\item L'Admin sta visualizzando lo storico ordini completo
	\end{itemize}
	\item \textbf{Postcondizioni}:
	\begin{itemize}
		\item L'Admin visualizza lo storico ordini di un Cliente specifico
	\end{itemize}
	\item \textbf{User Story}: L'Admin vuole visualizzare lo storico ordini relativo ad un Cliente specifico
\end{itemize}

\subsubsubsubsection{UC\_36.1.2: Seleziona filtro Data}
\label{uc_36.1.2}
\begin{itemize}
	\item \textbf{Attore principale}: Utente
	\item \textbf{Scenario principale}:
	\begin{enumerate}
		\item L'Utente si trova nella pagina dedicata per la visualizzazione dello storico ordini
		\item L'Utente imposta il filtro Data
	\end{enumerate}
	\item \textbf{Precondizioni}:
	\begin{itemize}
		\item Il sistema è online
		\item L'Utente sta visualizzando lo storico ordini completo
	\end{itemize}
	\item \textbf{Postcondizioni}:
	\begin{itemize}
		\item L'Utente visualizza solo gli ordini presenti nello storico compresi nel range temporale scelto
	\end{itemize}
	\item \textbf{User Story}: L'Utente vuole visualizzare solo gli ordini compresi in un range temporale scelto
\end{itemize}

\subsubsubsubsection{UC\_36.1.3: Seleziona filtro Prodotto}
\label{uc_36.1.3}
\begin{itemize}
	\item \textbf{Attore principale}: Utente
	\item \textbf{Scenario principale}:
	\begin{enumerate}
		\item L'Utente si trova nella pagina dedicata per la visualizzazione dello storico ordini
		\item L'Utente imposta il filtro Prodotto
	\end{enumerate}
	\item \textbf{Precondizioni}:
	\begin{itemize}
		\item Il sistema è online
		\item L'Utente sta visualizzando lo storico ordini completo
	\end{itemize}
	\item \textbf{Postcondizioni}:
	\begin{itemize}
		\item L'Utente visualizza solo gli ordini presenti nello storico che comprendono il prodotto o i prodotti scelti
	\end{itemize}
	\item \textbf{User Story}: L'Utente vuole visualizzare solo gli ordini in cui sono presenti i prodotti scelti
\end{itemize}


\subsubsection{UC\_37: Visualizza dettaglio ordine}
\label{uc_37}

L'UC\_37 ha 3 ulteriori inclusioni indicate in figura.

\begin{figure}[H]
	\centering
	\includegraphics[width=0.8\textwidth]{AnalisiRequisiti/UC_37}
	\caption{UC\_37: Visualizza dettaglio ordine }
\end{figure}

\begin{itemize}
	\item \textbf{Attore principale}: Utente
	\item \textbf{Scenario principale}:
	\begin{itemize}
		\item l'Utente clicca su un ordine specifico
		\item l'Utente visualizza i dettagli di un ordine specifico
		\item l'Utente visualizza il Cliente dell'ordine specifico. $\rightarrow$ Vedi [UC\_37.1, §\ref{uc_37.1}]
		\item L'Utente visualizza il Cliente dell'ordine specifico. $\rightarrow$ Vedi [UC\_37.2, §\ref{uc_37.2}]
		\item l'Utente visualizza il Cliente dell'ordine specifico. $\rightarrow$ Vedi [UC\_37.3, §\ref{uc_37.3}]
	\end{itemize}
	\item \textbf{Precondizioni}:
	\begin{itemize}
		\item il Sistema è online
		\item l'Utente sta visualizzando la pagina dello storico ordini
	\end{itemize}
	\item \textbf{Postcondizioni}:
	\begin{itemize}
		\item il Sistema mostra la pagina dedicata alla visualizzazione di un ordine
	\end{itemize}
	\item \textbf{Inclusioni}:
	\begin{itemize}
		\item $[$UC\_37.1, §\ref{uc_37.1}$]$
		\item $[$UC\_37.2, §\ref{uc_37.2}$]$
		\item $[$UC\_37.3, §\ref{uc_37.3}$]$
	\end{itemize}
	\item \textbf{Estensioni}:
	\begin{itemize}
		\item $[$UC\_38, §\ref{uc_38}$]$
	\end{itemize}
	\item \textbf{User Story}: l'Utente vuole visualizzare i dettagli di un ordine specifico
\end{itemize}

\subsubsubsection{UC\_37.1: Visualizza Cliente}
\label{uc_37.1}

\begin{itemize}
	\item \textbf{Attore principale}: Utente
	\item \textbf{Scenario principale}:
	\begin{itemize}
		\item l'Utente sta visualizzando il Cliente di quell'ordine
	\end{itemize}
	\item \textbf{Precondizioni}:
	\begin{itemize}
		\item il Sistema è online
		\item l'Utente sta visualizzando i dettagli di un ordine specifico
	\end{itemize}
	\item \textbf{Postcondizioni}:
	\begin{itemize}
		\item il Sistema sta mostrando il Cliente di quell'ordine
	\end{itemize}
	\item \textbf{User Story}: l'Utente vuole visualizzare il Cliente nella pagina dei dettagli di un ordine specifico
\end{itemize}

\subsubsubsection{UC\_37.2: Visualizza prodotti}
\label{uc_37.2}

\begin{itemize}
	\item \textbf{Attore principale}: Utente
	\item \textbf{Scenario principale}:
	\begin{itemize}
		\item l'Utente sta visualizzando i prodotti di quell'ordine
		\item l'Utente visualizza il nome dei prodotti. $\rightarrow$ Vedi [UC\_37.2.1, §\ref{uc_37.2.1}]
		\item l'Utente sta visualizzando la descrizione dei prodotti. $\rightarrow$ Vedi [UC\_37.2.2, §\ref{uc_37.2.1}]
	\end{itemize}
	\item \textbf{Precondizioni}:
	\begin{itemize}
		\item il Sistema è online
		\item l'Utente sta visualizzando i dettagli di un ordine specifico
	\end{itemize}
	\item \textbf{Postcondizioni}:
	\begin{itemize}
		\item il Sistema sta mostrando i prodotti di quell'ordine
	\end{itemize}
	\item \textbf{Inclusioni}:
	\begin{itemize}
		\item $[$UC\_37.2.1, §\ref{uc_37.2.1}$]$
		\item $[$UC\_37.2.2, §\ref{uc_37.2.2}$]$
	\end{itemize}
	\item \textbf{User Story}: l'Utente vuole visualizzare il Cliente nella pagina dei dettagli di un ordine specifico
\end{itemize}

\subsubsubsection{UC\_37.2.1: Visualizza nome prodotti}
\label{uc_37.2.1}

\begin{itemize}
	\item \textbf{Attore principale}: Utente
	\item \textbf{Scenario principale}:
	\begin{itemize}
		\item l'Utente sta visualizzando i prodotti di quell'ordine
		\item l'Utente visualizza il nome dei prodotti
	\end{itemize}
	\item \textbf{Precondizioni}:
	\begin{itemize}
		\item il Sistema è online
		\item l'Utente sta visualizzando i dettagli di un ordine specifico
	\end{itemize}
	\item \textbf{Postcondizioni}:
	\begin{itemize}
		\item il Sistema sta mostrando i prodotti di quell'ordine, comprensivo di nome e descrizione
	\end{itemize}
	\item \textbf{User Story}: l'Utente vuole visualizzare i prodotti nella pagina dei dettagli di un ordine specifico
\end{itemize}

\subsubsubsection{UC\_37.2.2: Visualizza descrizione prodotti}
\label{uc_37.2.2}

\begin{itemize}
	\item \textbf{Attore principale}: Utente
	\item \textbf{Scenario principale}:
	\begin{itemize}
		\item l'Utente sta visualizzando i prodotti di quell'ordine
		\item l'Utente visualizza la descrizione di quei prodotti
	\end{itemize}
	\item \textbf{Precondizioni}:
	\begin{itemize}
		\item il Sistema è online
		\item l'Utente sta visualizzando i dettagli di un ordine specifico
	\end{itemize}
	\item \textbf{Postcondizioni}:
	\begin{itemize}
		\item il Sistema sta mostrando i prodotti di quell'ordine, comprensivo di nome e descrizione
	\end{itemize}
	\item \textbf{User Story}: l'Utente vuole visualizzare i prodotti nella pagina dei dettagli di un ordine specifico
\end{itemize}

\subsubsubsection{UC\_37.3: Visualizza data ordine}
\label{uc_37.3}

\begin{itemize}
	\item \textbf{Attore principale}: Utente
	\item \textbf{Scenario principale}:
	\begin{itemize}
		\item l'Utente sta visualizzando la data di quell'ordine
	\end{itemize}
	\item \textbf{Precondizioni}:
	\begin{itemize}
		\item il Sistema è online
		\item l'Utente sta visualizzando i dettagli di un ordine specifico
	\end{itemize}
	\item \textbf{Postcondizioni}:
	\begin{itemize}
		\item il Sistema sta mostrando la data di quell'ordine
	\end{itemize}
	\item \textbf{User Story}: l'Utente vuole visualizzare la data nella pagina dei dettagli di un ordine specifico
\end{itemize}

\subsubsection{UC\_38: Duplicazione ordine tramite pagina storico}
\label{uc_38}

\begin{figure}[H]
	\centering
	\includegraphics[width=0.8\textwidth]{AnalisiRequisiti/UC_38}
	\caption{UC\_38: Duplica ordine}
\end{figure}

\begin{itemize}
	\item \textbf{Attore principale}: Cliente
	\item \textbf{Scenario principale}:
	\begin{enumerate}
		\item Il Cliente clicca il pulsante per duplicare un ordine dalla pagina del proprio storico o dettaglio ordine
		\item Il Cliente conferma la duplicazione ordine $\rightarrow$ Vedi \hyperref[uc_38.1]{UC\_38.1}
		\item Il sistema crea una copia dell'ordine selezionato e la aggiunge al carrello del Cliente
	\end{enumerate}
	\item \textbf{Precondizioni}:
	\begin{itemize}
		\item Il sistema è online
		\item Il Cliente è autenticato
		\item Il Cliente sta visualizzando la pagina del proprio storico ordini o il dettaglio di un proprio ordine
		\item Esiste almeno un ordine precedente del Cliente
	\end{itemize}
	\item \textbf{Postcondizioni}:
	\begin{itemize}
		\item Una copia dell'ordine selezionato è stata aggiunta al carrello del Cliente
	\end{itemize}
	\item \textbf{Scenari alternativi}:
	\begin{itemize}
		\item Il Cliente riceve un messaggio di errore di duplicazione ordine $\rightarrow$ Vedi \hyperref[uc_39]{UC\_39}
	\end{itemize}
	\item \textbf{Inclusioni}:
	\begin{itemize}
		\item \hyperref[uc_38.1]{UC\_38.1}
	\end{itemize}
	\item \textbf{Estensioni}:
	\begin{itemize}
		\item \hyperref[uc_39]{UC\_39}
	\end{itemize}
	\item \textbf{Trigger}: Il Cliente vuole duplicare un proprio ordine precedente per ripeterlo
\end{itemize}

\subsubsubsection{UC\_38.1: Conferma duplicazione ordine}
\label{uc_38.1}

\begin{itemize}
	\item \textbf{Attore principale}: Cliente
	\item \textbf{Scenario principale}:
	\begin{enumerate}
		\item Il Cliente conferma la volontà di procedere con la duplicazione dell'ordine
	\end{enumerate}
	\item \textbf{Precondizioni}:
	\begin{itemize}
		\item Il Cliente ha avviato la procedura di duplicazione (clic sul pulsante in UC\_38)
	\end{itemize}
	\item \textbf{Postcondizioni}:
	\begin{itemize}
		\item La conferma di duplicazione è stata inviata al sistema
	\end{itemize}
\end{itemize}

\subsubsection{UC\_39: Errore duplicazione ordine}
\label{uc_39}

\begin{itemize}
    \item \textbf{Attore$^G$ Principale}: Cliente
	 \item \textbf{Scenario Principale}:
	 \begin{enumerate}
		\item Il Cliente tenta di duplicare un ordine che ha prodotti non più disponibili nel sistema
	 \end{enumerate}
	\item \textbf{Precondizioni}:
	\begin{itemize}
		\item Il sistema è online
		\item Il Cliente è autenticato
	\end{itemize}
	\item \textbf{Postcondizioni}:
	\begin{itemize}
		\item Il sistema visualizza un messaggio di errore che informa dell'indisponibilità dei prodotti
		\item La duplicazione dell'ordine non viene eseguita
		\item Il carrello del Cliente non viene modificato
	\end{itemize}
\end{itemize}

\subsubsection{UC\_40: Visualizza pagina statistiche}
\label{uc_40}

\begin{figure}[H]
	\centering
	\includegraphics[width=0.8\textwidth]{AnalisiRequisiti/UC_40}
	\caption{UC\_40: Visualizza pagina statistiche}
\end{figure}

\begin{itemize}
	\item \textbf{Attore principale}: Admin
	\item \textbf{Scenario principale}:
	\begin{itemize}
		\item l'Admin visualizza la sezione statistiche dedicata compresa nella pagina delle performance
		\item l'Admin visualizza il numero di utenti. $\rightarrow$ Vedi [UC\_40.1, §\ref{uc_40.1}]
		\item l'Admin visualizza il numero di acquisti completati. $\rightarrow$ Vedi [UC\_40.2, §\ref{uc_40.2}]
		\item l'Admin visualizza il modello utilizzato. $\rightarrow$ Vedi [UC\_40.3, §\ref{uc_40.3}]
	\end{itemize}
	\item \textbf{Precondizioni}:
	\begin{itemize}
		\item il Sistema è online
		\item l'Admin sta visualizzando la pagina delle performance. $\rightarrow$ Vedi [UC\_25, §\ref{uc_25}]
	\end{itemize}
	\item \textbf{Postcondizioni}:
	\begin{itemize}
		\item l'Admin sta visualizzando la parte della pagina delle performance dedicata alle statistiche
	\end{itemize}
	\item \textbf{Inclusioni}:
	\begin{itemize}
		\item $[$UC\_40.1, §\ref{uc_40.1}$]$
		\item $[$UC\_40.2, §\ref{uc_40.2}$]$
		\item $[$UC\_40.3, §\ref{uc_40.3}$]$
	\end{itemize}
	\item \textbf{User Story}: l'Admin vuole visualizzare le statistiche della piattaforma
\end{itemize}

\subsubsubsection{UC\_40.1: Visualizza numero di utenti}
\label{uc_40.1}

\begin{itemize}
	\item \textbf{Attore principale}: Admin
	\item \textbf{Scenario principale}:
	\begin{itemize}
		\item l'Admin visualizza il numero di utenti.
	\end{itemize}
	\item \textbf{Precondizioni}:
	\begin{itemize}
		\item il Sistema è online
		\item l'Admin sta visualizzando la sezione delle statistiche
	\end{itemize}
	\item \textbf{Postcondizioni}:
	\begin{itemize}
		\item l'Admin sta visualizzando il numero di utenti nella pagina delle statistiche
	\end{itemize}
	\item \textbf{User Story}: l'Admin vuole visualizzare il numero di utenti presenti in quel momento nella piattaforma
\end{itemize}

\subsubsubsection{UC\_40.2: Visualizza numero di acquisti completati}
\label{uc_40.2}

\begin{itemize}
	\item \textbf{Attore principale}: Admin
	\item \textbf{Scenario principale}:
	\begin{itemize}
		\item l'Admin visualizza il numero di acquisti completati.
	\end{itemize}
	\item \textbf{Precondizioni}:
	\begin{itemize}
		\item il Sistema è online
		\item l'Admin sta visualizzando la sezione delle statistiche
	\end{itemize}
	\item \textbf{Postcondizioni}:
	\begin{itemize}
		\item l'Admin sta visualizzando il numero di acquisti completati tramite la piattaforma nella pagina delle statistiche
	\end{itemize}
	\item \textbf{User Story}: l'Admin vuole visualizzare il numero di acquisti completati fino a quel momento tramite la piattaforma
\end{itemize}

\subsubsubsection{UC\_40.3: Visualizza modello utilizzato}
\label{uc_40.3}

\begin{itemize}
	\item \textbf{Attore principale}: Admin
	\item \textbf{Scenario principale}:
	\begin{itemize}
		\item l'Admin visualizza il modello utilizzato in quel momento.
	\end{itemize}
	\item \textbf{Precondizioni}:
	\begin{itemize}
		\item il Sistema è online
		\item l'Admin sta visualizzando la sezione delle statistiche
		\item modello AI attivo
	\end{itemize}
	\item \textbf{Postcondizioni}:
	\begin{itemize}
		\item l'Admin sta visualizzando il modello utilizzato in quel momento
	\end{itemize}
	\item \textbf{User Story}: l'Admin vuole visualizzare il modello utilizzato in quel momento tramite la piattaforma
\end{itemize}



\subsubsection{UC\_41 Visualizzare manuale utente}
\label{uc_41}
\begin{figure}[H]
    \centering
    \includegraphics[width=0.8\textwidth]{AnalisiRequisiti/UC_41.png}
    \caption{UC\_41: Visualizzare manuale utente}
\end{figure}
\begin{itemize}
    \item \textbf{Attore principale}: Utente
    \item \textbf{Scenario principale}:
        \begin{enumerate}
            \item L'Utente apre la pagina dedicata al manuale utente 
            \item L'Utente visualizza la pagina \vr{Manuale Utente}
        \end{enumerate}
    \item \textbf{Precondizioni}: 
        \begin{itemize}
            \item Pagina \vr{Manuale Utente} esistente
        \end{itemize}
    \item \textbf{Postcondizioni}:
        \begin{itemize}
            \item Pagina \vr{Manuale Utente} visualizzata
        \end{itemize}
    \item \textbf{User Story}: L'Utente vuole visualizzare il manuale utente 
\end{itemize}

\subsubsection{UC\_42 Visualizzare FAQ}
\label{uc_42}
\begin{figure}[H]
    \centering
    \includegraphics[width=0.8\textwidth]{AnalisiRequisiti/UC_42.png}
    \caption{UC\_42: Visualizzare FAQ}
\end{figure}
\begin{itemize}
    \item \textbf{Attore principale}: Utente
    \item \textbf{Scenario principale}: 
        \begin{enumerate}
             \item L'Utente apre la pagina dedicata alle FAQ (Frequently Asked Questions) 
             \item L'Utente visualizza la pagina dedicata alle FAQ
        \end{enumerate}
    \item \textbf{Precondizioni}: 
        \begin{itemize}
            \item Pagina \vr{FAQ} esistente
        \end{itemize}
    \item \textbf{Postcondizioni}:         
         \begin{itemize}
            \item Pagina \vr{FAQ} visualizzata
        \end{itemize}
    \item \textbf{User Story}: L'Utente vuole visualizzare le FAQ 
\end{itemize}

\subsubsection{UC\_43 Visualizzare FAQ}
\label{uc_43}
\begin{figure}[H]
    \centering
    \includegraphics[width=0.8\textwidth]{AnalisiRequisiti/UC_43.pdf}
    \caption{UC\_43: Visualizzare FAQ}
\end{figure}
\begin{itemize}
    \item Attore principale: Utente
    \item Scenario principale: 
        \begin{enumerate}
             \item Viene aperta la pagina dedicata alle FAQ (Frequently Asked Questions) 
        \end{enumerate}
    \item Pre Condizioni: 
        \begin{itemize}
            \item Pagina "FAQ" esistente
        \end{itemize}
    \item Post Condizioni:         
         \begin{itemize}
            \item Pagina "FAQ" visualizzata
        \end{itemize}
    \item User Story: Un utente desidera visualizzare le FAQ 
\end{itemize}

\subsubsection{UC\_44: Invio di un comando nella chat}
\label{uc_44}

\begin{figure}[H]
	\centering
	\includegraphics[width=0.8\textwidth]{AnalisiRequisiti/UC_44}
	\caption{UC\_44: Invio di un comando nella chat}
\end{figure}

\begin{itemize}
    \item \textbf{Attore$^G$ Principale}: Cliente
	\item \textbf{Scenario principale}: 
	\begin{enumerate}
		\item Il Cliente digita un comando nella chat
		\item Il Cliente visualizza l'effetto del comando
	\end{enumerate}
	\item \textbf{Precondizioni}: 
	\begin{itemize}
		\item Il sistema è online
		\item Il Cliente è autenticato
		\item Il Cliente può scrivere nella chat
	\end{itemize}
	\item \textbf{Postcondizioni}:
	\begin{itemize}
		\item Il Cliente visualizza l'elenco dei comandi con spiegazione nella chat
	\end{itemize}
	\item \textbf{Generalizzazioni}:
	\begin{itemize}
		\item \hyperref[uc_44.1]{UC\_44.1}
		\item \hyperref[uc_44.2]{UC\_44.2}
		\item \hyperref[uc_44.3]{UC\_44.3}
		\item \hyperref[uc_44.4]{UC\_44.4}
		\item \hyperref[uc_44.5]{UC\_44.5}
		\item \hyperref[uc_44.6]{UC\_44.6}
	\end{itemize}
	\item \textbf{Trigger}: Il Cliente vuole inviare un comando nella chat
\end{itemize}

\subsubsubsection{UC\_44.1: Invio comando \vr{/duplica}}
\label{uc_44.1}
\begin{itemize}
    \item \textbf{Attore$^G$ Principale}: Cliente
	\item \textbf{Scenario principale}:
	\begin{enumerate}
		\item Il Cliente digita \texttt{/duplica} nella chat
		\item Il Cliente duplica l'ultimo ordine effettuato
	\end{enumerate}
	\item \textbf{Precondizioni}:
	\begin{itemize}
		\item Il sistema è online
		\item Il Cliente è autenticato
		\item Il Cliente può scrivere nella chat
	\end{itemize}
	\item \textbf{Postcondizioni}:
	\begin{itemize}
		\item Il sistema duplica l'ultimo ordine
	\end{itemize}
	\item \textbf{Trigger}: Il Cliente vuole duplicare l'ultimo ordine effettuato
\end{itemize}

\subsubsubsection{UC\_44.2: Invio comando \vr{/carrello}}
\label{uc_44.2}
\begin{itemize}
    \item \textbf{Attore$^G$ Principale}: Cliente
	\item \textbf{Scenario principale}:
	\begin{enumerate}
		\item Il Cliente digita \texttt{/carrello} nella chat
		\item Il Cliente visualizza gli articoli presenti nel carrello
	\end{enumerate}
	\item \textbf{Precondizioni}:
	\begin{itemize}
		\item Il sistema è online 
		\item Il Cliente è autenticato
		\item Il Cliente può scrivere nella chat
	\end{itemize}
	\item \textbf{Postcondizioni}:
	\begin{itemize}
		\item Il Cliente visualizza il riepilogo
	\end{itemize}
	\item \textbf{Trigger}: Il Cliente vuole vedere gli articoli presenti nel carrello
\end{itemize}

\subsubsubsection{UC\_44.3: Invio comando \vr{/duplica\{xx\}}}
\label{uc_44.3}
\begin{itemize}
    \item \textbf{Attore$^G$ Principale}: Cliente
	\item \textbf{Scenario principale}:
	\begin{enumerate}
		\item Il Cliente digita \texttt{/duplica\{xx\}} nella chat
		\item Il Cliente duplica l'ordine con codice xx
	\end{enumerate}
	\item \textbf{Precondizioni}:
	\begin{itemize}
		\item Il sistema è online 
		\item Il Cliente è autenticato
		\item Il Cliente può scrivere nella chat
	\end{itemize}
	\item \textbf{Postcondizioni}:
	\begin{itemize}
		\item Il sistema duplica l'ordine con codice xx
	\end{itemize}
	\item \textbf{Trigger}: Il Cliente vuole duplicare l'ordine con il codice xx
\end{itemize}

\subsubsubsection{UC\_44.4: Invio comando \vr{/invia}}
\label{uc_44.4}
\begin{itemize}
    \item \textbf{Attore$^G$ Principale}: Cliente
	\item \textbf{Scenario principale}:
	\begin{enumerate}
		\item Il Cliente digita \texttt{/invia} nella chat
		\item Il Cliente visualizza il riepilogo del carrello
		\item Il Cliente conferma l'invio
		\item Il Cliente invia l'ordine con gli articoli presenti nel carrello
	\end{enumerate}
	\item \textbf{Precondizioni}:
	\begin{itemize}
		\item Il sistema è online 
		\item Il Cliente è autenticato
		\item Il Cliente può scrivere nella chat
	\end{itemize}
	\item \textbf{Postcondizioni}:
	\begin{itemize}
		\item Il sistema riceve l'ordine
	\end{itemize}
	\item \textbf{Trigger}: Il Cliente vuole inviare l'ordine
\end{itemize}

\subsubsubsection{UC\_44.5: Invio comando \vr{/annulla}}
\label{uc_44.5}
\begin{itemize}
    \item \textbf{Attore$^G$ Principale}: Cliente
	\item \textbf{Scenario principale}:
	\begin{enumerate}
		\item Il Cliente digita \texttt{/annulla} nella chat
		\item Il Cliente annulla l'operazione che stava effettuando
	\end{enumerate}
	\item \textbf{Precondizioni}:
	\begin{itemize}
		\item Il sistema è online 
		\item Il Cliente è autenticato
		\item Il Cliente può scrivere nella chat
	\end{itemize}
	\item \textbf{Postcondizioni}:
	\begin{itemize}
		\item Il sistema annulla la disambiguazione
		\item Il Cliente può inserire un nuovo input
	\end{itemize}
	\item \textbf{Trigger}: Il Cliente vuole annullare la disambiguazione
\end{itemize}

\subsubsubsection{UC\_44.6: Invio comando \vr{/comandi}}
\label{uc_44.6}
\begin{itemize}
    \item \textbf{Attore$^G$ Principale}: Cliente
	\item \textbf{Scenario principale}:
	\begin{enumerate}
		\item Il Cliente digita \texttt{/comandi} nella chat
		\item Il Cliente visualizza i comandi con la relativa spiegazione nella chat
	\end{enumerate}
	\item \textbf{Precondizioni}:
	\begin{itemize}
		\item Il sistema è online 
		\item Il Cliente è autenticato
		\item Il Cliente può scrivere nella chat
	\end{itemize}
	\item \textbf{Postcondizioni}:
	\begin{itemize}
		\item Il Cliente visualizza i comandi con la relativa spiegazione
	\end{itemize}
	\item \textbf{Trigger}: Il Cliente vuole visualizzare i comandi e la loro spiegazione
\end{itemize}

\subsubsection{UC\_45 Visualizzare manuale utente}
\label{UC_45}
\begin{figure}[H]
    \centering
    \includegraphics[width=0.8\textwidth]{AnalisiRequisiti/UC_45.png}
    \caption{UC\_45: Visualizzare manuale utente}
\end{figure}
\begin{itemize}
    \item Attore principale: Utente
    \item Scenario principale: Viene aperta la pagina dedicata al manuale utente. 
    \item Pre Condizioni: Pagina "Manuale Utente" esistente
    \item Post Condizioni: Pagina "Manuale Utente" visualizzata
    \item User Story: Un utente desidera visualizzare il manuale utente. 
\end{itemize}

\subsubsection{UC\_46 Consultare i comandi disponibili}
\label{uc_46}
\begin{figure}[H]
    \centering
    \includegraphics[width=0.8\textwidth]{AnalisiRequisiti/UC_46.pdf}
    \caption{UC\_46: Consultare i comandi disponibili}
\end{figure}
\begin{itemize}
    \item Attore principale: Utente
    \item Scenario principale: 
        \begin{enumerate}
            \item Viene visualizzato un elenco dei comandi attualmente supportati, che l'Utente può svolgere
        \end{enumerate}
    \item Pre Condizioni: 
        \begin{itemize}
            \item Utente ha accesso alla chat
        \end{itemize}
    \item Post Condizioni: 
        \begin{itemize}
            \item Menù dei comandi visualizzato con un menù a comparsa sopra la chat
        \end{itemize}
    \item User Story: Un Utente vuole visualizzare l'elenco dei comandi che attualmente può eseguire
\end{itemize}

\subsubsection{UC\_47 Visualizzare elenco dei comandi in un'altra pagina}
\label{UC_47}
\begin{figure}[H]
    \centering
    \includegraphics[width=0.8\textwidth]{AnalisiRequisiti/UC_47.png}
    \caption{UC\_47: Visualizzare elenco dei comandi in un'altra pagina}
\end{figure}
\begin{itemize}
    \item Attore principale: Utente.
    \item Scenario principale: Viene visualizzato l'elenco dei comandi in una pagina separata da quella corrente.
    \item Pre Condizioni: Pagina esistente.
    \item Post Condizioni: Pagina dei comandi visualizzata.
    \item User Story: Un Utente vuole visualizzare l'elenco dei comandi.
\end{itemize}

\subsubsection{UC\_48: Visualizza storico ordini personale}
\label{uc_48}

\begin{figure}[H]
	\centering
	\includegraphics[width=0.8\textwidth]{AnalisiRequisiti/UC_48}
	\caption{UC\_48: Visualizza storico ordini personale}
\end{figure}

\begin{itemize}
	\item \textbf{Attore principale}: Cliente
	\item \textbf{Scenario principale}:
	\begin{enumerate}
		\item Il Cliente clicca il pulsante per visualizzare lo storico ordini
		\item Il Cliente visualizza l'elenco di tutti i suoi ordini
		\item Il Cliente visualizza il codice ordine $\rightarrow$ Vedi \hyperref[uc_48.1]{UC\_48.1}
		\item Il Cliente visualizza la data dell'ordine $\rightarrow$ Vedi \hyperref[uc_48.2]{UC\_48.2}
	\end{enumerate}
	\item \textbf{Precondizioni}:
	\begin{itemize}
		\item Il sistema è online
		\item Il Cliente è autenticato
	\end{itemize}
	\item \textbf{Postcondizioni}:
	\begin{itemize}
		\item Il Cliente si trova nella pagina web dedicata alla visualizzazione del suo storico ordini
	\end{itemize}
	\item \textbf{Inclusioni}:
	\begin{itemize}
		\item \hyperref[uc_48.1]{UC\_48.1}
		\item \hyperref[uc_48.2]{UC\_48.2}
	\end{itemize}
	\item \textbf{User Story}: Il Cliente vuole visualizzare tutto lo storico dei propri ordini effettuati fino a quel momento
\end{itemize}

\subsubsubsection{UC\_48.1: Visualizza codice ordine}
\label{uc_48.1}
\begin{itemize}
	\item \textbf{Attore principale}: Cliente
	\item \textbf{Scenario principale}:
	\begin{enumerate}
		\item Il Cliente visualizza i codici dei propri ordini
	\end{enumerate}
	\item \textbf{Precondizioni}:
	\begin{itemize}
		\item Il sistema è online
		\item Il Cliente è autenticato
	\end{itemize}
	\item \textbf{Postcondizioni}
	\begin{itemize}
		\item Vengono visualizzati i codici dei propri ordini
	\end{itemize}
	\item \textbf{User Story}: Il Cliente vuole visualizzare i codici ordini dei propri ordini presenti
\end{itemize}

\subsubsubsubsection{UC\_48.2: Visualizza data ordine}
\label{uc_48.2}
\begin{itemize}
	\item \textbf{Attore principale}: Cliente
	\item \textbf{Scenario principale}:
	\begin{enumerate}
		\item Il Cliente visualizza le date dei propri ordini
	\end{enumerate}
	\item \textbf{Precondizioni}:
	\begin{itemize}
		\item Il sistema è online
		\item Il Cliente è autenticato
	\end{itemize}
	\item \textbf{Postcondizioni}
	\begin{itemize}
		\item Vengono visualizzate le date dei propri ordini
	\end{itemize}
	\item \textbf{User Story}: Il Cliente vuole visualizzare la data ordine nei propri ordini presenti
\end{itemize}

\subsubsection{UC\_49: Errore invio comandi nella chat}
\label{uc_49}

\hyperref[uc_44]{UC\_44} ha un'ulteriore estensione indicata nel seguente diagramma dei casi d'uso.

\begin{figure}[H]
	\centering
	\includegraphics[width=0.8\textwidth]{AnalisiRequisiti/UC_49}
	\caption{Estensione UC\_44: UC\_49}
\end{figure}

\begin{itemize}
	\item \textbf{Attore$^G$ Principale}: Cliente
	\item \textbf{Precondizioni}:
	\begin{itemize}
		\item Il sistema è online
		\item Il Cliente è autenticato
		\item Una sessione chat del Cliente è attiva
		\item Il Cliente ha inviato due comandi nello stesso input oppure il Cliente ha inviato del testo non consentito con un comando
	\end{itemize}
	\item \textbf{Postcondizioni}:
	\begin{itemize}
		\item Il Cliente visualizza un messaggio di errore
	\end{itemize}
\end{itemize}

\subsubsection{UC\_50: Visualizza le chat}
\label{uc_50}

\begin{figure}[H]
	\centering
	\includegraphics[width=0.8\textwidth]{AnalisiRequisiti/UC_50}
	\caption{UC\_50: Visualizza lista chat}
\end{figure}

\begin{itemize}
	\item \textbf{Attore$^G$ Principale}: Cliente
	\item \textbf{Scenario principale}:
	\begin{enumerate}
		\item Il Cliente visualizza la lista delle chat
	\end{enumerate}
	\item \textbf{Precondizioni}:
	\begin{itemize}
		\item Il sistema è online
		\item Il Cliente è autenticato
	\end{itemize}
	\item \textbf{Postcondizioni}:
	\begin{itemize}
		\item Il Cliente visualizza la lista delle chat
	\end{itemize}
	\item \textbf{Inclusioni}:
	\begin{itemize}
		\item \hyperref[uc_50.1]{UC\_50.1}
	\end{itemize}
	\item \textbf{Trigger}: Il Cliente vuole visualizzare l'elenco delle chat
\end{itemize}

\subsubsubsection{UC\_50.1: Visualizzazione singola chat}
\label{uc_50.1}

\begin{itemize}
	\item \textbf{Attore$^G$ principale}: Cliente
	\item \textbf{Scenario principale}:
	\begin{enumerate}
		\item Il Cliente visualizza una singola chat 
	\end{enumerate}
	\item \textbf{Precondizioni}:
	\begin{itemize}
		\item Il Cliente ha effettuato almeno una chat
	\end{itemize}
	\item \textbf{Postcondizioni}:
	\begin{itemize}
		\item Il Cliente visualizza una singola chat
	\end{itemize}
	\item \textbf{Inclusioni}:
	\begin{itemize}
		\item \hyperref[uc_50.1.1]{UC\_50.1.1}
	\end{itemize}
	\item \textbf{Trigger}: Il Cliente vuole visualizzare una singola chat 
\end{itemize}

\subsubsubsection{UC\_50.1.1: Visualizzazione lista messaggi}
\label{uc_50.1.1}

\begin{itemize}
	\item \textbf{Attore$^G$ principale}: Cliente
	\item \textbf{Scenario principale}:
	\begin{enumerate}
		\item Il Cliente visualizza la lista dei messaggi di una singola chat
	\end{enumerate}
	\item \textbf{Precondizioni}:
	\begin{itemize}
		\item Il Cliente ha effettuato almeno una chat e scritto almeno un messaggio in essa
	\end{itemize}
	\item \textbf{Postcondizioni}:
	\begin{itemize}
		\item Il Cliente visualizza la lista messaggi di una singola chat 
	\end{itemize}
	\item \textbf{Inclusioni}:
	\begin{itemize}
		\item \hyperref[uc_50.1.1.1]{UC\_50.1.1.1}
	\end{itemize}
	\item \textbf{Trigger}: Il Cliente vuole visualizzare la lista messaggi di una singola chat
\end{itemize}

\subsubsubsection{UC\_50.1.1.1: Visualizzazione messaggio singolo}
\label{uc_50.1.1.1}

\begin{itemize}
	\item \textbf{Attore$^G$ principale}: Cliente
	\item \textbf{Scenario principale}:
	\begin{enumerate}
		\item Il Cliente visualizza un messaggio singolo all'interno di una singola chat 
	\end{enumerate}
	\item \textbf{Precondizioni}:
	\begin{itemize}
		\item Il Cliente ha effettuato almeno una chat e scritto almeno un messaggio in essa
	\end{itemize}
	\item \textbf{Postcondizioni}:
	\begin{itemize}
		\item Il Cliente visualizza un messaggio singolo all'interno di una singola chat 
	\end{itemize}
	\item \textbf{Specializzazioni}:
	\begin{itemize}
		\item \hyperref[uc_50.1.1.1.1]{UC\_50.1.1.1.1}
		\item \hyperref[uc_50.1.1.1.2]{UC\_50.1.1.1.2}
	\end{itemize}
	\item \textbf{Trigger}: Il Cliente vuole visualizzare un messaggio singolo all'interno di una chat 
\end{itemize}

\subsubsubsection{UC\_50.1.1.1.1: Visualizza messaggio Utente}
\label{uc_50.1.1.1.1}

\begin{itemize}
	\item \textbf{Attore$^G$ principale}: Cliente
	\item \textbf{Scenario principale}:
	\begin{enumerate}
		\item Il Cliente visualizza un messaggio singolo dell'Utente all'interno di una singola chat 
	\end{enumerate}
	\item \textbf{Precondizioni}:
	\begin{itemize}
		\item Il Cliente ha effettuato almeno una chat e scritto almeno un messaggio in essa
	\end{itemize}
	\item \textbf{Postcondizioni}:
	\begin{itemize}
		\item Il Cliente visualizza un messaggio singolo dell'Utente all'interno di una singola chat 
	\end{itemize}
	\item \textbf{Trigger}: Il Cliente vuole visualizzare un messaggio singolo dell'Utente all'interno di una chat 
\end{itemize}

\subsubsubsection{UC\_50.1.1.1.2: Visualizza messaggio AI}
\label{uc_50.1.1.1.2}

\begin{itemize}
	\item \textbf{Attore$^G$ principale}: Cliente
	\item \textbf{Scenario principale}:
	\begin{enumerate}
		\item Il Cliente visualizza un messaggio singolo dell'AI all'interno di una singola chat 
	\end{enumerate}
	\item \textbf{Precondizioni}:
	\begin{itemize}
		\item Il Cliente ha effettuato almeno una chat
		\item Il Cliente ha ricevuto almeno una risposta dall'AI
	\end{itemize}
	\item \textbf{Postcondizioni}:
	\begin{itemize}
		\item Il Cliente visualizza un messaggio singolo dell'AI all'interno di una singola chat 
	\end{itemize}
	\item \textbf{Trigger}: Il Cliente vuole visualizzare un messaggio singolo dell'AI all'interno di una chat 
\end{itemize}

\subsubsection{UC\_51: Selezione di una chat}
\label{uc_51}

\begin{figure}[H]
	\centering
	\includegraphics[width=0.8\textwidth]{AnalisiRequisiti/UC_51}
	\caption{UC\_51: Selezione di una chat}
\end{figure}

\begin{itemize}
	\item \textbf{Attore$^G$ Principale}: Cliente
	\item \textbf{Scenario principale}:
	\begin{enumerate}
		\item Il Cliente visualizza le chat $\rightarrow$ Vedi \hyperref[uc_50]{UC\_50}
		\item Il Cliente seleziona la chat che vuole visualizzare
		\item Il sistema mostra la chat
	\end{enumerate}
	\item \textbf{Precondizioni}:
	\begin{itemize}
		\item Il sistema è online
		\item Il Cliente è autenticato
	\end{itemize}
	\item \textbf{Postcondizioni}:
	\begin{itemize}
		\item Il sistema mostra la chat selezionata
	\end{itemize}
	\item \textbf{Trigger}: Il Cliente vuole selezionare una chat che aveva avviato in precedenza
\end{itemize}

\subsubsection{UC\_52: Cancella chat}
\label{uc_52}

\begin{figure}[H]
	\centering
	\includegraphics[width=0.8\textwidth]{AnalisiRequisiti/UC_52}
	\caption{UC\_52: Cancella chat}
\end{figure}

\begin{itemize}
	\item \textbf{Attore$^G$ Principale}: Cliente
	\item \textbf{Scenario principale}:
	\begin{enumerate}
		\item Il Cliente visualizza il pulsante per cancellare la chat
		\item Il Cliente preme il pulsante per cancellare la chat
		\item Il Cliente conferma la cancellazione della chat
	\end{enumerate}
	\item \textbf{Scenari alternativi}:
	\begin{itemize}
		\item Il Cliente decide di non iniziare una nuova chat; la sessione corrente rimane attiva
	\end{itemize}
	\item \textbf{Precondizioni}:
	\begin{itemize}
		\item Il sistema è online
		\item Il Cliente è autenticato
		\item Una sessione chat del Cliente è attiva
	\end{itemize}
	\item \textbf{Postcondizioni}:
	\begin{itemize}
		\item La chat cancellata non è più visibile
		\item Il Cliente visualizza una chat vuota
	\end{itemize}
	\item \textbf{Trigger}: Il Cliente vuole cancellare la chat
\end{itemize}

\subsubsection{UC\_53: Invio ordine dal carrello}
\label{uc_53}

\begin{figure}[H]
	\centering
	\includegraphics[width=0.8\textwidth]{AnalisiRequisiti/UC_53}
	\caption{UC\_53: Invio ordine dal carrello}
\end{figure}

\begin{itemize}
    \item \textbf{Attore$^G$ Principale}: Cliente
    \item \textbf{Scenario principale}:
	\begin{enumerate}
		\item Il Cliente visualizza il pulsante per l'invio dell'ordine
		\item Il Cliente preme il pulsante per l'invio dell'ordine
		\item Il Cliente riceve conferma dell'invio dell'ordine $\rightarrow$ Vedi \hyperref[uc_26.4]{UC\_26.4}
	\end{enumerate}
	\item \textbf{Precondizioni}:
	\begin{itemize}
		\item Il sistema è online
		\item Il Cliente è autenticato
		\item La sessione chat del Cliente è attiva
		\item Sono presenti articoli nel carrello
	\end{itemize}
	\item \textbf{Postcondizioni}:
	\begin{itemize}
		\item Il Cliente visualizza la conferma dell'invio dell'ordine
	\end{itemize}
	\item \textbf{Inclusioni}:
	\begin{itemize}
		\item \hyperref[uc_26.4]{UC\_26.4}
	\end{itemize}
    \item \textbf{Trigger}: Il Cliente vuole inviare l'ordine tramite pulsante
\end{itemize}

\subsubsection{UC\_54: Visualizza conferma ordine}
\label{uc_54}

\begin{figure}[H]
	\centering
	\includegraphics[width=0.8\textwidth]{AnalisiRequisiti/UC_54}
	\caption{UC\_54: Visualizza conferma ordine}
\end{figure}

\begin{itemize}
    \item \textbf{Attore$^G$ Principale}: Cliente
    \item \textbf{Scenario principale}:
	\begin{enumerate}
		\item Il Cliente riceve conferma dell'invio dell'ordine
	\end{enumerate}
	\item \textbf{Precondizioni}:
	\begin{itemize}
		\item Il sistema è online
		\item Il Cliente è autenticato
		\item La sessione chat del Cliente è attiva
		\item Sono presenti articoli nel carrello
		\item Il Cliente richiede l'invio dell'ordine tramite carrello
	\end{itemize}
	\item \textbf{Postcondizioni}:
	\begin{itemize}
		\item Il Cliente visualizza la conferma dell'invio dell'ordine
	\end{itemize}
	\item \textbf{Inclusioni}:
	\begin{itemize}
		\item \hyperref[uc_54]{UC\_54}
	\end{itemize}
    \item \textbf{Trigger}: Il Cliente dopo aver inviato l'ordine dal carrello visualizza una notifica
\end{itemize}

\subsubsection{UC\_55: Notifica di conferma annullamento}
\label{uc_55}

\begin{figure}[H]
	\centering
	\includegraphics[width=0.8\textwidth]{AnalisiRequisiti/UC_55}
	\caption{UC\_55: Notifica di conferma annullamento}
\end{figure}

\begin{itemize}
    \item \textbf{Attore$^G$ Principale}: Cliente
	\item \textbf{Scenario principale}:
	\begin{enumerate}
		\item Il Cliente visualizza la notifica di conferma dell'avvenuto annullamento della disambiguazione
	\end{enumerate}
	\item \textbf{Precondizioni}:
	\begin{itemize}
		\item Il sistema è online
		\item Il Cliente è autenticato
		\item La sessione chat del Cliente è attiva
		\item Il Cliente ha richiesto l'annullamento del processo$^G$ di disambiguazione $\rightarrow$ Vedi \hyperref[uc_22.1]{UC\_22.1}
	\end{itemize}
	\item \textbf{Postcondizioni}:
	\begin{itemize}
		\item Il Cliente è informato del corretto annullamento del processo$^G$
	\end{itemize}
	\item \textbf{Trigger}: Il Cliente vuole visualizzare una notifica di conferma di annullamento dell'ordine
\end{itemize}