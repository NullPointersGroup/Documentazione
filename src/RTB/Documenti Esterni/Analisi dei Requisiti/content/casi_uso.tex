\section{Casi d'Uso}

\subsection{Descrizione associata ai Casi d'Uso}
Ogni caso d'uso si compone di tutte le informazioni presenti nella seguente tabella, salvo i casi in cui uno specifico campo non risulti rilevante.\\

\begin{table}[h]
	\centering
	\renewcommand{\arraystretch}{1.15}
	\resizebox{\textwidth}{!}{
		\begin{tabular}{|p{0.25\textwidth}|p{0.65\textwidth}|}
			\hline
			\rowcolor[gray]{0.9}
			\textbf{Campo} & \textbf{Descrizione} \\
			\hline
			Grafico UML & Rappresenta lo scenario dei Casi d'Uso in oggetto. \\
			\hline
			Attore & Rappresenta coloro che interagiscono in quel sistema, senza il controllo da parte del sistema. \\
			\hline
			Scenario principale & La sequenza ragionevole delle operazioni che l'Attore deve effettuare per portare a compimento lo scenario. \\
			\hline
			Precondizioni & Lista di elementi necessari per far sì che l'Attore possa soddisfare il Caso d'Uso in oggetto. \\
			\hline
			Postcondizioni & Lista di elementi che descrive le modifiche effettuate internamente dopo il corretto avvenimento dello scenario principale. \\
			\hline
			Scenario alternativo & Rappresenta un comportamento valido ma non principale, che devia dal flusso base a causa di condizioni diverse, errori o scelte dell’attore. \\
			\hline
			Inclusioni & Ulteriori Casi d'Uso che l'Attore deve compiere per portare a termine lo scenario. \\
			\hline
			Estensioni & Relazione che aggiunge comportamento opzionale o alternativo a un caso d’uso completo, attivata solo al verificarsi di una condizione specifica.  \\
			\hline
			User Story & Descrizione sintetica di una funzionalità dal punto di vista dell’utente, focalizzata sugli obiettivi. \\
			\hline
		\end{tabular}
	}
	\caption{Campi dei Casi d'Uso}
	\vspace{0.3cm}
\end{table}

\subsection{Attori}
\begin{figure}[H]
	\centering
	\includegraphics[width=0.5\textwidth]{AnalisiRequisiti/Utenti}
	\caption{Attori principali}
\end{figure}


\begin{table}[h]
	\centering
	\renewcommand{\arraystretch}{1.15}
	\resizebox{\textwidth}{!}{
		\begin{tabular}{|p{0.25\textwidth}|p{0.65\textwidth}|}
			\hline
			\rowcolor[gray]{0.9}
			\textbf{Autore} & \textbf{Descrizione} \\
			\hline
			Utente & Rappresenta qualsiasi persona che può interagire con il sistema. \\
			\hline
			Cliente & Utente registrato con accesso a funzionalità dedicate agli utenti finali. \\
			\hline
			Admin & Utente con privilegi speciali per gestire e configurare il sistema. \\
			\hline
		\end{tabular}
	}
	\caption{Descrizione Attori}
	\vspace{0.3cm}
\end{table}

\newpage

\subsection{Lista dei Casi d'Uso}

%\subsubsection{UC\_01: Registrazione}
\label{uc_01}

\begin{figure}[H]
    \centering 
    \includegraphics[width=0.8\textwidth]{AnalisiRequisiti/UC_01}
    \caption{UC\_01: Registrazione}
\end{figure}

\begin{itemize}
    \item \textbf{Attore Principale}: Cliente
    \item \textbf{Scenario Principale}: 
    \begin{enumerate}
        \item Inserimento dello Username $\rightarrow$ Vedi \hyperref[uc_01.1]{UC\_01.1}
        \item Inserimento della Password $\rightarrow$ Vedi \hyperref[uc_01.2]{UC\_01.2}
        \item Inserimento della Password nel campo di \vr{conferma password} $\rightarrow$ Vedi \hyperref[uc_01.3]{UC\_01.3}
        \item Inserimento della email $\rightarrow$ Vedi \hyperref[uc_01.4]{UC\_01.4}
    \end{enumerate}
    \item \textbf{Precondizioni}:
    \begin{itemize}
        \item Il sistema è online
        \item Il Cliente non è registrato nel sistema
    \end{itemize}
    \item \textbf{Postcondizioni}:
    \begin{itemize}
        \item Il Cliente è ora registrato ed è riconosciuto come tale
    \end{itemize}
    \item \textbf{Inclusioni}:
    \begin{itemize}
        \item \hyperref[uc_01.1]{UC\_01.1}
        \item \hyperref[uc_01.2]{UC\_01.2}
        \item \hyperref[uc_01.3]{UC\_01.3}
        \item \hyperref[uc_01.4]{UC\_01.4}
    \end{itemize}
    \item \textbf{User Story}: Il Cliente vuole registrarsi nel sistema
\end{itemize}

\subsubsubsection{UC\_01.1: Inserimento username}
\label{uc_01.1}

\begin{figure}[H]
    \centering 
    \includegraphics[width=0.8\textwidth]{AnalisiRequisiti/UC_01.1}
    \caption{Estensioni UC\_01.1: UC\_01.1.1}
\end{figure}

\begin{itemize}
    \item \textbf{Attore Principale}: Cliente
    \item \textbf{Scenario Principale}:
    \begin{enumerate}
        \item Il Cliente inserisce uno username con il quale vuole registrarsi
    \end{enumerate}
    \item \textbf{Precondizioni}: 
    \begin{itemize}
        \item Il sistema è online
        \item Il Cliente non è registrato nel sistema
        \item Lo username che Il Cliente inserisce non deve essere già presente nel sistema
    \end{itemize}
    \item \textbf{Postcondizioni}:
    \begin{itemize}
        \item Username inserito, pronto per essere registrato
    \end{itemize}
    \item \textbf{Scenari Alternativi}:
    \begin{itemize}
        \item Il Cliente ha inserito uno username esistente $\rightarrow$ Vedi \hyperref[uc_01.1.1]{UC\_01.1.1}
    \end{itemize}
    \item \textbf{Estensioni}:
    \begin{itemize}
        \item \hyperref[uc_01.1.1]{UC\_01.1.1}
    \end{itemize}
    \item \textbf{User Story}: Il Cliente vuole registrarsi nel sistema
\end{itemize}

\subsubsubsubsection{UC\_01.1.1: Errore inserimento username}
\label{uc_01.1.1}

\begin{itemize}
    \item \textbf{Attore Principale}: Cliente
    \item \textbf{Precondizioni}:
    \begin{itemize}
        \item Il sistema è online
        \item Il Cliente non è registrato nel sistema
        \item Il Cliente ha inserito uno username già esistente come proprio username
    \end{itemize}
    \item \textbf{Postcondizioni}:
    \begin{itemize}
        \item Link HTML per andare alla pagina di login
    \end{itemize}
\end{itemize}

\subsubsubsection{UC\_01.2: Inserimento password}
\label{uc_01.2}

\begin{figure}[H]
    \centering 
    \includegraphics[width=0.8\textwidth]{AnalisiRequisiti/UC_01.2}
    \caption{Estensioni UC\_01.2: UC\_01.2.1}
\end{figure}

\begin{itemize}
    \item \textbf{Attore Principale}: Cliente
    \item \textbf{Scenario Principale}:
    \begin{enumerate}
        \item Il Cliente inserisce la password, che insieme allo username userà per autenticarsi
    \end{enumerate}
    \item \textbf{Precondizioni}: 
    \begin{itemize}
        \item Il sistema è online
        \item Il Cliente non è registrato nel sistema
        \item Il sistema non conosce la password
    \end{itemize}
    \item \textbf{Postcondizioni}:
    \begin{itemize}
        \item Il sistema riceve la password e ne conserva l'hash
    \end{itemize}
    \item \textbf{Scenari Alternativi}:
    \begin{itemize}
        \item Il Cliente ha inserito una password che non è conforme $\rightarrow$ Vedi \hyperref[uc_01.2.1]{UC\_01.2.1}
    \end{itemize}
    \item \textbf{Estensioni}:
    \begin{itemize}
        \item \hyperref[uc_01.2.1]{UC\_01.2.1}
    \end{itemize}
    \item \textbf{User Story}: Il Cliente vuole registrarsi nel sistema
\end{itemize}


\subsubsubsubsection{UC\_01.2.1: Errore inserimento password}
\label{uc_01.2.1}

\begin{itemize}
    \item \textbf{Attore Principale}: Cliente
    \item \textbf{Precondizioni}:
    \begin{itemize}
        \item Il sistema è online
        \item Il Cliente non è registrato nel sistema
        \item Il Cliente ha inserito una password e quest'ultima non rispetta i criteri di sicurezza
    \end{itemize}
    \item \textbf{Postcondizioni}:
    \begin{itemize}
        \item Messaggio di errore visibile al Cliente
    \end{itemize}
\end{itemize}

\subsubsubsection{UC\_01.3: Conferma password}
\label{uc_01.3}

\begin{figure}[H]
    \centering 
    \includegraphics[width=0.8\textwidth]{AnalisiRequisiti/UC_01.3}
    \caption{Estensioni UC\_01.3: UC\_01.3.1}
\end{figure}

\begin{itemize}
    \item \textbf{Attore Principale}: Cliente
    \item \textbf{Scenario Principale}:
    \begin{enumerate}
        \item Il Cliente inserisce nuovamente la password nel campo \vr{conferma password}
    \end{enumerate}
    \item \textbf{Precondizioni}:
    \begin{itemize}
        \item Il sistema è online
        \item Il Cliente non è registrato nel sistema
        \item Il sistema conosce la password inserita in precedenza
    \end{itemize}
    \item \textbf{Postcondizioni}:
    \begin{itemize}
        \item Password confermata, la registrazione può essere conclusa
    \end{itemize}
    \item \textbf{Scenari Alternativi}:
    \begin{itemize}
        \item Il Cliente ha inserito nel campo \vr{conferma password} una password che non coincide con quella inserita precedentemente $\rightarrow$ Vedi \hyperref[uc_01.3.1]{UC\_01.3.1}
    \end{itemize}
    \item \textbf{Estensioni}:
    \begin{itemize}
        \item \hyperref[uc_01.3.1]{UC\_01.3.1}
    \end{itemize}
    \item \textbf{User Story}: Il Cliente vuole registrarsi nel sistema
\end{itemize}

\subsubsubsubsection{UC\_01.3.1: Errore conferma password}
\label{uc_01.3.1}

\begin{itemize}
    \item \textbf{Attore Principale}: Cliente
    \item \textbf{Precondizioni}:
    \begin{itemize}
        \item Il sistema è online
        \item Il Cliente non è registrato nel sistema
        \item Il sistema conosce la password inserita in precedenza
        \item Il Cliente ha inserito una password non coincidente con quella inserita precedentemente
    \end{itemize}
    \item \textbf{Postcondizioni}:
    \begin{itemize}
        \item Messaggio di errore visibile al Cliente
    \end{itemize}
\end{itemize}

\subsubsubsection{UC\_01.4: Inserimento email}
\label{uc_01.4}

\begin{figure}[H]
    \centering 
    \includegraphics[width=0.8\textwidth]{AnalisiRequisiti/UC_01.4}
    \caption{Estensioni UC\_01.4: UC\_01.4.1, UC\_01.4.2}
\end{figure}

\begin{itemize}
    \item \textbf{Attore Principale}: Cliente
    \item \textbf{Scenario Principale}:
    \begin{enumerate}
        \item Il Cliente inserisce l’email che utilizzerà per il recupero della password. $\rightarrow$ Vedi \hyperref[uc_04]{UC\_04}
    \end{enumerate}
    \item \textbf{Precondizioni}:
    \begin{itemize}
        \item Il sistema è online
        \item Il Cliente non è registrato nel sistema
        \item Il sistema non conosce l'email
    \end{itemize}
    \item \textbf{Postcondizioni}:
    \begin{itemize}
        \item Email inserita, pronta per essere registrata
    \end{itemize}
    \item \textbf{Scenari Alternativi}:
    \begin{itemize}
        \item Il Cliente ha inserito una email non valida $\rightarrow$ Vedi \hyperref[uc_01.4.1]{UC\_01.4.1}
        \item Il Cliente ha inserito una email già presente nel sistema $\rightarrow$ Vedi \hyperref[uc_01.4.2]{UC\_01.4.2}
    \end{itemize}
    \item \textbf{Estensioni}:
    \begin{itemize}
        \item \hyperref[uc_01.4.1]{UC\_01.4.1}
        \item \hyperref[uc_01.4.2]{UC\_01.4.2}
    \end{itemize}
    \item \textbf{User Story}: Il Cliente vuole registrarsi nel sistema
\end{itemize}

\subsubsubsubsection{UC\_01.4.1: Email non valida}
\label{uc_01.4.1}

\begin{itemize} 
    \item \textbf{Attore Principale}: Cliente
    \item \textbf{Precondizioni}:
    \begin{itemize}
        \item Il sistema è online
        \item Il Cliente non è registrato nel sistema
        \item Il Cliente ha inserito una email in un formato non corretto
    \end{itemize}
    \item \textbf{Postcondizioni}:
    \begin{itemize}
        \item Messaggio di errore visibile al Cliente 
    \end{itemize}
\end{itemize}


\subsubsubsubsection{UC\_01.4.2: Email già presente}
\label{uc_01.4.2}

\begin{itemize}
    \item \textbf{Attore Principale}: Cliente
    \item \textbf{Precondizioni}:
    \begin{itemize}
        \item Il sistema è online
        \item Il Cliente non è registrato nel sistema
        \item Il Cliente ha inserito una email già presente nel sistema
    \end{itemize}
    \item \textbf{Postcondizioni}:
    \begin{itemize}
        \item Messaggio di errore visibile al Cliente 
    \end{itemize}
\end{itemize}

%\subsubsection{UC\_02: Autenticazione}
\label{uc_02}

\begin{figure}[H]
    \centering 
    \includegraphics[width=0.8\textwidth]{AnalisiRequisiti/UC_02}
    \caption{UC\_02: Autenticazione}
\end{figure}

\begin{itemize}
    \item \textbf{Attore$^G$ Principale}: Utente
    \item \textbf{Scenario Principale}:
    \begin{enumerate}
        \item L'Utente ha selezionato dal menù la possibilità di autenticarsi nel sistema 
        \item L'Utente inserisce lo username $\rightarrow$ Vedi \hyperref[uc_02.1]{UC\_02.1}
        \item L'Utente inserisce la password $\rightarrow$ Vedi \hyperref[uc_02.2]{UC\_02.2}
    \end{enumerate}
    \item \textbf{Precondizioni}:
    \begin{itemize}
        \item Il sistema è online
        \item L'Utente non è autenticato nel sistema
    \end{itemize}
    \item \textbf{Postcondizioni}:
    \begin{itemize}
        \item L'Utente è ora riconosciuto come Cliente
    \end{itemize}
    \item \textbf{Scenari Alternativi}:
    \begin{itemize}
        \item L'autenticazione fallisce $\rightarrow$ Vedi \hyperref[uc_03]{UC\_03}
    \end{itemize}
    \item \textbf{Inclusioni}:
    \begin{itemize}
        \item \hyperref[uc_02.1]{UC\_02.1}
        \item \hyperref[uc_02.2]{UC\_02.2}
    \end{itemize}
    \item \textbf{Estensioni}:
    \begin{itemize}
        \item \hyperref[uc_03]{UC\_03}
    \end{itemize}
    \item \textbf{Trigger}: L'Utente vuole autenticarsi nel sistema
\end{itemize}


\subsubsubsection{UC\_02.1: Inserimento username}
\label{uc_02.1}

\begin{itemize}
    \item \textbf{Attore$^G$ Principale}: Utente
    \item \textbf{Scenario Principale}:
    \begin{enumerate}
        \item L'Utente inserisce lo username con il quale vuole autenticarsi
    \end{enumerate}
    \item \textbf{Precondizioni}:
    \begin{itemize}
        \item Il sistema è online
        \item L'Utente non è ancora autenticato
    \end{itemize}
    \item \textbf{Postcondizioni}:
    \begin{itemize}
        \item Username inserito correttamente
    \end{itemize}
    \item \textbf{Trigger}: L'Utente vuole inserire lo username per l'autenticazione
\end{itemize}


\subsubsubsection{UC\_02.2: Inserimento password}
\label{uc_02.2}

\begin{itemize}
    \item \textbf{Attore$^G$ Principale}: Utente
    \item \textbf{Scenario Principale}:
    \begin{enumerate}
        \item L'Utente inserisce la propria password
    \end{enumerate}
    \item \textbf{Precondizioni}:
    \begin{itemize}
        \item Il sistema è online
        \item L'Utente non è ancora autenticato
    \end{itemize}
    \item \textbf{Postcondizioni}:
    \begin{itemize}
        \item Password inserita correttamente
    \end{itemize}
    \item \textbf{Trigger}: L'Utente vuole inserire la password per l'autenticazione
\end{itemize}