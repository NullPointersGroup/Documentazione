\subsection{Requisiti Funzionali}

\subsubsection{Requisiti Funzionali obbligatori}
\renewcommand{\arraystretch}{1.15}
\begin{longtable}{|>{\raggedright}p{0.15\textwidth}|>{\raggedright\arraybackslash}p{0.6\textwidth}|>{\raggedright\arraybackslash}p{0.25\textwidth}|}
	\hline
	\rowcolor[gray]{0.9}
	\textbf{Codice} & \textbf{Descrizione} & \textbf{Fonti} \\
	\hline
	\endfirsthead
	
	\hline
	\rowcolor[gray]{0.9}
	\textbf{Codice} & \textbf{Descrizione} & \textbf{Fonti}  \\
	\hline
	\endhead
	
	\hline
	X & L'Utente deve poter visualizzare il campo di input per inserire lo username & UC\_01.1 \\
	\hline
	X & L'Utente deve poter inserire lo username & UC\_01.1 \\
	\hline
	X & Lo username deve essere univoco & UC\_01.1.2 \\
	\hline
	X & L'Utente deve poter visualizzare una notifica di errore se lo username è già presente nel sistema & UC\_01.1.2 \\
	\hline
	X & Lo username deve avere un massimo di 24 caratteri. & UC\_01.1.1\newline UC\_08 \\
	\hline
	X & L'Utente deve poter visualizzare una notifica di errore se lo username eccede il numero massimo di caratteri & UC\_01.1.1 \\
	\hline
	X & L'Utente deve poter visualizzare il campo di input per inserire la password & UC\_01.2 \\
	\hline
	X & L'Utente deve poter inserire la password & UC\_01.2\\
	\hline
	X & La password deve avere almeno 1 lettera maiuscola, 1 lettera minuscola, 1 numero, 1 carattere speciale e almeno 8 caratteri & UC\_01.2\newline UC\_01.2.1\newline UC\_08 \\
	\hline
	X & La password deve avere un massimo di 24 caratteri & UC\_01.2.1 \\
	\hline
	X & L'Utente deve poter visualizzare una notifica di errore se la password non è conforme ai criteri & UC\_01.2.1 \\
	\hline
	X & L'Utente deve poter visualizzare una notifica di errore se la password supera 24 caratteri & UC\_01.2.1 \\
	\hline
	X & L'Utente deve poter tornare alla login via link HTML. & UC\_01.1.2\newline UC\_01.4.2 \\
	\hline
	X & L'Utente deve poter visualizzare il campo di input della nuova password & UC\_01.3 \\
	\hline
	X & L'Utente deve poter confermare la password tramite nuovo campo di input & UC\_01.3 \\
	\hline
	X & L'Utente deve poter visualizzare una notifica di errore se la password e la sua conferma non coincidono & UC\_01.3.1 \\
	\hline
	X & L'Utente deve poter visualizzare il campo di input per l'inserimento dell'email & UC\_01.4 \\
	\hline
	X & L'Utente deve poter inserire l'email & UC\_01.4 \\
	\hline
	X & L'email deve essere univoca & UC\_01.4.2 \\
	\hline
	X & L'Utente deve poter visualizzare l'errore nel caso in cui l'email non fosse nel formato corretto & UC\_01.4.1 \\
	\hline
	X & L'Utente deve poter visualizzare l'errore nel caso in cui l'email sia già presente nel sistema & UC\_01.4.2 \\
	\hline
	X & L'Utente deve essere indirizzato alla pagina login una volta che la registrazione ha avuto successo. & UC\_01 \\
	\hline
	X & L'Utente deve poter visualizzare il campo per inserire lo username & UC\_02.1 \\
	\hline
	X & L'Utente deve poter inserire lo username & UC\_02.1 \\
	\hline
	X & L'Utente deve poter visualizzare il campo per inserire la password & UC\_02.2 \\
	\hline
	X & L'Utente deve poter inserire la password & UC\_02.2 \\
	\hline
	X & Dev'essere data la possibilità di tornare alla registrazione via link. & UC\_02 \\
	\hline
	X & L'Utente deve poter visualizzare il messaggio \vr{Username o password errati} se l'autenticazione fallisce & UC\_03 \\
	\hline 
	X & Deve poter avvenire un logout automatico a seguito della cancellazione account & UC\_05\\
	\hline
	X & L'Utente deve poter visualizzare il pulsante \vr{Logout} nella pagina del profilo & UC\_06\\
	\hline
	X & L'Utente deve poter effettuare il logout tramite il pulsante \vr{Logout} & UC\_06 \\
	\hline
	X & L'Utente può inserire input testuali con lunghezza massima di 4096 caratteri. & UC\_09\newline UC\_10 \\
	\hline
	X & L'Utente deve poter visualizzare un messaggio di errore al superamento di 4096 caratteri. & UC\_10 \\
	\hline
	X & L'Utente deve poter visualizzare il campo di input. & UC\_09\newline UC\_13\newline UC\_15\newline ......................................UC\_17 \\
	\hline
	RF-OB\_35 & L'Utente deve poter visualizzare il tasto di invio & UC\_09\newline UC\_11\newline UC\_13\newline UC\_15\newline ......................................UC\_17 \\
	\hline
	RF-OB\_37 & L'Utente in assenza di contenuto di input non potrà premere il tasto di invio che sarà disabilitato & UC\_09\newline UC\_11\newline UC\_13\newline UC\_15\newline ......................................UC\_17 \\
	\hline
	RF-OB\_38 & L'Utente deve poter inserire, con qualsiasi forma di input, un massimo di 10 articoli distinti. &  ......................................UC\_13\newline UC\_15\newline UC\_17 \\
	\hline
	RF-OB\_10 & L'Utente deve ricevere una richiesta di chiarimento da parte del chatbot nel caso in cui non capisse l'articolo di riferimento oppure non ci fossero articoli corrispondenti. & ......................................\newline UC\_12\newline UC\_13 \\
	\hline
	X RF-OB\_11 & L'Utente dopo l'invio dovrà visualizzare il campo di testo vuoto & UC\_09\newline UC\_11\newline UC\_13 \\
	\hline
	RF-OB\_12 & L'Utente deve poter vedere un pulsante con l'icona del microfono & UC\_13 \\
	\hline
	RF-OB\_13 & L'Utente deve poter cliccare il pulsante con l'icona del microfono per registrare messaggi audio & UC\_13\newline UC\_13.1 \\
	\hline
	RF-OB\_14 & L'Utente deve poter vedere la durata della registrazione mentre registra & UC\_13\newline UC\_13.1 \\
	\hline
	RF-OB\_15 & L'Utente premendo nuovamente l'icona del microfono ferma la registrazione & UC\_13\newline UC\_13.1 \\
	\hline
	RF-OB\_16 & L'Utente deve vedere un pulsante con icona \vr{clip} & UC\_11.2\newline UC\_13.2 \\
	\hline
	RF-OB\_17 & L'Utente deve poter premere il pulsante per selezionare il file da caricare dal dispositivo & UC\_11.2\newline UC\_13.2 \\
	\hline
	RF-OB\_18 & L'Utente cliccando sul pulsante apre il dialogo di selezione file del sistema operativo. & UC\_11.2\newline UC\_13.2 \\
	\hline
	RF-OB\_19 & L'Utente deve poter selezionare i file & UC\_11.2\newline UC\_13.2 \\
	\hline
	RF-OB\_20 & L'Utente deve poter vedere il tasto \vr{Seleziona} nella selezione file & UC\_11.2\newline UC\_13.2 \\
	\hline
	RF-OB\_21 & L'Utente deve poter inviare audio solo se <= 120 secondi. & UC\_13.1\newline UC\_13.2 \\
	\hline
	RF-OB\_22 & Se il messaggio vocale supera i 120 secondi compare un messaggio di errore & UC\_13.1\newline UC\_13.2 \\
	\hline
	RF-OB\_23 & L'Utente può inviare file audio con dimensione massima di 10MB in caso contrario ne sarà impedito l'invio & UC\_13.1\newline UC\_13.2 \\
	\hline
	RF-OB\_24 & Se L'Utente prova ad inviare audio di dimensioni superiori ai 10MB deve esserci la comparsa dell'errore con spiegazione & UC\_14 \\
	\hline
	RF-OB\_25 & Il dialogo di selezione file deve filtrare e mostrare solo file .mp3, .m4a, .m4p, .wav. & UC\_13.2.1 \newline UC\_13.2.2 \newline UC\_13.2.3 \newline UC\_13.2.4 \\
	\hline
	RF-OB\_26 & L'Utente deve vedere il pulsante di rimozione dell'audio durante la registrazione & UC\_13.1 \\
	\hline
	RF-OB\_27 & L'Utente deve poter premere il pulsante di rimozione dell'audio durante la registrazione per eliminarlo e dunque impedire l'invio. & UC\_13.1 \\
	\hline
	RF-OB\_28 & L'Utente può aggiungere articoli all'ordine tramite comandi testuali o vocali & UC\_15\newline UC\_9\newline UC\_13 \\
	\hline
	RF-OB\_29 & Il Cliente deve ricevere conferma dell'aggiunta dei prodotti & UC\_15\newline UC\_21.2 \\
	\hline
	RF-OB\_30 & L'Utente deve essere notificato nel caso in cui i prodotti di cui ha richiesto l'inserimento non siano stati trovati nel catalogo o se c'è stato un errore. & UC\_16 \\
	\hline
	RF-OB\_31 & L'Utente deve poter rimuovere articoli tramite comandi testuali o vocali & UC\_17 \\
	\hline
	RF-OB\_32 & L'Utente deve ricevere una conferma della rimozione. & UC\_17 \\
	\hline
	RF-OB\_33 & L'Utente deve essere avvisato nel caso in cui i prodotti dei quali ha richiesto la rimozione non fossero presenti nel carrello o se c'è stato un errore. & UC\_18 \\
	\hline
	RF-OB\_35 & Il Cliente deve poter inviare il comando testuale \verb|/carrello|. & UC\_19.1 \\
	\hline
	RF-OB\_34 & Il Cliente visualizza l'anteprima del carrello dopo aver inserito il comando testuale \verb|/carrello|. & UC\_19.2 \\
	\hline
	X & Il Cliente deve poter visualizzare i prezzi di ogni articolo & UC\_19.1.2\newline UC\_26.2.1 \\
	\hline
	X & Il Cliente deve poter visualizzare il nome di ogni articolo & UC\_19.1.1\newline UC\_26.2.2 \\
	\hline
	X & Il Cliente visualizza la quantità di ogni articolo & UC\_19.2.3\newline UC\_26.2.3 \\
	\hline
	RF-OB\_39 & Il Cliente se digita \verb|/carrello| riceve un avviso nel caso in cui nel carrello non ci fossero elementi. & UC\_20 \\
	\hline
	RF-OB\_40 & Il Cliente deve visualizzare un avviso riguardante l'ambiguità dei prodotti & UC\_21 \\
	\hline
	RF-OB\_40 & Il Cliente deve poter riformulare l'ordine per prodotti ambigui & UC\_21.1 \\
	\hline
	RF-OB\_43 & Il Cliente può rispondere \vr{annulla} per uscire dalla disambiguazione. & UC\_22.1 \\
	\hline
	RF-OB\_44 & Il Cliente riceve conferma dell'annullamento della disambiguazione & UC\_22.2 \\
	\hline
	RF-OB\_46 & Il Cliente deve vedere nella home il tasto \vr{Segnala un problema} & UC\_24 \\
	\hline
	RF-OB\_47 & Il Cliente cliccando sul pulsante \vr{Segnala un problema} apre in una nuova pagina un modulo dedicato & UC\_24 \\
	\hline
	RF-OB\_48 & Il Cliente deve visualizzare il dropdown: \vr{Bug}, \vr{Richiesta di supporto}, \vr{Suggerimento} & UC\_24.1 \\
	\hline
	RF-OB\_49 & Il Cliente può selezionare il tipo da dropdown & UC\_24.1 \\ %il 24.2 si rifà al 23.2
	\hline
	RF-OB\_51 & Il Cliente può visualizzare il tasto \vr{Invia segnalazione} & UC\_24.3 \\
	\hline
	RF-OB\_52 & Il Cliente può inviare il form cliccando \vr{Invia segnalazione} & UC\_24.3 \\
	\hline
	RF-OB\_53 & Il Cliente riceve conferma dell'invio con un messaggio \vr{Ticket creato con successo} & UC\_24.4 \\
	\hline
	X & Il Cliente può inserire il comando \vr{/invia} nella chat per avviare la conferma dell'ordine. & UC\_26.1 \\
	\hline
	X & Se il carrello è vuoto dopo il comando \vr{/invia} il Cliente visualizza un messaggio di errore che avvisa che il carrello è vuoto. & UC\_27 \\
	\hline
	X & Il Cliente deve poter visualizzare il singolo elemento nella chat & UC\_26.2 \\
	\hline
	X & Il Cliente a seguito del messaggio riepilogativo deve decidere se annullare o confermare l'invio dell'ordine & UC\_26.2 \\
	\hline
	X & Il Cliente deve poter confermare l'invio dell'ordine con un input & UC\_26.3\newline UC\_34.1 \\
	\hline
	X & Il Cliente deve poter visualizzare la conferma dell'invio dell'ordine & UC\_26.4\\
	\hline
	X & Il Cliente deve poter annullare l'invio dell'ordine con un input & UC\_26.5\\
	\hline
	X & Il Cliente a seguito della conferma visualizza un riepilogo dell'ordine & UC\_26.2 \\
	\hline
	X & Il Cliente deve poter vedere il pulsante \vr{Nuova chat} per iniziare una nuova sessione & UC\_28 \\
	\hline
	X & L'Utente deve poter cliccare sul pulsante \vr{Nuova chat} per iniziare una nuova sessione. & UC\_28 \\
	\hline
	X & Il Cliente richiede la duplicazione al chatbot indicando il codice dell'ordine dentro il comando comando \vr{/duplica\{xx\}}. & UC\_34 \\
	\hline
	X & Il Cliente deve poter richiedere al chatbot di duplicare l'ultimo ordine effettuato indicando il comando "/duplica" & UC\_34 \\
	\hline
	X & Il Cliente deve essere informato che non è possibile procedere con la richiesta. & UC\_34 \\
	\hline
	X & Il Cliente deve poter visualizzare una notifica di conferma della duplicazione. & UC\_34.2 \\
	\hline
	X & Il Cliente deve poter visualizzare una notifica di errore nel caso in cui il prodotto non venga duplicato correttamente & UC\_35 \\
	\hline
	X & L'Utente deve poter visualizzare un'icona \vr{visualizza storico ordini} & UC\_36 \newline UC\_45 \\
	\hline
	X & L'Utente deve poter schiacciare l'icona \vr{visualizza storico ordini} & UC\_36 \newline UC\_45 \\
	\hline
	X & L'Admin deve poter visualizzare una pagina separata contenente lo storico totale degli ordini & UC\_36 \\
	\hline
	X & La lista completa degli ordini deve essere visualizzata in pagine contenenti al massimo 10 ordini ciascuna & UC\_36 \newline UC\_45 \\
	\hline
	X & Il caricamento di una pagina non implica il caricamento delle pagine successive & UC\_36 \newline UC\_45 \\
	\hline
	X & L'Utente deve poter visualizzare il singolo ordine nello storico & UC\_36.1 \newline UC\_45.1 \\
	\hline
	X & L'Utente deve poter visualizzare il codice ordine & UC\_36.1.1 \newline UC\_45.1.1 \\
	\hline
	X & L'Utente deve poter visualizzare la data dell'ordine & UC\_36.1.2 \newline UC\_45.1.2 \\
	\hline
	X & L'Admin deve poter visualizzare lo username del Cliente & UC\_36.1.3 \\
	\hline
	X & L'Utente deve poter visualizzare il dettaglio di un ordine specifico & UC\_37 \newline UC\_46 \\
	\hline
	X & L'Admin deve poter visualizzare lo username del Cliente & UC\_37.1 \\
	\hline
	X & L'Utente deve poter visualizzare i prodotti dell'ordine & UC\_37.2 \newline UC\_46.1 \\
	\hline
	X & L'Utente deve poter visualizzare il nome dei prodotti & UC\_37.2.1 \newline UC\_46.1.1 \\
	\hline
	X & L'Utente deve poter visualizzare la descrizione dei prodotti & UC\_37.2.2 \newline UC\_46.1.2 \\
	\hline
	X & L'Utente deve poter visualizzare la quantità dei prodotti & UC\_37.2.3 \newline UC\_46.1.3 \\
	\hline
	X & L'Utente deve poter visualizzare la data dell'ordine & UC\_37.3 \newline UC\_46.2 \\
	\hline
	X & L'Utente deve poter visualizzare il numero dell'ordine & UC\_37.4 \newline UC\_46.3 \\
	\hline
	X & Il Cliente deve poter visualizzare il pulsante per duplicare l'ordine & UC\_38 \\
	\hline
	X & Il Cliente deve poter cliccare sul pulsante per duplicare l'ordine & UC\_38 \\
	\hline
	X & Il Cliente deve poter confermare la duplicazione dell'ordine & UC\_38.1 \\
	\hline
	X & Il Cliente deve poter visualizzare la notifica di errore duplicazione ordine & UC\_39 \\
	\hline
	\multicolumn{3}{c}{} \\
	\caption{Requisiti Funzionali obbligatori}\label{tab:funzionali_obbligatori}\\
\end{longtable}

\subsubsection{Requisiti Funzionali desiderabili}
\renewcommand{\arraystretch}{1.15}
\begin{longtable}{|>{\raggedright}p{0.18\textwidth}|>{\raggedright\arraybackslash}p{0.6\textwidth}|>{\raggedright\arraybackslash}p{0.25\textwidth}|}
	\hline
	\rowcolor[gray]{0.9}
	\textbf{Codice} & \textbf{Descrizione} & \textbf{Fonti} \\
	\hline
	\endfirsthead
	
	\hline
	\rowcolor[gray]{0.9}
	\textbf{Codice} & \textbf{Descrizione} & \textbf{Fonti}  \\
	\hline
	\endhead
	
	\hline
	X & L'Utente deve poter visualizzare il link HTML per la password dimenticata & UC\_04 \\
	\hline
	X & L'Utente deve poter cliccare il link HTML per la password dimenticata & UC\_04 \\
	\hline
	X & L'Utente deve poter visualizzare il form per inserire l'indirizzo mail che ha usato quando si è registrato & UC\_04.1 \\
	\hline
	X & L'Utente deve ricevere via mail la password generata automaticamente & UC\_04.1 \\
	\hline
	X & L'Utente deve ricevere un messaggio di errore nel caso in cui l'email non fosse valida & UC\_04.1.1 \\
	\hline
	X & L'Utente deve ricevere un messaggio di errore nel caso in cui l'email non fosse presente nel sistema & UC\_04.1.2 \\
	\hline
	X & L'Utente deve poter visualizzare il bottone \vr{Elimina Account} nella pagina delle informazioni del profilo & UC\_05 \\
	\hline
	X & L'Utente deve poter premere il bottone per eliminare il suo account & UC\_05 \\
	\hline
	X & L'Utente deve poter visualizzare la conferma dell'eliminazione dell'account & UC\_05.1 \\
	\hline
	X & L'Utente deve poter visualizzare nella home un'icona che porta alla pagina delle info del profilo & UC\_07 \\
	\hline
	X & L'Utente deve poter cliccare sul link, rappresentato da un'icona, che porta sulla pagina delle info del proprio profilo & UC\_07 \\
	\hline
	X & L'Utente deve poter visualizzare il link alla pagina dello storico ordini nelle info del proprio profilo & UC\_07 \\
	\hline
	X & L'Utente deve poter visualizzare il proprio username nella pagina delle info del proprio profilo & UC\_07.1 \\
	\hline
	X & L'Utente deve poter visualizzare la propria ragione sociale nelle info del proprio profilo & UC\_07.2 \\
	\hline
	X & L'Utente deve poter visualizzare il proprio indirizzo email nelle info del proprio profilo & UC\_07.3 \\
	\hline
	X & L'Utente deve poter visualizzare il bottone \vr{Reimposta password} & UC\_08 \\
	\hline
	X & L'Utente deve poter cliccare il bottone \vr{Reimposta password} & UC\_08 \\
	\hline
	X & L'Utente deve poter visualizzare il campo di input \vr{Vecchia password} & UC\_08.1 \\
	\hline
	X & L'Utente deve poter inserire la vecchia password nel campo di input \vr{Vecchia password} & UC\_08.1 \\
	\hline
	X & L'Utente deve poter visualizzare un messaggio di errore nel caso in cui l'inserimento della vecchia password non corrispondesse con la password presente nel sistema & UC\_08.1.1 \\
	\hline
	X & L'Utente deve poter visualizzare il campo di input \vr{Nuova password} & UC\_08.2 \\
	\hline
	X & L'Utente deve poter inserire la nuova password nel campo di input \vr{Nuova password} & UC\_08.2 \\
	\hline
	X & L'Utente deve poter visualizzare un messaggio di errore nel caso in cui la nuova password inserita non fosse conforme ai criteri imposti & UC\_08.2.1 \\
	\hline
	X & L'Utente deve poter visualizzare il campo di input \vr{Conferma nuova password} & UC\_08.3 \\
	\hline
	X & L'Utente deve poter confermare la nuova password nel campo di input \vr{Conferma nuova password} & UC\_08.3 \\
	\hline
	X & L'Utente deve poter visualizzare un messaggio di errore nel caso in cui la conferma della nuova password non corrisponda alla nuova password inserita nel campo precedente & UC\_08.3.1 \\
	\hline
	X & L'Utente deve essere notificato qualora il cambio password vada a buon fine tramite un messaggio di conferma. & UC\_08.3 \\
	\hline
	X & Durante la digitazione L'Utente deve poter visualizzare un contatore x/4096 vicino al campo di testo & UC\_09 \\
	\hline
	X & L'Utente deve poter visualizzare il pulsante di caricamento file immagine (icona fotocamera). & UC\_11 \\
	\hline
	X & L'Utente deve poter cliccare l'icona della fotocamera & UC\_11 \\
	\hline
	X & L'Utente a seguito del click sull'icona a forma di clip deve poter vedere un dropdown tra cui: "Seleziona dal dispositivo" oppure "Scatta la foto". & UC\_11 \\
	\hline
	X & L'Utente deve poter selezionare \vr{Seleziona dal dispositivo} &  UC\_11.2 \\
	\hline
	X & Se L'Utente seleziona \vr{Seleziona dal dispositivo} deve poter visualizzare i media nel dispositivo &  UC\_11.2 \\
	\hline
	X & L'Utente deve poter selezionare \vr{Scatta foto}. &  UC\_11.1 \\
	\hline
	X & Alla selezione \vr{Scatta foto} L'Utente visualizza l'attivazione della fotocamera. &  UC\_11.1 \\
	\hline
	X & L'Utente deve poter inviare immagini con dimensione massima di 15MB e con risoluzione pari o superiore a 800x600 pixel. & UC\_11 \\
	\hline
	X & Nel caso in cui le immagini superino i 15MB abbiano risoluzione inferiore di 800x600 pixel sarà visibile al Cliente un messaggio di errore \vr{L'immagine non rispetta i requisiti richiesti} & UC\_11\newline UC\_12 \\
	\hline
	X & L'Utente deve premere il pulsante di invio per confermare l'invio dell'immagine. & UC\_11 \\
	\hline
	X & L'Utente deve poter visualizza un messaggio \vr{Non sono riuscito a rilevare il testo nell'immagine} se il modello non riesce a rilevare testo & UC\_12 \\
	\hline
	X & L'Utente deve poter selezionare immagini nei formati .jpg & UC\_11.2.1 \\
	\hline
	X & L'Utente deve poter selezionare immagini nei formati .jpeg  & UC\_11 \\
	\hline
	X & L'Utente deve poter selezionare immagini nei formati .png & UC\_11 \\
	\hline
	X & L'Utente può visionare l'anteprima dell'immagine caricata. & UC\_11 \\
	\hline
	RF-DE\_58 & Il Cliente può visualizzare il pulsante a fianco all'ultima risposta \vr{lascia Feedback}. & UC\_23 \\
	\hline
	RF-DE\_59 & Il Cliente può premere il pulsante affianco ad ogni risposta \vr{lascia Feedback} per fornire feedback all'AI & UC\_23 \\
	\hline
	RF-DE\_60 & Il Cliente premendo su \vr{lascia Feedback} deve visualizzare l'apertura di un form in una nuova pagina & UC\_23 \\
	\hline
	RF-DE\_61 & Il Cliente deve vedere il dropdown: \vr{Prodotto sbagliato}, \vr{Quantità errata}, \vr{Incomprensione}, \vr{Altro}. & UC\_23.1 \\
	\hline
	RF-DE\_62 & Il Cliente deve selezionare obbligatoriamente una categoria: \vr{Prodotto sbagliato}, \vr{Quantità errata}, \vr{Incomprensione}, \vr{Altro}. & UC\_23.1 \\
	\hline
	RF-DE\_63 & Il Cliente visualizza l'input testuale della descrizione. & UC\_23.2 \newline UC\_24.2 \\
	\hline
	RF-DE\_63 & Il Cliente deve inserire obbligatoriamente una descrizione. & UC\_23.2 \newline UC\_24.2\\
	\hline
	RF-DE\_63 & Il Cliente nella descrizione può inserire un massimo di 300 caratteri. & UC\_23.2 \newline UC\_24.2\\
	\hline
	RF-DE\_64 & Il Cliente deve vedere un contatore x/300 vicino al campo di testo & UC\_23.2 \newline UC\_24.2\\
	\hline
	RF-DE\_65 & Al superamento dei 300 caratteri il Clienite non potrà più scrivere nell'input & UC\_23.2 \newline UC\_24.2\\
	\hline
	RF-DE\_66 & Il Cliente visualizza in fondo alla pagina il pulsante \vr{Invia feedback} & UC\_23.3 \\
	\hline
	RF-DE\_67 & Il Cliente deve poter premere il pulsante \vr{Invia feedback} & UC\_23.3 \\
	\hline
	RF-DE\_68 & Il Cliente riceve la conferma dell'invio del feedback. & UC\_23.3 \\
	\hline
	X & L'Admin deve poter visualizzare la pagina web delle performance. & UC\_25 \\
	\hline
	X & L'Admin deve poter visualizzare il link per la pagina delle performance in qualsiasi pagina & UC\_25 \\
	\hline
	X & L'Admin deve poter schiacciare il link per la pagina delle performance in qualsiasi pagina & UC\_25 \\
	\hline
	X & L'Admin deve poter visualizzare il tempo medio di risposta del chatbot & UC\_25.1 \\
	\hline
	X & L'Admin deve poter visualizzare il tempo medio di permanenza nella web-app & UC\_25.2 \\
	\hline
	X & L'Admin deve poter visualizzare un bottone relativo all'esportazione del log & UC\_29 \\
	\hline
	X & L'Admin deve poter schiacciare il bottone relativo all'esportazione del log & UC\_29 \\
	\hline
	X & L'Admin deve poter ottenere il file di log in formato JSON & UC\_29 \\
	\hline
	X & L'Admin deve poter ottenere il file JSON dello storico ordini & UC\_30 \\
	\hline
	X & L'Admin deve poter visualizzare il tasto \vr{Esporta} nella pagina dello storico ordini & UC\_30 \\
	\hline
	X & L'Admin deve poter schiacciare il tasto \vr{Esporta} nella pagina dello storico ordini & UC\_30 \\
	\hline
	X & L'Admin deve poter visualizzare la conferma dell'esportazione dello storico ordini & UC\_30.1 \\
	\hline
	X & L'Admin deve poter visualizzare un errore nel caso in cui non fosse stato possibile scaricare nel proprio dispositivo il file JSON relativo allo storico ordini. & UC\_31 \\
	\hline
	X & Il file JSON esportato deve contenere il campo id [integer]: identificativo univoco dell'ordine & UC\_30 \\
	\hline
	X & Il file JSON esportato deve contenere il campo cod\_cli [integer]: identificativo univoco del cliente & UC\_30 \\
	\hline
	X & Il file JSON esportato deve contenere il campo cod\_art [varchar(13)]: identificativo univoco dell'articolo & UC\_30 \\
	\hline
	X & Il file JSON esportato deve contenere il campo data\_ord [date]: data di inserimento dell'ordine & UC\_30 \\
	\hline
	X & Il file JSON esportato deve contenere il campo qta\_ordinata [float]: quantità che l'Utente ha ordinato nell'unità di misura dell'articolo & UC\_30 \\
	\hline
	X & Il file JSON esportato deve contenere il campo rif [integer/string]: rappresenta un riferimento al testo originario & UC\_30 \\
	\hline
	X & L'Utente deve poter inserire più filtri nella stessa pagina di esportazione & UC\_36 \\
	\hline
	X & L'Admin deve poter visualizzare le statistiche & UC\_40 \\
	\hline
	X & L'Admin deve poter visualizzare il numero di utenti presenti in quel momento & UC\_40.1 \\
	\hline
	X & L'Admin deve poter visualizzare il numero di acquisti completati tramite la piattaforma & UC\_40.2 \\
	\hline
	X & L'Admin deve poter visualizzare, in forma di testo, il modello AI utilizzato in quel momento nella pagina delle performance & UC\_40.3 \\
	\hline
	X & L'Utente deve poter visualizzare l'elenco dei comandi in una pagina web dedicata & UC\_41 \\
	\hline
	X & L'Utente deve poter visualizzare il link per la pagina dell'elenco dei comandi in qualsiasi pagina & UC\_41 \\
	\hline
	X & L'Utente deve poter schiacciare il link per la pagina dell'elenco dei comandi in qualsiasi pagina & UC\_41 \\
	\hline
	X & Il Cliente deve poter visualizzare l'elenco dei comandi con la relativa spiegazione & UC\_41 \\
	\hline
	X & Il Cliente deve poter digitare \vr{/comandi} nella chat & UC\_42 \\
	\hline
	X & Il Cliente deve poter visualizzare il comando \vr{/duplica} con la relativa spiegazione & UC\_41.1 \newline UC\_42.1 \\
	\hline
	X & Il Cliente deve poter visualizzare il comando \vr{/carrello} con la relativa spiegazione & UC\_41.2 \newline UC\_42.2 \\
	\hline
	X & Il Cliente deve poter visualizzare il comando \vr{/duplica\{xx\}} con la relativa spiegazione & UC\_41.3 \newline UC\_42.3 \\
	\hline
	X & Il Cliente deve poter visualizzare il comando \vr{/invia} con la relativa spiegazione & UC\_41.4 \newline UC\_42.4 \\
	\hline
	X & Il Cliente deve poter visualizzare il comando \vr{/annulla} con la relativa spiegazione & UC\_41.5 \newline UC\_42.5 \\
	\hline
	X & Il Cliente deve poter visualizzare il comando \vr{/comandi} con la relativa spiegazione & UC\_41.6 \newline UC\_42.6 \\
	\hline
	X & Il Cliente deve poter visualizzare i comandi inline & UC\_43 \\
	\hline
	X & Il Cliente deve poter visualizzare il menù a comparsa dei comandi disponibili & UC\_43.1 \\
	\hline
	X & Il Cliente deve poter visualizzare il comando \vr{/duplica} nel menù a comparsa & UC\_43.1.1 \\
	\hline
	X & Il Cliente deve poter visualizzare il comando \vr{/carrello} nel menù a comparsa & UC\_43.1.2 \\
	\hline
	X & Il Cliente deve poter visualizzare il comando \vr{/duplica\{xx\}} nel menù a comparsa & UC\_43.1.3 \\
	\hline
	X & Il Cliente deve poter visualizzare il comando \vr{/invia} nel menù a comparsa & UC\_43.1.4 \\
	\hline
	X & Il Cliente deve poter visualizzare il comando \vr{/annulla} nel menù a comparsa & UC\_43.1.5 \\
	\hline
	X & Il Cliente deve poter visualizzare il comando \vr{/comandi} nel menù a comparsa & UC\_43.1.6 \\
	\hline
	X & Il Cliente deve poter inviare un comando nella chat & UC\_44 \\
	\hline
	X & Il Cliente deve poter inviare il comando \vr{/duplica} nella chat & UC\_44.1 \\
	\hline
	X & Il Cliente deve poter inviare il comando \vr{/carrello} nella chat & UC\_44.2 \\
	\hline
	X & Il Cliente deve poter inviare il comando \vr{/duplica\{xx\}} nella chat & UC\_44.3 \\
	\hline
	X & Il Cliente deve poter inviare il comando \vr{/invia} nella chat & UC\_44.4 \\
	\hline
	X & Il Cliente deve poter inviare il comando \vr{/annulla} nella chat & UC\_44.5 \\
	\hline
	X & Il Cliente deve poter inviare il comando \vr{/comandi} nella chat & UC\_44.6 \\
	\hline
	X & L'Utente deve poter impostare dei filtri & UC\_47 \newline UC\_48 \\
	\hline
	X & L'Utente deve poter filtrare per data gli ordini presenti nello storico ordini & UC\_47.1 \newline UC\_48.2 \\
	\hline
	X & L'Utente deve poter visualizzare l'icona del filtro data a forma di calendario & UC\_47.1 \newline UC\_48.2 \\
	\hline
	X & L'Utente deve poter cliccare sull’icona del filtro data & UC\_47.1 \newline UC\_48.2 \\
	\hline
	X & Il filtro per data deve poter permettere di visualizzare tutti gli ordini successivi o uguali ad una data scelta & UC\_47.1 \newline UC\_48.2 \\
	\hline
	X & Il filtro per data deve poter permettere di visualizzare tutti gli ordini compresi tra due date scelte & UC\_47.1 \newline UC\_48.2 \\
	\hline
	X & L’Utente deve poter vedere il campo testuale del filtro prodotti & UC\_48.1 \\
	\hline
	X & L’Utente deve poter scrivere nel campo testuale del filtro prodotti & UC\_48.1 \\
	\hline
	X & L’Utente deve poter vedere un riquadro di prodotti presenti che iniziano o sono uguali alla sequenza di caratteri inserita & UC\_47.2 \newline UC\_48.3 \\
	\hline
	X & L’Utente, digitando il nome del prodotto, deve poter vedere un riquadro dei prodotti presenti che contengono quella sequenza di caratteri & UC\_47.2 \newline UC\_48.3 \\
	\hline
	X & L’Admin deve poter vedere il campo testuale del filtro Cliente & UC\_48.1 \\
	\hline
	X & L’Admin deve poter scrivere nel campo testuale del filtro Cliente & UC\_48.1 \\
	\hline 
	X & L’Admin, digitando il codice del Cliente, deve poter vedere un riquadro di Clienti presenti che iniziano con quella sequenza di caratteri & UC\_48.1 \\
	\hline
	X & L’Admin, digitando il codice del Cliente, deve poter scegliere dal riquadro dei Clienti presenti L'Utente desiderato & UC\_48.1 \\
	\hline
	X & L’Admin, digitando lo username del Cliente, deve poter vedere un riquadro di Clienti presenti che contengono quella sequenza di caratteri & UC\_48.1 \\
	\hline
	X & L’Admin, digitando lo username del Cliente, deve poter scegliere dal riquadro dei Clienti presenti lo username del Cliente desiderato & UC\_48.1 \\
	\hline
	\multicolumn{3}{c}{} \\
	\caption{Requisiti Funzionali desiderabili}\label{tab:funzionali_desiderabili}\\
\end{longtable}

\subsubsection{Requisiti Funzionali opzionali}
\renewcommand{\arraystretch}{1.15}
\begin{longtable}{|>{\raggedright}p{0.15\textwidth}|>{\raggedright\arraybackslash}p{0.6\textwidth}|>{\raggedright\arraybackslash}p{0.25\textwidth}|}
	\hline
	\rowcolor[gray]{0.9}
	\textbf{Codice} & \textbf{Descrizione} & \textbf{Fonti} \\
	\hline
	\endfirsthead
	
	\hline
	\rowcolor[gray]{0.9}
	\textbf{Codice} & \textbf{Descrizione} & \textbf{Fonti}  \\
	\hline
	\endhead
	
	\hline
	X & Tutte le informazioni riguardanti l'Utente utili ai fini del logging devono essere anonimizzate & UC\_5 \\
	\hline
	X & L'Admin deve poter visualizzare creare nuovi utenti & UC\_32 \\
	\hline
	X & L'Admin deve poter schiacciare il bottone relativo alla creazione di un nuovo Utente & UC\_32 \\
	\hline
	X & L'Admin deve poter selezionare il ruolo di quest'ultimo & UC\_32.1 \\
	\hline
	X & L'Admin deve poter selezionare il ruolo Cliente & UC\_32.1 \\
	\hline
	X & L'Admin deve poter selezionare il ruolo Admin & UC\_32.1 \\
	\hline
	X & Deve essere visualizzata una notifica a schermo che informa della corretta creazione dell'Utente & UC\_32.2 \\
	\hline
	X & L'Admin deve poter visualizzare la notifica di errore creazione utente & UC\_33\\
	\hline
	\multicolumn{3}{c}{} \\
	\caption{Requisiti Funzionali opzionali}\label{tab:funzionali_opzionali}\\
\end{longtable}