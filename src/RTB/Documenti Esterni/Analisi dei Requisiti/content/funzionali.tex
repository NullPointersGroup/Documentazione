\subsection{Requisiti Funzionali}

\subsubsection{Requisiti Funzionali obbligatori}
\renewcommand{\arraystretch}{1.15}
\begin{longtable}{|>{\raggedright}p{0.15\textwidth}|>{\raggedright\arraybackslash}p{0.6\textwidth}|>{\raggedright\arraybackslash}p{0.25\textwidth}|}
	\hline
	\rowcolor[gray]{0.9}
	\textbf{Codice} & \textbf{Descrizione} & \textbf{Fonti} \\
	\hline
	\endfirsthead
	
	\hline
	\rowcolor[gray]{0.9}
	\textbf{Codice} & \textbf{Descrizione} & \textbf{Fonti}  \\
	\hline
	\endhead
	
	\hline
	RF-OB\_01 & L'Utente deve poter inserire lo username & UC\_02.1 \\
	\hline
	RF-OB\_02 & L'Utente deve poter inserire la password & UC\_02.2 \\
	\hline
	RF-OB\_03 & L'Utente deve poter vedere il messaggio \vr{Username o password errati} se l'autenticazione fallisce & UC\_03 \\
	\hline
	RF-OB\_04 & L'Utente può inserire input testuali con lunghezza massima di 4096 caratteri. & UC\_09\newline UC\_10 \\
	\hline
	RF-OB\_05 & L'Utente visualizza un messaggio di errore al superamento di 4096 caratteri. & UC\_09\newline UC\_10 \\
	\hline
	RF-OB\_06 & L'Utente deve poter visualizzare il tasto di invio & UC\_09\newline UC\_11\newline UC\_13\newline UC\_15\newline UC\_17 \\
	\hline
	RF-OB\_07 & L'Utente deve poter premere il tasto di invio & UC\_09\newline UC\_11\newline UC\_13\newline UC\_15\newline UC\_17 \\
	\hline
	RF-OB\_08 & L'Utente in assenza di contenuto di input non potrà premere il tasto di invio che sarà disabilitato & UC\_09\newline UC\_11\newline UC\_13\newline UC\_15\newline UC\_17 \\
	\hline
	RF-OB\_09 & L'Utente deve poter inserire, con qualsiasi forma di input, un massimo di 10 articoli distinti. & UC\_09\newline UC\_11\newline UC\_13\newline UC\_15\newline UC\_17 \\
	\hline
	RF-OB\_10 & L'Utente deve ricevere una richiesta di chiarimento da parte del chatbot nel caso in cui non capisse l'articolo di riferimento oppure non ci fossero articoli corrispondenti. & UC\_9\newline UC\_10\newline UC\_11\newline UC\_12\newline UC\_13 \\
	\hline
	RF-OB\_11 & L'Utente dopo l'invio dovrà visualizzare il campo di testo vuoto & UC\_09\newline UC\_11\newline UC\_13 \\
	\hline
	RF-OB\_12 & L'Utente deve poter vedere un pulsante con l'icona del microfono & UC\_13 \\
	\hline
	RF-OB\_13 & L'Utente deve poter cliccare il pulsante con l'icona del microfono per registrare messaggi audio & UC\_13\newline UC\_13.1 \\
	\hline
	RF-OB\_14 & L'Utente deve poter vedere la durata della registrazione mentre registra & UC\_13\newline UC\_13.1 \\
	\hline
	RF-OB\_15 & L'Utente premendo nuovamente l'icona del microfono ferma la registrazione & UC\_13\newline UC\_13.1 \\
	\hline
	RF-OB\_16 & L'Utente deve vedere un pulsante con icona \vr{clip} & UC\_11.2\newline UC\_13.2 \\
	\hline
	RF-OB\_17 & L'Utente deve poter premere il pulsante per selezionare il file da caricare dal dispositivo & UC\_11.2\newline UC\_13.2 \\
	\hline
	RF-OB\_18 & L'Utente cliccando sul pulsante apre il dialogo di selezione file del sistema operativo. & UC\_11.2\newline UC\_13.2 \\
	\hline
	RF-OB\_19 & L'Utente deve poter selezionare i file & UC\_11.2\newline UC\_13.2 \\
	\hline
	RF-OB\_20 & L'Utente deve poter vedere il tasto \vr{Seleziona} nella selezione file & UC\_11.2\newline UC\_13.2 \\
	\hline
	RF-OB\_21 & L'Utente deve poter inviare audio solo se <= 120 secondi. & UC\_13.1\newline UC\_13.2 \\
	\hline
	RF-OB\_22 & Se il messaggio vocale supera i 120 secondi compare un messaggio di errore & UC\_13.1\newline UC\_13.2 \\
	\hline
	RF-OB\_23 & L'Utente può inviare file audio con dimensione massima di 10MB in caso contrario ne sarà impedito l'invio & UC\_13.1\newline UC\_13.2 \\
	\hline
	RF-OB\_24 & Se L'Utente prova ad inviare audio di dimensioni superiori ai 10MB deve esserci la comparsa dell'errore con spiegazione & UC\_14 \\
	\hline
	RF-OB\_25 & Il dialogo di selezione file deve filtrare e mostrare solo file .mp3, .m4a, .m4p, .wav. & UC\_13.2.1 \newline UC\_13.2.2 \newline UC\_13.2.3 \newline UC\_13.2.4 \\
	\hline
	RF-OB\_26 & L'Utente deve vedere il pulsante di rimozione dell'audio durante la registrazione & UC\_13.1 \\
	\hline
	RF-OB\_27 & L'Utente deve poter premere il pulsante di rimozione dell'audio durante la registrazione per eliminarlo e dunque impedire l'invio. & UC\_13.1 \\
	\hline
	RF-OB\_28 & L'Utente può aggiungere articoli all'ordine tramite comandi testuali o vocali & UC\_15\newline UC\_9\newline UC\_13 \\
	\hline
	RF-OB\_29 & Il Cliente deve ricevere conferma dell'aggiunta dei prodotti & UC\_15\newline UC\_21.2 \\
	\hline
	RF-OB\_30 & L'Utente deve essere notificato nel caso in cui i prodotti di cui ha richiesto l'inserimento non siano stati trovati nel catalogo o se c'è stato un errore. & UC\_16 \\
	\hline
	RF-OB\_31 & L'Utente deve poter rimuovere articoli tramite comandi testuali o vocali & UC\_17 \\
	\hline
	RF-OB\_32 & L'Utente deve ricevere una conferma della rimozione. & UC\_17 \\
	\hline
	RF-OB\_33 & L'Utente deve essere avvisato nel caso in cui i prodotti dei quali ha richiesto la rimozione non fossero presenti nel carrello o se c'è stato un errore. & UC\_18 \\
	\hline
	RF-OB\_35 & Il Cliente deve poter inviare il comando testuale \verb|/carrello|. & UC\_19.1 \\
	\hline
	RF-OB\_34 & Il Cliente visualizza l'anteprima del carrello dopo aver inserito il comando testuale \verb|/carrello|. & UC\_19.2 \\
	\hline
	RF-OB\_36 & Il Cliente visualizza i prezzi di ogni articolo & UC\_19.2.2\newline UC\_26.2.1 \\
	\hline
	RF-OB\_37 & Il Cliente visualizza il nome di ogni articolo & UC\_19.2.1\newline UC\_26.2.2 \\
	\hline
	RF-OB\_38 & Il Cliente visualizza la quantità di ogni articolo & UC\_19.2.3\newline UC\_26.2.3 \\
	\hline
	RF-OB\_39 & Il Cliente se digita \verb|/carrello| riceve un avviso nel caso in cui nel carrello non ci fossero elementi. & UC\_20 \\
	\hline
	RF-OB\_40 & Il Cliente deve poter riformulare l'ordine per prodotti ambigui & UC\_21.1 \\
	\hline
	RF-OB\_43 & Il Cliente può rispondere \vr{annulla} per uscire dalla disambiguazione. & UC\_22.1 \\
	\hline
	RF-OB\_44 & Il Cliente riceve conferma dell'annullamento della disambiguazione & UC\_22.2 \\
	\hline
	RF-OB\_46 & Il Cliente deve vedere nella home il tasto \vr{Segnala un problema} & UC\_24 \\
	\hline
	RF-OB\_47 & Il Cliente cliccando sul pulsante \vr{Segnala un problema} apre in una nuova pagina un modulo dedicato & UC\_24 \\
	\hline
	RF-OB\_48 & Il Cliente deve visualizzare il dropdown: \vr{Bug}, \vr{Richiesta di supporto}, \vr{Suggerimento} & UC\_24.1 \\
	\hline
	RF-OB\_49 & Il Cliente può selezionare il tipo da dropdown & UC\_24.1 \\
	\hline
	RF-OB\_50 & Il Cliente deve inserire una descrizione in un campo di testo (es. massimo 300 caratteri). & UC\_24.2 \\
	\hline
	RF-OB\_51 & Il Cliente può visualizzare il tasto \vr{Invia segnalazione} & UC\_24.3 \\
	\hline
	RF-OB\_52 & Il Cliente può inviare il form cliccando \vr{Invia segnalazione} & UC\_24.3 \\
	\hline
	RF-OB\_53 & Il Cliente riceve conferma dell'invio con un messaggio \vr{Ticket creato con successo} & UC\_24.4 \\
	\hline
	RF-OB\_54 & Il Cliente può inserire il comando \verb|/invia| nella chat per avviare la conferma dell'ordine. & UC\_26.1 \\
	\hline
	RF-OB\_55 & Se il carrello è vuoto dopo il comando \verb|/invia| il Cliente visualizza un messaggio di errore che avvisa che il carrello è vuoto. & UC\_27 \\
	\hline
	RF-OB\_56 & Il Cliente deve inviare l'ordine digitando \verb|/invia| nella chat & UC\_26.1 \\
	\hline
	RF-OB\_57 & Il Cliente deve visualizzare il messaggio riepilogativo con i dettagli dell'ordine quando invia /invia nella chat & UC\_26.2 \\
	\hline
	RF-OB\_58 & Il Cliente a seguito del messaggio riepilogativo deve decidere se annullare o confermare l'invio dell'ordine & UC\_26 \\
	\hline
	RF-OB\_59 & Il Cliente deve poter confermare l'invio dell'ordine con un input & UC\_26.3\newline UC\_34 \\
	\hline
	RF-OB\_61 & Il Cliente deve visualizzare la conferma dell'invio dell'ordine & UC\_26.5\\
	\hline
	RF-OB\_61 & Il Cliente deve poter annullare l'invio dell'ordine con un input & UC\_26.5\\
	\hline
	RF-OB\_63 & Il Cliente a seguito della conferma visualizza un riepilogo con data ordine e un numero d'ordine univoco a livello di sistema & UC\_26\newline UC\_26.2\newline UC\_34 \\ %i sottocasi di 26.2 sono nel 19.qualcosa! (più su)
	\hline
	RF-OB\_65 & Il Cliente deve poter vedere il pulsante \vr{nuova chat} per iniziare una nuova sessione (sempre presente) & UC\_28 \\
	\hline
	RF-OB\_66 & L'Utente deve poter cliccare sul pulsante \vr{nuova chat} per iniziare una nuova sessione. & UC\_28 \\
	\hline
	RF-OB\_67 & Il Cliente richiede la duplicazione al chatbot indicando il codice dell'ordine desiderato preceduto dal comando "/duplica". & UC\_34 \\
	\hline
	RF-OB\_68 & Il Cliente deve poter richiedere al chatbot di duplicare l'ultimo ordine effettuato indicando il comando "/duplica" & UC\_34 \\
	\hline
	RF-OB\_69 & Il chatbot deve inizializzare la procedura di duplicazione dell'ordine. & UC\_34 \\
	\hline
	RF-OB\_70 & Il chatbot, a seguito della richiesta di duplicazione di un ordine da parte di un Cliente con ordine da duplicare non esistente o non disponibile, deve informare il Cliente che non è possibile procedere con la richiesta. & UC\_34 \\
	\hline
	RF-OB\_71 & Il chatbot manda il dettaglio riepilogativo dell'ordine desiderato da duplicare e chiede conferma & UC\_34.1 \\
	\hline
	RF-OB\_72 & Il chatbot mostra al Cliente una notifica di conferma della duplicazione. & UC\_34.2 \\
	\hline
	RF-OB\_73 & L'Utente deve poter visualizzare un'icona \vr{visualizza storico ordini} & UC\_36 \\
	\hline
	RF-OB\_74 & L'Utente deve poter schiacciare l'icona \vr{visualizza storico ordini} & UC\_36 \\
	\hline
	RF-OB\_75 & L'Utente deve poter visualizzare una pagina separata contenente lo storico totale dei suoi ordini & UC\_36 \\
	\hline
	RF-OB\_76 & La lista completa degli ordini deve essere visualizzata in pagine contenenti al massimo 10 ordini ciascuna & UC\_36 \\
	\hline
	RF-OB\_77 & Il caricamento di una pagina non implica il caricamento delle pagine successive & UC\_36 \\
	\hline
	RF-OB\_78 & L'Utente può visualizzare il dettaglio di un ordine specifico & UC\_37 \\
	\hline
	RF-OB\_79 & L'Utente può visualizzare chi è colui che acquista & UC\_37.1 \\
	\hline
	RF-OB\_80 & L'Utente può visualizzare i prodotti dell'ordine & UC\_37.2 \\
	\hline
	RF-OB\_81 & L'Utente può visualizzare il nome dei prodotti & UC\_37.2.1 \\
	\hline
	RF-OB\_82 & L'Utente può visualizzare la descrizione dei prodotti & UC\_37.2.2 \\
	\hline
	RF-OB\_83 & L'Utente può visualizzare la quantità dei prodotti & UC\_37.2.3 \\
	\hline
	RF-OB\_84 & L'Utente può visualizzare la data dell'ordine & UC\_37.3 \\
	\hline
	RF-OB\_85 & L'Utente deve poter cliccare sul pulsante per duplicare l'ordine nella schermata dei dettagli ordine & UC\_38 \\
	\hline
	RF-OB\_86 & L'Utente deve poter confermare la duplicazione dell'ordine & UC\_38.1 \\
	\hline
	RF-OB\_87 & L'Utente deve poter annullare la duplicazione dell'ordine. & UC\_39 \\
	\hline
	RF-OB\_88 & L'Utente deve vedere, tramite un pop-up, la conferma della duplicazione dell'ordine & UC\_38.1 \\
	\hline
	RF-OB\_89 & L'Utente deve vedere, tramite un pop-up, l'annullamento della duplicazione dell'ordine & UC\_39 \\
	\hline
	RF-OB\_90 & L'Utente deve poter digitare il comando /comandi nella chat & UC\_44 \\
	\hline
	RF-OB\_91 & L'Utente deve poter visualizzare l'elenco dei comandi nella chat con la relativa spiegazione dopo aver digitato /comandi & UC\_44 \\
	\hline
	RF-OB\_92 & Deve comparire un menù a comparsa sopra la chat con l'elenco dei comandi disponibili dopo aver digitato / & UC\_45 \\
	\hline
	\multicolumn{3}{c}{} \\
	\caption{Requisiti Funzionali obbligatori}\label{tab:funzionali_obbligatori}\\
\end{longtable}
	
\subsubsection{Requisiti Funzionali desiderabili}
\renewcommand{\arraystretch}{1.15}
\begin{longtable}{|>{\raggedright}p{0.18\textwidth}|>{\raggedright\arraybackslash}p{0.6\textwidth}|>{\raggedright\arraybackslash}p{0.25\textwidth}|}
	\hline
	\rowcolor[gray]{0.9}
	\textbf{Codice} & \textbf{Descrizione} & \textbf{Fonti} \\
	\hline
	\endfirsthead
	
	\hline
	\rowcolor[gray]{0.9}
	\textbf{Codice} & \textbf{Descrizione} & \textbf{Fonti}  \\
	\hline
	\endhead
	
	\hline
	RF-DE\_01 & Lo username deve essere univoco & UC\_01.1\newline UC\_01.1.1\newline UC\_32.3 \\
	\hline
	RF-DE\_02 & L'email deve essere univoca & UC\_01.4\newline UC\_01.4.2\newline UC\_32.4 \\
	\hline
	RF-DE\_03 & L'Utente deve poter inserire lo username & UC\_01.1\newline UC\_32.3 \\
	\hline
	RF-DE\_04 & L'Utente visualizza una notifica di errore se l'username è già presente nel sistema & UC\_01.1.1 \\
	\hline
	RF-DE\_05 & Dev'essere data la possibilità di tornare alla login via link HTML. & UC\_01.1.1\newline UC\_01.4.1 \\
	\hline
	RF-DE\_06 & L'Utente deve poter inserire la password & UC\_01.2\newline UC\_01.2.1 \\
	\hline
	RF-DE\_07 & La password deve avere almeno 1 lettera maiuscola, 1 lettera minuscola, 1 numero, 1 carattere speciale e almeno 8 caratteri & UC\_01.2\newline UC\_01.2.1\newline UC\_08 \\
	\hline
	RF-DE\_08 & Lo username deve avere un massimo di 24 caratteri. & UC\_01.2\newline UC\_01.1.1\newline UC\_08\newline UC\_32.3 \\
	\hline
	RF-DE\_09 & La password deve avere un massimo di 24 caratteri & UC\_01.2\newline UC\_01.2.1\newline UC\_08 \\
	\hline
	RF-DE\_10 & L'Utente visualizza una notifica di errore se la password non è conforme ai criteri & UC\_01.2.1 \\
	\hline
	RF-DE\_11 & Possibilità conferma password tramite nuovo campo di input & UC\_01.3 \\
	\hline
	RF-DE\_12 & La password e la sua conferma devono coincidere & UC\_01.2\newline UC\_01.3.1\newline UC\_01.3 \\
	\hline
	RF-DE\_13 & L'Utente visualizza una notifica di errore se le password non coincidono & UC\_01.3.1 \\
	\hline
	RF-DE\_14 & Possibilità inserimento email & UC\_01.4 \\
	\hline
	RF-DE\_15 & L'Utente visualizza l'errore nel caso la mail non è nel formato corretto (@ non presente per esempio) & UC\_01.4.1 \\
	\hline
	RF-DE\_16 & L'Utente visualizza l'errore nel caso la mail sia già presente nel sistema & UC\_01.4.2 \\
	\hline
	RF-DE\_17 & L'Utente deve essere indirizzato alla pagina login una volta che la registrazione ha avuto successo. & UC\_01 \\
	\hline
	RF-DE\_18 & L'Utente deve poter vedere dove ha sbagliato: le password non coincidono & UC\_01 \\
	\hline
	RF-DE\_19 & L'Utente deve poter vedere dove ha sbagliato: il nome utente esiste già. & UC\_01 \\
	\hline
	RF-DE\_20 & Dev'essere data la possibilità di tornare alla registrazione via link. & UC\_02 \\
	\hline
	RF-DE\_21 & L'Utente deve poter vedere il link HTML per la password dimenticata & UC\_04 \\
	\hline
	RF-DE\_22 & L'Utente deve poter cliccare il link HTML per la password dimenticata & UC\_04 \\
	\hline
	RF-DE\_23 & L'Utente deve ricevere via mail la password generata automaticamente & UC\_04.1 \\
	\hline
	RF-DE\_24 & L'Utente deve poter vedere il form per inserire l'indirizzo mail che ha usato quando si è registrato & UC\_04.1 \\
	\hline
	RF-DE\_25 & Nella home è presente un'icona che porta sulla pagina delle info del profilo & UC\_07 \\
	\hline
	RF-DE\_26 & L'Utente deve poter vedere il bottone \vr{Elimina Account} nella pagina delle informazioni del profilo & UC\_05 \\
	\hline
	RF-DE\_27 & L'Utente deve poter eliminare il suo account & UC\_05 \\
	\hline
	RF-DE\_28 & Deve poter avvenire un logout automatico a seguito della cancellazione account & UC\_05\newline UC\_06 \\
	\hline
	RF-DE\_29 & L'Utente deve poter vedere il pulsante \vr{Logout} nella pagina del profilo & UC\_06\newline UC\_07 \\
	\hline
	RF-DE\_30 & L'Utente deve poter effettuare il logout tramite il pulsante \vr{Logout} & UC\_06 \\
	\hline
	RF-DE\_31 & L'Utente deve poter cliccare sul link, rappresentato da un'icona, che porta sulla pagina delle info del proprio profilo & UC\_07 \\
	\hline
	RF-DE\_32 & L'Utente deve poter vedere il link alla pagina dello storico ordini nelle info del proprio profilo & UC\_07 \\
	\hline
	RF-DE\_33 & L'Utente deve poter vedere il proprio username nella pagina delle info del proprio profilo & UC\_07.1 \\
	\hline
	RF-DE\_34 & L'Utente deve poter vedere la propria ragione sociale nelle info del proprio profilo & UC\_07.2 \\
	\hline
	RF-DE\_35 & L'Utente deve poter vedere il bottone \vr{Reimposta password} & UC\_08 \\
	\hline
	RF-DE\_36 & L'Utente deve poter cliccare il bottone \vr{Reimposta password} & UC\_08 \\
	\hline
	RF-DE\_37 & L'Utente deve poter vedere i nuovi campi di input \vr{Vecchia password}, \vr{Nuova password} e \vr{Conferma nuova password} & UC\_08 \\
	\hline
	RF-DE\_38 & L'Utente deve poter vedere il pulsante "Cambia password" & UC\_08 \\
	\hline
	RF-DE\_39 & L'Utente deve poter reimpostare la propria password, a patto che conosca quella precedente. & UC\_08 \\
	\hline
	RF-DE\_40 & L'Utente deve poter essere notificato qualora il cambio password vada a buon fine tramite un messaggio di conferma. & UC\_08 \\
	\hline
	RF-DE\_41 & Durante la digitazione L'Utente può visualizzare un contatore x/4096 vicino al campo di testo & UC\_09 \\
	\hline
	RF-DE\_42 & L'Utente può vedere il pulsante di caricamento file immagine (icona fotocamera). & UC\_11 \\
	\hline
	RF-DE\_43 & L'Utente deve poter cliccare l'icona della fotocamera & UC\_11 \\
	\hline
	RF-DE\_44 & L'Utente a seguito del click sull'icona a forma di clip deve poter vedere un dropdown tra cui: "Seleziona dal dispositivo" oppure "Scatta la foto". & UC\_11\newline UC\_11.1\newline UC\_11.2 \\
	\hline
	RF-DE\_45 & L'Utente deve poter selezionare \vr{Seleziona dal dispositivo} & UC\_11\newline UC\_11.2 \\
	\hline
	RF-DE\_46 & Se L'Utente seleziona \vr{Seleziona dal dispositivo} deve poter visualizzare i media nel dispositivo & UC\_11\newline UC\_11.2 \\
	\hline
	RF-DE\_47 & L'Utente deve poter selezionare \vr{Scatta foto}. & UC\_11\newline UC\_11.1 \\
	\hline
	RF-DE\_48 & Alla selezione \vr{Scatta foto} L'Utente visualizza l'attivazione della fotocamera. & UC\_11\newline UC\_11.1 \\
	\hline
	RF-DE\_49 & L'Utente può inviare immagini con dimensione massima di 15MB e con risoluzione pari o superiore a 800x600 pixel. & UC\_11 \\
	\hline
	RF-DE\_50 & L'Utente deve poter visualizzare che le immagini non devono essere superiori ai 15MB oppure inferiori ad 800x600 pixel. & UC\_11 \\
	\hline
	RF-DE\_51 & Nel caso in cui le immagini superino i 15MB abbiano risoluzione inferiore di 800x600 pixel sarà visibile al Cliente un messaggio di errore \vr{L'immagine non rispetta i requisiti richiesti} & UC\_11\newline UC\_12 \\
	\hline
	RF-DE\_52 & L'Utente, prima dell'invio dell'immagine deve premere esplicitamente il pulsante di invio per confermare l'invio. & UC\_11 \\
	\hline
	RF-DE\_53 & L'Utente successivamente all'invio visualizza che l'immagine è in elaborazione & UC\_11 \\
	\hline
	RF-DE\_54 & L'Utente visualizza un messaggio \vr{Non sono riuscito a rilevare il testo nell'immagine} se il modello non riesce a rilevare testo & UC\_12 \\
	\hline
	RF-DE\_55 & L'Utente può inviare immagini nei formati .jpg .jpeg e .png & UC\_11\newline UC\_11.2.1 \newline UC\_11.2.2 \newline UC\_11.2.3 \\
	\hline
	RF-DE\_56 & L'Utente può visionare l'anteprima dell'immagine caricata. & UC\_11 \\
	\hline
	RF-DE\_57 & L'Utente cliccando sull'immagine caricata può visionarla a schermo intero. & UC\_11 \\
	\hline
	RF-DE\_58 & Il Cliente può visualizzare il pulsante a fianco all'ultima risposta \vr{lascia Feedback}. & UC\_23 \\
	\hline
	RF-DE\_59 & Il Cliente può premere il pulsante affianco ad ogni risposta \vr{lascia Feedback} per fornire feedback all'AI & UC\_23 \\
	\hline
	RF-DE\_60 & Il Cliente premendo su \vr{lascia Feedback} deve visualizzare l'apertura di un form in una nuova pagina & UC\_23 \\
	\hline
	RF-DE\_61 & Il Cliente deve vedere il dropdown: \vr{Prodotto sbagliato}, \vr{Quantità errata}, \vr{Incomprensione}, \vr{Altro}. & UC\_23.1 \\
	\hline
	RF-DE\_62 & Il Cliente deve selezionare obbligatoriamente una categoria: \vr{Prodotto sbagliato}, \vr{Quantità errata}, \vr{Incomprensione}, \vr{Altro}. & UC\_23.1 \\
	\hline
	RF-DE\_63 & Il Cliente deve inserire obbligatoriamente una descrizione tramite input testuale con un massimo di 300 caratteri. & UC\_23.2 \\
	\hline
	RF-DE\_64 & Il Cliente deve vedere un contatore x/300 vicino al campo di testo & UC\_23.2 \\
	\hline
	RF-DE\_65 & Al superamento dei 300 caratteri il Clienite non potrà più scrivere nell'input & UC\_23.2 \\
	\hline
	RF-DE\_66 & Il Cliente visualizza in fondo alla pagina il pulsante \vr{Invia feedback} & UC\_23.3 \\
	\hline
	RF-DE\_67 & Il Cliente deve poter premere il pulsante \vr{Invia feedback} & UC\_23.3 \\
	\hline
	RF-DE\_68 & Il Cliente riceve la conferma dell'invio del feedback. & UC\_23.3 \\
	\hline
	RF-DE\_69 & L'Admin deve visualizzare la pagina web delle performance. & UC\_25 \\
	\hline
	RF-DE\_70 & L'Admin può visualizzare le metriche per le performance (tempo risposta, durata media sessione). & UC\_25\\
	\hline
	RF-DE\_72 & L'Admin può visualizzare il tempo medio di risposta del chatbot & UC\_25.1 \\
	\hline
	RF-DE\_73 & L'Admin può visualizzare il tempo medio di permanenza nella web-app & UC\_25.2 \\
	\hline
	RF-DE\_75 & L'Admin deve poter visualizzare un bottone relativo all'esportazione del log & UC\_29 \\
	\hline
	RF-DE\_76 & L'Admin deve poter schiacciare il bottone relativo all'esportazione del log & UC\_29 \\
	\hline
	RF-DE\_77 & L'Admin deve poter ottenere il file di log in formato JSON dell'intero lasso di tempo & UC\_29 \\
	\hline
	RF-DE\_78 & L'Admin deve poter ottenere il file di log in formato JSON del lasso di tempo specificato & UC\_29.1 \\
	\hline
	RF-DE\_79 & L'Admin deve poter scaricare il file di log in formato JSON nel proprio dispositivo & UC\_29\newline UC\_29.1 \\
	\hline
	RF-DE\_80 & L'Admin deve poter esportare il registro ordini filtrato per cliente. & UC\_29.2 \\
	\hline
	RF-DE\_81 & L'Admin deve poter ottenere il file JSON del registro ordini & UC\_30 \\
	\hline
	RF-DE\_82 & L'Admin deve poter visualizzare il tasto \vr{Esporta} nella pagina dello storico ordini & UC\_30 \\
	\hline
	RF-DE\_83 & L'Admin deve poter schiacciare il tasto \vr{Esporta} nella pagina dello storico ordini & UC\_30 \\
	\hline
	RF-DE\_84 & L'Admin deve vedere la conferma dell'esportazione dello storico ordini & UC\_30.1 \\
	\hline
	RF-DE\_87 & L'Admin deve poter visualizzare un errore nel caso in cui non fosse stato possibile scaricare nel proprio dispositivo il file JSON relativo allo storico ordini. & UC\_31 \\
	\hline
	RF-DE\_88 & Il file JSON esportato deve contenere il campo id [integer]: identificativo univoco dell'ordine & UC\_30 \\
	\hline
	RF-DE\_89 & Il file JSON esportato deve contenere il campo cod\_cli [integer]: identificativo univoco del cliente & UC\_30 \\
	\hline
	RF-DE\_90 & Il file JSON esportato deve contenere il campo cod\_art [varchar(13)]: identificativo univoco dell'articolo & UC\_30 \\
	\hline
	RF-DE\_91 & Il file JSON esportato deve contenere il campo data\_ord [date]: data di inserimento dell'ordine & UC\_30 \\
	\hline
	RF-DE\_92 & Il file JSON esportato deve contenere il campo qta\_ordinata [float]: quantità che l'Utente ha ordinato nell'unità di misura dell'articolo & UC\_30 \\
	\hline
	RF-DE\_93 & Il file JSON esportato deve contenere il campo rif [integer/string]: rappresenta un riferimento al testo originario & UC\_30 \\
	\hline
	RF-DE\_94 & L'Utente deve poter filtrare per data gli ordini presenti nello Storico Ordini & UC\_36.1.2 \\
	\hline
	RF-DE\_95 & L'Utente deve poter visualizzare l'icona del filtro data "a forma di calendario" & UC\_36.1.2 \\
	\hline
	RF-DE\_96 & L'Utente deve poter cliccare sull'icona del filtro data & UC\_36.1.2 \\
	\hline
	RF-DE\_97 & Il filtro per data deve poter permettere di visualizzare tutti gli ordini antecedenti o uguali ad una data scelta & UC\_29.1\newline UC\_25.1\newline UC\_25.2\newline UC\_36.1.2 \\
	\hline
	RF-DE\_98 & Il filtro per data deve poter permettere di visualizzare tutti gli ordini successivi o uguali ad una data scelta & UC\_29.1\newline UC\_25.1\newline UC\_25.2\newline UC\_36.1.2 \\
	\hline
	RF-DE\_99 & Il filtro per data deve poter permettere di visualizzare tutti gli ordini compresi tra due date scelte & UC\_29.1\newline UC\_25.1\newline UC\_25.2\newline UC\_36.1.2 \\
	\hline
	RF-DE\_100 & L'Admin deve poter filtrare per Cliente gli ordini presenti nello Storico Ordini & UC\_36.1.1 \\
	\hline
	RF-DE\_101 & L'Admin deve poter visualizzare gli ordini di un Cliente specifico & UC\_29.2\newline UC\_36.1.1 \\
	\hline
	RF-DE\_102 & L'Admin deve poter vedere il campo testuale del filtro Cliente (sia nome che codice) & UC\_29.2\newline UC\_36.1.1 \\
	\hline
	RF-DE\_103 & L'Admin deve poter scrivere nel campo testuale del filtro Cliente & UC\_29.2\newline UC\_36.1.1 \\
	\hline
	RF-DE\_104 & L'Admin, digitando il codice del Cliente, deve poter vedere un riquadro di Clienti presenti che iniziano con quella sequenza di caratteri & UC\_29.2\newline UC\_36.1.1 \\
	\hline
	RF-DE\_105 & L'Admin, digitando il codice del Cliente, deve poter scegliere dal riquadro dei Clienti presenti L'Utente desiderato & UC\_29.2\newline UC\_36.1.1 \\
	\hline
	RF-DE\_106 & L'Admin, digitando lo username del Cliente, deve poter vedere un riquadro di Clienti presenti che contengono quella sequenza di caratteri & UC\_29.2\newline UC\_36.1.1 \\
	\hline
	RF-DE\_107 & L'Admin, digitando lo username del Cliente, deve poter scegliere dal riquadro dei Clienti presenti lo username del Cliente desiderato & UC\_29.2\newline UC\_36.1.1 \\
	\hline
	RF-DE\_108 & L'Utente, digitando il codice del prodotto, deve poter vedere un riquadro di prodotti presenti che iniziano o sono uguali a quella sequenza di caratteri &  UC\_36.1.3 \\
	\hline
	RF-DE\_109 & L'Utente, digitando il nome del prodotto, deve poter vedere un riquadro dei prodotti presenti che contengono quella sequenza di caratteri & UC\_36.1.3 \\
	\hline
	RF-DE\_110 & L'Utente deve poter inserire più filtri nella stessa pagina di esportazione & UC\_36 \\
	\hline
	RF-DE\_111 & L'Admin deve poter visualizzare le statistiche nella pagina web delle performance & UC\_25\newline UC\_40 \\
	\hline
	RF-DE\_112 & L'Admin deve poter visualizzare il link per la pagina delle performance in qualsiasi pagina & UC\_25 \\
	\hline
	RF-DE\_113 & L'Admin deve poter schiacciare il link per la pagina delle performance in qualsiasi pagina & UC\_25 \\
	\hline
	RF-DE\_114 & L'Admin deve poter visualizzare il numero di utenti presenti in quel momento & UC\_40.1 \\
	\hline
	RF-DE\_115 & L'Admin deve poter visualizzare il numero di acquisti completati tramite la piattaforma & UC\_40.2 \\
	\hline
	RF-DE\_116 & L'Admin deve poter visualizzare, in forma di testo, il modello AI utilizzato in quel momento nella pagina delle performance & UC\_40.3 \\
	\hline
	RF-DE\_117 & L'Utente deve poter visualizzare il Manuale Utente in una pagina web dedicata & UC\_41 \\
	\hline
	RF-DE\_118 & L'Utente deve poter visualizzare il link per la pagina del Manuale Utente in qualsiasi pagina & UC\_41 \\
	\hline
	RF-DE\_119 & L'Utente deve poter schiacciare il link per la pagina del Manuale Utente in qualsiasi pagina & UC\_41 \\
	\hline
	RF-DE\_120 & L'Utente deve poter visualizzare le FAQ in una pagina web dedicata & UC\_42 \\
	\hline
	RF-DE\_121 & L'Utente deve poter visualizzare il link per la pagina delle FAQ nella home & UC\_42 \\
	\hline
	RF-DE\_122 & L'Utente deve poter schiacciare il link per la pagina delle FAQ nella home & UC\_42 \\
	\hline
	RF-DE\_123 & L'Utente deve poter visualizzare l'elenco dei comandi in una pagina web dedicata & UC\_43 \\
	\hline
	RF-DE\_124 & L'Utente deve poter visualizzare il link per la pagina dell'elenco dei comandi in qualsiasi pagina & UC\_43 \\
	\hline
	RF-DE\_125 & L'Utente deve poter schiacciare il link per la pagina dell'elenco dei comandi in qualsiasi pagina & UC\_43 \\
	\hline
	RF-DE\_125 & L'Utente deve poter schiacciare il link per la pagina dell'elenco dei comandi in qualsiasi pagina & UC\_47.2, \newline UC\_25.1, \newline UC\_25.2\\
	\hline
	\multicolumn{3}{c}{} \\
	\caption{Requisiti Funzionali desiderabili}\label{tab:funzionali_desiderabili}\\
\end{longtable}
	
\subsubsection{Requisiti Funzionali opzionali}
\renewcommand{\arraystretch}{1.15}
\begin{longtable}{|>{\raggedright}p{0.15\textwidth}|>{\raggedright\arraybackslash}p{0.6\textwidth}|>{\raggedright\arraybackslash}p{0.25\textwidth}|}
	\hline
	\rowcolor[gray]{0.9}
	\textbf{Codice} & \textbf{Descrizione} & \textbf{Fonti} \\
	\hline
	\endfirsthead
	
	\hline
	\rowcolor[gray]{0.9}
	\textbf{Codice} & \textbf{Descrizione} & \textbf{Fonti}  \\
	\hline
	\endhead
	
	\hline
	RF-OP\_01 & Tutte le informazioni riguardanti l'Utente utili ai fini del loggimng devono essere anonimizzate & UC\_5 \\
	\hline
	RF-OP\_02 & L'Admin deve poter visualizzare la pagina \vr{Gestione} & UC\_32 \\
	\hline
	RF-OP\_03 & L'Admin deve poter schiacciare il bottone relativo alla creazione di un nuovo Utente & UC\_32 \\
	\hline
	RF-OP\_04 & L'Admin deve poter visualizzare un form relativo alla creazione di un nuovo Utente & UC\_32.1 \\
	\hline
	RF-OP\_05 & L'Admin, all'interno del form relativo alla creazione di un nuovo Utente, deve poter selezionare il ruolo di quest'ultimo & UC\_32.2 \\
	\hline
	RF-OP\_06 & L'Admin può selezionare il ruolo Cliente & UC\_32.2 \\
	\hline
	RF-OP\_07 & L'Admin può selezionare il ruolo Admin & UC\_32.2 \\
	\hline
	RF-OP\_08 & L'Admin, all'interno del form relativo alla creazione di un nuovo Utente, deve poter scrivere lo username di quest'ultimo & UC\_32.3 \\
	\hline
	RF-OP\_10 & L'Admin deve poter visualizzare una tabella di tutti i Clienti nella pagina "Gestione" & UC\_32.1 \\
	\hline
	RF-OP\_11 & Deve essere visualizzata una notifica a schermo che informa della corretta creazione dell'Utente & UC\_33\newline UC\_32.5 \\
	\hline
	RF-OP\_12 & Successivamente alla mancata creazione di un Utente da parte di un Admin deve essere visualizzata una notifica a schermo che informa sull'errore che ha impedito la creazione dell'Utente & UC\_33\newline UC\_33 \\
	\hline
	\multicolumn{3}{c}{} \\
	\caption{Requisiti Funzionali opzionali}\label{tab:funzionali_opzionali}\\
\end{longtable}