\subsection{Requisiti Funzionali}

\subsubsection{Requisiti Funzionali obbligatori}
\renewcommand{\arraystretch}{1.15}
\begin{longtable}{|>{\raggedright}p{0.18\textwidth}|>{\raggedright\arraybackslash}p{0.5\textwidth}|>{\raggedright\arraybackslash}p{0.2\textwidth}|}

	\hline
	\rowcolor[gray]{0.9}
	\textbf{Codice} & \textbf{Descrizione} & \textbf{Fonti} \\
	\hline
	\endfirsthead
	
	\hline
	\rowcolor[gray]{0.9}
	\textbf{Codice} & \textbf{Descrizione} & \textbf{Fonti}  \\
	\hline
	\endhead

	\hline
	RF-OB\_01 \hypertarget{rf-ob_1} & L'Utente deve poter visualizzare il campo di input per inserire lo username & \hyperref[uc_01.1]{UC\_01.1} \\
	\hline
	RF-OB\_02 \hypertarget{rf-ob_2} & L'Utente deve poter inserire lo username & \hyperref[uc_01.1]{UC\_01.1} \\
	\hline
	RF-OB\_03 \hypertarget{rf-ob_3} & Lo username deve essere univoco & \hyperref[uc_01.1.2]{UC\_01.1.2} \\
	\hline
	RF-OB\_04 \hypertarget{rf-ob_4} & L'Utente deve poter visualizzare una notifica di errore se lo username è già presente nel sistema & \hyperref[uc_01.1.2]{UC\_01.1.2} \\
	\hline
	RF-OB\_05 \hypertarget{rf-ob_5} & Lo username deve avere un massimo di 24 caratteri & \hyperref[uc_01.1.1]{UC\_01.1.1}\newline \hyperref[uc_08]{UC\_08} \newline \hyperref[uc_32.2.1]{UC\_32.2.1}\\
	\hline
	RF-OB\_06 \hypertarget{rf-ob_6} & L'Utente deve poter visualizzare una notifica di errore se lo username eccede il numero massimo di caratteri & \hyperref[uc_01.1.1]{UC\_01.1.1} \newline \hyperref[uc_32.2.1]{UC\_32.2.1}\\
	\hline
	RF-OB\_07 \hypertarget{rf-ob_7} & L'Utente deve poter visualizzare il campo di input per inserire la password & \hyperref[uc_01.2]{UC\_01.2} \\
	\hline
	RF-OB\_08 \hypertarget{rf-ob_8} & L'Utente deve poter inserire la password & \hyperref[uc_01.2]{UC\_01.2}\\
	\hline
	RF-OB\_09 \hypertarget{rf-ob_9} & La password deve avere almeno 1 lettera maiuscola, 1 lettera minuscola, 1 numero, 1 carattere speciale e almeno 8 caratteri & \hyperref[uc_01.2]{UC\_01.2}\newline \hyperref[uc_01.2.1]{UC\_01.2.1}\newline \hyperref[uc_08]{UC\_08} \\
	\hline
	RF-OB\_10 \hypertarget{rf-ob_10} & La password deve avere un massimo di 24 caratteri & \hyperref[uc_01.2.1]{UC\_01.2.1} \\
	\hline
	RF-OB\_11 \hypertarget{rf-ob_11} & L'Utente deve poter visualizzare una notifica di errore se la password non è conforme ai criteri & \hyperref[uc_01.2.1]{UC\_01.2.1} \\
	\hline
	RF-OB\_12 \hypertarget{rf-ob_12} & L'Utente deve poter visualizzare una notifica di errore se la password supera 24 caratteri & \hyperref[uc_01.2.1]{UC\_01.2.1} \\
	\hline
	RF-OB\_13 \hypertarget{rf-ob_13} & L'Utente deve poter tornare alla login via link HTML & \hyperref[uc_01.1.2]{UC\_01.1.2}\newline \hyperref[uc_01.4.2]{UC\_01.4.2} \\
	\hline
	RF-OB\_14 \hypertarget{rf-ob_14} & L'Utente deve poter visualizzare il campo di input della nuova password & \hyperref[uc_01.3]{UC\_01.3} \\
	\hline
	RF-OB\_15 \hypertarget{rf-ob_15} & L'Utente deve poter confermare la password tramite nuovo campo di input & \hyperref[uc_01.3]{UC\_01.3} \\
	\hline
	RF-OB\_16 \hypertarget{rf-ob_16} & L'Utente deve poter visualizzare una notifica di errore se la password e la sua conferma non coincidono & \hyperref[uc_01.3.1]{UC\_01.3.1} \\
	\hline
	RF-OB\_17 \hypertarget{rf-ob_17} & L'Utente deve poter visualizzare il campo di input per l'inserimento dell'email & \hyperref[uc_01.4]{UC\_01.4} \\
	\hline
	RF-OB\_18 \hypertarget{rf-ob_18} & L'Utente deve poter inserire l'email & \hyperref[uc_01.4]{UC\_01.4} \\
	\hline
	RF-OB\_19 \hypertarget{rf-ob_19} & L'email deve essere univoca & \hyperref[uc_01.4.2]{UC\_01.4.2} \newline \hyperref[uc_32.3.2]{UC\_32.3.2} \\
	\hline
	RF-OB\_20 \hypertarget{rf-ob_20} & L'Utente deve poter visualizzare l'errore nel caso in cui l'email non fosse nel formato corretto & \hyperref[uc_01.4.1]{UC\_01.4.1} \\
	\hline
	RF-OB\_21 \hypertarget{rf-ob_21} & L'Utente deve poter visualizzare l'errore nel caso in cui l'email sia già presente nel sistema & \hyperref[uc_01.4.2]{UC\_01.4.2} \\
	\hline
	RF-OB\_22 \hypertarget{rf-ob_22} & L'Utente deve essere indirizzato alla pagina login una volta che la registrazione ha avuto successo & \hyperref[uc_01]{UC\_01} \\
	\hline
	RF-OB\_23 \hypertarget{rf-ob_23} & L'Utente deve poter visualizzare il campo per inserire lo username & \hyperref[uc_02.1]{UC\_02.1} \\
	\hline
	RF-OB\_24 \hypertarget{rf-ob_24} & L'Utente deve poter inserire lo username & \hyperref[uc_02.1]{UC\_02.1} \\
	\hline
	RF-OB\_25 \hypertarget{rf-ob_25} & L'Utente deve poter visualizzare il campo per inserire la password & \hyperref[uc_02.2]{UC\_02.2} \\
	\hline
	RF-OB\_26 \hypertarget{rf-ob_26} & L'Utente deve poter inserire la password & \hyperref[uc_02.2]{UC\_02.2} \\
	\hline
	RF-OB\_27 \hypertarget{rf-ob_27} & Dev'essere data la possibilità di tornare alla registrazione via link & \hyperref[uc_02]{UC\_02} \\
	\hline
	RF-OB\_28 \hypertarget{rf-ob_28} & L'Utente deve poter visualizzare il messaggio \vr{Username o password errati} se l'autenticazione fallisce & \hyperref[uc_03]{UC\_03} \\
	\hline
	RF-OB\_29 \hypertarget{rf-ob_29} & Deve poter avvenire un logout automatico a seguito della cancellazione account & \hyperref[uc_05]{UC\_05}\\
	\hline
	RF-OB\_30 \hypertarget{rf-ob_30} & L'Utente deve poter visualizzare il pulsante \vr{Logout} nella pagina del profilo & \hyperref[uc_06]{UC\_06}\\
	\hline
	RF-OB\_31 \hypertarget{rf-ob_31} & L'Utente deve poter effettuare il logout tramite il pulsante \vr{Logout} & \hyperref[uc_06]{UC\_06} \\
	\hline
	RF-OB\_32 \hypertarget{rf-ob_32} & Il Cliente deve poter inserire input testuali con lunghezza massima di 4096 caratteri & \hyperref[uc_09]{UC\_09}\newline \hyperref[uc_10]{UC\_10} \\
	\hline
	RF-OB\_33 \hypertarget{rf-ob_33} & Il Cliente deve poter vedere il tasto di invio bloccato al superamento di 4096 caratteri & \hyperref[uc_10]{UC\_10} \\
	\hline
	RF-OB\_34 \hypertarget{rf-ob_34} & Il Cliente deve poter visualizzare il campo di input & \hyperref[uc_09]{UC\_09}\newline \hyperref[uc_13]{UC\_13}\newline \hyperref[uc_15]{UC\_15}\newline \hyperref[uc_17]{UC\_17} \\
	\hline
	RF-OB\_35 \hypertarget{rf-ob_35} & Il Cliente deve poter visualizzare il tasto di invio & \hyperref[uc_09]{UC\_09}\newline \hyperref[uc_11]{UC\_11}\newline \hyperref[uc_13]{UC\_13}\newline \hyperref[uc_15]{UC\_15}\newline \hyperref[uc_17]{UC\_17} \\
	\hline
	RF-OB\_36 \hypertarget{rf-ob_36} & Il Cliente deve poter premere il tasto di invio & \hyperref[uc_09]{UC\_09}\newline \hyperref[uc_11]{UC\_11}\newline \hyperref[uc_13]{UC\_13}\newline \hyperref[uc_15]{UC\_15}\newline \hyperref[uc_17]{UC\_17} \\
	\hline
	RF-OB\_37 \hypertarget{rf-ob_37} & Il Cliente in assenza di contenuto di input non potrà premere il tasto di invio che sarà disabilitato & \hyperref[uc_09]{UC\_09}\newline \hyperref[uc_11]{UC\_11}\newline \hyperref[uc_13]{UC\_13}\newline \hyperref[uc_15]{UC\_15}\newline \hyperref[uc_17]{UC\_17} \\
	\hline
	RF-OB\_38 \hypertarget{rf-ob_38} & Il Cliente dopo l'invio dovrà visualizzare il campo di testo vuoto & \hyperref[uc_09]{UC\_09}\newline \hyperref[uc_11]{UC\_11}\newline \hyperref[uc_13]{UC\_13} \\
	\hline
	RF-OB\_39 \hypertarget{rf-ob_39} & Il Cliente deve poter visualizzare un pulsante con l'icona del microfono & \hyperref[uc_13]{UC\_13} \\
	\hline
	RF-OB\_40 \hypertarget{rf-ob_40} & Il Cliente deve poter cliccare il pulsante con l'icona del microfono per registrare messaggi audio & \hyperref[uc_13.1]{UC\_13.1} \\
	\hline
	RF-OB\_41 \hypertarget{rf-ob_41} & Il Cliente deve poter visualizzare la durata della registrazione & \hyperref[uc_13.1]{UC\_13.1} \\
	\hline
	RF-OB\_42 \hypertarget{rf-ob_42} & Il Cliente premendo nuovamente l'icona del microfono ferma la registrazione & \hyperref[uc_13]{UC\_13}\newline \hyperref[uc_13.1]{UC\_13.1} \\
	\hline
	RF-OB\_43 \hypertarget{rf-ob_43} & Il Cliente deve poter visualizzare un pulsante con icona \vr{clip} & \hyperref[uc_11.2]{UC\_11.2}\newline \hyperref[uc_13.2]{UC\_13.2} \\
	\hline
	RF-OB\_44 \hypertarget{rf-ob_44} & Il Cliente deve poter premere il pulsante per selezionare il file da caricare dal dispositivo & \hyperref[uc_11.2]{UC\_11.2}\newline \hyperref[uc_13.2]{UC\_13.2} \\
	\hline
	RF-OB\_45 \hypertarget{rf-ob_45} & Il Cliente cliccando sul pulsante deve poter aprire il dialogo di selezione file del sistema operativo & \hyperref[uc_11.2]{UC\_11.2}\newline \hyperref[uc_13.2]{UC\_13.2} \\
	\hline
	RF-OB\_46 \hypertarget{rf-ob_46} & Il Cliente deve poter selezionare i file & \hyperref[uc_11.2]{UC\_11.2}\newline \hyperref[uc_13.2]{UC\_13.2} \\
	\hline
	RF-OB\_47 \hypertarget{rf-ob_47} & Il Cliente deve poter vedere il tasto \vr{Seleziona} nella selezione file & \hyperref[uc_11.2]{UC\_11.2}\newline \hyperref[uc_13.2]{UC\_13.2} \\
	\hline
	RF-OB\_48 \hypertarget{rf-ob_48} & Il Cliente deve poter inviare audio solo se <= 120 secondi & \hyperref[uc_13]{UC\_13} \\
	\hline
	RF-OB\_49 \hypertarget{rf-ob_49} & Il Cliente deve poter visualizzare un messaggio di errore se il messaggio vocale supera i 120 secondi & \hyperref[uc_14]{UC\_14} \\
	\hline
	RF-OB\_50 \hypertarget{rf-ob_50} & Il Cliente deve poter inviare file audio con dimensione massima di 10MB in caso contrario ne sarà impedito l'invio & \hyperref[uc_13.1]{UC\_13.1}\newline \hyperref[uc_13.2]{UC\_13.2} \\
	\hline
	RF-OB\_51 \hypertarget{rf-ob_51} & Il Cliente deve poter visualizzare un messaggio di errore se le dimensioni dell'audio superano i 10MB & \hyperref[uc_14]{UC\_14} \\
	\hline
	RF-OB\_52 \hypertarget{rf-ob_52} & Il Cliente deve poter visualizzare un messaggio di errore se il messaggio vocale supera i 10MB & \hyperref[uc_14]{UC\_14} \\
	\hline
	RF-OB\_53 \hypertarget{rf-ob_53} & Il Cliente deve poter selezionare file audio in formato .mp3 & \hyperref[uc_13.2.1]{UC\_13.2.1}\\
	\hline
	RF-OB\_54 \hypertarget{rf-ob_54} & Il Cliente deve poter selezionare file audio in formato .m4a & \hyperref[uc_13.2.2]{UC\_13.2.2} \\
	\hline
	RF-OB\_55 \hypertarget{rf-ob_55} & Il Cliente deve poter selezionare file audio in formato .m4p & \hyperref[uc_13.2.3]{UC\_13.2.3} \\
	\hline
	RF-OB\_56 \hypertarget{rf-ob_56} & Il Cliente deve poter selezionare file audio in formato .wav & \hyperref[uc_13.2.4]{UC\_13.2.4} \\
	\hline
	RF-OB\_57 \hypertarget{rf-ob_57} & Il Cliente deve poter visualizzare un messaggio di errore qualora i formati non rientrassero tra quelli accettati & \hyperref[uc_14]{UC\_14} \\
	\hline
	RF-OB\_58 \hypertarget{rf-ob_58} & Il Cliente deve poter visualizzare un messaggio di errore qualora il chatbot non riconosca gli articoli & \hyperref[uc_14]{UC\_14} \\
	\hline
	RF-OB\_59 \hypertarget{rf-ob_59} & Il Cliente deve poter aggiungere una massimo di 10 prodotti per volta al carrello tramite input & \hyperref[uc_15]{UC\_15} \\
	\hline
	RF-OB\_60 \hypertarget{rf-ob_60} & Il Cliente deve ricevere conferma dell'aggiunta dei prodotti & \hyperref[uc_15]{UC\_15} \\
	\hline
	RF-OB\_61 \hypertarget{rf-ob_61} & Il Cliente deve visualizzare un messaggio di errore nel caso in cui tenti di aggiungere più di 10 articoli & \hyperref[uc_16]{UC\_16} \\
	\hline
	RF-OB\_62 \hypertarget{rf-ob_62} & Il Cliente deve poter rimuovere articoli tramite input & \hyperref[uc_17]{UC\_17} \\
	\hline
	RF-OB\_63 \hypertarget{rf-ob_63} & Il Cliente deve ricevere una conferma della rimozione degli articoli & \hyperref[uc_17]{UC\_17} \\
	\hline
	RF-OB\_64 \hypertarget{rf-ob_64} & Il Cliente deve essere avvisato nel caso in cui richiedesse la rimozione di alcuni articoli dal carrello ma quest'ultimo fosse vuoto & \hyperref[uc_18]{UC\_18} \\
	\hline
	RF-OB\_65 \hypertarget{rf-ob_65} & Il Cliente deve poter inviare il comando testuale \vr{/carrello} & \hyperref[uc_19]{UC\_19} \newline \hyperref[uc_44.2]{UC\_44.2}  \\
	\hline
	RF-OB\_66 \hypertarget{rf-ob_66} & Il Cliente deve poter visualizzare il prezzo di ogni articolo & \hyperref[uc_19.1.1]{UC\_19.1.1} \newline \hyperref[uc_26.2.1.1]{UC\_26.2.1.1} \newline \hyperref[uc_34.1.1.1]{UC\_34.1.1.1} \\
	\hline
	RF-OB\_67 \hypertarget{rf-ob_67} & Il Cliente deve poter visualizzare il nome di ogni articolo & \hyperref[uc_19.1.2]{UC\_19.1.2} \newline \hyperref[uc_26.2.1.2]{UC\_26.2.1.2}\newline \hyperref[uc_34.1.1.2]{UC\_34.1.1.2}\\
	\hline
	RF-OB\_68 \hypertarget{rf-ob_68} & Il Cliente deve poter visualizzare la quantità di ogni articolo & \hyperref[uc_19.1.3]{UC\_19.1.3} \newline \hyperref[uc_26.2.1.3]{UC\_26.2.1.3} \newline \hyperref[uc_34.1.1.3]{UC\_34.1.1.3}\\
	\hline
	RF-OB\_69 \hypertarget{rf-ob_69} & Il Cliente deve poter visualizzare i singoli elementi del carrello & \hyperref[uc_19.1]{UC\_19.1} \\
	\hline
	RF-OB\_70 \hypertarget{rf-ob_70} & Il Cliente deve poter visualizzare l'anteprima del carrello & \hyperref[uc_19]{UC\_19} \\
	\hline
	RF-OB\_71 \hypertarget{rf-ob_71} & Il Cliente deve poter ricevere riceve un avviso nel caso in cui nel carrello non ci fossero elementi & \hyperref[uc_20]{UC\_20} \\
	\hline
	RF-OB\_72 \hypertarget{rf-ob_72} & Il Cliente deve poter inviare l'ordine & \hyperref[uc_26]{UC\_26} \\
	\hline
	RF-OB\_73 \hypertarget{rf-ob_73} & Il Cliente deve poter inserire il comando \vr{/invia} nella chat per avviare la conferma dell'ordine & \hyperref[uc_26.1]{UC\_26.1} \\
	\hline
	RF-OB\_74 \hypertarget{rf-ob_74} & Il Cliente deve poter visualizzare il riepilogo ordine nella chat & \hyperref[uc_26.2]{UC\_26.2} \newline \hyperref[uc_34.1]{UC\_34.1}\\
	\hline
	RF-OB\_75 \hypertarget{rf-ob_75} & Il Cliente a seguito del messaggio riepilogativo deve decidere se annullare o confermare l'invio dell'ordine & \hyperref[uc_26.2]{UC\_26.2} \\
	\hline
	RF-OB\_76 \hypertarget{rf-ob_74} & Il Cliente deve poter visualizzare i singoli elementi del riepilogo ordine dalla chat & \hyperref[uc_26.2.1]{UC\_26.2.1} \newline \hyperref[uc_34.1.1]{UC\_34.1.1} \\
	\hline
	RF-OB\_77 \hypertarget{rf-ob_76} & Il Cliente deve poter confermare l'invio dell'ordine con un input & \hyperref[uc_26.3]{UC\_26.3}\newline \hyperref[uc_34.2]{UC\_34.2} \\
	\hline
	RF-OB\_78 \hypertarget{rf-ob_77} & Il Cliente deve poter visualizzare la conferma dell'invio dell'ordine & \hyperref[uc_26.4]{UC\_26.4}\\
	\hline
	RF-OB\_79 \hypertarget{rf-ob_78} & Il Cliente deve poter annullare l'invio dell'ordine con un input & \hyperref[uc_26.5]{UC\_26.5}\\
	\hline
	RF-OB\_80 \hypertarget{rf-ob_79} & Se il carrello è vuoto dopo il comando \vr{/invia} il Cliente visualizza un messaggio di errore che avvisa che il carrello è vuoto & \hyperref[uc_27]{UC\_27} \\
	\hline
	RF-OB\_81 \hypertarget{rf-ob_80} & Il Cliente deve poter visualizzare un messaggio riguardante l'ambiguità dei prodotti & \hyperref[uc_21]{UC\_21} \\
	\hline
	RF-OB\_82 \hypertarget{rf-ob_81} & Il Cliente deve poter riformulare l'ordine per prodotti ambigui & \hyperref[uc_21.1]{UC\_21.1} \\
	\hline
	RF-OB\_83 \hypertarget{rf-ob_82} & Il Cliente deve poter annullare la disambiguazione & \hyperref[uc_22]{UC\_22} \\
	\hline
	RF-OB\_84 \hypertarget{rf-ob_83} & Il Cliente deve poter rispondere \vr{/annulla} per uscire dalla disambiguazione & \hyperref[uc_22.1]{UC\_22.1} \\
	\hline
	RF-OB\_85 \hypertarget{rf-ob_84} & Il Cliente deve poter ricevere conferma dell'annullamento della disambiguazione & \hyperref[uc_22.2]{UC\_22.2} \\
	\hline
	RF-OB\_86 \hypertarget{rf-ob_85} & Il Cliente deve poter vedere il pulsante \vr{Nuova chat} & \hyperref[uc_28]{UC\_28} \\
	\hline
	RF-OB\_87 \hypertarget{rf-ob_86} & L'Utente deve poter cliccare sul pulsante \vr{Nuova chat} & \hyperref[uc_28]{UC\_28} \\
	\hline
	RF-OB\_88 \hypertarget{rf-ob_87} & Il Cliente deve poter visualizzare una notifica di errore nel caso in cui il prodotto non venga duplicato correttamente & \hyperref[uc_35]{UC\_35} \\
	\hline
	RF-OB\_89 \hypertarget{rf-ob_88} & L'Utente deve poter visualizzare il pulsante \vr{visualizza storico ordini} & \hyperref[uc_36]{UC\_36} \newline \hyperref[uc_45]{UC\_45} \\
	\hline
	RF-OB\_90 \hypertarget{rf-ob_89} & L'Utente deve poter schiacciare il pulsante \vr{visualizza storico ordini} & \hyperref[uc_36]{UC\_36} \newline \hyperref[uc_45]{UC\_45} \\
	\hline
	RF-OB\_91 \hypertarget{rf-ob_90} & L'Admin deve poter visualizzare una pagina separata contenente lo storico totale degli ordini di tutti i Clienti & \hyperref[uc_36]{UC\_36} \\
	\hline
	RF-OB\_92 \hypertarget{rf-ob_91} & Il Cliente deve poter visualizzare una pagina separata contente lo storico totale dei suoi ordini & \hyperref[uc_45]{UC\_45} \\
	\hline
	RF-OB\_93 \hypertarget{rf-ob_92} & La lista completa degli ordini deve essere visualizzata in pagine contenenti al massimo 10 ordini ciascuna & \hyperref[uc_36]{UC\_36} \newline \hyperref[uc_45]{UC\_45} \\
	\hline
	RF-OB\_94 \hypertarget{rf-ob_93} & Il caricamento di una pagina non implica il caricamento delle pagine successive & \hyperref[uc_36]{UC\_36} \newline \hyperref[uc_45]{UC\_45} \\
	\hline
	RF-OB\_95 \hypertarget{rf-ob_94} & L'Utente deve poter visualizzare il singolo ordine nello storico & \hyperref[uc_36.1]{UC\_36.1} \newline \hyperref[uc_45.1]{UC\_45.1} \\
	\hline
	RF-OB\_96 \hypertarget{rf-ob_95} & L'Utente deve poter visualizzare il codice ordine & \hyperref[uc_36.1.1]{UC\_36.1.1} \newline \hyperref[uc_45.1.1]{UC\_45.1.1} \\
	\hline
	RF-OB\_97 \hypertarget{rf-ob_96} & L'Utente deve poter visualizzare la data dell'ordine & \hyperref[uc_36.1.2]{UC\_36.1.2} \newline \hyperref[uc_45.1.2]{UC\_45.1.2} \\
	\hline
	RF-OB\_98 \hypertarget{rf-ob_97} & L'Admin deve poter visualizzare lo username del Cliente & \hyperref[uc_36.1.3]{UC\_36.1.3} \\
	\hline
	RF-OB\_99 \hypertarget{rf-ob_98} & L'Utente deve poter visualizzare il dettaglio di un ordine specifico & \hyperref[uc_37]{UC\_37} \newline \hyperref[uc_46]{UC\_46} \\
	\hline
	RF-OB\_100 \hypertarget{rf-ob_99} & L'Admin deve poter visualizzare lo username del Cliente & \hyperref[uc_37.1]{UC\_37.1} \\
	\hline
	RF-OB\_101 \hypertarget{rf-ob_100} & L'Utente deve poter visualizzare i prodotti dell'ordine & \hyperref[uc_37.2]{UC\_37.2} \newline \hyperref[uc_46.1]{UC\_46.1} \\
	\hline
	RF-OB\_102 \hypertarget{rf-ob_101} & L'Utente deve poter visualizzare il nome dei prodotti & \hyperref[uc_37.2.1]{UC\_37.2.1} \newline \hyperref[uc_46.1.1]{UC\_46.1.1} \\
	\hline
	RF-OB\_103 \hypertarget{rf-ob_102} & L'Utente deve poter visualizzare la descrizione dei prodotti & \hyperref[uc_37.2.2]{UC\_37.2.2} \newline \hyperref[uc_46.1.2]{UC\_46.1.2} \\
	\hline
	RF-OB\_104 \hypertarget{rf-ob_103} & L'Utente deve poter visualizzare la quantità dei prodotti & \hyperref[uc_37.2.3]{UC\_37.2.3} \newline \hyperref[uc_46.1.3]{UC\_46.1.3} \\
	\hline
	RF-OB\_105 \hypertarget{rf-ob_104} & L'Utente deve poter visualizzare la data dell'ordine & \hyperref[uc_37.3]{UC\_37.3} \newline \hyperref[uc_46.2]{UC\_46.2} \\
	\hline
	RF-OB\_106 \hypertarget{rf-ob_105} & L'Utente deve poter visualizzare il numero dell'ordine & \hyperref[uc_37.4]{UC\_37.4} \newline \hyperref[uc_46.3]{UC\_46.3} \\
	\hline
	RF-OB\_107 \hypertarget{rf-ob_106} & Il Cliente deve poter visualizzare il pulsante per duplicare l'ordine & \hyperref[uc_38]{UC\_38} \\
	\hline
	RF-OB\_108 \hypertarget{rf-ob_107} & Il Cliente deve poter cliccare sul pulsante per duplicare l'ordine & \hyperref[uc_38]{UC\_38} \\
	\hline
	RF-OB\_109 \hypertarget{rf-ob_108} & Il Cliente deve poter confermare la duplicazione dell'ordine & \hyperref[uc_38.1]{UC\_38.1} \\
	\hline
	RF-OB\_110 \hypertarget{rf-ob_109} & Il Cliente deve poter visualizzare la notifica di errore duplicazione ordine & \hyperref[uc_39]{UC\_39} \\
	\hline
	\multicolumn{3}{c}{} \\
	\caption{Requisiti Funzionali obbligatori}\hypertarget{tab:funzionali_obbligatori}\\
\end{longtable}

	
\subsubsection{Requisiti Funzionali desiderabili}
\renewcommand{\arraystretch}{1.15}
\begin{longtable}{|>{\raggedright}p{0.18\textwidth}|>{\raggedright\arraybackslash}p{0.5\textwidth}|>{\raggedright\arraybackslash}p{0.2\textwidth}|}
	\hline
	\rowcolor[gray]{0.9}
	\textbf{Codice} & \textbf{Descrizione} & \textbf{Fonti} \\
	\hline
	\endfirsthead
	
	\hline
	\rowcolor[gray]{0.9}
	\textbf{Codice} & \textbf{Descrizione} & \textbf{Fonti}  \\
	\hline
	\endhead
	
	\hline
	RF-DE\_01 \hypertarget{rf-de_1} & L'Utente deve poter visualizzare il link HTML per la password dimenticata & \hyperref[uc_04]{UC\_04} \\
	\hline
	RF-DE\_02 \hypertarget{rf-de_2} & L'Utente deve poter cliccare il link HTML per la password dimenticata & \hyperref[uc_04]{UC\_04} \\
	\hline
	RF-DE\_03 \hypertarget{rf-de_3} & L'Utente deve poter visualizzare il form per inserire l'indirizzo mail che ha usato quando si è registrato & \hyperref[uc_04.1]{UC\_04.1} \\
	\hline
	RF-DE\_04 \hypertarget{rf-de_4} & L'Utente deve ricevere via mail la password generata automaticamente & \hyperref[uc_04.1]{UC\_04.1} \\
	\hline
	RF-DE\_05 \hypertarget{rf-de_5} & L'Utente deve ricevere un messaggio di errore nel caso in cui l'email non fosse valida & \hyperref[uc_04.1.1]{UC\_04.1.1} \\
	\hline
	RF-DE\_06 \hypertarget{rf-de_6} & L'Utente deve ricevere un messaggio di errore nel caso in cui l'email non fosse presente nel sistema & \hyperref[uc_04.1.2]{UC\_04.1.2} \\
	\hline
	RF-DE\_07 \hypertarget{rf-de_7} & L'Utente deve poter visualizzare il pulsante \vr{Elimina Account} nella pagina delle informazioni del profilo & \hyperref[uc_05]{UC\_05} \\
	\hline
	RF-DE\_08 \hypertarget{rf-de_8} & L'Utente deve poter premere il pulsante per eliminare il suo account & \hyperref[uc_05]{UC\_05} \\
	\hline
	RF-DE\_09 \hypertarget{rf-de_9} & L'Utente deve poter visualizzare la conferma dell'eliminazione dell'account & \hyperref[uc_05.1]{UC\_05.1} \\
	\hline
	RF-DE\_10 \hypertarget{rf-de_10} & L'Utente deve poter visualizzare nella home un'icona che porta alla pagina delle info del profilo & \hyperref[uc_07]{UC\_07} \\
	\hline
	RF-DE\_11 \hypertarget{rf-de_11} & L'Utente deve poter cliccare sul link, rappresentato da un'icona, che porta sulla pagina delle info del proprio profilo & \hyperref[uc_07]{UC\_07} \\
	\hline
	RF-DE\_12 \hypertarget{rf-de_12} & L'Utente deve poter visualizzare il link alla pagina dello storico ordini nelle info del proprio profilo & \hyperref[uc_07]{UC\_07} \\
	\hline
	RF-DE\_13 \hypertarget{rf-de_13} & L'Utente deve poter visualizzare il proprio username nella pagina delle info del proprio profilo & \hyperref[uc_07.1]{UC\_07.1} \\
	\hline
	RF-DE\_14 \hypertarget{rf-de_14} & L'Utente deve poter visualizzare la propria ragione sociale nelle info del proprio profilo & \hyperref[uc_07.2]{UC\_07.2} \\
	\hline
	RF-DE\_15 \hypertarget{rf-de_15} & L'Utente deve poter visualizzare il proprio indirizzo email nelle info del proprio profilo & \hyperref[uc_07.3]{UC\_07.3} \\
	\hline
	RF-DE\_16 \hypertarget{rf-de_16} & L'Utente deve poter visualizzare il pulsante \vr{Reimposta password} & \hyperref[uc_08]{UC\_08} \\
	\hline
	RF-DE\_17 \hypertarget{rf-de_17} & L'Utente deve poter cliccare il pulsante \vr{Reimposta password} & \hyperref[uc_08]{UC\_08} \\
	\hline
	RF-DE\_18 \hypertarget{rf-de_18} & L'Utente deve poter visualizzare il campo di input \vr{Vecchia password} & \hyperref[uc_08.1]{UC\_08.1} \\
	\hline
	RF-DE\_19 \hypertarget{rf-de_19} & L'Utente deve poter inserire la vecchia password nel campo di input \vr{Vecchia password} & \hyperref[uc_08.1]{UC\_08.1} \\
	\hline
	RF-DE\_20 \hypertarget{rf-de_20} & L'Utente deve poter visualizzare un messaggio di errore nel caso in cui l'inserimento della vecchia password non corrispondesse con la password presente nel sistema & \hyperref[uc_08.1.1]{UC\_08.1.1} \\
	\hline
	RF-DE\_21 \hypertarget{rf-de_21} & L'Utente deve poter visualizzare il campo di input \vr{Nuova password} & \hyperref[uc_08.2]{UC\_08.2} \\
	\hline
	RF-DE\_22 \hypertarget{rf-de_22} & L'Utente deve poter inserire la nuova password nel campo di input \vr{Nuova password} & \hyperref[uc_08.2]{UC\_08.2} \\
	\hline
	RF-DE\_23 \hypertarget{rf-de_23} & L'Utente deve poter visualizzare un messaggio di errore nel caso in cui la nuova password inserita non fosse conforme ai criteri imposti & \hyperref[uc_08.2.1]{UC\_08.2.1} \\
	\hline
	RF-DE\_24 \hypertarget{rf-de_24} & L'Utente deve poter visualizzare il campo di input \vr{Conferma nuova password} & \hyperref[uc_08.3]{UC\_08.3} \\
	\hline
	RF-DE\_25 \hypertarget{rf-de_25} & L'Utente deve poter confermare la nuova password nel campo di input \vr{Conferma nuova password} & \hyperref[uc_08.3]{UC\_08.3} \\
	\hline
	RF-DE\_26 \hypertarget{rf-de_26} & L'Utente deve poter visualizzare un messaggio di errore nel caso in cui la conferma della nuova password non corrisponda alla nuova password inserita nel campo precedente & \hyperref[uc_08.3.1]{UC\_08.3.1} \\
	\hline
	RF-DE\_27 \hypertarget{rf-de_27} & L'Utente deve essere notificato qualora il cambio password vada a buon fine tramite un messaggio di conferma & \hyperref[uc_08.3]{UC\_08.3} \\
	\hline
	RF-DE\_28 \hypertarget{rf-de_28} & Durante la digitazione il Cliente deve poter visualizzare un contatore x/4096 vicino al campo di testo & \hyperref[uc_09]{UC\_09} \\
	\hline
	RF-DE\_29 \hypertarget{rf-de_29} & Il Cliente deve poter visualizzare il pulsante di caricamento file immagine (icona fotocamera) & \hyperref[uc_11]{UC\_11} \\
	\hline
	RF-DE\_30 \hypertarget{rf-de_30} & Il Cliente deve poter cliccare l'icona della fotocamera & \hyperref[uc_11]{UC\_11} \\
	\hline
	RF-DE\_31 \hypertarget{rf-de_31} & Il Cliente a seguito del click sull'icona a forma di clip deve poter vedere un dropdown tra cui: "Seleziona dal dispositivo" oppure "Scatta la foto" & \hyperref[uc_11]{UC\_11} \\
	\hline
	RF-DE\_32 \hypertarget{rf-de_32} & Il Cliente deve poter selezionare \vr{Seleziona dal dispositivo} & \hyperref[uc_11.2]{UC\_11.2} \\
	\hline
	RF-DE\_33 \hypertarget{rf-de_33} & Se Il Cliente seleziona \vr{Seleziona dal dispositivo} deve poter visualizzare i media nel dispositivo & \hyperref[uc_11.2]{UC\_11.2} \\
	\hline
	RF-DE\_34 \hypertarget{rf-de_34} & Il Cliente deve poter selezionare \vr{Scatta foto} & \hyperref[uc_11.1]{UC\_11.1} \\
	\hline
	RF-DE\_35 \hypertarget{rf-de_35} & Alla selezione \vr{Scatta foto} il Cliente visualizza l'attivazione della fotocamera & \hyperref[uc_11.1]{UC\_11.1} \\
	\hline
	RF-DE\_36 \hypertarget{rf-de_36} & Il Cliente deve poter inviare immagini con dimensione massima di 15MB e con risoluzione pari o superiore a 800x600 pixel & \hyperref[uc_11]{UC\_11} \\
	\hline
	RF-DE\_37 \hypertarget{rf-de_37} & Nel caso in cui le immagini superino i 15MB abbiano risoluzione inferiore di 800x600 pixel sarà visibile al Cliente un messaggio di errore \vr{L'immagine non rispetta i requisiti richiesti} & \hyperref[uc_11]{UC\_11}\newline \hyperref[uc_12]{UC\_12} \\
	\hline
	RF-DE\_38 \hypertarget{rf-de_38} & Il Cliente deve premere il pulsante di invio per confermare l'invio dell'immagine & \hyperref[uc_11]{UC\_11} \\
	\hline
	RF-DE\_39 \hypertarget{rf-de_39} & Il Cliente deve poter visualizza un messaggio \vr{Non sono riuscito a rilevare il testo nell'immagine} se il modello non riesce a rilevare testo & \hyperref[uc_12]{UC\_12} \\
	\hline
	RF-DE\_40 \hypertarget{rf-de_40} & Il Cliente deve poter selezionare immagini nei formati .jpg & \hyperref[uc_11.2.1]{UC\_11.2.1} \\
	\hline
	RF-DE\_41 \hypertarget{rf-de_41} & Il Cliente deve poter selezionare immagini nei formati .jpeg & \hyperref[uc_11.2.2]{UC\_11.2.2} \\
	\hline
	RF-DE\_42 \hypertarget{rf-de_42} & Il Cliente deve poter selezionare immagini nei formati .png & \hyperref[uc_11.2.3]{UC\_11.2.3} \\
	\hline
	RF-DE\_43 \hypertarget{rf-de_43} & Il Cliente deve poter visionare l'anteprima dell'immagine caricata & \hyperref[uc_11]{UC\_11} \\
	\hline
	RF-DE\_44 \hypertarget{rf-de_44} & Il Cliente deve poter visualizzare il pulsante a fianco all'ultima risposta \vr{lascia Feedback} & \hyperref[uc_23]{UC\_23} \\
	\hline
	RF-DE\_45 \hypertarget{rf-de_45} & Il Cliente deve poter premere il pulsante affianco ad ogni risposta \vr{lascia Feedback} per fornire feedback all'AI & \hyperref[uc_23]{UC\_23} \\
	\hline
	RF-DE\_46 \hypertarget{rf-de_46} & Il Cliente premendo su \vr{lascia Feedback} deve visualizzare l'apertura di un form in una nuova pagina & \hyperref[uc_23]{UC\_23} \\
	\hline
	RF-DE\_47 \hypertarget{rf-de_47} & Il Cliente deve vedere il dropdown relativo alla tipologia di feedback: \vr{Prodotto sbagliato}, \vr{Quantità errata}, \vr{Incomprensione}, \vr{Altro} & \hyperref[uc_23.1]{UC\_23.1} \\
	\hline
	RF-DE\_48 \hypertarget{rf-de_48} & Il Cliente deve selezionare obbligatoriamente una tipologia: \vr{Prodotto sbagliato}, \vr{Quantità errata}, \vr{Incomprensione}, \vr{Altro} & \hyperref[uc_23.1]{UC\_23.1} \\
	\hline
	RF-DE\_49 \hypertarget{rf-de_49} & Il Cliente visualizza l'input testuale della descrizione & \hyperref[uc_23.2]{UC\_23.2}\newline \hyperref[uc_24.2]{UC\_24.2} \\
	\hline
	RF-DE\_50 \hypertarget{rf-de_50} & Il Cliente deve inserire obbligatoriamente una descrizione & \hyperref[uc_23.2]{UC\_23.2}\newline \hyperref[uc_24.2]{UC\_24.2} \\
	\hline
	RF-DE\_51 \hypertarget{rf-de_51} & Il Cliente nella descrizione deve poter inserire un massimo di 300 caratteri & \hyperref[uc_23.2]{UC\_23.2}\newline \hyperref[uc_24.2]{UC\_24.2} \\
	\hline
	RF-DE\_52 \hypertarget{rf-de_52} & Il Cliente deve vedere un contatore x/300 vicino al campo di testo & \hyperref[uc_23.2]{UC\_23.2}\newline \hyperref[uc_24.2]{UC\_24.2} \\
	\hline
	RF-DE\_53 \hypertarget{rf-de_53} & Al superamento dei 300 caratteri il Cliente non potrà più scrivere nell'input & \hyperref[uc_23.2]{UC\_23.2}\newline \hyperref[uc_24.2]{UC\_24.2} \\
	\hline
	RF-DE\_54 \hypertarget{rf-de_54} & Il Cliente visualizza in fondo alla pagina il pulsante \vr{Invia feedback} & \hyperref[uc_23.3]{UC\_23.3} \\
	\hline
	RF-DE\_55 \hypertarget{rf-de_55} & Il Cliente deve poter premere il pulsante \vr{Invia feedback} & \hyperref[uc_23.3]{UC\_23.3} \\
	\hline
	RF-DE\_56 \hypertarget{rf-de_56} & Il Cliente deve poter ricevere la notifica di la conferma di invio feedback & \hyperref[uc_23.3]{UC\_23.4} \\
	\hline
	RF-DE\_57 \hypertarget{rf-de_57} & Il Cliente deve poter vedere nella home il pulsante \vr{Segnala un problema} & \hyperref[uc_24]{UC\_24} \\
	\hline
	RF-DE\_58 \hypertarget{rf-de_58} & Il Cliente deve poter cliccare sul pulsante \vr{Segnala un problema} & \hyperref[uc_24]{UC\_24} \\
	\hline
	RF-DE\_59 \hypertarget{rf-de_59} & Il Cliente deve visualizzare il dropdown: \vr{Bug$^G$}, \vr{Richiesta di supporto}, \vr{Suggerimento} & \hyperref[uc_24.1]{UC\_24.1} \\
	\hline
	RF-DE\_60 \hypertarget{rf-de_60} & Il Cliente deve poter selezionare il tipo da dropdown & \hyperref[uc_24.1]{UC\_24.1} \\
	\hline
	RF-DE\_61 \hypertarget{rf-de_61} & Il Cliente deve poter visualizzare il tasto \vr{Invia segnalazione} & \hyperref[uc_24.3]{UC\_24.3} \\
	\hline
	RF-DE\_62 \hypertarget{rf-de_62} & Il Cliente deve poter inviare il form cliccando \vr{Invia segnalazione} & \hyperref[uc_24.3]{UC\_24.3} \\
	\hline
	RF-DE\_63 \hypertarget{rf-de_63} & Il Cliente deve poter ricevere una notifica di conferma dell'invio con un messaggio \vr{Ticket creato con successo} & \hyperref[uc_24.4]{UC\_24.4} \\
	\hline
	RF-DE\_64 \hypertarget{rf-de_64} & L'Admin deve poter visualizzare la pagina web delle performance & \hyperref[uc_25]{UC\_25} \\
	\hline
	RF-DE\_65 \hypertarget{rf-de_65} & L'Admin deve poter visualizzare il link per la pagina delle performance in qualsiasi pagina & \hyperref[uc_25]{UC\_25} \\
	\hline
	RF-DE\_66 \hypertarget{rf-de_66} & L'Admin deve poter schiacciare il link per la pagina delle performance in qualsiasi pagina & \hyperref[uc_25]{UC\_25} \\
	\hline
	RF-DE\_67 \hypertarget{rf-de_67} & L'Admin deve poter visualizzare il tempo medio di risposta del chatbot & \hyperref[uc_25.1]{UC\_25.1} \\
	\hline
	RF-DE\_68 \hypertarget{rf-de_68} & L'Admin deve poter visualizzare il tempo medio di permanenza nella web-app & \hyperref[uc_25.2]{UC\_25.2} \\
	\hline
	RF-DE\_69 \hypertarget{rf-de_69} & L'Admin deve poter visualizzare un pulsante relativo all'esportazione del log & \hyperref[uc_29]{UC\_29} \\
	\hline
	RF-DE\_70 \hypertarget{rf-de_70} & L'Admin deve poter schiacciare il pulsante relativo all'esportazione del log & \hyperref[uc_29]{UC\_29} \\
	\hline
	RF-DE\_71 \hypertarget{rf-de_71} & L'Admin deve poter ottenere il file di log in formato JSON$^G$ & \hyperref[uc_29]{UC\_29} \\
	\hline
	RF-DE\_72 \hypertarget{rf-de_72} & L'Admin deve poter ottenere il file JSON$^G$ dello storico ordini & \hyperref[uc_30]{UC\_30} \\
	\hline
	RF-DE\_73 \hypertarget{rf-de_73} & L'Admin deve poter visualizzare il tasto \vr{Esporta} nella pagina dello storico ordini & \hyperref[uc_30]{UC\_30} \\
	\hline
	RF-DE\_74 \hypertarget{rf-de_74} & L'Admin deve poter schiacciare il tasto \vr{Esporta} nella pagina dello storico ordini & \hyperref[uc_30]{UC\_30} \\
	\hline
	RF-DE\_75 \hypertarget{rf-de_75} & L'Admin deve poter visualizzare la conferma dell'esportazione dello storico ordini & \hyperref[uc_30.1]{UC\_30.1} \\
	\hline
	RF-DE\_76 \hypertarget{rf-de_76} & L'Admin deve poter visualizzare un errore nel caso in cui non fosse stato possibile scaricare nel proprio dispositivo il file & \hyperref[uc_31]{UC\_31} \\
	\hline
	RF-DE\_77 \hypertarget{rf-de_77} & Il file JSON$^G$ esportato deve contenere il campo id [integer]: identificativo univoco dell'ordine & \hyperref[uc_30]{UC\_30} \\
	\hline
	RF-DE\_78 \hypertarget{rf-de_78} & Il file JSON$^G$ esportato deve contenere il campo cod\_cli [integer]: identificativo univoco del cliente & \hyperref[uc_30]{UC\_30} \\
	\hline
	RF-DE\_79 \hypertarget{rf-de_79} & Il file JSON$^G$ esportato deve contenere il campo cod\_art [varchar(13)]: identificativo univoco dell'articolo & \hyperref[uc_30]{UC\_30} \\
	\hline
	RF-DE\_80 \hypertarget{rf-de_80} & Il file JSON$^G$ esportato deve contenere il campo data\_ord [date]: data di inserimento dell'ordine & \hyperref[uc_30]{UC\_30} \\
	\hline
	RF-DE\_81 \hypertarget{rf-de_81} & Il file JSON$^G$ esportato deve contenere il campo qta\_ordinata [float]: quantità che l'Utente ha ordinato nell'unità di misura dell'articolo & \hyperref[uc_30]{UC\_30} \\
	\hline
	RF-DE\_82 \hypertarget{rf-de_82} & Il file JSON$^G$ esportato deve contenere il campo rif [integer/string]: rappresenta un riferimento al testo originario & \hyperref[uc_30]{UC\_30} \\
	\hline
	RF-DE\_83 \hypertarget{rf-de_83} & Il Cliente deve poter richiedere la duplicazione indicando il codice dell'ordine dentro il comando comando \vr{/duplica\{xx\}} & \hyperref[uc_34]{UC\_34} \newline \hyperref[uc_43.1.3]{UC\_43.1.3} \\
	\hline
	RF-DE\_84 \hypertarget{rf-de_84} & Il Cliente deve poter richiedere la duplicazione dell'ultimo ordine effettuato indicando il comando "/duplica" & \hyperref[uc_34]{UC\_34} \newline \hyperref[uc_43.1.2]{UC\_43.1.2} \\
	\hline
	RF-DE\_85 \hypertarget{rf-de_85} & Il Cliente deve poter richiedere la duplicazione di un ordine tramite un input testuale, immagine o di audio & \hyperref[uc_34]{UC\_34} \newline \hyperref[uc_09]{UC\_09} \newline \hyperref[uc_11]{UC\_11} \newline \hyperref[uc_13]{UC\_13} \\
	\hline
	RF-DE\_86 \hypertarget{rf-de_86} & Il Cliente deve poter visualizzare una notifica di conferma della duplicazione & \hyperref[uc_34.3]{UC\_34.3} \\
	\hline
	RF-DE\_87 \hypertarget{rf-de_87} & L'Admin deve poter visualizzare le statistiche & \hyperref[uc_40]{UC\_40} \\
	\hline
	RF-DE\_88 \hypertarget{rf-de_88} & L'Admin deve poter visualizzare il numero di utenti presenti in quel momento & \hyperref[uc_40.1]{UC\_40.1} \\
	\hline
	RF-DE\_89 \hypertarget{rf-de_89} & L'Admin deve poter visualizzare il numero di acquisti completati tramite la piattaforma & \hyperref[uc_40.2]{UC\_40.2} \\
	\hline
	RF-DE\_90 \hypertarget{rf-de_90} & L'Admin deve poter visualizzare, in forma di testo, il modello AI utilizzato in quel momento nella pagina delle performance & \hyperref[uc_40.3]{UC\_40.3} \\
	\hline
	RF-DE\_91 \hypertarget{rf-de_91} & L'Utente deve poter visualizzare l'elenco dei comandi in una pagina web dedicata & \hyperref[uc_41]{UC\_41} \\
	\hline
	RF-DE\_92 \hypertarget{rf-de_92} & L'Utente deve poter visualizzare il pulsante contenente il per la pagina dell'elenco dei comandi in qualsiasi pagina & \hyperref[uc_41]{UC\_41} \\
	\hline
	RF-DE\_93 \hypertarget{rf-de_93} & L'Utente deve poter schiacciare il pulsante contenente il link per la pagina dell'elenco dei comandi in qualsiasi pagina & \hyperref[uc_41]{UC\_41} \\
	\hline
	RF-DE\_94 \hypertarget{rf-de_94} & Il Cliente deve poter visualizzare l'elenco dei comandi con la relativa spiegazione & \hyperref[uc_41]{UC\_41} \\
	\hline
	RF-DE\_95 \hypertarget{rf-de_95} & Il Cliente deve poter visualizzare l'elenco dei comandi con la relativa spiegazione & \hyperref[uc_42]{UC\_42} \\
	\hline
	RF-DE\_96 \hypertarget{rf-de_96} & Il Cliente deve poter visualizzare il comando \vr{/duplica} con la relativa spiegazione & \hyperref[uc_41.1]{UC\_41.1}\newline \hyperref[uc_42.1]{UC\_42.1} \\
	\hline
	RF-DE\_97 \hypertarget{rf-de_97} & Il Cliente deve poter visualizzare il comando \vr{/carrello} con la relativa spiegazione & \hyperref[uc_41.2]{UC\_41.2}\newline \hyperref[uc_42.2]{UC\_42.2} \\
	\hline
	RF-DE\_98 \hypertarget{rf-de_98} & Il Cliente deve poter visualizzare il comando \vr{/duplica\{xx\}} con la relativa spiegazione & \hyperref[uc_41.3]{UC\_41.3}\newline \hyperref[uc_42.3]{UC\_42.3} \\
	\hline
	RF-DE\_99 \hypertarget{rf-de_99} & Il Cliente deve poter visualizzare il comando \vr{/invia} con la relativa spiegazione & \hyperref[uc_41.4]{UC\_41.4}\newline \hyperref[uc_42.4]{UC\_42.4} \\
	\hline
	RF-DE\_100 \hypertarget{rf-de_100} & Il Cliente deve poter visualizzare il comando \vr{/annulla} con la relativa spiegazione & \hyperref[uc_41.5]{UC\_41.5}\newline \hyperref[uc_42.5]{UC\_42.5} \\
	\hline
	RF-DE\_101 \hypertarget{rf-de_101} & Il Cliente deve poter visualizzare il comando \vr{/comandi} con la relativa spiegazione & \hyperref[uc_41.6]{UC\_41.6}\newline \hyperref[uc_42.6]{UC\_42.6} \\
	\hline
	RF-DE\_102 \hypertarget{rf-de_102} & Il Cliente deve poter visualizzare i comandi inline & \hyperref[uc_43]{UC\_43} \\
	\hline
	RF-DE\_103 \hypertarget{rf-de_103} & Il Cliente deve poter visualizzare il menù a comparsa dei comandi disponibili & \hyperref[uc_43.1]{UC\_43.1} \\
	\hline
	RF-DE\_104 \hypertarget{rf-de_104} & Il Cliente deve poter visualizzare il comando \vr{/duplica} nel menù a comparsa & \hyperref[uc_43.1.1]{UC\_43.1.1} \\
	\hline
	RF-DE\_105 \hypertarget{rf-de_105} & Il Cliente deve poter visualizzare il comando \vr{/carrello} nel menù a comparsa & \hyperref[uc_43.1.2]{UC\_43.1.2} \\
	\hline
	RF-DE\_106 \hypertarget{rf-de_106} & Il Cliente deve poter visualizzare il comando \vr{/duplica\{xx\}} nel menù a comparsa & \hyperref[uc_43.1.3]{UC\_43.1.3} \\
	\hline
	RF-DE\_107 \hypertarget{rf-de_107} & Il Cliente deve poter visualizzare il comando \vr{/invia} nel menù a comparsa & \hyperref[uc_43.1.4]{UC\_43.1.4} \\
	\hline
	RF-DE\_108 \hypertarget{rf-de_108} & Il Cliente deve poter visualizzare il comando \vr{/annulla} nel menù a comparsa & \hyperref[uc_43.1.5]{UC\_43.1.5} \\
	\hline
	RF-DE\_109 \hypertarget{rf-de_109} & Il Cliente deve poter visualizzare il comando \vr{/comandi} nel menù a comparsa & \hyperref[uc_43.1.6]{UC\_43.1.6} \\
	\hline
	RF-DE\_110 \hypertarget{rf-de_110} & Il Cliente deve poter inviare un comando nella chat & \hyperref[uc_44]{UC\_44} \\
	\hline
	RF-DE\_111 \hypertarget{rf-de_111} & Il Cliente deve poter inviare il comando \vr{/duplica} nella chat & \hyperref[uc_34]{UC\_34} \newline \hyperref[uc_44.1]{UC\_44.1} \\
	\hline
	RF-DE\_112 \hypertarget{rf-de_112} & Il Cliente deve poter inviare il comando \vr{/duplica\{xx\}} nella chat & \hyperref[uc_34]{UC\_34} \newline\hyperref[uc_44.3]{UC\_44.3} \\
	\hline
	RF-DE\_113 \hypertarget{rf-de_113} & Il Cliente deve poter inviare il comando \vr{/invia} nella chat & \hyperref[uc_26]{UC\_26} \newline \hyperref[uc_44.4]{UC\_44.4} \\
	\hline
	RF-DE\_114 \hypertarget{rf-de_114} & Il Cliente deve poter inviare il comando \vr{/annulla} nella chat & \hyperref[uc_22.1]{UC\_22.1} \newline \hyperref[uc_44.5]{UC\_44.5} \\
	\hline
	RF-DE\_115 \hypertarget{rf-de_115} & Il Cliente deve poter inviare il comando \vr{/comandi} nella chat & \hyperref[uc_44.6]{UC\_44.6} \\
	\hline
	RF-DE\_116 \hypertarget{rf-de_116} & L'Utente deve poter impostare dei filtri & \hyperref[uc_47]{UC\_47}\newline \hyperref[uc_48]{UC\_48} \\
	\hline
	RF-DE\_117 \hypertarget{rf-de_117} & L'Utente deve poter filtrare per data gli ordini presenti nello storico ordini & \hyperref[uc_47.1]{UC\_47.1}\newline \hyperref[uc_48.2]{UC\_48.2} \\
	\hline
	RF-DE\_118 \hypertarget{rf-de_118} & L'Utente deve poter visualizzare l'icona del filtro data a forma di calendario & \hyperref[uc_47.1]{UC\_47.1}\newline \hyperref[uc_48.2]{UC\_48.2} \\
	\hline
	RF-DE\_119 \hypertarget{rf-de_119} & L'Utente deve poter cliccare sull'icona del filtro data & \hyperref[uc_47.1]{UC\_47.1}\newline \hyperref[uc_48.2]{UC\_48.2} \\
	\hline
	RF-DE\_120 \hypertarget{rf-de_120} & Il filtro per data deve poter permettere di visualizzare tutti gli ordini successivi o uguali ad una data scelta & \hyperref[uc_47.1]{UC\_47.1}\newline \hyperref[uc_48.2]{UC\_48.2} \\
	\hline
	RF-DE\_121 \hypertarget{rf-de_121} & Il filtro per data deve poter permettere di visualizzare tutti gli ordini compresi tra due date scelte & \hyperref[uc_47.1]{UC\_47.1}\newline \hyperref[uc_48.2]{UC\_48.2} \\
	\hline
	RF-DE\_122 \hypertarget{rf-de_122} & L'Utente deve poter vedere il campo testuale del filtro prodotti & \hyperref[uc_47.2]{UC\_47.2}\newline \hyperref[uc_48.1]{UC\_48.1} \\
	\hline
	RF-DE\_123 \hypertarget{rf-de_123} & L'Utente deve poter scrivere nel campo testuale del filtro prodotti & \hyperref[uc_47.2]{UC\_47.2}\newline \hyperref[uc_48.1]{UC\_48.1} \\
	\hline
	RF-DE\_124 \hypertarget{rf-de_124} & L'Utente deve poter vedere un riquadro di prodotti presenti che iniziano o sono uguali alla sequenza di caratteri inserita & \hyperref[uc_47.2]{UC\_47.2}\newline \hyperref[uc_48.3]{UC\_48.3} \\
	\hline
	RF-DE\_125 \hypertarget{rf-de_125} & L'Utente, digitando il nome del prodotto, deve poter vedere un riquadro dei prodotti presenti che contengono quella sequenza di caratteri & \hyperref[uc_47.2]{UC\_47.2}\newline \hyperref[uc_48.3]{UC\_48.3} \\
	\hline
	RF-DE\_126 \hypertarget{rf-de_126} & L'Admin deve poter vedere il campo testuale del filtro Cliente & \hyperref[uc_48.1]{UC\_48.1} \\
	\hline
	RF-DE\_127 \hypertarget{rf-de_127} & L'Admin deve poter scrivere nel campo testuale del filtro Cliente & \hyperref[uc_48.1]{UC\_48.1} \\
	\hline
	RF-DE\_128 \hypertarget{rf-de_128} & L'Admin deve poter scegliere dal riquadro dei Clienti presenti il Cliente desiderato & \hyperref[uc_48.1]{UC\_48.1} \\
	\hline
	RF-DE\_129 \hypertarget{rf-de_129} & L'Admin deve poter vedere un riquadro di Clienti presenti che iniziano con la sequenza di caratteri inserita & \hyperref[uc_48.1]{UC\_48.1} \\
	\hline
	RF-DE\_130 \hypertarget{rf-de_130} & L'Admin deve poter vedere un riquadro di Clienti presenti che contengono la sequenza di caratteri inserita & \hyperref[uc_48.1]{UC\_48.1} \\
	\hline
	RF-DE\_131 \hypertarget{rf-de_131} & L'Admin deve poter scegliere dal riquadro dei Clienti presenti lo username del Cliente desiderato & \hyperref[uc_48.1]{UC\_48.1} \\
	\hline
	RF-DE\_132 \hypertarget{rf-de_132} & Il Cliente deve poter ricevere un avviso nel caso in cui invii più comandi nello stesso input & \hyperref[uc_49]{UC\_49} \\
	\hline
	RF-DE\_133 \hypertarget{rf-de_133} & Il Cliente deve poter ricevere un avviso nel caso in cui invii il comando con altro testo & \hyperref[uc_49]{UC\_49} \\
	\hline
	RF-DE\_134 \hypertarget{rf-de_134} & Il Cliente deve poter visualizzare le chat & \hyperref[uc_50]{UC\_50} \\
	\hline
	RF-DE\_135 \hypertarget{rf-de_135} & Il Cliente deve poter selezionare una chat & \hyperref[uc_51]{UC\_51} \\
	\hline
	RF-DE\_136 \hypertarget{rf-de_136} & Il Cliente deve poter cancellare una chat & \hyperref[uc_52]{UC\_52} \\
	\hline
	RF-DE\_137 \hypertarget{rf-de_137} & Il Cliente deve poter vedere il pulsante per cancellare la chat & \hyperref[uc_52]{UC\_52} \\
	\hline
	RF-DE\_138 \hypertarget{rf-de_138} & Il Cliente deve poter premere il pulsante per cancellare la chat & \hyperref[uc_52]{UC\_52} \\
	\hline
	RF-DE\_139 \hypertarget{rf-de_139} & Il Cliente deve poter confermare la cancellazione della chat & \hyperref[uc_52]{UC\_52} \\
	\hline
	RF-DE\_140 \hypertarget{rf-de_140} & Il Cliente deve poter visualizzare il pulsante di invio ordine dal carrello & \hyperref[uc_53]{UC\_53} \\
	\hline
	RF-DE\_141 \hypertarget{rf-de_141} & Il Cliente deve poter premere il pulsante abilitato di invio ordine & \hyperref[uc_53]{UC\_53} \\
	\hline
	\multicolumn{3}{c}{} \\
	\caption{Requisiti Funzionali desiderabili}\hypertarget{tab:funzionali_desiderabili}\\
\end{longtable}

	
\subsubsection{Requisiti Funzionali opzionali}
\renewcommand{\arraystretch}{1.15}
\begin{longtable}{|>{\raggedright}p{0.18\textwidth}|>{\raggedright\arraybackslash}p{0.5\textwidth}|>{\raggedright\arraybackslash}p{0.2\textwidth}|}
	\hline
	\rowcolor[gray]{0.9}
	\textbf{Codice} & \textbf{Descrizione} & \textbf{Fonti} \\
	\hline
	\endfirsthead
	
	\hline
	\rowcolor[gray]{0.9}
	\textbf{Codice} & \textbf{Descrizione} & \textbf{Fonti}  \\
	\hline
	\endhead
	
	\hline
	RF-OP\_01 \hypertarget{rf-op_1} & Tutte le informazioni riguardanti l'Utente utili ai fini del logging devono essere anonimizzate & \hyperref[uc_05]{UC\_5} \\
	\hline
	RF-OP\_02 \hypertarget{rf-op_2} & L'Admin deve poter creare nuovi utenti & \hyperref[uc_32]{UC\_32} \\
	\hline
	RF-OP\_03 \hypertarget{rf-op_3} & L'Admin deve poter visualizzare il pulsante relativo alla creazione di un nuovo Utente & \hyperref[uc_32]{UC\_32} \\
	\hline
	RF-OP\_04 \hypertarget{rf-op_4} & L'Admin deve poter schiacciare il pulsante relativo alla creazione di un nuovo Utente & \hyperref[uc_32]{UC\_32} \\
	\hline
	RF-OP\_05 \hypertarget{rf-op_5} & L'Admin deve poter selezionare il ruolo di quest'ultimo & \hyperref[uc_32.1]{UC\_32.1} \\
	\hline
	RF-OP\_06 \hypertarget{rf-op_6} & L'Admin deve poter selezionare il ruolo Cliente & \hyperref[uc_32.1]{UC\_32.1} \\
	\hline
	RF-OP\_07 \hypertarget{rf-op_7} & L'Admin deve poter selezionare il ruolo Admin & \hyperref[uc_32.1]{UC\_32.1} \\
	\hline
	RF-OP\_08 \hypertarget{rf-op_08} & L'Admin deve poter visualizzare il campo di input per inserire lo username & \hyperref[uc_32.2]{UC\_32.2} \\
	\hline
	RF-OP\_09 \hypertarget{rf-op_09} & L'Admin deve poter inserire lo username & \hyperref[uc_32.2]{UC\_32.2} \\
	\hline
	RF-OP\_10 \hypertarget{rf-op_10} & Lo username deve essere univoco & \hyperref[uc_32.2.2]{UC\_32.2.2} \\
	\hline
	RF-OP\_11 \hypertarget{rf-op_11} & L'Admin deve poter visualizzare una notifica di errore se lo username è già presente nel sistema & \hyperref[uc_32.2.2]{UC\_32.2.2} \\
	\hline
	RF-OP\_12 \hypertarget{rf-op_12} & L'Admin deve poter visualizzare il campo di input per l'inserimento dell'email & \hyperref[uc_32.3]{UC\_32.3} \\
	\hline
	RF-OP\_13 \hypertarget{rf-op_13} & L'Admin deve poter inserire l'email & \hyperref[uc_32.3]{UC\_32.3} \\
	\hline
	RF-OP\_14 \hypertarget{rf-op_14} & L'Admin deve poter visualizzare l'errore nel caso in cui l'email non fosse nel formato corretto & \hyperref[uc_32.3.1]{UC\_32.3.1} \\
	\hline
	RF-OP\_15 \hypertarget{rf-op_15} & L'Admin deve poter visualizzare l'errore nel caso in cui l'email sia già presente nel sistema & \hyperref[uc_32.3.2]{UC\_32.3.2} \\
	\hline
	RF-OP\_16 \hypertarget{rf-op_16} & Deve essere visualizzata una notifica a schermo che informa della corretta creazione dell'Utente & \hyperref[uc_32.4]{UC\_32.4} \\
	\hline
	RF-OP\_17 \hypertarget{rf-op_17} & L'Admin deve poter visualizzare la notifica di errore creazione utente & \hyperref[uc_33]{UC\_33}\\
	\hline
	\multicolumn{3}{c}{} \\
	\caption{Requisiti Funzionali opzionali}\hypertarget{tab:funzionali_opzionali}\\

\end{longtable}
