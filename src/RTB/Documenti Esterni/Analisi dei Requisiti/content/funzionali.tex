\subsection{Requisiti Funzionali}

\subsubsection{Requisiti Funzionali obbligatori}
\renewcommand{\arraystretch}{1.15}
\begin{longtable}{|>{\raggedright}p{0.15\textwidth}|>{\raggedright\arraybackslash}p{0.6\textwidth}|>{\raggedright\arraybackslash}p{0.25\textwidth}|}
	\hline
	\rowcolor[gray]{0.9}
	\textbf{Codice} & \textbf{Descrizione} & \textbf{Fonti} \\
	\hline
	\endfirsthead
	
	\hline
	\rowcolor[gray]{0.9}
	\textbf{Codice} & \textbf{Descrizione} & \textbf{Fonti}  \\
	\hline
	\endhead

	\hline
	\phantomsection
	\label{rf-ob_1}
	RF-OB\_01 & L'Utente deve poter visualizzare il campo di input per inserire lo username & \hyperref[uc_01.1]{UC\_01.1} \\
	\hline
	\phantomsection
	\label{rf-ob_2}
	RF-OB\_02 & L'Utente deve poter inserire lo username & \hyperref[uc_01.1]{UC\_01.1} \\
	\hline
	\phantomsection
	\label{rf-ob_3}
	RF-OB\_03 & Lo username deve essere univoco & \hyperref[uc_01.1.2]{UC\_01.1.2} \\
	\hline
	\phantomsection
	\label{rf-ob_4}
	RF-OB\_04 & L'Utente deve poter visualizzare una notifica di errore se lo username è già presente nel sistema & \hyperref[uc_01.1.2]{UC\_01.1.2} \\
	\hline
	\phantomsection
	\label{rf-ob_5}
	RF-OB\_05 & Lo username deve avere un massimo di 24 caratteri. & \hyperref[uc_01.1.1]{UC\_01.1.1}\newline \hyperref[uc_08]{UC\_08} \\
	\hline
	\phantomsection
	\label{rf-ob_6}
	RF-OB\_06 & L'Utente deve poter visualizzare una notifica di errore se lo username eccede il numero massimo di caratteri & \hyperref[uc_01.1.1]{UC\_01.1.1} \\
	\hline
	\phantomsection
	\label{rf-ob_7}
	RF-OB\_07 & L'Utente deve poter visualizzare il campo di input per inserire la password & \hyperref[uc_01.2]{UC\_01.2} \\
	\hline
	\phantomsection
	\label{rf-ob_8}
	RF-OB\_08 & L'Utente deve poter inserire la password & \hyperref[uc_01.2]{UC\_01.2}\\
	\hline
	\phantomsection
	\label{rf-ob_9}
	RF-OB\_09 & La password deve avere almeno 1 lettera maiuscola, 1 lettera minuscola, 1 numero, 1 carattere speciale e almeno 8 caratteri & \hyperref[uc_01.2]{UC\_01.2}\newline \hyperref[uc_01.2.1]{UC\_01.2.1}\newline \hyperref[uc_08]{UC\_08} \\
	\hline
	\phantomsection
	\label{rf-ob_10}
	RF-OB\_10 & La password deve avere un massimo di 24 caratteri & \hyperref[uc_01.2.1]{UC\_01.2.1} \\
	\hline
	\phantomsection
	\label{rf-ob_11}
	RF-OB\_11 & L'Utente deve poter visualizzare una notifica di errore se la password non è conforme ai criteri & \hyperref[uc_01.2.1]{UC\_01.2.1} \\
	\hline
	\phantomsection
	\label{rf-ob_12}
	RF-OB\_12 & L'Utente deve poter visualizzare una notifica di errore se la password supera 24 caratteri & \hyperref[uc_01.2.1]{UC\_01.2.1} \\
	\hline
	\phantomsection
	\label{rf-ob_13}
	RF-OB\_13 & L'Utente deve poter tornare alla login via link HTML. & \hyperref[uc_01.1.2]{UC\_01.1.2}\newline \hyperref[uc_01.4.2]{UC\_01.4.2} \\
	\hline
	\phantomsection
	\label{rf-ob_14}
	RF-OB\_14 & L'Utente deve poter visualizzare il campo di input della nuova password & \hyperref[uc_01.3]{UC\_01.3} \\
	\hline
	\phantomsection
	\label{rf-ob_15}
	RF-OB\_15 & L'Utente deve poter confermare la password tramite nuovo campo di input & \hyperref[uc_01.3]{UC\_01.3} \\
	\hline
	\phantomsection
	\label{rf-ob_16}
	RF-OB\_16 & L'Utente deve poter visualizzare una notifica di errore se la password e la sua conferma non coincidono & \hyperref[uc_01.3.1]{UC\_01.3.1} \\
	\hline
	\phantomsection
	\label{rf-ob_17}
	RF-OB\_17 & L'Utente deve poter visualizzare il campo di input per l'inserimento dell'email & \hyperref[uc_01.4]{UC\_01.4} \\
	\hline
	\phantomsection
	\label{rf-ob_18}
	RF-OB\_18 & L'Utente deve poter inserire l'email & \hyperref[uc_01.4]{UC\_01.4} \\
	\hline
	\phantomsection
	\label{rf-ob_19}
	RF-OB\_19 & L'email deve essere univoca & \hyperref[uc_01.4.2]{UC\_01.4.2} \\
	\hline
	\phantomsection
	\label{rf-ob_20}
	RF-OB\_20 & L'Utente deve poter visualizzare l'errore nel caso in cui l'email non fosse nel formato corretto & \hyperref[uc_01.4.1]{UC\_01.4.1} \\
	\hline
	\phantomsection
	\label{rf-ob_21}
	RF-OB\_21 & L'Utente deve poter visualizzare l'errore nel caso in cui l'email sia già presente nel sistema & \hyperref[uc_01.4.2]{UC\_01.4.2} \\
	\hline
	\phantomsection
	\label{rf-ob_22}
	RF-OB\_22 & L'Utente deve essere indirizzato alla pagina login una volta che la registrazione ha avuto successo. & \hyperref[uc_01]{UC\_01} \\
	\hline
	\phantomsection
	\label{rf-ob_23}
	RF-OB\_23 & L'Utente deve poter visualizzare il campo per inserire lo username & \hyperref[uc_02.1]{UC\_02.1} \\
	\hline
	\phantomsection
	\label{rf-ob_24}
	RF-OB\_24 & L'Utente deve poter inserire lo username & \hyperref[uc_02.1]{UC\_02.1} \\
	\hline
	\phantomsection
	\label{rf-ob_25}
	RF-OB\_25 & L'Utente deve poter visualizzare il campo per inserire la password & \hyperref[uc_02.2]{UC\_02.2} \\
	\hline
	\phantomsection
	\label{rf-ob_26}
	RF-OB\_26 & L'Utente deve poter inserire la password & \hyperref[uc_02.2]{UC\_02.2} \\
	\hline
	\phantomsection
	\label{rf-ob_27}
	RF-OB\_27 & Dev'essere data la possibilità di tornare alla registrazione via link. & \hyperref[uc_02]{UC\_02} \\
	\hline
	\phantomsection
	\label{rf-ob_28}
	RF-OB\_28 & L'Utente deve poter visualizzare il messaggio \vr{Username o password errati} se l'autenticazione fallisce & \hyperref[uc_03]{UC\_03} \\
	\hline
	\phantomsection
	\label{rf-ob_29}
	RF-OB\_29 & Deve poter avvenire un logout automatico a seguito della cancellazione account & \hyperref[uc_05]{UC\_05}\\
	\hline
	\phantomsection
	\label{rf-ob_30}
	RF-OB\_30 & L'Utente deve poter visualizzare il pulsante \vr{Logout} nella pagina del profilo & \hyperref[uc_06]{UC\_06}\\
	\hline
	\phantomsection
	\label{rf-ob_31}
	RF-OB\_31 & L'Utente deve poter effettuare il logout tramite il pulsante \vr{Logout} & \hyperref[uc_06]{UC\_06} \\
	\hline
	\phantomsection
	\label{rf-ob_32}
	RF-OB\_32 & Il Cliente deve poter inserire input testuali con lunghezza massima di 4096 caratteri. & \hyperref[uc_09]{UC\_09}\newline \hyperref[uc_10]{UC\_10} \\
	\hline
	\phantomsection
	\label{rf-ob_33}
	RF-OB\_33 & Il Cliente deve poter visualizzare un messaggio di errore al superamento di 4096 caratteri. & \hyperref[uc_10]{UC\_10} \\
	\hline
	\phantomsection
	\label{rf-ob_34}
	RF-OB\_34 & Il Cliente deve poter visualizzare il campo di input. & \hyperref[uc_09]{UC\_09}\newline \hyperref[uc_13]{UC\_13}\newline \hyperref[uc_15]{UC\_15}\newline \hyperref[uc_17]{UC\_17} \\
	\hline
	\phantomsection
	\label{rf-ob_35}
	RF-OB\_35 & Il Cliente deve poter visualizzare il tasto di invio & \hyperref[uc_09]{UC\_09}\newline \hyperref[uc_11]{UC\_11}\newline \hyperref[uc_13]{UC\_13}\newline \hyperref[uc_15]{UC\_15}\newline \hyperref[uc_17]{UC\_17} \\
	\hline
	\phantomsection
	\label{rf-ob_36}
	RF-OB\_36 & Il Cliente deve poter premere il tasto di invio & \hyperref[uc_09]{UC\_09}\newline \hyperref[uc_11]{UC\_11}\newline \hyperref[uc_13]{UC\_13}\newline \hyperref[uc_15]{UC\_15}\newline \hyperref[uc_17]{UC\_17} \\
	\hline
	\phantomsection
	\label{rf-ob_37}
	RF-OB\_37 & Il Cliente in assenza di contenuto di input non potrà premere il tasto di invio che sarà disabilitato & \hyperref[uc_09]{UC\_09}\newline \hyperref[uc_11]{UC\_11}\newline \hyperref[uc_13]{UC\_13}\newline \hyperref[uc_15]{UC\_15}\newline \hyperref[uc_17]{UC\_17} \\
	\hline
	\phantomsection
	\label{rf-ob_38}
	RF-OB\_38 & Il Cliente dopo l'invio dovrà visualizzare il campo di testo vuoto & \hyperref[uc_09]{UC\_09}\newline \hyperref[uc_11]{UC\_11}\newline \hyperref[uc_13]{UC\_13} \\
	\hline
	\phantomsection
	\label{rf-ob_39}
	RF-OB\_39 & Il Cliente deve poter visualizzare un pulsante con l'icona del microfono & \hyperref[uc_13]{UC\_13} \\
	\hline
	\phantomsection
	\label{rf-ob_40}
	RF-OB\_40 & Il Cliente deve poter cliccare il pulsante con l'icona del microfono per registrare messaggi audio & \hyperref[uc_13.1]{UC\_13.1} \\
	\hline
	\phantomsection
	\label{rf-ob_41}
	RF-OB\_41 & Il Cliente deve poter visualizzare la durata della registrazione & \hyperref[uc_13.1]{UC\_13.1} \\
	\hline
	\phantomsection
	\label{rf-ob_42}
	RF-OB\_42 & Il Cliente premendo nuovamente l'icona del microfono ferma la registrazione & \hyperref[uc_13]{UC\_13}\newline \hyperref[uc_13.1]{UC\_13.1} \\
	\hline
	\phantomsection
	\label{rf-ob_43}
	RF-OB\_43 & Il Cliente deve poter visualizzare un pulsante con icona \vr{clip} & \hyperref[uc_11.2]{UC\_11.2}\newline \hyperref[uc_13.2]{UC\_13.2} \\
	\hline
	\phantomsection
	\label{rf-ob_44}
	RF-OB\_44 & Il Cliente deve poter premere il pulsante per selezionare il file da caricare dal dispositivo & \hyperref[uc_11.2]{UC\_11.2}\newline \hyperref[uc_13.2]{UC\_13.2} \\
	\hline
	\phantomsection
	\label{rf-ob_45}
	RF-OB\_45 & Il Cliente cliccando sul pulsante deve poter aprire il dialogo di selezione file del sistema operativo. & \hyperref[uc_11.2]{UC\_11.2}\newline \hyperref[uc_13.2]{UC\_13.2} \\
	\hline
	\phantomsection
	\label{rf-ob_46}
	RF-OB\_46 & Il Cliente deve poter selezionare i file & \hyperref[uc_11.2]{UC\_11.2}\newline \hyperref[uc_13.2]{UC\_13.2} \\
	\hline
	\phantomsection
	\label{rf-ob_47}
	RF-OB\_47 & Il Cliente deve poter vedere il tasto \vr{Seleziona} nella selezione file & \hyperref[uc_11.2]{UC\_11.2}\newline \hyperref[uc_13.2]{UC\_13.2} \\
	\hline
	\phantomsection
	\label{rf-ob_48}
	RF-OB\_48 & Il Cliente deve poter inviare audio solo se <= 120 secondi. & \hyperref[uc_13]{UC\_13} \\
	\hline
	\phantomsection
	\label{rf-ob_49}
	RF-OB\_49 & Il Cliente deve poter visualizzare un messaggio di errore se il messaggio vocale supera i 120 secondi & \hyperref[uc_14]{UC\_14} \\
	\hline
	\phantomsection
	\label{rf-ob_50}
	RF-OB\_50 & Il Cliente deve poter inviare file audio con dimensione massima di 10MB in caso contrario ne sarà impedito l'invio & \hyperref[uc_13.1]{UC\_13.1}\newline \hyperref[uc_13.2]{UC\_13.2} \\
	\hline
	\phantomsection
	\label{rf-ob_51}
	RF-OB\_51 & Il Cliente deve poter visualizzare un messaggio di errore se le dimensioni dell'audio superano i 10MB & \hyperref[uc_14]{UC\_14} \\
	\hline
	\phantomsection
	\label{rf-ob_52}
	RF-OB\_52 & Il Cliente deve poter visualizzare un messaggio di errore se il messaggio vocale supera i 10MB & \hyperref[uc_14]{UC\_14} \\
	\hline
	\phantomsection
	\label{rf-ob_53}
	RF-OB\_53 & Il Cliente deve poter selezionare file audio in formato .mp3 & \hyperref[uc_13.2.1]{UC\_13.2.1}\\
	\hline
	\phantomsection
	\label{rf-ob_54}
	RF-OB\_54 & Il Cliente deve poter selezionare file audio in formato .m4a & \hyperref[uc_13.2.2]{UC\_13.2.2} \\
	\hline
	\phantomsection
	\label{rf-ob_55}
	RF-OB\_55 & Il Cliente deve poter selezionare file audio in formato .m4p & \hyperref[uc_13.2.3]{UC\_13.2.3} \\
	\hline
	\phantomsection
	\label{rf-ob_56}
	RF-OB\_56 & Il Cliente deve poter selezionare file audio in formato .wav. & \hyperref[uc_13.2.4]{UC\_13.2.4} \\
	\hline
	\phantomsection
	\label{rf-ob_57}
	RF-OB\_57 & Il Cliente deve poter visualizzare un messaggio di errore qualora i formati non rientrassero tra quelli accettati & \hyperref[uc_14]{UC\_14} \\
	\hline
	\phantomsection
	\label{rf-ob_58}
	RF-OB\_58 & Il Cliente deve poter visualizzare un messaggio di errore qualora il chatbot non riconosca gli articoli & \hyperref[uc_14]{UC\_14} \\
	\hline
	\phantomsection
	\label{rf-ob_59}
	RF-OB\_59 & Il Cliente deve poter aggiungere una massimo di 10 prodotti per volta al carrello tramite input & \hyperref[uc_15]{UC\_15} \\
	\hline
	\phantomsection
	\label{rf-ob_60}
	RF-OB\_60 & Il Cliente deve ricevere conferma dell'aggiunta dei prodotti & \hyperref[uc_15]{UC\_15} \\
	\hline
	\phantomsection
	\label{rf-ob_61}
	RF-OB\_61 & Il Cliente deve visualizzare un messaggio di errore nel caso in cui tenti di aggiungere più di 10 articoli & \hyperref[uc_16]{UC\_16} \\
	\hline
	\phantomsection
	\label{rf-ob_62}
	RF-OB\_62 & Il Cliente deve poter rimuovere articoli tramite input & \hyperref[uc_17]{UC\_17} \\
	\hline
	\phantomsection
	\label{rf-ob_63}
	RF-OB\_63 & Il Cliente deve ricevere una conferma della rimozione degli articoli & \hyperref[uc_17]{UC\_17} \\
	\hline
	\phantomsection
	\label{rf-ob_64}
	RF-OB\_64 & Il Cliente deve essere avvisato nel caso in cui richiedesse la rimozione di alcuni articoli dal carrello ma quest'ultimo fosse vuoto. & \hyperref[uc_18]{UC\_18} \\
	\hline
	\phantomsection
	\label{rf-ob_65}
	RF-OB\_65 & Il Cliente deve poter inviare il comando testuale \vr{/carrello} & \hyperref[uc_19.1]{UC\_19.1} \\
	\hline
	\phantomsection
	\label{rf-ob_66}
	RF-OB\_66 & Il Cliente deve poter visualizzare il prezzo di ogni articolo & \hyperref[uc_19.1.1]{UC\_19.1.1} \\
	\hline
	\phantomsection
	\label{rf-ob_67}
	RF-OB\_67 & Il Cliente deve poter visualizzare il nome di ogni articolo & \hyperref[uc_19.1.2]{UC\_19.1.2} \\
	\hline
	\phantomsection
	\label{rf-ob_68}
	RF-OB\_68 & Il Cliente deve poter visualizzare la quantità di ogni articolo & \hyperref[uc_19.1.3]{UC\_19.1.3} \\
	\hline
	\phantomsection
	\label{rf-ob_69}
	RF-OB\_69 & Il Cliente deve poter visualizzare i singoli elementi del carrello & \hyperref[uc_19.1]{UC\_19.1} \\
	\hline
	\phantomsection
	\label{rf-ob_70}
	RF-OB\_70 & Il Cliente deve poter visualizzare l'anteprima del carrello & \hyperref[uc_19]{UC\_19} \\
	\hline
	\phantomsection
	\label{rf-ob_71}
	RF-OB\_71 & Il Cliente deve poter ricevere riceve un avviso nel caso in cui nel carrello non ci fossero elementi & \hyperref[uc_20]{UC\_20} \\
	\hline
	\phantomsection
	\label{rf-ob_72}
	RF-OB\_72 & Il Cliente deve poter inviare l'ordine & \hyperref[uc_26]{UC\_26} \\
	\hline
	\phantomsection
	\label{rf-ob_73}
	RF-OB\_73 & Il Cliente deve poter inserire il comando \vr{/invia} nella chat per avviare la conferma dell'ordine & \hyperref[uc_26.1]{UC\_26.1} \\
	\hline
	\phantomsection
	\label{rf-ob_74}
	RF-OB\_74 & Il Cliente deve poter visualizzare il riepilogo ordine nella chat & \hyperref[uc_26.2]{UC\_26.2} \\
	\hline
	\phantomsection
	\label{rf-ob_75}
	RF-OB\_75 & Il Cliente a seguito del messaggio riepilogativo deve decidere se annullare o confermare l'invio dell'ordine & \hyperref[uc_26.2]{UC\_26.2} \\
	\hline
	\phantomsection
	\label{rf-ob_76}
	RF-OB\_76 & Il Cliente deve poter confermare l'invio dell'ordine con un input & \hyperref[uc_26.3]{UC\_26.3}\newline \hyperref[uc_34.1]{UC\_34.1} \\
	\hline
	\phantomsection
	\label{rf-ob_77}
	RF-OB\_77 & Il Cliente deve poter visualizzare la conferma dell'invio dell'ordine & \hyperref[uc_26.4]{UC\_26.4}\\
	\hline
	\phantomsection
	\label{rf-ob_78}
	RF-OB\_78 & Il Cliente deve poter annullare l'invio dell'ordine con un input & \hyperref[uc_26.5]{UC\_26.5}\\
	\hline
	\phantomsection
	\label{rf-ob_79}
	RF-OB\_79 & Se il carrello è vuoto dopo il comando \vr{/invia} il Cliente visualizza un messaggio di errore che avvisa che il carrello è vuoto. & \hyperref[uc_27]{UC\_27} \\
	\hline
	\phantomsection
	\label{rf-ob_80}
	RF-OB\_80 & Il Cliente deve poter visualizzare un messaggio riguardante l'ambiguità dei prodotti & \hyperref[uc_21]{UC\_21} \\
	\hline
	\phantomsection
	\label{rf-ob_81}
	RF-OB\_81 & Il Cliente deve poter riformulare l'ordine per prodotti ambigui & \hyperref[uc_21.1]{UC\_21.1} \\
	\hline
	\phantomsection
	\label{rf-ob_82}
	RF-OB\_82 & Il Cliente deve poter rispondere \vr{/annulla} per uscire dalla disambiguazione. & \hyperref[uc_22.1]{UC\_22.1} \\
	\hline
	\phantomsection
	\label{rf-ob_83}
	RF-OB\_83 & Il Cliente deve poter ricevere conferma dell'annullamento della disambiguazione & \hyperref[uc_22.2]{UC\_22.2} \\
	\hline
	\phantomsection
	\label{rf-ob_84}
	RF-OB\_84 & Il Cliente deve poter vedere il pulsante \vr{Nuova chat} per iniziare una nuova sessione & \hyperref[uc_28]{UC\_28} \\
	\hline
	\phantomsection
	\label{rf-ob_85}
	RF-OB\_85 & L'Utente deve poter cliccare sul pulsante \vr{Nuova chat} per iniziare una nuova sessione. & \hyperref[uc_28]{UC\_28} \\
	\hline
	\phantomsection
	\label{rf-ob_86}
	RF-OB\_86 & Il Cliente deve poter visualizzare una notifica di errore nel caso in cui il prodotto non venga duplicato correttamente & \hyperref[uc_35]{UC\_35} \\
	\hline
	\phantomsection
	\label{rf-ob_87}
	RF-OB\_87 & L'Utente deve poter visualizzare il pulsante \vr{visualizza storico ordini} & \hyperref[uc_36]{UC\_36} \newline \hyperref[uc_45]{UC\_45} \\
	\hline
	\phantomsection
	\label{rf-ob_88}
	RF-OB\_88 & L'Utente deve poter schiacciare il pulsante \vr{visualizza storico ordini} & \hyperref[uc_36]{UC\_36} \newline \hyperref[uc_45]{UC\_45} \\
	\hline
	\phantomsection
	\label{rf-ob_89}
	RF-OB\_89 & L'Admin deve poter visualizzare una pagina separata contenente lo storico totale degli ordini di tutti i Clienti & \hyperref[uc_36]{UC\_36} \\
	\hline
	\phantomsection
	\label{rf-ob_90}
	RF-OB\_90 & Il Cliente deve poter visualizzare una pagina separata contente lo storico totale dei suoi ordini & \hyperref[uc_45]{UC\_45} \\
	\hline
	\phantomsection
	\label{rf-ob_91}
	RF-OB\_91 & La lista completa degli ordini deve essere visualizzata in pagine contenenti al massimo 10 ordini ciascuna & \hyperref[uc_36]{UC\_36} \newline \hyperref[uc_45]{UC\_45} \\
	\hline
	\phantomsection
	\label{rf-ob_92}
	RF-OB\_92 & Il caricamento di una pagina non implica il caricamento delle pagine successive & \hyperref[uc_36]{UC\_36} \newline \hyperref[uc_45]{UC\_45} \\
	\hline
	\phantomsection
	\label{rf-ob_93}
	RF-OB\_93 & L'Utente deve poter visualizzare il singolo ordine nello storico & \hyperref[uc_36.1]{UC\_36.1} \newline \hyperref[uc_45.1]{UC\_45.1} \\
	\hline
	\phantomsection
	\label{rf-ob_94}
	RF-OB\_94 & L'Utente deve poter visualizzare il codice ordine & \hyperref[uc_36.1.1]{UC\_36.1.1} \newline \hyperref[uc_45.1.1]{UC\_45.1.1} \\
	\hline
	\phantomsection
	\label{rf-ob_95}
	RF-OB\_95 & L'Utente deve poter visualizzare la data dell'ordine & \hyperref[uc_36.1.2]{UC\_36.1.2} \newline \hyperref[uc_45.1.2]{UC\_45.1.2} \\
	\hline
	\phantomsection
	\label{rf-ob_96}
	RF-OB\_96 & L'Admin deve poter visualizzare lo username del Cliente & \hyperref[uc_36.1.3]{UC\_36.1.3} \\
	\hline
	\phantomsection
	\label{rf-ob_97}
	RF-OB\_97 & L'Utente deve poter visualizzare il dettaglio di un ordine specifico & \hyperref[uc_37]{UC\_37} \newline \hyperref[uc_46]{UC\_46} \\
	\hline
	\phantomsection
	\label{rf-ob_98}
	RF-OB\_98 & L'Admin deve poter visualizzare lo username del Cliente & \hyperref[uc_37.1]{UC\_37.1} \\
	\hline
	\phantomsection
	\label{rf-ob_99}
	RF-OB\_99 & L'Utente deve poter visualizzare i prodotti dell'ordine & \hyperref[uc_37.2]{UC\_37.2} \newline \hyperref[uc_46.1]{UC\_46.1} \\
	\hline
	\phantomsection
	\label{rf-ob_100}
	RF-OB\_100 & L'Utente deve poter visualizzare il nome dei prodotti & \hyperref[uc_37.2.1]{UC\_37.2.1} \newline \hyperref[uc_46.1.1]{UC\_46.1.1} \\
	\hline
	\phantomsection
	\label{rf-ob_101}
	RF-OB\_101 & L'Utente deve poter visualizzare la descrizione dei prodotti & \hyperref[uc_37.2.2]{UC\_37.2.2} \newline \hyperref[uc_46.1.2]{UC\_46.1.2} \\
	\hline
	\phantomsection
	\label{rf-ob_102}
	RF-OB\_102 & L'Utente deve poter visualizzare la quantità dei prodotti & \hyperref[uc_37.2.3]{UC\_37.2.3} \newline \hyperref[uc_46.1.3]{UC\_46.1.3} \\
	\hline
	\phantomsection
	\label{rf-ob_103}
	RF-OB\_103 & L'Utente deve poter visualizzare la data dell'ordine & \hyperref[uc_37.3]{UC\_37.3} \newline \hyperref[uc_46.2]{UC\_46.2} \\
	\hline
	\phantomsection
	\label{rf-ob_104}
	RF-OB\_104 & L'Utente deve poter visualizzare il numero dell'ordine & \hyperref[uc_37.4]{UC\_37.4} \newline \hyperref[uc_46.3]{UC\_46.3} \\
	\hline
	\phantomsection
	\label{rf-ob_105}
	RF-OB\_105 & Il Cliente deve poter visualizzare il pulsante per duplicare l'ordine & \hyperref[uc_38]{UC\_38} \\
	\hline
	\phantomsection
	\label{rf-ob_106}
	RF-OB\_106 & Il Cliente deve poter cliccare sul pulsante per duplicare l'ordine & \hyperref[uc_38]{UC\_38} \\
	\hline
	\phantomsection
	\label{rf-ob_107}
	RF-OB\_107 & Il Cliente deve poter confermare la duplicazione dell'ordine & \hyperref[uc_38.1]{UC\_38.1} \\
	\hline
	\phantomsection
	\label{rf-ob_108}
	RF-OB\_108 & Il Cliente deve poter visualizzare la notifica di errore duplicazione ordine & \hyperref[uc_39]{UC\_39} \\
	\hline
	\multicolumn{3}{c}{} \\
	\caption{Requisiti Funzionali obbligatori}\label{tab:funzionali_obbligatori}\\
\end{longtable}

	
\subsubsection{Requisiti Funzionali desiderabili}
\renewcommand{\arraystretch}{1.15}
\begin{longtable}{|>{\raggedright}p{0.18\textwidth}|>{\raggedright\arraybackslash}p{0.6\textwidth}|>{\raggedright\arraybackslash}p{0.25\textwidth}|}
	\hline
	\rowcolor[gray]{0.9}
	\textbf{Codice} & \textbf{Descrizione} & \textbf{Fonti} \\
	\hline
	\endfirsthead
	
	\hline
	\rowcolor[gray]{0.9}
	\textbf{Codice} & \textbf{Descrizione} & \textbf{Fonti}  \\
	\hline
	\endhead
	
	\hline
	\phantomsection
	\label{rf-de_1}
	RF-DE\_01 & L'Utente deve poter visualizzare il link HTML per la password dimenticata & \hyperref[uc_04]{UC\_04} \\
	\hline
	\phantomsection
	\label{rf-de_2}
	RF-DE\_02 & L'Utente deve poter cliccare il link HTML per la password dimenticata & \hyperref[uc_04]{UC\_04} \\
	\hline
	\phantomsection
	\label{rf-de_3}
	RF-DE\_03 & L'Utente deve poter visualizzare il form per inserire l'indirizzo mail che ha usato quando si è registrato & \hyperref[uc_04.1]{UC\_04.1} \\
	\hline
	\phantomsection
	\label{rf-de_4}
	RF-DE\_04 & L'Utente deve ricevere via mail la password generata automaticamente & \hyperref[uc_04.1]{UC\_04.1} \\
	\hline
	\phantomsection
	\label{rf-de_5}
	RF-DE\_05 & L'Utente deve ricevere un messaggio di errore nel caso in cui l'email non fosse valida & \hyperref[uc_04.1.1]{UC\_04.1.1} \\
	\hline
	\phantomsection
	\label{rf-de_6}
	RF-DE\_06 & L'Utente deve ricevere un messaggio di errore nel caso in cui l'email non fosse presente nel sistema & \hyperref[uc_04.1.2]{UC\_04.1.2} \\
	\hline
	\phantomsection
	\label{rf-de_7}
	RF-DE\_07 & L'Utente deve poter visualizzare il pulsante \vr{Elimina Account} nella pagina delle informazioni del profilo & \hyperref[uc_05]{UC\_05} \\
	\hline
	\phantomsection
	\label{rf-de_8}
	RF-DE\_08 & L'Utente deve poter premere il pulsante per eliminare il suo account & \hyperref[uc_05]{UC\_05} \\
	\hline
	\phantomsection
	\label{rf-de_9}
	RF-DE\_09 & L'Utente deve poter visualizzare la conferma dell'eliminazione dell'account & \hyperref[uc_05.1]{UC\_05.1} \\
	\hline
	\phantomsection
	\label{rf-de_10}
	RF-DE\_10 & L'Utente deve poter visualizzare nella home un'icona che porta alla pagina delle info del profilo & \hyperref[uc_07]{UC\_07} \\
	\hline
	\phantomsection
	\label{rf-de_11}
	RF-DE\_11 & L'Utente deve poter cliccare sul link, rappresentato da un'icona, che porta sulla pagina delle info del proprio profilo & \hyperref[uc_07]{UC\_07} \\
	\hline
	\phantomsection
	\label{rf-de_12}
	RF-DE\_12 & L'Utente deve poter visualizzare il link alla pagina dello storico ordini nelle info del proprio profilo & \hyperref[uc_07]{UC\_07} \\
	\hline
	\phantomsection
	\label{rf-de_13}
	RF-DE\_13 & L'Utente deve poter visualizzare il proprio username nella pagina delle info del proprio profilo & \hyperref[uc_07.1]{UC\_07.1} \\
	\hline
	\phantomsection
	\label{rf-de_14}
	RF-DE\_14 & L'Utente deve poter visualizzare la propria ragione sociale nelle info del proprio profilo & \hyperref[uc_07.2]{UC\_07.2} \\
	\hline
	\phantomsection
	\label{rf-de_15}
	RF-DE\_15 & L'Utente deve poter visualizzare il proprio indirizzo email nelle info del proprio profilo & \hyperref[uc_07.3]{UC\_07.3} \\
	\hline
	\phantomsection
	\label{rf-de_16}
	RF-DE\_16 & L'Utente deve poter visualizzare il pulsante \vr{Reimposta password} & \hyperref[uc_08]{UC\_08} \\
	\hline
	\phantomsection
	\label{rf-de_17}
	RF-DE\_17 & L'Utente deve poter cliccare il pulsante \vr{Reimposta password} & \hyperref[uc_08]{UC\_08} \\
	\hline
	\phantomsection
	\label{rf-de_18}
	RF-DE\_18 & L'Utente deve poter visualizzare il campo di input \vr{Vecchia password} & \hyperref[uc_08.1]{UC\_08.1} \\
	\hline
	\phantomsection
	\label{rf-de_19}
	RF-DE\_19 & L'Utente deve poter inserire la vecchia password nel campo di input \vr{Vecchia password} & \hyperref[uc_08.1]{UC\_08.1} \\
	\hline
	\phantomsection
	\label{rf-de_20}
	RF-DE\_20 & L'Utente deve poter visualizzare un messaggio di errore nel caso in cui l'inserimento della vecchia password non corrispondesse con la password presente nel sistema & \hyperref[uc_08.1.1]{UC\_08.1.1} \\
	\hline
	\phantomsection
	\label{rf-de_21}
	RF-DE\_21 & L'Utente deve poter visualizzare il campo di input \vr{Nuova password} & \hyperref[uc_08.2]{UC\_08.2} \\
	\hline
	\phantomsection
	\label{rf-de_22}
	RF-DE\_22 & L'Utente deve poter inserire la nuova password nel campo di input \vr{Nuova password} & \hyperref[uc_08.2]{UC\_08.2} \\
	\hline
	\phantomsection
	\label{rf-de_23}
	RF-DE\_23 & L'Utente deve poter visualizzare un messaggio di errore nel caso in cui la nuova password inserita non fosse conforme ai criteri imposti & \hyperref[uc_08.2.1]{UC\_08.2.1} \\
	\hline
	\phantomsection
	\label{rf-de_24}
	RF-DE\_24 & L'Utente deve poter visualizzare il campo di input \vr{Conferma nuova password} & \hyperref[uc_08.3]{UC\_08.3} \\
	\hline
	\phantomsection
	\label{rf-de_25}
	RF-DE\_25 & L'Utente deve poter confermare la nuova password nel campo di input \vr{Conferma nuova password} & \hyperref[uc_08.3]{UC\_08.3} \\
	\hline
	\phantomsection
	\label{rf-de_26}
	RF-DE\_26 & L'Utente deve poter visualizzare un messaggio di errore nel caso in cui la conferma della nuova password non corrisponda alla nuova password inserita nel campo precedente & \hyperref[uc_08.3.1]{UC\_08.3.1} \\
	\hline
	\phantomsection
	\label{rf-de_27}
	RF-DE\_27 & L'Utente deve essere notificato qualora il cambio password vada a buon fine tramite un messaggio di conferma. & \hyperref[uc_08.3]{UC\_08.3} \\
	\hline
	\phantomsection
	\label{rf-de_28}
	RF-DE\_28 & Durante la digitazione il Cliente deve poter visualizzare un contatore x/4096 vicino al campo di testo & \hyperref[uc_09]{UC\_09} \\
	\hline
	\phantomsection
	\label{rf-de_29}
	RF-DE\_29 & Il Cliente deve poter visualizzare il pulsante di caricamento file immagine (icona fotocamera). & \hyperref[uc_11]{UC\_11} \\
	\hline
	\phantomsection
	\label{rf-de_30}
	RF-DE\_30 & Il Cliente deve poter cliccare l'icona della fotocamera & \hyperref[uc_11]{UC\_11} \\
	\hline
	\phantomsection
	\label{rf-de_31}
	RF-DE\_31 & Il Cliente a seguito del click sull'icona a forma di clip deve poter vedere un dropdown tra cui: "Seleziona dal dispositivo" oppure "Scatta la foto". & \hyperref[uc_11]{UC\_11} \\
	\hline
	\phantomsection
	\label{rf-de_32}
	RF-DE\_32 & Il Cliente deve poter selezionare \vr{Seleziona dal dispositivo} & \hyperref[uc_11.2]{UC\_11.2} \\
	\hline
	\phantomsection
	\label{rf-de_33}
	RF-DE\_33 & Se Il Cliente seleziona \vr{Seleziona dal dispositivo} deve poter visualizzare i media nel dispositivo & \hyperref[uc_11.2]{UC\_11.2} \\
	\hline
	\phantomsection
	\label{rf-de_34}
	RF-DE\_34 & Il Cliente deve poter selezionare \vr{Scatta foto}. & \hyperref[uc_11.1]{UC\_11.1} \\
	\hline
	\phantomsection
	\label{rf-de_35}
	RF-DE\_35 & Alla selezione \vr{Scatta foto} il Cliente visualizza l'attivazione della fotocamera. & \hyperref[uc_11.1]{UC\_11.1} \\
	\hline
	\phantomsection
	\label{rf-de_36}
	RF-DE\_36 & Il Cliente deve poter inviare immagini con dimensione massima di 15MB e con risoluzione pari o superiore a 800x600 pixel. & \hyperref[uc_11]{UC\_11} \\
	\hline
	\phantomsection
	\label{rf-de_37}
	RF-DE\_37 & Nel caso in cui le immagini superino i 15MB abbiano risoluzione inferiore di 800x600 pixel sarà visibile al Cliente un messaggio di errore \vr{L'immagine non rispetta i requisiti richiesti} & \hyperref[uc_11]{UC\_11}\newline \hyperref[uc_12]{UC\_12} \\
	\hline
	\phantomsection
	\label{rf-de_38}
	RF-DE\_38 & Il Cliente deve premere il pulsante di invio per confermare l'invio dell'immagine. & \hyperref[uc_11]{UC\_11} \\
	\hline
	\phantomsection
	\label{rf-de_39}
	RF-DE\_39 & Il Cliente deve poter visualizza un messaggio \vr{Non sono riuscito a rilevare il testo nell'immagine} se il modello non riesce a rilevare testo & \hyperref[uc_12]{UC\_12} \\
	\hline
	\phantomsection
	\label{rf-de_40}
	RF-DE\_40 & Il Cliente deve poter selezionare immagini nei formati .jpg & \hyperref[uc_11.2.1]{UC\_11.2.1} \\
	\hline
	\phantomsection
	\label{rf-de_41}
	RF-DE\_41 & Il Cliente deve poter selezionare immagini nei formati .jpeg & \hyperref[uc_11.2.2]{UC\_11.2.2} \\
	\hline
	\phantomsection
	\label{rf-de_42}
	RF-DE\_42 & Il Cliente deve poter selezionare immagini nei formati .png & \hyperref[uc_11.2.3]{UC\_11.2.3} \\
	\hline
	\phantomsection
	\label{rf-de_43}
	RF-DE\_43 & Il Cliente deve poter visionare l'anteprima dell'immagine caricata. & \hyperref[uc_11]{UC\_11} \\
	\hline
	\phantomsection
	\label{rf-de_44}
	RF-DE\_44 & Il Cliente deve poter visualizzare il pulsante a fianco all'ultima risposta \vr{lascia Feedback}. & \hyperref[uc_23]{UC\_23} \\
	\hline
	\phantomsection
	\label{rf-de_45}
	RF-DE\_45 & Il Cliente deve poter premere il pulsante affianco ad ogni risposta \vr{lascia Feedback} per fornire feedback all'AI & \hyperref[uc_23]{UC\_23} \\
	\hline
	\phantomsection
	\label{rf-de_46}
	RF-DE\_46 & Il Cliente premendo su \vr{lascia Feedback} deve visualizzare l'apertura di un form in una nuova pagina & \hyperref[uc_23]{UC\_23} \\
	\hline
	\phantomsection
	\label{rf-de_47}
	RF-DE\_47 & Il Cliente deve vedere il dropdown relativo alla tipologia di feedback: \vr{Prodotto sbagliato}, \vr{Quantità errata}, \vr{Incomprensione}, \vr{Altro}. & \hyperref[uc_23.1]{UC\_23.1} \\
	\hline
	\phantomsection
	\label{rf-de_48}
	RF-DE\_48 & Il Cliente deve selezionare obbligatoriamente una tipologia: \vr{Prodotto sbagliato}, \vr{Quantità errata}, \vr{Incomprensione}, \vr{Altro}. & \hyperref[uc_23.1]{UC\_23.1} \\
	\hline
	\phantomsection
	\label{rf-de_49}
	RF-DE\_49 & Il Cliente visualizza l'input testuale della descrizione. & \hyperref[uc_23.2]{UC\_23.2}\newline \hyperref[uc_24.2]{UC\_24.2} \\
	\hline
	\phantomsection
	\label{rf-de_50}
	RF-DE\_50 & Il Cliente deve inserire obbligatoriamente una descrizione. & \hyperref[uc_23.2]{UC\_23.2}\newline \hyperref[uc_24.2]{UC\_24.2} \\
	\hline
	\phantomsection
	\label{rf-de_51}
	RF-DE\_51 & Il Cliente nella descrizione deve poter inserire un massimo di 300 caratteri. & \hyperref[uc_23.2]{UC\_23.2}\newline \hyperref[uc_24.2]{UC\_24.2} \\
	\hline
	\phantomsection
	\label{rf-de_52}
	RF-DE\_52 & Il Cliente deve vedere un contatore x/300 vicino al campo di testo & \hyperref[uc_23.2]{UC\_23.2}\newline \hyperref[uc_24.2]{UC\_24.2} \\
	\hline
	\phantomsection
	\label{rf-de_53}
	RF-DE\_53 & Al superamento dei 300 caratteri il Cliente non potrà più scrivere nell'input & \hyperref[uc_23.2]{UC\_23.2}\newline \hyperref[uc_24.2]{UC\_24.2} \\
	\hline
	\phantomsection
	\label{rf-de_54}
	RF-DE\_54 & Il Cliente visualizza in fondo alla pagina il pulsante \vr{Invia feedback} & \hyperref[uc_23.3]{UC\_23.3} \\
	\hline
	\phantomsection
	\label{rf-de_55}
	RF-DE\_55 & Il Cliente deve poter premere il pulsante \vr{Invia feedback} & \hyperref[uc_23.3]{UC\_23.3} \\
	\hline
	\phantomsection
	\label{rf-de_56}
	RF-DE\_56 & Il Cliente deve poter ricevere la notifica di la conferma di invio feedback. & \hyperref[uc_23.3]{UC\_23.3} \\
	\hline
	\phantomsection
	\label{rf-de_57}
	RF-DE\_57 & Il Cliente deve poter vedere nella home il pulsante \vr{Segnala un problema} & \hyperref[uc_24]{UC\_24} \\
	\hline
	\phantomsection
	\label{rf-de_58}
	RF-DE\_58 & Il Cliente deve poter cliccare sul pulsante \vr{Segnala un problema} & \hyperref[uc_24]{UC\_24} \\
	\hline
	\phantomsection
	\label{rf-de_59}
	RF-DE\_59 & Il Cliente deve visualizzare il dropdown: \vr{Bug}, \vr{Richiesta di supporto}, \vr{Suggerimento} & \hyperref[uc_24.1]{UC\_24.1} \\
	\hline
	\phantomsection
	\label{rf-de_60}
	RF-DE\_60 & Il Cliente deve poter selezionare il tipo da dropdown & \hyperref[uc_24.1]{UC\_24.1} \\
	\hline
	\phantomsection
	\label{rf-de_61}
	RF-DE\_61 & Il Cliente deve poter visualizzare il tasto \vr{Invia segnalazione} & \hyperref[uc_24.3]{UC\_24.3} \\
	\hline
	\phantomsection
	\label{rf-de_62}
	RF-DE\_62 & Il Cliente deve poter inviare il form cliccando \vr{Invia segnalazione} & \hyperref[uc_24.3]{UC\_24.3} \\
	\hline
	\phantomsection
	\label{rf-de_63}
	RF-DE\_63 & Il Cliente deve poter ricevere una notifica di conferma dell'invio con un messaggio \vr{Ticket creato con successo} & \hyperref[uc_24.4]{UC\_24.4} \\
	\hline
	\phantomsection
	\label{rf-de_64}
	RF-DE\_64 & L'Admin deve poter visualizzare la pagina web delle performance. & \hyperref[uc_25]{UC\_25} \\
	\hline
	\phantomsection
	\label{rf-de_65}
	RF-DE\_65 & L'Admin deve poter visualizzare il link per la pagina delle performance in qualsiasi pagina & \hyperref[uc_25]{UC\_25} \\
	\hline
	\phantomsection
	\label{rf-de_66}
	RF-DE\_66 & L'Admin deve poter schiacciare il link per la pagina delle performance in qualsiasi pagina & \hyperref[uc_25]{UC\_25} \\
	\hline
	\phantomsection
	\label{rf-de_67}
	RF-DE\_67 & L'Admin deve poter visualizzare il tempo medio di risposta del chatbot & \hyperref[uc_25.1]{UC\_25.1} \\
	\hline
	\phantomsection
	\label{rf-de_68}
	RF-DE\_68 & L'Admin deve poter visualizzare il tempo medio di permanenza nella web-app & \hyperref[uc_25.2]{UC\_25.2} \\
	\hline
	\phantomsection
	\label{rf-de_69}
	RF-DE\_69 & L'Admin deve poter visualizzare un pulsante relativo all'esportazione del log & \hyperref[uc_29]{UC\_29} \\
	\hline
	\phantomsection
	\label{rf-de_70}
	RF-DE\_70 & L'Admin deve poter schiacciare il pulsante relativo all'esportazione del log & \hyperref[uc_29]{UC\_29} \\
	\hline
	\phantomsection
	\label{rf-de_71}
	RF-DE\_71 & L'Admin deve poter ottenere il file di log in formato JSON & \hyperref[uc_29]{UC\_29} \\
	\hline
	\phantomsection
	\label{rf-de_72}
	RF-DE\_72 & L'Admin deve poter ottenere il file JSON dello storico ordini & \hyperref[uc_30]{UC\_30} \\
	\hline
	\phantomsection
	\label{rf-de_73}
	RF-DE\_73 & L'Admin deve poter visualizzare il tasto \vr{Esporta} nella pagina dello storico ordini & \hyperref[uc_30]{UC\_30} \\
	\hline
	\phantomsection
	\label{rf-de_74}
	RF-DE\_74 & L'Admin deve poter schiacciare il tasto \vr{Esporta} nella pagina dello storico ordini & \hyperref[uc_30]{UC\_30} \\
	\hline
	\phantomsection
	\label{rf-de_75}
	RF-DE\_75 & L'Admin deve poter visualizzare la conferma dell'esportazione dello storico ordini & \hyperref[uc_30.1]{UC\_30.1} \\
	\hline
	\phantomsection
	\label{rf-de_76}
	RF-DE\_76 & L'Admin deve poter visualizzare un errore nel caso in cui non fosse stato possibile scaricare nel proprio dispositivo il file JSON relativo allo storico ordini. & \hyperref[uc_31]{UC\_31} \\
	\hline
	\phantomsection
	\label{rf-de_77}
	RF-DE\_77 & Il file JSON esportato deve contenere il campo id [integer]: identificativo univoco dell'ordine & \hyperref[uc_30]{UC\_30} \\
	\hline
	\phantomsection
	\label{rf-de_78}
	RF-DE\_78 & Il file JSON esportato deve contenere il campo cod\_cli [integer]: identificativo univoco del cliente & \hyperref[uc_30]{UC\_30} \\
	\hline
	\phantomsection
	\label{rf-de_79}
	RF-DE\_79 & Il file JSON esportato deve contenere il campo cod\_art [varchar(13)]: identificativo univoco dell'articolo & \hyperref[uc_30]{UC\_30} \\
	\hline
	\phantomsection
	\label{rf-de_80}
	RF-DE\_80 & Il file JSON esportato deve contenere il campo data\_ord [date]: data di inserimento dell'ordine & \hyperref[uc_30]{UC\_30} \\
	\hline
	\phantomsection
	\label{rf-de_81}
	RF-DE\_81 & Il file JSON esportato deve contenere il campo qta\_ordinata [float]: quantità che l'Utente ha ordinato nell'unità di misura dell'articolo & \hyperref[uc_30]{UC\_30} \\
	\hline
	\phantomsection
	\label{rf-de_82}
	RF-DE\_82 & Il file JSON esportato deve contenere il campo rif [integer/string]: rappresenta un riferimento al testo originario & \hyperref[uc_30]{UC\_30} \\
	\hline
	\phantomsection
	\label{rf-de_83}
	RF-DE\_83 & Il Cliente deve poter richiedere la duplicazione indicando il codice dell'ordine dentro il comando comando \vr{/duplica\{xx\}}. & \hyperref[uc_34]{UC\_34} \\
	\hline
	\phantomsection
	\label{rf-de_84}
	RF-DE\_84 & Il Cliente deve poter richiedere la duplicazione dell'ultimo ordine effettuato indicando il comando "/duplica" & \hyperref[uc_34]{UC\_34} \\
	\hline
	\phantomsection
	\label{rf-de_85}
	RF-DE\_85 & Il Cliente deve poter visualizzare una notifica di conferma della duplicazione. & \hyperref[uc_34.2]{UC\_34.2} \\
	\hline
	\phantomsection
	\label{rf-de_86}
	RF-DE\_86 & L'Admin deve poter visualizzare le statistiche & \hyperref[uc_40]{UC\_40} \\
	\hline
	\phantomsection
	\label{rf-de_87}
	RF-DE\_87 & L'Admin deve poter visualizzare il numero di utenti presenti in quel momento & \hyperref[uc_40.1]{UC\_40.1} \\
	\hline
	\phantomsection
	\label{rf-de_88}
	RF-DE\_88 & L'Admin deve poter visualizzare il numero di acquisti completati tramite la piattaforma & \hyperref[uc_40.2]{UC\_40.2} \\
	\hline
	\phantomsection
	\label{rf-de_89}	
	RF-DE\_89 & L'Admin deve poter visualizzare, in forma di testo, il modello AI utilizzato in quel momento nella pagina delle performance & \hyperref[uc_40.3]{UC\_40.3} \\
	\hline
	\phantomsection
	\label{rf-de_90}	
	RF-DE\_90 & L'Utente deve poter visualizzare l'elenco dei comandi in una pagina web dedicata & \hyperref[uc_41]{UC\_41} \\
	\hline
	\phantomsection
	\label{rf-de_91}	
	RF-DE\_91 & L'Utente deve poter visualizzare il pulsante contenente il per la pagina dell'elenco dei comandi in qualsiasi pagina & \hyperref[uc_41]{UC\_41} \\
	\hline
	\phantomsection
	\label{rf-de_92}	
	RF-DE\_92 & L'Utente deve poter schiacciare il pulsante contenente il link per la pagina dell'elenco dei comandi in qualsiasi pagina & \hyperref[uc_41]{UC\_41} \\
	\hline
	\phantomsection
	\label{rf-de_93}	
	RF-DE\_93 & Il Cliente deve poter visualizzare l'elenco dei comandi con la relativa spiegazione & \hyperref[uc_41]{UC\_41} \\
	\hline
	\phantomsection
	\label{rf-de_94}	
	RF-DE\_94 & Il Cliente deve poter visualizzare l'elenco dei comandi con la relativa spiegazione & \hyperref[uc_42]{UC\_42} \\
	\hline
	\phantomsection
	\label{rf-de_95}	
	RF-DE\_95 & Il Cliente deve poter visualizzare il comando \vr{/duplica} con la relativa spiegazione & \hyperref[uc_41.1]{UC\_41.1}\newline \hyperref[uc_42.1]{UC\_42.1} \\
	\hline
	\phantomsection
	\label{rf-de_96}	
	RF-DE\_96 & Il Cliente deve poter visualizzare il comando \vr{/carrello} con la relativa spiegazione & \hyperref[uc_41.2]{UC\_41.2}\newline \hyperref[uc_42.2]{UC\_42.2} \\
	\hline
	\phantomsection
	\label{rf-de_97}	
	RF-DE\_97 & Il Cliente deve poter visualizzare il comando \vr{/duplica\{xx\}} con la relativa spiegazione & \hyperref[uc_41.3]{UC\_41.3}\newline \hyperref[uc_42.3]{UC\_42.3} \\
	\hline
	\phantomsection
	\label{rf-de_98}	
	RF-DE\_98 & Il Cliente deve poter visualizzare il comando \vr{/invia} con la relativa spiegazione & \hyperref[uc_41.4]{UC\_41.4}\newline \hyperref[uc_42.4]{UC\_42.4} \\
	\hline
	\phantomsection
	\label{rf-de_99}	
	RF-DE\_99 & Il Cliente deve poter visualizzare il comando \vr{/annulla} con la relativa spiegazione & \hyperref[uc_41.5]{UC\_41.5}\newline \hyperref[uc_42.5]{UC\_42.5} \\
	\hline
	\phantomsection
	\label{rf-de_100}	
	RF-DE\_100 & Il Cliente deve poter visualizzare il comando \vr{/comandi} con la relativa spiegazione & \hyperref[uc_41.6]{UC\_41.6}\newline \hyperref[uc_42.6]{UC\_42.6} \\
	\hline
	\phantomsection
	\label{rf-de_101}	
	RF-DE\_101 & Il Cliente deve poter visualizzare i comandi inline & \hyperref[uc_43]{UC\_43} \\
	\hline
	\phantomsection
	\label{rf-de_102}	
	RF-DE\_102 & Il Cliente deve poter visualizzare il menù a comparsa dei comandi disponibili & \hyperref[uc_43.1]{UC\_43.1} \\
	\hline
	\phantomsection
	\label{rf-de_103}	
	RF-DE\_103 & Il Cliente deve poter visualizzare il comando \vr{/duplica} nel menù a comparsa & \hyperref[uc_43.1.1]{UC\_43.1.1} \\
	\hline
	\phantomsection
	\label{rf-de_104}	
	RF-DE\_104 & Il Cliente deve poter visualizzare il comando \vr{/carrello} nel menù a comparsa & \hyperref[uc_43.1.2]{UC\_43.1.2} \\
	\hline
	\phantomsection
	\label{rf-de_105}	
	RF-DE\_105 & Il Cliente deve poter visualizzare il comando \vr{/duplica\{xx\}} nel menù a comparsa & \hyperref[uc_43.1.3]{UC\_43.1.3} \\
	\hline
	\phantomsection
	\label{rf-de_106}	
	RF-DE\_106 & Il Cliente deve poter visualizzare il comando \vr{/invia} nel menù a comparsa & \hyperref[uc_43.1.4]{UC\_43.1.4} \\
	\hline
	\phantomsection
	\label{rf-de_107}	
	RF-DE\_107 & Il Cliente deve poter visualizzare il comando \vr{/annulla} nel menù a comparsa & \hyperref[uc_43.1.5]{UC\_43.1.5} \\
	\hline
	\phantomsection
	\label{rf-de_108}	
	RF-DE\_108 & Il Cliente deve poter visualizzare il comando \vr{/comandi} nel menù a comparsa & \hyperref[uc_43.1.6]{UC\_43.1.6} \\
	\hline
	\phantomsection
	\label{rf-de_109}	
	RF-DE\_109 & Il Cliente deve poter inviare un comando nella chat & \hyperref[uc_44]{UC\_44} \\
	\hline
	\phantomsection
	\label{rf-de_110}	
	RF-DE\_110 & Il Cliente deve poter inviare il comando \vr{/duplica} nella chat & \hyperref[uc_44.1]{UC\_44.1} \\
	\hline
	\phantomsection
	\label{rf-de_111}	
	RF-DE\_111 & Il Cliente deve poter inviare il comando \vr{/carrello} nella chat & \hyperref[uc_44.2]{UC\_44.2} \\
	\hline
	\phantomsection
	\label{rf-de_112}	
	RF-DE\_112 & Il Cliente deve poter inviare il comando \vr{/duplica\{xx\}} nella chat & \hyperref[uc_44.3]{UC\_44.3} \\
	\hline
	\phantomsection
	\label{rf-de_113}	
	RF-DE\_113 & Il Cliente deve poter inviare il comando \vr{/invia} nella chat & \hyperref[uc_44.4]{UC\_44.4} \\
	\hline
	\phantomsection
	\label{rf-de_114}	
	RF-DE\_114 & Il Cliente deve poter inviare il comando \vr{/annulla} nella chat & \hyperref[uc_44.5]{UC\_44.5} \\
	\hline
	\phantomsection
	\label{rf-de_115}	
	RF-DE\_115 & Il Cliente deve poter inviare il comando \vr{/comandi} nella chat & \hyperref[uc_44.6]{UC\_44.6} \\
	\hline
	\phantomsection
	\label{rf-de_116}	
	RF-DE\_116 & L'Utente deve poter impostare dei filtri & \hyperref[uc_47]{UC\_47}\newline \hyperref[uc_48]{UC\_48} \\
	\hline
	\phantomsection
	\label{rf-de_117}	
	RF-DE\_117 & L'Utente deve poter filtrare per data gli ordini presenti nello storico ordini & \hyperref[uc_47.1]{UC\_47.1}\newline \hyperref[uc_48.2]{UC\_48.2} \\
	\hline
	\phantomsection
	\label{rf-de_118}	
	RF-DE\_118 & L'Utente deve poter visualizzare l'icona del filtro data a forma di calendario & \hyperref[uc_47.1]{UC\_47.1}\newline \hyperref[uc_48.2]{UC\_48.2} \\
	\hline
	\phantomsection
	\label{rf-de_119}	
	RF-DE\_119 & L'Utente deve poter cliccare sull’icona del filtro data & \hyperref[uc_47.1]{UC\_47.1}\newline \hyperref[uc_48.2]{UC\_48.2} \\
	\hline
	\phantomsection
	\label{rf-de_120}
	RF-DE\_120 & Il filtro per data deve poter permettere di visualizzare tutti gli ordini successivi o uguali ad una data scelta & \hyperref[uc_47.1]{UC\_47.1}\newline \hyperref[uc_48.2]{UC\_48.2} \\
	\hline
	\phantomsection
	\label{rf-de_121}
	RF-DE\_121 & Il filtro per data deve poter permettere di visualizzare tutti gli ordini compresi tra due date scelte & \hyperref[uc_47.1]{UC\_47.1}\newline \hyperref[uc_48.2]{UC\_48.2} \\
	\hline	
	\label{rf-de_122}	
	RF-DE\_122 & L'Utente deve poter vedere il campo testuale del filtro prodotti & \hyperref[uc_47.2]{UC\_47.2}\newline \hyperref[uc_48.1]{UC\_48.1} \\
	\hline	
	\label{rf-de_123}	
	RF-DE\_123 & L'Utente deve poter scrivere nel campo testuale del filtro prodotti & \hyperref[uc_47.2]{UC\_47.2}\newline \hyperref[uc_48.1]{UC\_48.1} \\
	\hline	
	\label{rf-de_124}	
	RF-DE\_124 & L'Utente deve poter vedere un riquadro di prodotti presenti che iniziano o sono uguali alla sequenza di caratteri inserita & \hyperref[uc_47.2]{UC\_47.2}\newline \hyperref[uc_48.3]{UC\_48.3} \\
	\hline	
	\label{rf-de_125}	
	RF-DE\_125 & L'Utente, digitando il nome del prodotto, deve poter vedere un riquadro dei prodotti presenti che contengono quella sequenza di caratteri & \hyperref[uc_47.2]{UC\_47.2}\newline \hyperref[uc_48.3]{UC\_48.3} \\
	\hline	
	\label{rf-de_126}	
	RF-DE\_126 & L'Admin deve poter vedere il campo testuale del filtro Cliente & \hyperref[uc_48.1]{UC\_48.1} \\
	\hline	
	\label{rf-de_127}	
	RF-DE\_127 & L'Admin deve poter scrivere nel campo testuale del filtro Cliente & \hyperref[uc_48.1]{UC\_48.1} \\
	\hline
	\phantomsection
	\label{rf-de_128}	
	RF-DE\_128 & L'Admin deve poter scegliere dal riquadro dei Clienti presenti il Cliente desiderato & \hyperref[uc_48.1]{UC\_48.1} \\
	\hline	
	\label{rf-de_129}	
	RF-DE\_129 & L'Admin deve poter vedere un riquadro di Clienti presenti che iniziano con la sequenza di caratteri inserita & \hyperref[uc_48.1]{UC\_48.1} \\
	\hline
	\phantomsection
	\label{rf-de_130}	
	RF-DE\_130 & L'Admin deve poter vedere un riquadro di Clienti presenti che contengono la sequenza di caratteri inserita & \hyperref[uc_48.1]{UC\_48.1} \\
	\hline	
	\label{rf-de_131}	
	RF-DE\_131 & L'Admin deve poter scegliere dal riquadro dei Clienti presenti lo username del Cliente desiderato & \hyperref[uc_48.1]{UC\_48.1} \\
	\hline
	\multicolumn{3}{c}{} \\
	\caption{Requisiti Funzionali desiderabili}\label{tab:funzionali_desiderabili}\\
\end{longtable}

	
\subsubsection{Requisiti Funzionali opzionali}
\renewcommand{\arraystretch}{1.15}
\begin{longtable}{|>{\raggedright}p{0.15\textwidth}|>{\raggedright\arraybackslash}p{0.6\textwidth}|>{\raggedright\arraybackslash}p{0.25\textwidth}|}
	\hline
	\rowcolor[gray]{0.9}
	\textbf{Codice} & \textbf{Descrizione} & \textbf{Fonti} \\
	\hline
	\endfirsthead
	
	\hline
	\rowcolor[gray]{0.9}
	\textbf{Codice} & \textbf{Descrizione} & \textbf{Fonti}  \\
	\hline
	\endhead
	
	\hline
	\phantomsection
	\label{rf-op_1}
	RF-OP\_01 & Tutte le informazioni riguardanti l'Utente utili ai fini del logging devono essere anonimizzate & \hyperref[uc_05]{UC\_5} \\
	\hline
	\phantomsection
	\label{rf-op_2}
	RF-OP\_02 & L'Admin deve poter creare nuovi utenti & \hyperref[uc_32]{UC\_32} \\
	\hline
	\phantomsection
	\label{rf-op_3}
	RF-OP\_03 & L'Admin deve poter visualizzare il pulsante relativo alla creazione di un nuovo Utente & \hyperref[uc_32]{UC\_32} \\
	\hline
	\phantomsection
	\label{rf-op_4}
	RF-OP\_04 & L'Admin deve poter schiacciare il pulsante relativo alla creazione di un nuovo Utente & \hyperref[uc_32]{UC\_32} \\
	\hline
	\phantomsection
	\label{rf-op_5}
	RF-OP\_05 & L'Admin deve poter selezionare il ruolo di quest'ultimo & \hyperref[uc_32.1]{UC\_32.1} \\
	\hline
	\phantomsection
	\label{rf-op_6}
	RF-OP\_06 & L'Admin deve poter selezionare il ruolo Cliente & \hyperref[uc_32.1]{UC\_32.1} \\
	\hline
	\phantomsection
	\label{rf-op_7}
	RF-OP\_07 & L'Admin deve poter selezionare il ruolo Admin & \hyperref[uc_32.1]{UC\_32.1} \\
	\hline
	\phantomsection
	\label{rf-op_8}
	RF-OP\_08 & Deve essere visualizzata una notifica a schermo che informa della corretta creazione dell'Utente & \hyperref[uc_32.2]{UC\_32.2} \\
	\hline
	\phantomsection
	\label{rf-op_9}
	RF-OP\_09 & L'Admin deve poter visualizzare la notifica di errore creazione utente & \hyperref[uc_33]{UC\_33}\\
	\hline
	\multicolumn{3}{c}{} \\
	\caption{Requisiti Funzionali opzionali}\label{tab:funzionali_opzionali}\\
\end{longtable}
