\subsection{Requisiti di Qualità}
\subsubsection{Requisiti di Qualità obbligatori}
\begin{table}[h]
	\centering
	\renewcommand{\arraystretch}{1.15}
	\resizebox{\textwidth}{!}{
		\begin{tabular}{|p{0.15\textwidth}|p{0.7\textwidth}|}
			\hline
			\rowcolor[gray]{0.9}
			\textbf{Codice} & \textbf{Descrizione}\\
			\hline
            RN-OB\_1 & Il sistema deve essere modulare e scalabile per supportare
            variazioni di carico.\\
            \hline
            RN-OB\_2 & Deve garantire affidabilità nell'elaborazione, mantenendo un basso tasso di errori. \\
            \hline
            RN-OB\_3 & Deve garantire manutenibilità tramite struttura modulare, logging e separazione dei componenti. \\
            \hline
            RN-OB\_4 & Deve rispettare tempi di risposta accettabili in relazione al carico previsto (max 8s, min 4s). \\
            \hline
            RN-OB\_5 & Deve garantire sicurezza dei dati trattati (integrità, riservatezza, autenticazione). \\
            \hline
            RN-OB\_6 & Deve garantire alta disponibilità del servizio (operatività 24/7). \\
            \hline
            RN-OB\_7 & I microservizi devono poter essere aggiornati senza interrompere l'intero sistema (rolling updates). \\
			\hline
            RN-OB\_8 & Devono essere garantiti meccanismi di autenticazione sicura (password hashing, rate limiting tentativi). \\
			\hline
		\end{tabular}
	}
	\caption{Campi dei Requisiti di Qualità obbligatori}
	\vspace{0.3cm}
\end{table}
\FloatBarrier

\subsubsection{Requisiti di Qualità desiderabili}
\begin{table}[h]
	\centering
	\renewcommand{\arraystretch}{1.15}
	\resizebox{\textwidth}{!}{
		\begin{tabular}{|p{0.15\textwidth}|p{0.7\textwidth}|}
			\hline
			\rowcolor[gray]{0.9}
			\textbf{Codice} & \textbf{Descrizione}\\
			\hline
            RN-DE\_1 & Latenza ridotta anche sotto carichi elevati (scaling automatico consigliato). \\
			\hline
            RN-DE\_2 & Strumenti di monitoring dei componenti (Grafana/Kibana/Prometheus).\\ 
			\hline
            RN-DE\_3 & Supporto a meccanismi di caching per ridurre tempi di risposta in operazioni ripetute.\\ 
			\hline
            RN-DE\_4 & Logging distribuito e centralizzato per analisi avanzate.\\ 
			\hline
            RN-DE\_5 & Supporto a sistemi di alerting automatici (email, webhook, dashboard).\\ 
			\hline
            RN-DE\_6 & Possibilità di benchmark periodici sulle performance dei modelli AI.\\ 
			\hline
            RN-DE\_7 & Pubblicazione di benchmark pubblici sulle performance del sistema.\\ 
			\hline
            RN-DE\_8 & Supporto a backup e ripristino dei dati e delle configurazioni critiche.\\ 
			\hline
            RN-DE\_9 & Il sistema deve essere in grado di gestire correttamente errori e fallimenti della pipeline tramite meccanismi di recovery.\\ 
			\hline
		\end{tabular}
	}
	\caption{Campi dei Requisiti di Qualità desiderabili}
	\vspace{0.3cm}
\end{table}
\FloatBarrier

\subsubsection{Requisiti di Qualità opzionali}
\begin{table}[h]
	\centering
	\renewcommand{\arraystretch}{1.15}
	\resizebox{\textwidth}{!}{
		\begin{tabular}{|p{0.15\textwidth}|p{0.7\textwidth}|}
			\hline
			\rowcolor[gray]{0.9}
			\textbf{Codice} & \textbf{Descrizione}\\
			\hline
            RN-OP\_1 & Ottimizzazione avanzata per la riduzione dei costi computazionali.\\ 
            \hline
            RN-OP\_2 & Sistema di throttling intelligente per evitare sovraccarichi in input massivi.\\ 
            \hline
            RN-OP\_3 & Supporto multi-tenant a livello infrastrutturale (isolamento risorse).\\ 
            \hline
		\end{tabular}
	}
	\caption{Campi dei Requisiti di Qualità opzionali}
	\vspace{0.3cm}
\end{table}
\FloatBarrier