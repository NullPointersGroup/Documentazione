\subsubsection{UC\_46: Impostazioni filtro Cliente}
\label{uc_46}

\begin{figure}[H]
	\centering
	\includegraphics[width=0.8\textwidth]{AnalisiRequisiti/UC_46}
	\caption{UC\_46: Impostazione filtri Cliente}
\end{figure}

\begin{itemize}
	\item \textbf{Attore principale}: Cliente
	\item \textbf{Scenario principale}:
	\begin{enumerate}
		\item Il Cliente accede alle impostazioni dei filtri in una pagina
		\item Il Cliente imposta i filtri secondo le proprie preferenze
		\item Il Cliente imposta il filtro data $\rightarrow$ Vedi \hyperref[uc_46.1]{UC\_46.1}
		\item Il Cliente imposta il filtro prodotti $\rightarrow$ Vedi \hyperref[uc_46.2]{UC\_46.2}
	\end{enumerate}
	\item \textbf{Precondizioni}:
	\begin{itemize}
		\item Il sistema è online
		\item Il Cliente è autenticato
		\item Il Cliente si trova in una pagina che supporta i filtri
	\end{itemize}
	\item \textbf{Postcondizioni}:
	\begin{itemize}
		\item I filtri sono stati configurati secondo le preferenze del Cliente
		\item La pagina visualizzata viene aggiornata per riflettere i filtri applicati
	\end{itemize}
	\item \textbf{Inclusioni}:
	\begin{itemize}
		\item \hyperref[uc_46.1]{UC\_46.1}
		\item \hyperref[uc_46.2]{UC\_46.2}
	\end{itemize}
	\item \textbf{User Story}: Il Cliente vuole personalizzare la visualizzazione dei dati applicando filtri specifici, per trovare più facilmente le informazioni di interesse
\end{itemize}

\subsubsubsection{UC\_46.1: Imposta filtro data}
\label{uc_46.1}
\begin{itemize}
	\item \textbf{Attore principale}: Cliente
	\item \textbf{Scenario principale}:
	\begin{enumerate}
		\item Il Cliente imposta i parametri del filtro per data
	\end{enumerate}
	\item \textbf{Precondizioni}:
	\begin{itemize}
		\item Il sistema è online
		\item Il Cliente è autenticato
		\item Il Cliente sta configurando i filtri in una pagina
	\end{itemize}
	\item \textbf{Postcondizioni}
	\begin{itemize}
		\item Il filtro per data è stato impostato con i parametri scelti dal Cliente
	\end{itemize}
	\item \textbf{User Story}: Il Cliente vuole filtrare i dati per intervallo di date, per concentrarsi su periodi specifici
\end{itemize}

\subsubsubsection{UC\_46.2: Imposta filtro prodotti}
\label{uc_46.2}
\begin{itemize}
	\item \textbf{Attore principale}: Cliente
	\item \textbf{Scenario principale}:
	\begin{enumerate}
		\item Il Cliente imposta i parametri del filtro per prodotti
	\end{enumerate}
	\item \textbf{Precondizioni}:
	\begin{itemize}
		\item Il sistema è online
		\item Il Cliente è autenticato
		\item Il Cliente sta configurando i filtri in una pagina
	\end{itemize}
	\item \textbf{Postcondizioni}
	\begin{itemize}
		\item Il filtro per prodotti è stato impostato con i parametri scelti dal Cliente
	\end{itemize}
	\item \textbf{User Story}: Il Cliente vuole filtrare i dati per prodotto specifico, per rintracciare informazioni relative ad articoli particolari
\end{itemize}