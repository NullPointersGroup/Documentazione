\subsubsection{UC\_04: Segnalazione password dimenticata}
\label{uc_04}
\begin{figure}[H]
    \centering 
    \includegraphics[width=1\textwidth]{AnalisiRequisiti/UC_04.png}
    \caption{Estensione caso d'uso UC\_02: UC\_04}
\end{figure}
\begin{itemize}
    \item \textbf{Attore Principale}: Utente 
    \item \textbf{Scenario Principale}:
    \begin{itemize}
        \item L'utente seleziona l'opzione "Password dimenticata" nel sistema
    \end{itemize}
    \item \textbf{Pre-Condizioni}:
    \begin{itemize}
        \item Utente non autenticato
        \item Accesso alla pagina di login
    \end{itemize}
    \item \textbf{Post-Condizioni}:
    \begin{itemize}
        \item Richiesta di recupero password inviata al sistema
    \end{itemize}
    \item \textbf{Inclusioni}:
    \begin{itemize}
        \item UC\_04.1 (\ref{uc_04.1})
        \item UC\_04.2 (\ref{uc_04.2})
    \end{itemize}
    \item \textbf{Trigger}: L'utente vuole ripristinare la propria password
\end{itemize}

\subsubsubsection{UC\_04.1: Ricezione email con nuova password}
\label{uc_04.1}
\begin{itemize}
    \item \textbf{Attore Principale}: Utente
    \item \textbf{Scenario Principale}:
    \begin{itemize}
        \item L'utente riceve una email nel cui contenuto è presente la nuova password generata autoamticamente dal sistema
    \end{itemize}
    \item \textbf{Pre-Condizioni}:
    \begin{itemize}
        \item Utente non autenticato
        \item Procedura di recupero password avviata
        \item Email valida inserita
    \end{itemize}
    \item \textbf{Post-Condizioni}:
    \begin{itemize}
        \item Email con nuova password ricevuta
        \item Credenziali aggiornate nel sistema
    \end{itemize}
    \item \textbf{Trigger}: L'utente vuole ripristinare la propria password
\end{itemize}

\subsubsubsection{UC\_04.2: Inserimento email di contatto}
\label{uc_04.2}
\begin{itemize}
    \item \textbf{Attore Principale}: Utente
    \item \textbf{Scenario Principale}:
    \begin{itemize}
        \item L'utente inserisce la mail con cui si è registrato
    \end{itemize}
    \item \textbf{Pre-Condizioni}:
    \begin{itemize}
        \item Procedura di recupero password avviata
    \end{itemize}
    \item \textbf{Post-Condizioni}:
    \begin{itemize}
        \item Email inserita correttamente
    \end{itemize}
    \item \textbf{Scenari Alternativi}:
    \begin{itemize}
        \item L'utente ha inserito una email non valida
        \item L'utente ha inserito una email non presente nel sistema
    \end{itemize}
    \item \textbf{Estensioni}:
    \begin{itemize}
        \item UC\_04.2.1 (\ref{uc_04.2.1})
        \item UC\_04.2.2 (\ref{uc_04.2.2})
    \end{itemize}
    \item \textbf{Trigger}: L'utente vuole ripristinare la password
\end{itemize}
UC\_04.2 presenta ulteriori due estensioni, come mostrate in figura:
\begin{figure}[H]
    \centering 
    \includegraphics[width=1\textwidth]{AnalisiRequisiti/UC_04.2.png}
    \caption{Estensioni caso d'uso UC\_04.2: UC\_04.2.1, UC\_04.2.2}
\end{figure}

\subsubsubsubsection{UC\_04.2.1: Email non valida}
\label{uc_04.2.1}
\begin{itemize}
    \item \textbf{Attore Principale}: Utente
    \item \textbf{Pre-Condizioni}:
    \begin{itemize}
        \item Procedura di recupero password avviata
        \item L'utente ha inserito una mail in un formato non corretto
    \end{itemize}
    \item \textbf{Post-Condizioni}:
    \begin{itemize}
        \item Messaggio d'errore visibile all'utente
    \end{itemize}
\end{itemize}

\subsubsubsubsection{UC\_04.2.2: Email non presente}
\label{uc_04.2.2}
\begin{itemize}
    \item \textbf{Attore Principale}: Utente
    \item \textbf{Pre-Condizioni}:
    \begin{itemize}
        \item Procedura di recupero password avviata
        \item L'utente ha inserito una mail non presente nel sistema
    \end{itemize}
    \item \textbf{Post-Condizioni}:
    \begin{itemize}
        \item Messaggio d'errore visibile all'utente
    \end{itemize}
\end{itemize}