\subsubsection{UC\_36: Visualizza storico ordini}
\label{uc_36}

\begin{figure}[H]
	\centering
	\includegraphics[width=0.8\textwidth]{AnalisiRequisiti/UC_36}
	\caption{UC\_36: Visualizza storico ordini }
\end{figure}

\begin{itemize}
	\item \textbf{Attore principale}: Utente
	\item \textbf{Scenario principale}:
	\begin{enumerate}
		\item L'Utente clicca il pulsante per visualizzare lo storico ordini
	\end{enumerate}
	\item \textbf{Precondizioni}:
	\begin{itemize}
		\item Il sistema è online
		\item L'Utente si trova nella home
	\end{itemize}
	\item \textbf{Postcondizioni}:
	\begin{itemize}
		\item L'Utente si trova nella pagina web dedicata alla visualizzazione dello storico ordini
	\end{itemize}
	\item \textbf{Scenari alternativi}:
	\begin{itemize}
		\item L'Utente decide di impostare almeno un filtro per la visualizzazione di ordini specifici $\rightarrow$ Vedi \hyperref[uc_36.1]{UC\_36.1}
		\item L'Utente decide di visualizzare il dettaglio di un ordine partendo dalla pagina di visualizzazione storico ordini $\rightarrow$ Vedi \hyperref[uc_37]{UC\_37}
		\item L'Utente decide di duplicare un ordine partendo dalla pagina di visualizzazione storico ordini $\rightarrow$ Vedi \hyperref[uc_38]{UC\_38}
	\end{itemize}
	\item \textbf{Estensioni}:
	\begin{itemize}
		\item \hyperref[uc_36.1]{UC\_36.1}
		\item \hyperref[uc_37]{UC\_37}
		\item \hyperref[uc_38]{UC\_38}
	\end{itemize}
	\item \textbf{User Story}: L'Utente vuole visualizzare tutto lo storico dei suoi ordini effettuati fino a quel momento
\end{itemize}

Oltretutto UC\_36 ha un'ulteriore estensione descritta con il seguente diagramma UML, non descritto nel precedente per semplicità
\begin{figure}[H]
	\centering
	\includegraphics[width=0.8\textwidth]{AnalisiRequisiti/UC_36.1}
	\caption{UC\_36.1: Imposta filtri }
\end{figure}

\subsubsubsection{UC\_36.1: Imposta filtri}
\label{uc_36.1}
\begin{itemize}
	\item \textbf{Attore principale}: Utente
	\item \textbf{Scenario principale}:
	\begin{enumerate}
		\item L'Utente imposta almeno un filtro fra quelli disponibili
		\item L'Utente visualizza solo gli elementi che rispettano i filtri impostati
	\end{enumerate}
	\item \textbf{Precondizioni}:
	\begin{itemize}
		\item Il sistema è online
		\item L'Utente sta visualizzando la pagina di cui vuole filtrare gli elementi
	\end{itemize}
	\item \textbf{Postcondizioni}:
	\begin{itemize}
		\item L'Utente visualizza gli elementi filtrati
	\end{itemize}
	\item \textbf{Generalizzazioni}:
	\begin{itemize}
		\item \hyperref[uc_36.1.1]{UC\_36.1.1}
		\item \hyperref[uc_36.1.2]{UC\_36.1.2}
		\item \hyperref[uc_36.1.3]{UC\_36.1.3}
	\end{itemize}
	\item \textbf{User Story}: L'Utente vuole visualizzare lo storico ordini filtrato, possibili scelte di filtro sono per Cliente (accessibile solo dall'Admin), per data o per prodotti
\end{itemize}

\subsubsubsubsection{UC\_36.1.1: Seleziona filtro cliente}
\label{uc_36.1.1}
\begin{itemize}
	\item \textbf{Attore principale}: Admin
	\item \textbf{Scenario principale}:
	\begin{enumerate}
		\item L'Admin imposta il filtro Cliente
	\end{enumerate}
	\item \textbf{Precondizioni}:
	\begin{itemize}
		\item Il sistema è online
		\item L'Admin sta visualizzando lo storico ordini, filtrato o completo
	\end{itemize}
	\item \textbf{Postcondizioni}:
	\begin{itemize}
		\item L'Admin visualizza lo storico ordini di un Cliente specifico
	\end{itemize}
	\item \textbf{User Story}: L'Admin vuole visualizzare lo storico ordini relativo ad un Cliente specifico
\end{itemize}

\subsubsubsubsection{UC\_36.1.2: Seleziona filtro data}
\label{uc_36.1.2}
\begin{itemize}
	\item \textbf{Attore principale}: Utente
	\item \textbf{Scenario principale}:
	\begin{enumerate}
		\item L'Utente imposta il filtro Data
	\end{enumerate}
	\item \textbf{Precondizioni}:
	\begin{itemize}
		\item Il sistema è online
		\item L'Utente si trova in una pagina dedicata per la visualizzazione
		\item L'Utente sta visualizzando gli elementi, filtrati o completi
	\end{itemize}
	\item \textbf{Postcondizioni}:
	\begin{itemize}
		\item L'Utente visualizza solo gli elementi compresi nel range temporale scelto
	\end{itemize}
	\item \textbf{User Story}: L'Utente vuole visualizzare solo gli elementi compresi in un range temporale scelto
\end{itemize}

\subsubsubsubsection{UC\_36.1.3: Seleziona filtro prodotto}
\label{uc_36.1.3}
\begin{itemize}
	\item \textbf{Attore principale}: Utente
	\item \textbf{Scenario principale}:
	\begin{enumerate}
		\item L'Utente imposta il filtro Prodotto
	\end{enumerate}
	\item \textbf{Precondizioni}:
	\begin{itemize}
		\item Il sistema è online
		\item L'Utente si trova in una dedicata per la visualizzazione
		\item L'Utente sta visualizzando gli elementi, filtrati o completi
	\end{itemize}
	\item \textbf{Postcondizioni}:
	\begin{itemize}
		\item L'Utente visualizza gli elementi che comprendono il prodotto o i prodotti scelti, assieme agli altri filtri eventualmente inseriti
	\end{itemize}
	\item \textbf{User Story}: L'Utente vuole visualizzare solo gli elementi in cui sono presenti i prodotti scelti
\end{itemize}
