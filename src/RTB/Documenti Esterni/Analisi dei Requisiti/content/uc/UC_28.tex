\subsubsection{UC\_28: Inizio nuova chat}
\label{uc_28}

\begin{figure}[H]
	\centering
	\includegraphics[width=0.8\textwidth]{AnalisiRequisiti/UC_28}
	\caption{UC\_28: Inizio nuova chat}
\end{figure}

\begin{itemize}
	\item \textbf{Attore principale}: Cliente
	\item \textbf{Scenario principale}:
	\begin{itemize}
		\item Il Cliente visualizza il pulsante per iniziare una nuova chat
		\item Il Cliente accetta l'inizio di una nuova chat premendo il pulsante
		\item Il Cliente visualizza una nuova sessione di chat
	\end{itemize}
	\item \textbf{Scenari alternativi}:
	\begin{itemize}
		\item Il Cliente decide di non iniziare una nuova chat, la sessione corrente rimane attiva
	\end{itemize}
	\item \textbf{Precondizioni}:
	\begin{itemize}
		\item Il Cliente ha confermato l'invio di un ordine $\rightarrow$ Vedi \hyperref[uc_26]{UC\_26}
	\end{itemize}
	\item \textbf{Postcondizioni}:
	\begin{itemize}
		\item Il Cliente visualizza una nuova chat con contesto resettato.
	\end{itemize}
	\item \textbf{Trigger}: Il Cliente, dopo aver completato un ordine, vuole ricominciare da zero con una nuova chat
\end{itemize}

Il seguente diagramma di attività descrive in modo più chiaro il flusso delle azioni che il Cliente può compiere per l'invio di un ordine, evidenziando, al termine del flusso, il processo di avvio di una nuova sessione di chat o, in alternativa, il mantenimento della sessione corrente.
\begin{figure}[H]
	\centering
	\includegraphics[width=\textwidth]{AnalisiRequisiti/DA_UC28}
	\caption{Diagramma di attività: invio ordine}
\end{figure}