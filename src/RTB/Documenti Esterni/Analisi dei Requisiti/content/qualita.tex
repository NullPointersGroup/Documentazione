\subsection{Requisiti di Qualità}

\subsubsection{Requisiti di Qualità obbligatori}
\renewcommand{\arraystretch}{1.15}
\begin{longtable}{|>{\raggedright}p{0.15\textwidth}|>{\raggedright\arraybackslash}p{0.5\textwidth}|>{\raggedright\arraybackslash}p{0.3\textwidth}|}
    \hline
    \rowcolor[gray]{0.9}
    \textbf{Codice} & \textbf{Descrizione} & \textbf{Fonti} \\
    \hline
    \endfirsthead
    
    \hline
    \rowcolor[gray]{0.9}
    \textbf{Codice} & \textbf{Descrizione} & \textbf{Fonti} \\
    \hline
    \endhead
    
    RQ-OB\_1 & Il sistema deve essere modulare e scalabile per supportare variazioni di carico. & Esterno \\
    \hline
    RQ-OB\_2 & Deve garantire affidabilità nell'elaborazione, mantenendo un basso tasso di errori. & \href{https://www.math.unipd.it/~tullio/IS-1/2025/Progetto/C8.pdf}{Capitolato$^G$ di Progetto$^G$} \newline Sez. Architettura proposta \\
    \hline
    RQ-OB\_3 & Deve garantire manutenibilità tramite struttura modulare, logging e separazione dei componenti. & \href{https://www.math.unipd.it/~tullio/IS-1/2025/Progetto/C8.pdf}{Capitolato$^G$ di Progetto$^G$} \newline Sez. Architettura AI multimodale \\
    \hline
    RQ-OB\_4 & Deve rispettare tempi di risposta accettabili in relazione al carico previsto (max 8s, min 4s). & Esterno \\
    \hline
    RQ-OB\_5 & Deve garantire alta disponibilità del servizio (operatività 24/7). & Esterno \\
    \hline
    RQ-OB\_6 & I microservizi devono poter essere aggiornati senza interrompere l'intero sistema (rolling updates). & Interno \newline Esterno \\
    \hline
    RQ-OB\_7 & È necessario realizzare opportuni Test$^G$ di Unità & Interno \newline Esterno \\
    \hline
    RQ-OB\_8 & È necessario descrivere i Test$^G$ da effettuare nel Piano di Qualifica$^G$ ver 1.0.0 & Interno \\
    \hline
    RQ-OB\_9 & È necessario rispettare tutte le norme presenti nel documento Norme di Progetto$^G$ ver 1.0.0 & Interno \\
    \hline
    %RQ-OB\_10 & È necessario raggiungere gli obiettivi posti nel Piano di Qualifica$^G$ ver 1.0.0 & Interno \\
    %\hline
    \multicolumn{3}{c}{} \\
    \caption{Requisiti di Qualità obbligatori}\label{tab:qualita_obbligatori}\\
\end{longtable}

\subsubsection{Requisiti di Qualità desiderabili}
\renewcommand{\arraystretch}{1.15}
\begin{longtable}{|>{\raggedright}p{0.15\textwidth}|>{\raggedright\arraybackslash}p{0.5\textwidth}|>{\raggedright\arraybackslash}p{0.3\textwidth}|}
    \hline
    \rowcolor[gray]{0.9}
    \textbf{Codice} & \textbf{Descrizione} & \textbf{Fonti} \\
    \hline
    \endfirsthead
    
    \hline
    \rowcolor[gray]{0.9}
    \textbf{Codice} & \textbf{Descrizione} & \textbf{Fonti} \\
    \hline
    \endhead
    
    RQ-DE\_1 & Latenza ridotta anche sotto carichi elevati (scaling automatico consigliato). & Interno \newline Esterno \\
    \hline
    RQ-DE\_2 & Strumenti di monitoring dei componenti con Prometheus. & Interno \\
    \hline
    RQ-DE\_3 & Supporto a meccanismi di caching per ridurre tempi di risposta in operazioni ripetute. & Esterno \\
    \hline
    RQ-DE\_4 & Logging distribuito e centralizzato per analisi avanzate. & Interno \newline Esterno \\
    \hline
    RQ-DE\_5 & Supporto a sistemi di alerting automatici (email, webhook, dashboard). & Interno \\
    \hline
    RQ-DE\_6 & Possibilità di benchmark periodici sulle performance dei modelli AI. & Interno \newline Esterno \\
    \hline
    RQ-DE\_7 & Pubblicazione di benchmark pubblici sulle performance del sistema. & Interno \newline Esterno \\
    \hline
    RQ-DE\_8 & Supporto a backup e ripristino dei dati e delle configurazioni critiche. & Esterno \\
    \hline
    RQ-DE\_9 & Il sistema deve essere in grado di gestire correttamente errori e fallimenti della pipeline tramite meccanismi di recovery. & Esterno \\
    \hline
    
    \multicolumn{3}{c}{} \\
    \caption{Requisiti di Qualità desiderabili}\label{tab:qualita_desiderabili}\\
\end{longtable}

\subsubsection{Requisiti di Qualità opzionali}
\renewcommand{\arraystretch}{1.15}
\begin{longtable}{|>{\raggedright}p{0.15\textwidth}|>{\raggedright\arraybackslash}p{0.5\textwidth}|>{\raggedright\arraybackslash}p{0.3\textwidth}|}
    \hline
    \rowcolor[gray]{0.9}
    \textbf{Codice} & \textbf{Descrizione} & \textbf{Fonti} \\
    \hline
    \endfirsthead
    
    \hline
    \rowcolor[gray]{0.9}
    \textbf{Codice} & \textbf{Descrizione} & \textbf{Fonti} \\
    \hline
    \endhead
    
    RQ-OP\_1 & Ottimizzazione avanzata per la riduzione dei costi computazionali. & Interno \\
    \hline
    RQ-OP\_2 & Sistema di throttling intelligente per evitare sovraccarichi in input massivi. & Interno \\
    \hline
    RQ-OP\_3 & Supporto multi-tenant a livello infrastrutturale (isolamento risorse). & Interno \\
    \hline
    
    \multicolumn{3}{c}{} \\
    \caption{Requisiti di Qualità opzionali}\label{tab:qualita_opzionali}\\
\end{longtable}