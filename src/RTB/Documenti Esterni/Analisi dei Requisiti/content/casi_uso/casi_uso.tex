\section{Casi d'uso}
\subsection{Descrizione dei casi d'uso}
Ogni caso d'uso si compone di tutte le informazioni presenti nella seguente tabella, salvo i casi in cui lo specifico campo non risulta rilevante.\\

\begin{table}[h]
	\centering
	\renewcommand{\arraystretch}{1.15}
	\resizebox{\textwidth}{!}{
		\begin{tabular}{|p{0.35\textwidth}|p{0.65\textwidth}|}
			\hline
			Campo & Descrizione \\
			\hline
			Grafico UML & Rappresenta lo scenario dei Casi d'Uso in oggetto. \\
			\hline
			Attore & Rappresenta coloro che interagiscono in quel sistema, senza il controllo da parte del sistema. \\
			\hline
			Scenario principale & La sequenza ragionevole delle operazioni che l'Attore deve effettuare per portare a compimento lo scenario. \\
			\hline
			Precondizioni & Lista di elementi necessari per far sì che l'Attore possa soddisfare il Caso d'Uso in oggetto. \\
			\hline
			Postcondizioni & Lista di elementi che descrive le modifiche effettuate internamente dopo il corretto avvenimento dello scenario principale. \\
			\hline
			Scenario alternativo & Rappresenta un comportamento valido ma non principale, che devia dal flusso base a causa di condizioni diverse, errori o scelte dell’attore. \\
			\hline
			Inclusioni & Ulteriori Casi d'Uso che l'Attore deve compiere per portare a termine lo scenario. \\
			\hline
			Estensioni & Relazione che aggiunge comportamento opzionale o alternativo a un caso d’uso completo, attivata solo al verificarsi di una condizione specifica.  \\
			\hline
			User Story & Descrizione sintetica di una funzionalità dal punto di vista dell’utente, focalizzata sugli obiettivi. \\
			\hline
		\end{tabular}
	}
	\vspace{0.3cm}
\end{table}