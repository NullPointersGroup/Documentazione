\section{Introduzione}
	\subsection{Scopo del documento}
	Il presente documento di \vr{Analisi dei Requisiti$^G$} definisce in modo formale, completo e strutturato i requisiti funzionali e non funzionali del sistema software \textbf{SmartOrder}, da sviluppare nell'ambito del progetto$^G$ didattico del corso di Ingegneria del Software.\\
	Descrive i casi d'uso principali del sistema, corredati da diagrammi UML$^G$ per visualizzare le interazioni tra attori e funzionalità$^G$.\\
	Rappresenta il fondamento per le successive fasi di progettazione$^G$, implementazione, testing e validazione$^G$, garantendo l'allineamento del prodotto finale con le specifiche del proponente$^G$ \textbf{Ergon Informatica Srl} e con gli obiettivi delineati nel capitolato$^G$ C8.\\
	I requisiti identificati sono classificati secondo le seguenti priorità:
	\vspace{-0.3em}
	\begin{itemize}
	    \setlength\itemsep{-0.1em}
	    \item \textbf{Obbligatori:} essenziali per il funzionamento minimo del sistema e irrinunciabili per il proponente$^G$;
	    \item \textbf{Desiderabili:} non critici, ma in grado di apportare un valore aggiunto significativo all'utente finale;
	    \item \textbf{Opzionali:} implementabili in fasi successive o in estensioni future.
	\end{itemize}

	Il documento è redatto dal gruppo \textbf{\vr{NullPointers Group}} ed è destinato a:

	\vspace{-0.3em}
	\begin{itemize}
	    \setlength\itemsep{-0.1em}
	    \item il \textbf{Committente$^G$} (Ergon Informatica Srl), per la verifica$^G$ della corretta interpretazione delle richieste;
	    \item il \textbf{Team di Sviluppo$^G$} come linea guida per la progettazione$^G$ architetturale e la codifica;
	    \item il \textbf{Team di Verifica$^G$} per la definizione delle strategie di test$^G$ e validazione$^G$.
	\end{itemize}

    Il documento è inoltre destinato ad altre figure professionali coinvolte nello sviluppo$^G$, quali Amministratori e Responsabili di Progetto$^G$, per consentire loro di acquisire una piena comprensione delle specifiche di Sistema.

	\subsection{Prospettiva del prodotto}
	SmartOrder si propone come una piattaforma intelligente e multimodale per l'interpretazione automatica di ordini di acquisto provenienti da canali eterogenei, quali testo (email, chat), audio (chiamate, messaggi vocali) e immagini (foto, documenti), e la loro trasformazione in ordini strutturati pronti per l'inserimento in sistemi gestionali ERP$^G$.\\
    Il sistema è progettato secondo un'\textbf{architettura modulare} e \textbf{scalabile}, in grado di integrare modelli avanzati di intelligenza artificiale$^G$ (LLM$^G$, visione artificiale, speech-to-text) e di adattarsi a volumi elevati di dati mantenendo elevate prestazioni.\\
	L'obiettivo è ridurre drasticamente l'intervento umano in attività ripetitive e a basso valore aggiunto, minimizzando gli errori di interpretazione e migliorando l'efficienza$^G$ operativa e la soddisfazione del cliente finale.
	
	\subsection{Funzioni del prodotto}
	Il sistema dovrà offrire le seguenti funzionalità$^G$ principali:
	\begin{itemize}
	    \setlength\itemsep{-0.1em}
	    \item Acquisizione di input multimodali (testo, audio, immagini) da molteplici canali;
	    \item Estrazione automatica di informazioni tramite pipeline di NLP$^G$, visione artificiale e trascrizione audio;
	    \item Validazione$^G$, arricchimento semantico e normalizzazione dei dati estratti;
	    \item Fusione multimodale per un'interpretazione contestuale coerente;
	    \item Generazione di ordini strutturati in formati standard (JSON$^G$, XML$^G$) compatibili con ERP$^G$;
	    \item Integrazione con database$^G$ aziendali tramite API$^G$ REST$^G$;
	    \item Interfaccia web per il monitoraggio, la gestione e il feedback dei processi;
	    \item Meccanismi di logging e apprendimento continuo per il miglioramento del sistema.
	\end{itemize}
	I requisiti obbligatori sono stati definiti in accordo con le indicazioni del proponente$^G$ e con quanto emerso dai colloqui preliminari.

	\subsection{Caratteristiche dell'utente}
	Il sistema si rivolge principalmente a:
	\begin{itemize}
	    \setlength\itemsep{-0.1em}
	    \item \textbf{Operatori aziendali} addetti all'inserimento e alla validazione$^G$ degli ordini;
	    \item \textbf{Amministratori di sistema} per il monitoraggio, la configurazione e la manutenzione$^G$ della piattaforma;
	    \item \textbf{Clienti finali} che inviano ordini tramite canali non strutturati (es. email, WhatsApp$^G$).
	\end{itemize}
	Non sono richieste competenze tecniche avanzate per l'utilizzo delle funzionalità$^G$ base, mentre la configurazione avanzata e il monitoraggio sono riservati a utenti con ruolo amministrativo.

	\subsection{Definizioni, acronimi e abbreviazioni}
	Per tutti i termini tecnici, gli acronimi e le definizioni utilizzate nel documento si rimanda al \href{https://nullpointersgroup.github.io/Documentazione/output/RTB/Documentazione_interna/Glossario.pdf}{Glossario}, disponibile come documento separato. \\
	Ogni parola presente nel Glossario viene segnata come segue:
	\begin{center}
        termine
    \end{center}

	\subsection{Riferimenti}
	\subsubsection{Riferimenti normativi}
	\begin{itemize}[itemsep=5pt, parsep=5pt, label=$\scriptstyle\bullet$]
	
	\item \textbf{Norme di Progetto$^G$, versione 1.0.0} \\
	\url{https://nullpointersgroup.github.io/Documentazione/output/RTB/Documenti\%20Interni/Norme_di_Progetto.pdf}\\[3pt]
	\textbf{Ultima consultazione: 12 Dicembre 2025}
	
	\item \textbf{Capitolato$^G$ C8 - Ergon Informatica Srl - SmartOrder} \\
	\url{https://www.math.unipd.it/~tullio/IS-1/2025/Progetto/C8.pdf}\\[3pt]
	\textbf{Ultima consultazione: 7 Novembre 2025}
	
	\end{itemize}
	
	\subsubsection{Riferimenti informativi}
	\begin{itemize}[itemsep=5pt, parsep=5pt, label=$\scriptstyle\bullet$]
	
	\item \textbf{Lezione del prof. Tullio sull'Analisi dei Requisiti$^G$} \\
	\url{https://www.math.unipd.it/~tullio/IS-1/2025/Dispense/T05.pdf}\\[3pt]
	\textbf{Ultima consultazione: 16 Dicembre 2025}
	
	\item \textbf{Standard ISO/IEC 830:1998} \\
	\url{https://ieeexplore.ieee.org/document/720574} \\[3pt]
	\textbf{Ultima consultazione: 16 Dicembre 2025}
	
	\item \textbf{Approfondimenti standard ISO/IEC 12207:1995} \\
	\url{https://www.math.unipd.it/~tullio/IS-1/2009/Approfondimenti/ISO_12207-1995.pdf} \\[3pt]
	\textbf{Ultima consultazione: 16 Dicembre 2025}
    
    \item \textbf{Glossario, versione 1.0.0} \\
    \url{https://nullpointersgroup.github.io/Documentazione/output/RTB/Documentazione_interna/Glossario.pdf}\\[3pt]
    \textbf{Ultima consultazione: 16 Dicembre 2025}
    
	\end{itemize}