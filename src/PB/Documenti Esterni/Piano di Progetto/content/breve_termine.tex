\section{Pianificazione a breve termine}

\subsection{Introduzione}
NullPointers Group ha stabilito di procedere con un approccio Agile$^G$ allo svolgimento del progetto$^G$, definendo la durata dello Sprint$^G$ di 2 settimane per avere un'organizzazione migliore.\\
Seguendo questo principio, il gruppo si impegna a stabilire all’inizio di ogni sprint$^G$ i ruoli all'interno del gruppo, che ruotano bisettimanalmente, e le attività per le settimane successive, mantenendo però la possibilità di effettuare il cambiamento qualora le esigenze organizzative lo imponessero.\\
Per verificare la continua e corretta prosecuzione delle attività, rimarremo in contatto con l'azienda \textbf{Ergon Informatica s.r.l.} sia tramite incontri telematici su \textbf{Google Meet$^G$} sia tramite i canali di comunicazione diretti concordati.\\
Seguiranno ora le descrizioni dei vari periodi di lavoro, nelle quali verranno esposte le sezioni:
\begin{enumerate}
	\item Informazioni generali e attività da svolgere
	\item Rischi attesi
	\item Preventivo
	\item Consuntivo$^G$
	\item Aggiornamento delle risorse rimanenti
	\item Rischi incontrati
	\item Efficacia$^G$ delle strategie di gestione di mitigazione dei rischi
	\item Migliorie da attuare per le attività future
	\item Retrospettiva
\end{enumerate}

\newpage
\subsection{Requirements and Technology Baseline}

\input{sprint/sprint01}

\newpage
\subsubsection{Sprint$^G$ 2}

\subsubsubsection{Informazioni generali e attività da svolgere}

\begin{tabular}{p{0.25\textwidth} p{0.2\textwidth}}
	\textbf{Inizio} & 17/11/2025 \\
	\textbf{Fine prevista} & 01/12/2025 \\
	\textbf{Fine reale} & 01/12/2025 \\
	\textbf{Giorni di ritardo} & 0
\end{tabular}
\newline \newline 
Le nostre attività da svolgere, definite nel \underline{\href{https://nullpointersgroup.github.io/Documentazione/output/RTB/Verbali\%20Interni/2025-11-17_verbale_interno.pdf}{Verbale interno del 17/11/2025}} e nel \underline{\href{https://nullpointersgroup.github.io/Documentazione/output/RTB/Verbali\%20Interni/2025-11-24_verbale_interno.pdf}{Verbale interno del 24/11/2025}} sono:
\begin{itemize}[itemsep=5pt, parsep=1pt, label=$\scriptstyle\bullet$]
	\item Inizio stesura Piano di Progetto
	\item Continuazione stesura Norme di Progetto
	\item Stesura diario di bordo 24/11 e 1/12
	\item Aggiornamento Glossario
	\item Continuazione individuazione Casi d'Uso e Requisiti
\end{itemize}


\subsubsubsection{Rischi attesi}
I possibili rischi attesi sono:
\begin{itemize}[itemsep=5pt, parsep=1pt, label=$\scriptstyle\bullet$]
	\item RA1: un imprevisto a qualche componente
	\item RP3: rischio d'incomprensione di qualche task
\end{itemize}


\subsubsubsection{Preventivo}
Si prospetta l'utilizzo delle seguenti risorse:\newpage
\begin{table}[H]
	\centering
	% Prima tabella senza bordi per le intestazioni ruotate
	\begin{tabular}{p{0.2\textwidth}p{0.12\textwidth}p{0.12\textwidth}p{0.12\textwidth}p{0.12\textwidth}p{0.12\textwidth}p{0.12\textwidth}}
		& \rotatebox{45}{\textbf{Responsabile}} &
		\rotatebox{45}{\textbf{Amministratore}} & \rotatebox{45}{\textbf{Analista}} & \rotatebox{45}{\textbf{Programmatore}} & \rotatebox{45}{\textbf{Verificatore}} &
		\rotatebox{45}{\textbf{Progettista}} \\
	\end{tabular}
	
	\vspace{0.2cm} % Piccolo spazio tra le due tabelle
	
	% Seconda tabella con i dati e i bordi
	\begin{tabular}{|p{0.2\textwidth}|p{0.12\textwidth}|p{0.12\textwidth}|p{0.12\textwidth}|p{0.12\textwidth}|p{0.12\textwidth}|p{0.12\textwidth}|}
		\hline
		M. Mazzaretto & - & 2 & 3 & - & 1 & - \\ \hline
		T. Ceron      & - & - & 6 & - & 1 & - \\ \hline
		L. Pieripolli & - & 5 & - & - & 1 & - \\ \hline
		L. Marcuzzo   & - & - & 6 & - & 1 & - \\ \hline
		M. Brunello   & - & 5 & - & - & - & - \\ \hline
		L. Casagrande & 4 & - & - & - & - & - \\ \hline
	\end{tabular}
	\caption{Sprint$^G$ 2: Preventivo}
\end{table}

\begin{figure}[H]
	\centering
	\includegraphics[width=0.8\textwidth]{PianoProgetto/sprint02_preventivo}
	\caption{Sprint$^G$ 2: Preventivo}
\end{figure}

\subsubsubsection{Consuntivo}
In questo sprint$^G$ sono state utilizzate le seguenti risorse:

\begin{table}[H]
	\centering
	% Prima tabella senza bordi per le intestazioni ruotate
	\begin{tabular}{p{0.2\textwidth}p{0.12\textwidth}p{0.12\textwidth}p{0.12\textwidth}p{0.12\textwidth}p{0.12\textwidth}p{0.12\textwidth}}
		& \rotatebox{45}{\textbf{Responsabile}} &
		\rotatebox{45}{\textbf{Amministratore}} & \rotatebox{45}{\textbf{Analista}} & \rotatebox{45}{\textbf{Programmatore}} & \rotatebox{45}{\textbf{Verificatore}} &
		\rotatebox{45}{\textbf{Progettista}} \\
	\end{tabular}
	
	\vspace{0.2cm} % Piccolo spazio tra le due tabelle
	
	% Seconda tabella con i dati e i bordi
	\begin{tabular}{|p{0.2\textwidth}|p{0.12\textwidth}|p{0.12\textwidth}|p{0.12\textwidth}|p{0.12\textwidth}|p{0.12\textwidth}|p{0.12\textwidth}|}
		\hline
		M. Mazzaretto & - & 2.5 & 5 & - & 1 & - \\ \hline
		T. Ceron      & - & - & 5 & - & 0.5 & - \\ \hline
		L. Pieripolli & - & 3 & - & - & 0.5 & - \\ \hline
		L. Marcuzzo   & - & - & 5 & - & 0.5 & - \\ \hline
		M. Brunello   & - & 3.5 & - & - & 0.5 & - \\ \hline
		L. Casagrande & 3 & - & - & - & 1 & - \\ \hline
	\end{tabular}
	\caption{Sprint$^G$ 2: Consuntivo}
\end{table}

Si evidenzia come il processo di verifica$^G$ sia migliorato, nonostante ciò è stato necessario l'apporto di più persone. Per questo il gruppo ha l'obiettivo di migliorare i processi di verifica.

\begin{figure}[H]
	\centering
	\includegraphics[width=0.8\textwidth]{PianoProgetto/sprint02_consuntivo}
	\caption{Sprint$^G$ 2: Consuntivo}
\end{figure}

\subsubsubsection{Aggiornamento delle risorse rimanenti}

\begin{table}[H]
	\centering
	\begin{tabular}{|p{0.2\textwidth}|p{0.1\textwidth}|p{0.05\textwidth}|p{0.1\textwidth}|p{0.13\textwidth}|p{0.12\textwidth}|}
		\hline
		\rowcolor{gray!25}
		Ruolo & Costo & Ore & Costo \newline effettivo & Ore \newline rimanenti & Budget \newline rimanente \\ \hline
		Responsabile$^G$   & 30(€/h) & 3 & 90 & 44 (-3) & 1320 (-90) \\ \hline
		Amministratore$^G$ & 20(€/h) & 9 & 180 & 20 (-9) & 400 \newline (-180) \\ \hline
		Analista$^G$       & 25(€/h) & 15 & 375 & 72.5 (-15) & 1812.5 \newline (-375) \\ \hline
		Progettista$^G$    & 25(€/h) & - & - & 125 & 3125 \\ \hline
		Verificatore$^G$   & 15(€/h) & 4 & 60 & 107 (-4) & 1605 (-60) \\ \hline
		Programmatore  & 15(€/h) & - & - & 125 & 1875 \\ \hline
		Totale         & - & 31 & 705 & 493.5 (-31) & 10137.5 \newline (-705) \\ \hline
	\end{tabular}
	\caption{Sprint$^G$ 2: Aggiornamento risorse}
\end{table}

\subsubsubsection{Rischi incontrati}
Non è stato incontrato nessun rischio.

\subsubsubsection{Retrospettiva}
Sono state completate tutte le attività definite, seppur una di queste è stata continuata solo in un documento condiviso non pubblico, per questo non è ancora stato pubblicato in repo il risultato.

\newpage
\subsubsection{Sprint$^G$ 3}

\subsubsubsection{Informazioni generali e attività da svolgere}

\begin{tabular}{p{0.25\textwidth} p{0.2\textwidth}}
	\textbf{Inizio} & 01/12/2025 \\
	\textbf{Fine prevista} & 15/12/2025 \\
	\textbf{Fine reale} & \\
	\textbf{Giorni di ritardo} &
\end{tabular}
\newline \newline 
Le nostre attività da svolgere, definite nel \underline{\href{https://nullpointersgroup.github.io/Documentazione/output/RTB/Verbali\%20Interni/2025-12-01_verbale_interno.pdf}{Verbale interno del 01/12/2025}} sono:
\begin{itemize}[itemsep=5pt, parsep=1pt, label=$\scriptstyle\bullet$]
	\item Inizio stesura Piano di Qualifica
	\item Definizione metriche di qualità
	\item Continuazione stesura Norme di Progetto
	\item Stesura diario di bordo 15/12
	\item Terminazione individuazione Casi d'Uso e Requisiti
	\item Discussione con la proponente del punto sopra
\end{itemize}


\subsubsubsection{Rischi attesi}
I possibili rischi attesi sono:
\begin{itemize}[itemsep=5pt, parsep=1pt, label=$\scriptstyle\bullet$]
	\item RA1: un imprevisto a qualche componente
	\item RP3: rischio d'incomprensione di qualche task
\end{itemize}


\subsubsubsection{Preventivo}
Si prospetta l'utilizzo delle seguenti risorse:\newpage
\begin{table}[H]
	\centering
	% Prima tabella senza bordi per le intestazioni ruotate
	\begin{tabular}{p{0.2\textwidth}p{0.12\textwidth}p{0.12\textwidth}p{0.12\textwidth}p{0.12\textwidth}p{0.12\textwidth}p{0.12\textwidth}}
		& \rotatebox{45}{\textbf{Responsabile}} &
		\rotatebox{45}{\textbf{Amministratore}} & \rotatebox{45}{\textbf{Analista}} & \rotatebox{45}{\textbf{Programmatore}} & \rotatebox{45}{\textbf{Verificatore}} &
		\rotatebox{45}{\textbf{Progettista}} \\
	\end{tabular}
	
	\vspace{0.2cm} % Piccolo spazio tra le due tabelle
	
	% Seconda tabella con i dati e i bordi
	\begin{tabular}{|p{0.2\textwidth}|p{0.12\textwidth}|p{0.12\textwidth}|p{0.12\textwidth}|p{0.12\textwidth}|p{0.12\textwidth}|p{0.12\textwidth}|}
		\hline
		M. Mazzaretto & 4 & - & - & - & 1 & - \\ \hline
		T. Ceron      & - & - & 8 & - & - & - \\ \hline
		L. Pieripolli & - & - & 8 & - & - & - \\ \hline
		L. Marcuzzo   & - & 5 & - & - & 1 & - \\ \hline
		M. Brunello   & - & - & 8 & - & - & - \\ \hline
		L. Casagrande & - & 5 & - & - & 1 & - \\ \hline
	\end{tabular}
	\caption{Sprint$^G$ 3: Preventivo}
\end{table}

\begin{figure}[H]
	\centering
	\includegraphics[width=0.8\textwidth]{PianoProgetto/sprint03_preventivo}
	\caption{Sprint$^G$ 3: Preventivo}
\end{figure}

\subsubsubsection{Consuntivo}

\subsubsubsection{Aggiornamento delle risorse rimanenti}

\subsubsubsection{Rischi incontrati}

\subsubsubsection{Retrospettiva}


\newpage
\subsubsection{Sprint 4}

\subsubsubsection{Informazioni generali e attività da svolgere}

\begin{tabular}{p{0.25\textwidth} p{0.2\textwidth}}
	\textbf{Inizio} & 15/12/2025 \\
	\textbf{Fine prevista} & 29/12/2025 \\
	\textbf{Fine reale} & \\
	\textbf{Giorni di ritardo} &
\end{tabular}
\newline \newline 
Le nostre attività da svolgere, definite nel \href{https://nullpointersgroup.github$^G$.io/Documentazione/output/RTB/Verbali\%20Interni/2025-12-15\_verbale\_interno.pdf}{Verbale interno del 15/12/2025}, \href{https://nullpointersgroup.github$^G$.io/Documentazione/output/RTB/Verbali\%20Interni/2025-12-17\_verbale\_interno.pdf}{Verbale interno del 17/12/2025} e nel \href{https://nullpointersgroup.github$^G$.io/Documentazione/output/RTB/Verbali\%20Interni/2025-12-23\_verbale\_interno.pdf}{Verbale interno del 23/12/2025} sono:
\begin{itemize}[itemsep=5pt, parsep=1pt, label=$\scriptstyle\bullet$]
	\item Continuazione Analisi di Requisiti: sezione Attori, Requisiti di qualità e di vincolo
	\item Continuazione Analisi di Requisiti: sezione Casi d’Uso (1-29/36-51)
\end{itemize}


\subsubsubsection{Rischi attesi}
I possibili rischi attesi sono:
\begin{itemize}[itemsep=5pt, parsep=1pt, label=$\scriptstyle\bullet$]
	\item RP1: esame universitario a qualche componente
	\item RP2: attività extra-universitarie quali festività natalizie
	\item RP3: rischio d'incomprensione di qualche task
\end{itemize}


\subsubsubsection{Preventivo}
Si prospetta l'utilizzo delle seguenti risorse:\newpage
\begin{table}[H]
	\centering
	% Prima tabella senza bordi per le intestazioni ruotate
	\begin{tabular}{p{0.2\textwidth}p{0.12\textwidth}p{0.12\textwidth}p{0.12\textwidth}p{0.12\textwidth}p{0.12\textwidth}p{0.12\textwidth}}
		& \rotatebox{45}{\textbf{Responsabile}} &
		\rotatebox{45}{\textbf{Amministratore}} & \rotatebox{45}{\textbf{Analista}} & \rotatebox{45}{\textbf{Programmatore}} & \rotatebox{45}{\textbf{Verificatore}} &
		\rotatebox{45}{\textbf{Progettista}} \\
	\end{tabular}
	
	\vspace{0.2cm} % Piccolo spazio tra le due tabelle
	
	% Seconda tabella con i dati e i bordi
	\begin{tabular}{|p{0.2\textwidth}|p{0.12\textwidth}|p{0.12\textwidth}|p{0.12\textwidth}|p{0.12\textwidth}|p{0.12\textwidth}|p{0.12\textwidth}|}
		\hline
		M. Mazzaretto & - & - & 6 & - & - & - \\ \hline
		T. Ceron      & - & - & 6 & - & 1 & - \\ \hline
		L. Pieripolli & - & - & 6 & - & 1 & - \\ \hline
		L. Marcuzzo   & 4 & - & - & - & - & - \\ \hline
		M. Brunello   & - & - & 6 & - & 1 & - \\ \hline
		L. Casagrande & - & - & 6 & - & - & - \\ \hline
	\end{tabular}
	\caption{Sprint 4: Preventivo}
\end{table}

Il totale preventivato per lo sprint$^G$ è \textbf{915€}.\\

\begin{figure}[H]
	\centering
	\includegraphics[width=0.8\textwidth]{PianoProgetto/sprint04_preventivo}
	\caption{Sprint 4: Preventivo}
\end{figure}

\subsubsubsection{Consuntivo}

\subsubsubsection{Aggiornamento delle risorse rimanenti}

\subsubsubsection{Rischi incontrati}

\subsubsubsection{Efficacia delle strategie di gestione di mitigazione dei rischi}

\subsubsubsection{Migliorie da attuare per le attività future}

\subsubsubsection{Retrospettiva}


\newpage
\subsubsection{Sprint 5}

\subsubsubsection{Informazioni generali e attività da svolgere}

\begin{tabular}{p{0.25\textwidth} p{0.2\textwidth}}
	\textbf{Inizio} & 29/12/2025 \\
	\textbf{Fine prevista} & 12/01/2026 \\
	\textbf{Fine reale} &  12/01/2026\\
	\textbf{Giorni di ritardo} & 0
\end{tabular}
\newline \newline 
Le nostre attività da svolgere, definite nel \href{https://nullpointersgroup.github.io/Documentazione/output/RTB/Verbali\%20Interni/2025-12-29\_verbale\_interno.pdf}{Verbale interno del 29/12/2025} e nel \href{https://nullpointersgroup.github.io/Documentazione/output/RTB/Verbali\%20Interni/2026-01-05\_verbale\_interno.pdf}{Verbale interno del 05/01/2026} sono:
\begin{itemize}[itemsep=5pt, parsep=1pt, label=$\scriptstyle\bullet$]
	\item Continuazione Analisi dei Requisiti: sezione Casi d’Uso (30-35)
	\item Continuazione Analisi dei Requisiti: sezione requisiti funzionali
	\item Continuazione Analisi dei Requisiti: sezione tracciamento requisiti
\end{itemize}

\subsubsubsection{Rischi attesi}
I possibili rischi attesi sono:
\begin{itemize}[itemsep=5pt, parsep=1pt, label=$\scriptstyle\bullet$]
	\item RP1: Inizio sessione invernale
	\item RP2: attività extra-universitarie quali festività natalizie
	\item RP3: rischio d'incomprensione di qualche task$^G$
\end{itemize}

\subsubsubsection{Preventivo}
Si prospetta l'utilizzo delle seguenti risorse:\newpage
\begin{table}[H]
	\centering
	% Prima tabella senza bordi per le intestazioni ruotate
	\begin{tabular}{p{0.2\textwidth}p{0.12\textwidth}p{0.12\textwidth}p{0.12\textwidth}p{0.12\textwidth}p{0.12\textwidth}p{0.12\textwidth}}
		& \rotatebox{45}{\textbf{Responsabile$^G$}} &
		\rotatebox{45}{\textbf{Amministratore$^G$}} & \rotatebox{45}{\textbf{Analista$^G$}} & \rotatebox{45}{\textbf{Programmatore$^G$}} & \rotatebox{45}{\textbf{Verificatore$^G$}} &
		\rotatebox{45}{\textbf{Progettista$^G$}} \\
	\end{tabular}
	
	\vspace{0.2cm} % Piccolo spazio tra le due tabelle
	
	% Seconda tabella con i dati e i bordi
	\begin{tabular}{|p{0.2\textwidth}|p{0.12\textwidth}|p{0.12\textwidth}|p{0.12\textwidth}|p{0.12\textwidth}|p{0.12\textwidth}|p{0.12\textwidth}|}
		\hline
		M. Mazzaretto & - & - & 4 & - & - & - \\ \hline
		T. Ceron      & - & - & 4 & - & - & - \\ \hline
		L. Pieripolli & - & - & 3 & - & - & - \\ \hline
		L. Marcuzzo   & - & - & 5 & - & 2 & - \\ \hline
		M. Brunello   & 3 & - & - & - & - & - \\ \hline
		L. Casagrande & - & - & 2 & - & 3 & - \\ \hline
	\end{tabular}
	\caption{Sprint 5: Preventivo}
\end{table}

Il totale preventivato per lo sprint$^G$ è \textbf{615€}.\\

\begin{figure}[H]
	\centering
	\includegraphics[width=0.8\textwidth]{PianoProgetto/sprint05_preventivo}
	\caption{Sprint 5: Preventivo}
\end{figure}

\subsubsubsection{Consuntivo}
In questo sprint sono state utilizzate le seguenti risorse:
\begin{table}[H]
	\centering
	% Prima tabella senza bordi per le intestazioni ruotate
	\begin{tabular}{p{0.2\textwidth}p{0.12\textwidth}p{0.12\textwidth}p{0.12\textwidth}p{0.12\textwidth}p{0.12\textwidth}p{0.12\textwidth}}
		& \rotatebox{45}{\textbf{Responsabile$^G$}} &
		\rotatebox{45}{\textbf{Amministratore$^G$}} & \rotatebox{45}{\textbf{Analista$^G$}} & \rotatebox{45}{\textbf{Programmatore$^G$}} & \rotatebox{45}{\textbf{Verificatore$^G$}} &
		\rotatebox{45}{\textbf{Progettista$^G$}} \\
	\end{tabular}
	
	\vspace{0.2cm} % Piccolo spazio tra le due tabelle
	
	% Seconda tabella con i dati e i bordi
	\begin{tabular}{|p{0.2\textwidth}|p{0.12\textwidth}|p{0.12\textwidth}|p{0.13\textwidth}|p{0.12\textwidth}|p{0.12\textwidth}|p{0.12\textwidth}|}
		\hline
		M. Mazzaretto & - & - & 5 \textcolor{red}{(+1)} & - & - & - \\ \hline
		T. Ceron      & - & - & 3,5 \textcolor{green}{(-0.5)} & - & - & - \\ \hline
		L. Pieripolli & - & - & 3,5 \textcolor{red}{(+0.5)}& - & - & - \\ \hline
		L. Marcuzzo   & - & - & 3 \textcolor{green}{(-2)} & - & 2 & - \\ \hline
		M. Brunello   & 3 & - & - & - & - & - \\ \hline
		L. Casagrande & - & - & 4 \textcolor{red}{(+2)} & - & 4 \textcolor{green}{(+1)} & - \\ \hline
	\end{tabular}
	\caption{Sprint 5: Consuntivo}
\end{table}
In questo sprint la distribuzione delle ore non è stata accuratamente rispettata, a causa di una leggera sottostima dell'impegno richiesto per ultimare l'Analisi dei Requisiti.
\begin{figure}[H]
	\centering
	\includegraphics[width=0.8\textwidth]{PianoProgetto/sprint05_consuntivo}
	\caption{Sprint 4: Consuntivo}
\end{figure}



\subsubsubsection{Aggiornamento delle risorse rimanenti}
\begin{table}[H]
	\centering
	\begin{tabular}{|p{0.2\textwidth}|p{0.1\textwidth}|p{0.05\textwidth}|p{0.1\textwidth}|p{0.13\textwidth}|p{0.12\textwidth}|}
		\hline
		\rowcolor{gray!25}
		Ruolo & Costo & Ore & Costo \newline effettivo & Ore \newline rimanenti & Budget \newline rimanente \\ \hline
		Responsabile$^G$   & 30(€/h) & 3 & 90 & 34 \textcolor{red}{(-3)} & 1020 \textcolor{red}{(-90)} \\ \hline
		Amministratore$^G$ & 20(€/h) & - & - & 11 & 220 \\ \hline
		Analista$^G$       & 25(€/h) & 19 & 475 & 7.5 \textcolor{red}{(-19)} & 187.5 \newline \textcolor{red}{(-475)} \\ \hline
		Progettista$^G$    & 25(€/h) & - & - & 125 & 3125 \\ \hline
		Verificatore$^G$   & 15(€/h) & 6 & 90 & 95 \textcolor{red}{(-6)} & 1425 \textcolor{red}{(-95)} \\ \hline
		Programmatore$^G$  & 15(€/h) & - & - & 125 & 1875 \\ \hline
		Totale         & - & 28 & 655 & 397.5 \textcolor{red}{(-28)} & 7852.5 \newline \textcolor{red}{(-655)} \\ \hline
	\end{tabular}
	\caption{Sprint 4: Aggiornamento risorse}
\end{table}
\subsubsubsection{Rischi incontrati}
In questo sprint è stato incontrato il rischio RE2, dato dalla poca esperienza nell'attività di Analisi dei Requisiti, che ha portato a una stima errata del carico di lavoro per alcuni membri in fase di preventivo.

\subsubsubsection{Efficacia delle strategie di gestione di mitigazione dei rischi}
Per il rischio RE2 dobbiamo cercare di preventivare le ore in modo più accurato, anche se, come visto nei precedenti sprint, siamo ad un buon punto.\\
Il gruppo ha inoltre migliorato ulteriormente la gestione individuale delle task e la comunicazione interna, consentendo una gestione più efficiente del lavoro.

\subsubsubsection{Migliorie da attuare per le attività future}
È necessario migliorare la stima delle ore per evitare ulteriori sottostime o sovrastime.

\subsubsubsection{Retrospettiva}
Tutte le attività previste sono state completate nei tempi previsti.


\newpage
\subsubsection{Sprint 6}

\subsubsubsection{Informazioni generali e attività da svolgere}

\begin{tabular}{p{0.25\textwidth} p{0.2\textwidth}}
	\textbf{Inizio} & 12/01/2026 \\
	\textbf{Fine prevista} & 26/01/2026 \\
	\textbf{Fine reale} &  \\
	\textbf{Giorni di ritardo} &
\end{tabular}
\newline \newline 
Le nostre attività da svolgere, definite nel \href{https://nullpointersgroup.github.io/Documentazione/output/RTB/Verbali%20Interni/2026-01-12_verbale_interno.pdf}{Verbale interno del 12/01/2026} sono:
\begin{itemize}[itemsep=5pt, parsep=1pt, label=$\scriptstyle\bullet$]
    \item Definizione Test nelle Norme di Progetto.
    \item Scrittura Test di Sistema nel Piano di Qualifica.
    \item Scrittura Test di Regressione nel Piano di Qualifica.
    \item Aggiunta termini nel Glossario.
\end{itemize}
Inoltre, sebbene non tracciata nel verbale precedente, è emersa la seguente attività aggiuntiva:
\begin{itemize}
	\item Scrittura dei test di Accettazione nel Piano di Qualifica.
\end{itemize}

\subsubsubsection{Rischi attesi}
I possibili rischi attesi sono:
\begin{itemize}[itemsep=5pt, parsep=1pt, label=$\scriptstyle\bullet$]
	\item RP1: Sessione invernale in corso 
	\item RP3: rischio d'incomprensione di qualche task$^G$
	\item RA1: rischio di qualche imprevisto personale dovuto alle prove di esame
	\item RE1: rischio dovuto alla poca conoscenza sulle nuove tecnologie da affrontare
\end{itemize}

\subsubsubsection{Preventivo}
Si prospetta l'utilizzo delle seguenti risorse:\newpage
\begin{table}[H]
	\centering
	% Prima tabella senza bordi per le intestazioni ruotate
	\begin{tabular}{p{0.2\textwidth}p{0.12\textwidth}p{0.12\textwidth}p{0.12\textwidth}p{0.12\textwidth}p{0.12\textwidth}p{0.12\textwidth}}
		& \rotatebox{45}{\textbf{Responsabile$^G$}} &
		\rotatebox{45}{\textbf{Amministratore$^G$}} & \rotatebox{45}{\textbf{Analista$^G$}} & \rotatebox{45}{\textbf{Programmatore$^G$}} & \rotatebox{45}{\textbf{Verificatore$^G$}} &
		\rotatebox{45}{\textbf{Progettista$^G$}} \\
	\end{tabular}
	
	\vspace{0.2cm} % Piccolo spazio tra le due tabelle
	
	% Seconda tabella con i dati e i bordi
	\begin{tabular}{|p{0.2\textwidth}|p{0.12\textwidth}|p{0.12\textwidth}|p{0.12\textwidth}|p{0.12\textwidth}|p{0.12\textwidth}|p{0.12\textwidth}|}
		\hline
		M. Mazzaretto & - & - & 1 & 3 & - & - \\ \hline
		T. Ceron      & 3 & - & - & - & - & - \\ \hline
		L. Pieripolli & - & 3 & - & 3 & - & - \\ \hline
		L. Marcuzzo   & - & - & 1 & 3 & - & - \\ \hline
		M. Brunello   & - & - & - & 3 & 2 & - \\ \hline
		L. Casagrande & - & - & - & 3 & - & - \\ \hline
	\end{tabular}
	\caption{Sprint 6: Preventivo}
\end{table}

Il totale preventivato per lo sprint$^G$ è \textbf{455€}.\\

\begin{figure}[H]
	\centering
	\includegraphics[width=0.8\textwidth]{PianoProgetto/sprint06_preventivo}
	\caption{Sprint 6: Preventivo}
\end{figure}

\subsubsubsection{Consuntivo}



\subsubsubsection{Aggiornamento delle risorse rimanenti}

\subsubsubsection{Rischi incontrati}

\subsubsubsection{Efficacia delle strategie di gestione di mitigazione dei rischi}

\subsubsubsection{Migliorie da attuare per le attività future}

\subsubsubsection{Retrospettiva}


\newpage
\subsubsection{Sprint 7}

\subsubsubsection{Informazioni generali e attività da svolgere}

\begin{tabular}{p{0.25\textwidth} p{0.2\textwidth}}
	\textbf{Inizio} & 26/01/2026 \\
	\textbf{Fine prevista} & 08/02/2026 \\
	\textbf{Fine reale} &  08/02/2026\\
	\textbf{Giorni di ritardo} & 0
\end{tabular}
\newline \newline 
Le nostre attività da svolgere, definite nel \href{https://nullpointersgroup.github.io/Documentazione/output/RTB/Verbali%20Interni/2026-01-26_verbale_interno.pdf}{Verbale interno del 26/01/2026} e \href{https://nullpointersgroup.github.io/Documentazione/output/RTB/Verbali%20Interni/2026-02-04_verbale_interno.pdf}{Verbale interno del 04/02/2026} sono:
\begin{itemize}[itemsep=5pt, parsep=1pt, label=$\scriptstyle\bullet$]
    \item Inizio implementazione PoC
    \item Modifica Analisi dei Requisiti: sezione Casi d'Uso (50-53)
\end{itemize}

\subsubsubsection{Rischi attesi}
I possibili rischi attesi sono:
\begin{itemize}[itemsep=5pt, parsep=1pt, label=$\scriptstyle\bullet$]
	\item RP1: Sessione invernale in corso 
	\item RP3: rischio d'incomprensione di qualche task
	\item RA1: rischio di qualche imprevisto personale
	\item RE1: rischio dovuto alla poca conoscenza sulle nuove tecnologie da affrontare
\end{itemize}

\subsubsubsection{Preventivo}
Si prospetta l'utilizzo delle seguenti risorse:
\begin{table}[H]
	\centering
	\makebox[\textwidth][c]{%
		\begin{minipage}{1.2\textwidth}
			\centering
			% Prima tabella senza bordi per le intestazioni ruotate
			\begin{tabular}{p{0.2\textwidth}p{0.12\textwidth}p{0.12\textwidth}p{0.12\textwidth}p{0.12\textwidth}p{0.12\textwidth}p{0.12\textwidth}}
				& \rotatebox{45}{\textbf{Responsabile}} &
				\rotatebox{45}{\textbf{Amministratore}} & \rotatebox{45}{\textbf{Analista}} & \rotatebox{45}{\textbf{Programmatore}} & \rotatebox{45}{\textbf{Verificatore}} &
				\rotatebox{45}{\textbf{Progettista}} \\
			\end{tabular}
			
			\vspace{0.2cm} % Piccolo spazio tra le due tabelle
			
			% Seconda tabella con i dati e i bordi
			\begin{tabular}{|p{0.2\textwidth}|p{0.12\textwidth}|p{0.12\textwidth}|p{0.12\textwidth}|p{0.12\textwidth}|p{0.12\textwidth}|p{0.12\textwidth}|}
				\hline
				M. Mazzaretto & - & - & - & 3 & - & - \\ \hline
				T. Ceron      & - & - & - & 3 & - & - \\ \hline
				L. Pieripolli & 3 & - & - & - & - & - \\ \hline
				L. Marcuzzo   & - & - & - & 3 & - & - \\ \hline
				M. Brunello   & - & - & - & 3 & - & - \\ \hline
				L. Casagrande & - & - & - & 3 & - & - \\ \hline
			\end{tabular}
		\end{minipage}
	}
	\caption{Sprint 7: Preventivo}
\end{table}

Il totale preventivato per lo sprint è \textbf{330€}.\\

\begin{figure}[H]
	\centering
	\includegraphics[width=0.8\textwidth]{PianoProgetto/sprint07_preventivo}
	\caption{Sprint 7: Preventivo}
\end{figure}

\subsubsubsection{Consuntivo}
In questo sprint sono state utilizzate le seguenti risorse:
\begin{table}[H]
	\centering
	\makebox[\textwidth][c]{%
		\begin{minipage}{1.2\textwidth}
			\centering
			% Prima tabella senza bordi per le intestazioni ruotate
			\begin{tabular}{p{0.2\textwidth}p{0.12\textwidth}p{0.12\textwidth}p{0.12\textwidth}p{0.12\textwidth}p{0.12\textwidth}p{0.12\textwidth}}
				& \rotatebox{45}{\textbf{Responsabile}} &
				\rotatebox{45}{\textbf{Amministratore}} & \rotatebox{45}{\textbf{Analista}} & \rotatebox{45}{\textbf{Programmatore}} & \rotatebox{45}{\textbf{Verificatore}} &
				\rotatebox{45}{\textbf{Progettista}} \\
			\end{tabular}
			
			\vspace{0.2cm} % Piccolo spazio tra le due tabelle
			
			% Seconda tabella con i dati e i bordi
			\begin{tabular}{|p{0.2\textwidth}|p{0.12\textwidth}|p{0.12\textwidth}|p{0.13\textwidth}|p{0.12\textwidth}|p{0.12\textwidth}|p{0.12\textwidth}|}
				\hline
				M. Mazzaretto & - & 1\textcolor{red}{(+1)} & 1 \textcolor{red}{(+1)} & 3 & - & - \\ \hline
				T. Ceron      & - & - & - & 3 & 1 \textcolor{red}{(+1)} & - \\ \hline
				L. Pieripolli & 3 & - & - & - & - & - \\ \hline
				L. Marcuzzo   & - & - & - & 3 & - & - \\ \hline
				M. Brunello   & - & 1 \textcolor{red}{(+1)} & - & 4 \textcolor{red}{(+1)} & - & - \\ \hline
				L. Casagrande & - & - & - & 3 & - & - \\ \hline
			\end{tabular}
		\end{minipage}
	}
	\caption{Sprint 7: Consuntivo}
\end{table}
Il totale consuntivato per lo sprint è \textbf{410€}.\\

In questo sprint, il carico di lavoro effettivo ha superato leggermente le stime iniziali, in oltre il team ha speso diverse ore per la formazione individuale non inclusa nelle stime.

\begin{figure}[H]
	\centering
	\includegraphics[width=0.8\textwidth]{PianoProgetto/sprint07_consuntivo}
	\caption{Sprint 7: Consuntivo}
\end{figure}

\subsubsubsection{Aggiornamento delle risorse rimanenti}
\begin{table}[H]
	\centering
	\begin{tabular}{|p{0.2\textwidth}|p{0.1\textwidth}|p{0.05\textwidth}|p{0.1\textwidth}|p{0.15\textwidth}|p{0.17\textwidth}|}
		\hline
		\rowcolor{gray!25}
		Ruolo & Costo & Ore & Costo \newline effettivo & Ore \newline rimanenti & Budget \newline rimanente \\ \hline
		
		Responsabile
		& 30(€/h) & 3 & 90
		& 29 \textcolor{red}{(-3)}
		& 870 \textcolor{red}{(-90)} \\ \hline
		
		Amministratore
		& 20(€/h) & 2 & 40
		& 6 \textcolor{red}{(-2)}
		& 120 \textcolor{red}{(-40)} \\ \hline
		
		Analista
		& 25(€/h) & 1 & 25
		& 3.5 \textcolor{red}{(-1)}
		& 87.5 \textcolor{red}{(-25)} \\ \hline
		
		Progettista
		& 25(€/h) & - & -
		& 125
		& 3125 \\ \hline
		
		Verificatore
		& 15(€/h) & 1 & 15
		& 93 \textcolor{red}{(-1)}
		& 1395 \textcolor{red}{(-15)} \\ \hline
		
		Programmatore
		& 15(€/h) & 16 & 240
		& 109 \textcolor{red}{(-16)}
		& 1635 \textcolor{red}{(-240)} \\ \hline
		
		Totale
		& - & 23 & 410
		& 365.5 \textcolor{red}{(-23)}
		& 7232.5 \textcolor{red}{(-410)} \\ \hline
	\end{tabular}
	\caption{Sprint 7: Aggiornamento risorse}
\end{table}

\subsubsubsection{Rischi incontrati}
Durante questo sprint si è concretizzato il rischio RP1, che si è però manifestato con un'intensità inferiore rispetto al precedente periodo, grazie al fatto che il suo avveramento era già stato anticipato.
Inoltre, è emerso il rischio RE1, anch'esso già previsto nella valutazione iniziale. La sua insorgenza è stata mitigata da un maggiore impegno individuale nello studio e nell'approfondimento delle nuove tecnologie coinvolte.

\subsubsubsection{Efficacia delle strategie di gestione di mitigazione dei rischi}
Il gruppo ha stimato con buona accuratezza il monte ore da destinare al ruolo di programmatore, anche in considerazione del fatto che il periodo coincidente con la sessione d'esame ha influito sulla disponibilità temporale.

\subsubsubsection{Migliorie da attuare per le attività future}
Nei prossimi sprint, il gruppo dovrà pianificare con maggiore attenzione l'assegnazione dei ruoli necessari, assicurandosi di non sottostimare il carico di lavoro associato a ciascuno di essi.

\subsubsubsection{Retrospettiva}
Il team ha completato in modo efficiente le attività pianificate, riuscendo anche a completare il PoC. Ciò ha consentito il pieno soddisfacimento degli obiettivi prefissati nel Piano di Progetto.