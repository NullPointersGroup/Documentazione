\begin{center}
\section*{D}
\end{center}
\addcontentsline{toc}{section}{D}

\term{Database}
Un database (in italiano: base di dati) è un sistema per la conservazione e gestione dei dati. Comprende una componente fisica, che è il supporto di memorizzazione dei dati, e una componente software, detta DBMS (Database Management System), che gestisce l’accesso ai dati, ne garantisce la persistenza e fornisce un metodo sicuro ed efficiente per leggerli e modificarli.

\term{Debug}
Processo di individuazione, analisi e correzione degli errori presenti nel codice di un programma.

\term{Decisione}
Un accordo all'interno del Team, riguardante metodologie o convenzioni da adottare per soddisfare un determinato bisogno.

\term{Discord}
Piattaforma di videochiamate e messaggistica asincrona. È caratterizzata dalla possibilità di creare spazi indipendenti chiamati \textit{server}, all'interno dei quali è possibile istituire vari canali testuali (tipicamente suddivisi per argomento) e canali vocali indipendenti, utilizzati per chiamate di gruppo tra i membri del server.

\term{Distribuzione}
Processo finale dello sviluppo software in cui il prodotto viene reso disponibile in una versione stabile per l’utilizzo da parte degli utenti.

\term{Docker}
Software che permette di eseguire applicazioni in ambienti isolati chiamati container.

\term{Doxygen}
Software che permette la generazione automatica di documentazione a partire dal codice sorgente di un software.

\term{Draw.io}
Software utilizzato per la creazione di diagrammi UML.

