\section{Diario della riunione}
Discussione con l'azienda proponente riguardo il progetto SmartOrder guidata dalle nostre domande su diversi ambiti del progetto.
\\
\section*{Resoconto}
\subsection{Tecniche di tokenizzazione}
È stato richiesto un approfondimento sul funzionamento delle tecniche di tokenizzazione nell'ambito del progetto.\\
L'azienda ha spiegato che nel contesto degli LLM tutti i testi vengono tokenizzati, ovvossia suddivisi in sottoinsiemi di stringhe più piccole. \\
Questo processo consente di rappresentare qualsiasi tipo di dato in formato testuale, che gli LLM possono elaborare efficacemente.\\
La tokenizzazione permette quindi di scomporre frasi e testi in unità elementari gestibili dal modello e rappresenta il primo passaggio nell'interpretazione delle richieste del cliente finalizzata alla creazione di ordini strutturati.
\subsection{Database vettoriali}
Il gruppo ha richiesto un approfondimento sui database vettoriali e sul loro utilizzo nel progetto.\\
L'azienda ha chiarito che nell'ambito degli LLM viene affiancato un database vettoriale per la memorizzazione dei token che sono rappresentati come vettori numerici.\\
Quando viene effettuata una richiesta, il sistema la trasforma in token identificando le corrispondenze più simili. 
L'azienda ha precisato che la scelta del database vettoriale dipende dal particolare LLM utilizzato, in quanto esistono soluzioni più o meno ottimali in base al modello implementato.

\subsection{Gestione di incompletezza e ambiguità dei dati}
L'azienda proponente ha chiarito che la gestione dell'ambiguità dei dati avverrà mediante l'interfaccia web dell'applicazione. Il sistema elaborerà una proposta e, qualora non riesca a identificare con certezza il prodotto richiesto, presenterà al cliente un resoconto della situazione tramite la web app. L'interfaccia permetterà al cliente di visualizzare e modificare il carrello proposto. Nei casi più complessi, sarà possibile richiedere l'intervento di un operatore umano per fornire assistenza nella risoluzione del problema.

\subsection{Vincoli di sicurezza e protezione dei dati sensibili}
Considerando che il sistema dovrà gestire ordini e dati potenzialmente sensibili, sono state richieste informazioni circa i vincoli di sicurezza da rispettare, le misure di protezione dei dati e gli aspetti relativi alla privacy.\\
L'azienda ha specificato che tutte le comunicazioni in ingresso e in uscita dal sistema dovranno essere protette mediante protocollo HTTPS. Tutti i dati sensibili saranno gestiti in modo da proteggere i dati personali garantendo la riservatezza e l'integrità delle informazioni dell'utente.

\subsection{Test}
Per il progetto SmartOrder l'azienda proponente ha confermato che verranno forniti i dati necessari per effettuare test di stress della piattaforma, inclusi scenari simili a un carico elevato di richieste (ad esempio test tipo DDoS).
Tali test potranno essere eseguiti anche autonomamente.

\subsection{Hosting e deployment}
È stato richiesto di chiarire le modalità di hosting e messa in produzione della piattaforma, in particolare se fossero necessari server dedicati, container Docker o servizi cloud.\\
L'azienda ha specificato che la piattaforma dovrà essere sviluppata utilizzando Docker per garantire la portabilità del sistema. 

\subsection{Formazione}
Per quanto riguarda la progettazione e l’architettura, l’azienda ha dichiarato che offrirà supporto durante le fasi di sviluppo.\\
Sono già disponibili modelli e strutture di esempio che possono essere consultate per il progetto oltre che ai dataset forniti.\\
Inoltre, l'azienda si è resa disponibile sull'assistenza su diversi aspetti tecnici, inclusa la struttura Docker della piattaforma (front-end, back-end, intelligenza artificiale e database).

\subsection{Altro}
L'azienda ha chiarito che il sistema dovrà essere utilizzato solamente online ed che non è necessario il supporto e la gestione di lingue diverse dall'italiano.
